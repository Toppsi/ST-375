\section{Konklusion}
Ud fra design og implementering af hver enkelt blok og det samlet systemet er der blevet udarbejdet et system, som er i stand til at opsamle biologiske signaler fra et accelerometer og tilbagegive patienten feedback ift. patientens kropshældning. Systemet er blevet designet og implementeres således, at det er i stand til at detekterer i hvilket retning hældningen sker og ligeledes give patienten feedback på dette, så det vil være muligt for patienterne træne balancen ved udførelse af balanceøvelser. \\
Systemet er blevet designet og implementeret til at give analog og digital feedback. Den første del af systemet er en fælles analog del, som består af en opsamlingsblok, der indebære accelerometeret, offsetjustering og en forstærker. Desuden indgår en filter i den første del, inden signalet deles i en digital del samt en del til analog feedback. Til den digitale del bliver signalet sendt ind i en ADC-konverter, herefter en USB-isolator og til sidst bliver signalet opsamlet af Scopelogger, behandlet i Matlab og vist som digital feedback i form af et real-time plot. Inden den analoge feedback bliver signalet forstærket, for ikke at have et mindre arbejdsområde end $\pm90\{circ}$. Den analoge feedback gives i form at visuel og somatosensorisk feedback, hvor den visuelle er LEDer designet efter trafiklys princippet ift. patientens kropshældning. Den somatosensoriske feedback gives i form af vibration på den hånd, som svare til patientens hældningsretning og gives sammen med den gule og røde LED. \\
Systemets blokke er blevet evalueret igennem test, hvor der er blevet teste for om blokkene overholder kravene samt tolerancerne, som er opstillet for den enkelte blok. Testene viste at de enkelte blokke overholder blokkens krav. Der er desuden blevet foretaget af en test af det samlede implementeret system, hvor resultatet viste af det samlede system overholdte de overordnet krav for systemet også selvom testen viste at der forekom en større afvigelser for nogle af blokkene, end ved bloktest.   

\clearpage