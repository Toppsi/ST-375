% !TeX spellcheck = da_DK
\section{Konklusion}
I dette projekt er et biofeedbacksystem blevet udarbejdet. Dette kan opsample et biologisk signal fra et accelerometer, der informerer om en apopleksipatients kropshældning. Accelerometerets g-påvirkning ved den specifikke hældning konverteres til en elektrisk spænding, som bruges til at give en visuel og somatosensorisk feedback. Ud over den analoge del af systemet kan den digitale del af afbillede det biologiske signal, samt gemme patientdata vha. udviklet software. Apopleksipatienter kan med fordel anvende systemet til træning af balancen vha. forskellige balanceøvelser under rehabiliteringsprocessen, eftersom systemet giver information omkring kroppens position i det frontale plan. Systemet er designet således, at det måler patienternes kropshældning samt angiver, hvilken retning hældningen sker i vha. biofeedback. Ifølge studier, jævnfør \ref{MekBioFeed}, side \pageref{MekBioFeed} medfører en signifikant forbedring af balancefunktionen. Det vurderes derfor, at hvis patienterne anvender systemet i kombination med balanceøvelser under rehabiliteringsprocessen, vil dette kunne hjælpe dem til en forbedret balance. 

Et biofeedbacksystem med et accelerometer, der hjælper apopleksipatienters balancetræning under rehabilitering, kan designes ved, at systemet starter med en opsamlingsblok. Denne omfatter accelerometeret, offsetjustering og en forstærker. Herefter følger en filtreringsblok, der omfatter et lavpasfilter, inden systemet forgrenes i hhv. en analog og digital del. Næste blok i den analoge del af systemet er en tilpasningsblok, hvor signalet forstærkes og tilpasses et mindre range for hældningsgrader. Signalet ledes herefter over i feedbackblokken, der indeholder en komparatorkonfiguration tilkoblet de forskellige feedbackkomponenter, der aktiveres ved forskellige hældningsgrader og informerer patienten herom. Den visuelle feedback omfatter LED'er i en konfiguration designet efter princippet i et trafiklys, mens den somatosensoriske feedback omfatter vibration. 
I den digitale del af systemet bliver det analoge signal fra filtret sendt ind i en ADC-blok for at konvertere det til et digitalt signal. Herefter ledes signalet igennem en blok indeholdende en USB-isolator og til sidst ind i computeren, der opsamler og behandler signalet i MATLAB samt afbilleder som et realtime plot. Plottet kan herefter gemmes til senere brug og analyse af det fagkyndige personale.

Designet af systemets blokke er blevet evalueret igennem simuleringer og tests, hvor det er undersøgt, om blokkene overholder de opstillede kravspecifikationer og tolerancer. Alle blokkene er på baggrund af de udførte tests blevet godkendt til at indgå i det samlede system. Der bliver afslutningsvis foretaget en test af det samlede system, hvor resultatet viste, at de overordnede funktionelle krav for systemet er opfyldt. I denne test blev det vurderet, at der kunne tolereres større afvigelser end i de udførte tests af de enkelte blokke, da signalet her sendes igennem alle blokkene, der hver især har mindre afvigelser. For at optimere den information, som systemet giver vedrørende kropshældningen, bør tærskelværdierne i feedbackblokken evt. korrigeres for at minimere afvigelser.  

For at understøtte om dette system hjælper apopleksipatienters balancetræning under rehabiliteringen, bør yderligere forsøg med apopleksipatienter udføres, hvorfor dette kan testes.

 %Inden den analoge feedback bliver signalet forstærket, for ikke at have et mindre arbejdsområde end $\pm90\{circ}$.  

\clearpage