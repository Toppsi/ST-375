\section{Konklusion}
Igennem dette projekt er der blevet udarbejdet et biofeedbacksystem, som er i stand til at opsamle biologiske signaler i form af apopleksipatienters kropshældning vha. et accelerometer. Accelerometerets g påvirkning ved den specifikke hældning konverteres til en elektrisk spænding, hvilket vha. systemet konverteres til visuel og somatosensorisk feedback. Ud over den analoge del af kredsløbet kan den digitale del af systemet afbilde det biologiske signal, samt gemme patientdata vha. udviklet software. Apopleksipatienter kan med fordel anvende systemet til træning af balancen vha. forskellige balanceøvelser under rehabiliteringsprocessen, eftersom systemet giver information omkring kroppens position i det frontale plan. Systemet er designet således, at det måler patienternes kropshældning, samt angiver hvilken retning hældningen sker i vha. biofeedback, hvilket på baggrund af studier, jævnfør \ref{MekBioFeed} på side \pageref{MekBioFeed} medfører en signifikant forbedring af balancefunktionen. Det vurderes derfor, at hvis patienterne anvender systemet i kombination med balanceøvelser vil dette hjælpe dem til at forbedre balancenn under rehabiliteringsprocessen.  

 \\
Den analoge del af systemet starter med en opsamlingsblok, der omfatter accelerometeret, offsetjustering og en forstærker. Herefter følger en filtreringsblok der omfatter et lavpasfilter, inden systemet forgrenes i hhv. en analog og digital del. Næste trin i den analoge del af systemet er en tilpasningsblok, hvor signalet igen forstærkes, hvorefter det ledes over i feedbackblokken omfattende en komparatorkonfiguration tilkoblet de forskellige feedbackkomponenter, som udløses til patienten afhængigt af hældningsgraden. Den visuelle feedback omfatter LEDer i en konfiguration designet efter princippet i et trafiklys, imens den somatosensoriske feedback omfatter vibration på hånden i den side hældningen sker imod. 
I den digitale del af systemet bliver det analoge signal først sendt ind i en ADC-blok for at konvertere det til digital. Herefter løber signalet igennem en blok indeholdende en USB-isolator og til sidst ind i blokken for computeren, der omfatter opsamling med Scopelogger, behandling i Matlab og afbildning som et real-time plot. Plottet kan herefter gemmes til senere brug og analyse af det fagkyndige personale. \\
Designet af systemets blokke er blevet evalueret igennem simuleringer og tests, hvor det er undersøgt, om blokkene overholder de opstillede kravspecifikationer og tolerancer. Alle blokkene er på baggrund af de udførte tests blevet godkendt til at indgå i det samlede system. Der er afslutningsvis blevet foretaget en test af det samlede system, hvor resultatet viste de overordnede funktionelle krav for systemet er opfyldt. I denne test blev det vurderet, at der kunne tolereres større afvigelser end i de udførte tests af de enkelte blokke, da signalet her blev sendt igennem alle blokkene, der hver især viste mindre afvigelser. For at optimere den information systemet giver vedrørende kropshældningen bør tærskelværdierne i komparatorkonfigurationen evt. korrigeres, hvis systemet skal benyttes i sundhedssektoren for at minimere afvigelser.  

 %Inden den analoge feedback bliver signalet forstærket, for ikke at have et mindre arbejdsområde end $\pm90\{circ}$.  

\clearpage