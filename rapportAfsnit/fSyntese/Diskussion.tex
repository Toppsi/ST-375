% !TeX spellcheck = da_DK
\chapter{Syntese}
\section{Diskussion}
Projektet har til formål at udvikle et samlet system, som kan give visuelt og somatosensorisk biofeedback under rehabilitering af apopleskipatienter med balanceproblemer jævnfør \ref{formaal_anvendelse} på side \pageref{formaal_anvendelse}. Projektet er bygget op på baggrund af raske personer, da balancen er meget individuel vil der netop tages udgangspunkt i raske, hvor systemet herud fra kan videre udvikles. 

\subsubsection{Rehabilitering}
Systemet kan være problematisk at anvende til personer med apopleksipatienter, da de ofte har kognitive problemer. Det skal derfor vurderes, hvorvidt flere feedback former vil være forvirrende for patienterne herunder, hvordan de forskellige feedbackformer anvendes. Den visuelle feedback vil være nødvendig for de apopleksi patienter, hvor balancen er meget svækket, mens den somatosensorisk vil være at foretrække for patienter som er senere i rehabiliteringsforløbet og har forbedret balance. På denne måde er systemet mere anvendeligt, da LED er uafhængig af vibratorerne. Aktiveringen af de forskellige LED'er er udvalgt på baggrund af indlæring, hvilket vil sige at patienterne i forvejen kender sammensætningen af de forskellige farver på LED'erne.
LED vil påvirke patienter der er senere i rehabiliteringen mindre og derved er det den somatosensorisk feedback der benyttes. Det skal vurderes, hvilken side der er mest hensigtsmæssigt at sætte vibratorene i for, at forvirre patienterne mindst muligt. Vibratorene er i systemet sat på den side som patienter vil hælde til ved eventuelt svajning. Det kan diskuteres, hvorvidt denne måde er mest hensigtsmæssig ift. at have vibratorerne til den side som patienterne skal rette sig til. En løsning på dette vil være at sætte vibratorene omvendt når de placeres på patienten. 
I systemet er det kun muligt at anvende 5 bestemte  tærskelværdier, som symbolisere de grader patienten kan hælde til, dette giver nogle begrænsninger ift. den individuelle balance samt videre udvikling af systemet. En løsningen på dette kunne være at anvende en switch-knap  i systemet som gør det muligt at anvende systemet til flere stadier i rehabiliteringen samt flere øvelser. Dette kan bl.a. gøres ved .... Dette vil gøre at patienterne kan anvende det i saggitalplan, ved at bytte flere akserne på acceleorometeret samt ved ændring af tærskelværdier via switch-funktion ift. balanceniveau. På nuværende tidspunkt vil accelerometerets placering ikke have en påvirkning, men ved ændring fra frontalplan til saggitalplan skal der tages højde for flere faktorer som påvirkning fra hjerte og lunger samt tage højde for at accelerometeret vil sættes mere anteriort på patienten ved placering på sternum. 

\subsubsection{Implementering}
Systemet er bl.a. designet efter udført pilotforsøg \ref{Bilag:Pilotforsoeg}. Lokalets temperatur og støj fra omgivelserne vil kunne påvirke de målinger som blev målt i pilotforsøget. Dette vil bl.a. påvirke senstiteten af accelerometeret. Ydermere kan fremgangsmåden af forsøget havde betydning for afvigelser, hvor flere faktorer kan medføre dette. Selve opsætningen af accelerometeret i forbindelse med rotation vil f.eks påvirke filteret, da dæmpningen for $45$Hz vil ændres ved en mindre støj på signalet. ADC'en kunne også have påvirket de målinger der blev foretaget, da denne har en afvigelse fra det teoretiske. 
De anvendte måleapparat herunder oscilloskop, multimeter og optagelse med ADC vil kunne påvirke ift. afvigelser og aflæsning af decimaler. De forskellige apparater kan kun måle et hvis antal decimaler og derved kan denne varierer ift. teorien. Derudover kan der være afvigelse ift. de anvendte operationsforstærkere, komparatorer, modstande og kondensatorer. For operationsforstærkere og komparatorerne skal der tages højde for databladenes afvigelser, hvor modstande og kondensatorer kan måles for at opnå nogle med mindre afvigelse.
 
\subsubsection{Det samlede system}
Vurdering af testen generelt - dette afhænger af hvordan vi gør det om det er muligt at lave ændringer ved denne som kan give mindre afvigelser. Eller hvorvidt det er afvigelsen af de enkelte blokke. Der er flere faktorer som kan have indflydelse på dette bl.a. de enkelte komponenters afvigelse fra de teoretiske værdier osv....

I forbindelse med offsetjusteringen i opsamlingsblokken ville det være muligt at forstærke signalet allerede der, i stedet for efter, ved at ændre på modstandene i offsettet. Dette er ikke valgt af den årsag, at eventuelle fejl vil være nemmere at vurdere ved at holde kredsløbene adskilt.

ADC'ens afvigelse vil kunne påvirke signalet ift. den digitale signal....