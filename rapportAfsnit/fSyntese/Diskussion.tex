% !TeX spellcheck = da_DK
\chapter{Syntese}
\section{Diskussion}
Formålet med projektet er at udvikle et system, der kan give visuel og somatosensorisk biofeedback samt et digitalt output under apopleksipatienters balancetræning jævnfør afsnit \ref{formaal_anvendelse}, side \pageref{formaal_anvendelse}. Systemet er udviklet på baggrund af fastsatte hældningsgrader for risikozoner jævnfør afsnit \ref{KomparatorAfs}, side \pageref{KomparatorAfs}. På baggrund af teori, design, simulering, implementering og test af systemets enkelte kredsløb ses det, at kravspecifikationerne overholdes. Der er imidlertid nogle områder indenfor de enkelte blokke, hvor det kan overvejes, hvorvidt andre målemetoder og designs kan anvendes.

\subsection{Målemetoder og pilotforsøg}
Der er flere steder i processen, hvor de endelige resultater kan være  påvirket. Systemet og de enkelte blokke er primært designet på baggrund af det udførte pilotforsøg, jævnfør bilag \ref{Bilag:Pilotforsoeg}, side \pageref{Bilag:Pilotforsoeg}. De målte data under pilotforsøget kan være påvirket af forskellige faktorer, herunder lokalets temperatur på dagen for forsøgets udførelse samt støj fra omgivelserne. Dette kan have indflydelse på de beregnede værdier i systemets forskellige blokke.
Ydermere kan pilotforsøgets metode have væsentlig betydning ift. afvigelser i de beregnede værdier. Eksempelvis er spændingen ved de enkelte hældningsgrader udregnet ved at måle spændingen ved $0^{\circ}$ samt $\pm90^{\circ}$ og efterfølgende udregne spændingen for de enkelte hældningsgrader, hvor feedbacken skal udløses. Desuden er selve rotationsmålingerne udført ved, at en person har roteret accelerometeret med hånden til hhv. højre og venstre. Disse fejlkilder kunne have været begrænset ved anvendelse af mere præcist udstyr til opmåling af vinkler, hvor det desuden kunne være muligt at måle den præcise hældning ved de udvalgte grader frem for at udregne dem.  \\
De målte data kan desuden være påvirket af afvigelser i de anvendte måleapparater, herunder multimeter og oscilloskop. De forskellige apparater kan måle et begrænset antal decimaler, og værdierne kan således variere ift. teorien. Derudover vil der være afvigelse ift. de anvendte komponenter under implementeringen, da der arbejdes med reelle komponenter. Visse komponenter, herunder operationsforstærkere og komparatorer, har i forvejen oplyst en afvigelse i deres datablade. Operationsforstærkeren af typen TL$081$ har f.eks. en inputoffsetspænding på typisk $5$mV \cite{Corporation1995}.

Testen af feedbackblokken er foretaget ved brug af et oscilloskop. De målte tærskelværdier er sammenlignet med den værdi, hvor LED'en og vibratoren tænder, hvilket kan ses ved et spændingsfald. Oscilloskopet har indbygget en $8$-bit ADC til at afbillede målingerne digitalt. Denne konvertering gør, at oscilloskopet har en gråzone mellem målepunkterne, på ca. $0.04$V. I denne gråzone vil oscilloskopet ikke kunne vise den præcise værdi. Flere af tærskelværdierne samt spændingsfaldene for LED'erne og vibratorerne ligger i denne gråzone, hvorfor det er vanskeligt at måle den nøjagtige forskel på værdierne. Derved er testen af komparatoren foretaget ved at tilnærme tærskelværdierne samt værdien for spændingsfaldet til det nærmeste målepunkt. Metoden er problematisk ved tærskelværdier med en lavere værdi, da $0.04$V udgør en større del af tærskelværdien, hvilket ses i \tableref{Tab:test-taendsluk}, side \pageref{Tab:test-taendsluk}, hvor værdien -$2^{\circ}$ afviger med $15.72$\%. Ved denne måling er tærskelværdien -$0.2373$V, hvilket er placeret i gråzonen, som går fra $0.200$V til $0.240$V og det samme er spændingsfaldet. Her bliver spændingsfaldet tilnærmet målepunktet på $0.200$V og tærskelværdien tilnærmes $0.240$V. Tærskelværdien og værdien for spændingsfaldet ligger tæt på hinanden, men hvis de afrundes til nærmeste målepunkt, vil det betyde, at de har samme værdi på oscilloskopet, hvilket vil give en forkert aflæsning.  \\
En målemetode, hvori problematikken med oscilloskopets gråzone kan mindskes, er at opsætte et spændingstræ af tre modstande, hvor den midterste modstand byttes ud med et potentiometer. For at have en fast spænding kan spændingsreferencen på $2.5$V bruges. De to outputs fra spændingstræet kan udarbejdes således, at de danner et arbejdsområde, som manuelt kan ændres vha. potentiometeret. Et multimeter kan tilkobles for at aflæse den reelle værdi for arbejdsområdet mere nøjagtigt. F.eks. kan arbejdsområdet ligge mellem $0.200$V og $0.300$V, og da potentiometeret kan drejes $10$ omgange, justeres der pr. omgang med $0.010$V, hvorfor gråzonen bliver mindre end ved brugen af oscilloskopet. Metodens nøjagtighed afhænger af anvendte modstande i spændingstræet, spændingsreferencen, potentiometeret og multimetret. 

\subsection{Systemets design}
Der er blevet udarbejdet et system, som detekterer en kropshældning til hhv. højre eller venstre, men det kan diskuteres, hvorvidt designet af de enkelte blokke i systemet er den optimale løsning. I opsamlingsblokken justeres offsettet fra accelerometret ved $0$ g-påvirkning til $0$V, jævnfør afsnit \ref{Offset_Teori_Design}, side \pageref{Offset_Teori_Design}. Problemet med offsetjusteringen er, at accelerometerets offsetværdi kan ændre sig, hver gang det benyttes. I systemet er offsettet blevet fastsat til en bestemt værdi på baggrund af pilotforsøget. Hvis accelerometerets offsetværdi afviger fra den allerede fastsatte værdi hver gang det benyttes, vil dette påvirke tærskelværdierne. Problemet kan afhjælpes ved at benytte et højpasfilter, der designes til at dæmpe $0$Hz frekvens, hvorved offsettet dæmpes. Problemet ved denne løsning er, at det ikke er muligt at designe et ideelt højpasfilter, hvorfor det ønskede signal også kan blive dæmpet. Samtidig vil der opstå et problem, hvis patienten ikke er i bevægelse f.eks. ved en kropshældning på $8^{\circ}$, vil det ønskede signal blive dæmpet. 

Offsetkredsløbet forsynes af en referencespænding, da kredsløbet er afhængig af en konstant spænding ift. at få en korrekt offsetjustering. Hvis offsettet ikke justeres korrekt, har dette indflydelse for systemets detektion af kropshældning, da accelerometeret er centreret omkring den forkerte værdi. Ved benyttelse af en referencespænding tilføjes ekstra komponenter til det samlede system, hvilket giver risiko for afvigelser, men det vurderes at være den mest optimale løsning, da et batteri ikke kan levere en konstant spænding over tid. Der forekommer altså et gradvist fald i den spænding, som batterierne kan yde, hvormed der på et tidspunkt ikke bliver leveret en tilstrækkelig spænding til offsettet. Der er stadig en uløst problematik ift. referencespændingen, da det er batterier, der fungerer som spændingsforsyning, jævnfør afsnit \ref{subsec:Spaendingsref}, side \pageref{subsec:Spaendingsref}. Der kan tilføjes en funktion, som oplyser brugeren, når referencespændingen ikke længere fungerer efter hensigten. Problematikken kan derudover løses ved at tilkoble referencespændingen til elnettet, som forsyner med en konstant spænding. Hvis referencespændingen falder tilstrækkeligt, vil det have en påvirkning på hele systemet. Denne problemstilling gør sig også gældende for referencespændingen til feedbackblokken, hvorfor samme løsning til denne blok skal udføres. For feedbackblokken vil problematikken medføre en afvigelse i tærskelværdierne og dermed vil systemet ikke udløse feedback ved de korrekte hældningsgrader. 

Derudover designes offsetjusteringen og forstærkningen i opsamlingsblokken således, at det er to forskellige kredsløb, hvilket ikke er hensigtsmæssigt, da det tilføjer flere komponenter til systemet med afvigelser. Hvis disse to kredsløb samles i ét kredsløb, vil dette forsimple blokken og dermed begrænse afvigelser. Denne løsning kan også implementeres ved filteret, hvor forstærkeren i tilpasningsblokken kan designes som en del af filteret. Løsningen er imidlertid ikke valgt, da der i dette projekt ønskes en tydelig afgrænsning imellem de enkelte kredsløb for at gøre systemet overskueligt, hvorved identificering af eventuelle fejl vurderes at være nemmere.

For at øge systemets præcision ift. detektion af patienters kropshældning kan arbejdsområdet korrigeres. I tilpasningsblokken bliver signalet i systemet tilpasset til være $\pm3$V ved $\pm25^{\circ}$. Jævnfør afsnit \ref{Tilpasningsblok}, side \pageref{Tilpasningsblok} vurderes det, at en hældning over denne værdi ikke vil være relevant ift. at vurdere, hvorvidt patienten er faldet. Dette bør undersøges nærmere for at underbygge denne vurdering yderligere. En evt. variation fra denne fastsatte hældningsgrad vil betyde, at forstærkningsfaktoren skal ændres. Dette vil medføre, at det efterfølgende arbejdsområde vil blive enten større eller mindre, hvorfor tærskelværdierne i feedbackblokken også skal ændres.\fxnote{NTK: Løsning: variabel forstærker = potentiometer} Hvis der er evidens for at øge arbejdsområdet kan tærskelværdierne indstilles med større præcision, da en grads forskel i hældning derved vil give en større spændingsforskel. Hvorvidt dette vil mindske afvigelsen mellem den forventede og målte hældningsgrad, jævnfør afsnit \ref{samlet_systemtest_ref}, side \pageref{samlet_systemtest_ref} skal undersøges. 

\subsection{Brugervenlighed}
Det kan være kompliceret at give feedback til patienter med følger af apopleksi, da de kan have kognitive problemer jævnfør afsnit \ref{Krav_biofeedback}, side \pageref{Krav_biofeedback}. Det skal derfor vurderes, hvorvidt flere typer feedback ad gangen er hensigtsmæssigt for patienterne, samt hvordan feedbacken anvendes.  

Der bør ift. den somatosensoriske feedback vurderes hvilken side, der er mest hensigtsmæssig at udløse feedbacken i; Det kan i praksis undersøges, om det virker bedst for patienterne, at vibrationen udløses i samme side som hældningen sker til eller, om det fungerer bedre, at vibrationen sker i modsatte side. Hvis den sidste mulighed er mest optimal, vil løsningen være at sætte vibratorene omvendt på patienten. Det kan desuden overvejes, om der kan implementeres en tredje feedbackform, auditiv feedback, hvor patienten advares vha. lyde, der angiver hældningsgraden. Denne feedbackform kan umiddelbart være vanskelig at anvende, da apopleksipatienter, jævnfør afsnit \ref{Krav_biofeedback}, side \pageref{Krav_biofeedback}, ofte er ældre og derved kan have nedsat hørelse \cite{Sundhedsstyrelsen2011}. Derudover kan betydningen af lydene være svær at huske, da det ikke er et system, der i forvejen er indlært som ved den visuelle feedback. For at øge effekten af feedbacken kan systemet udvikles til at fravælge og tilvælge flere former for feedback, alt efter hvilken patienten responderer bedst på.

Softwaren er designet til at gemme en billedfil, men det kan diskuteres, hvorvidt det er mere hensigtsmæssigt at gemme samtlige datapunkters værdi. Fordelen ved at gemme alle patienternes datapunkter er, at det fagkyndige personale kan udføre yderligere beregninger for patientens øvelse, herunder gennemsnitshældning, antallet af udsving over tærskelværdier og det præcise maksimale udsving uden fald. Dette vil give mere information om patientens balancefunktion samt progression under rehabiliteringsforløbet. \\
Det er problematisk, at systemet er designet således, at patienten skal bevæge sig under øvelsen for at starte og stoppe optagelsen, da der optages unødvendige bevægelser. En løsning på denne problematik er at forsinke optagelsen med $15$ sekunder efter aktivering i interfacet, hvorved patienten har tid til at tage udgangsposition for øvelsen. Ellers kan brugeren efter øvelsen redigere på x-aksen, hvis redigering igennem MATLAB muliggøres. Det kan også være en mulighed, at stop- og startfunktion for interfacet placeres omkring accelerometeret på patienten, hvorved vedkommende selv kan trykke, når udgangspositionen for den pågældende øvelse er indtaget.

\subsection{Samlet systemtest}
I simuleringen af det samlede system aktiveres det forventede antal LED'er og vibratorer ved de enkelte tærskelværdier. Systemet er dermed simuleret og vurderet til at fungere efter hensigten, jævnfør afsnit \ref{FunkKrav}, side \pageref{FunkKrav}. Hvis afvigelsen mellem definerede og simulerede hældningsgrader undersøges, ses det, at feedbacken ikke udløses ved den korrekte hældningsgrad. Den største afvigelse ses for den røde LED i positiv retning, hvilket er problematisk, da denne LED lyser, når patienten er i fare. En afvigelse af denne størrelse er i højere grad acceptabel for den grønne LED, eftersom denne lyser, når patienten befinder sig indenfor risikozonerne. Det bør derfor undersøges og simuleres, hvorvidt en korrektion af offsettet og medkalkulering at operationsforstærkernes indbyggede offset vil have en indflydelse på det samlede system og dens detektion af definerede hældningsgrader. 

Når det samlede system implementeres og testes kan det diskuteres, hvorvidt aflæsningen af accelerometerets output er foretaget præcist, da outputtet aflæses ved forsøgsafviklernes vurdering af tidspunktet for udløst feedback. En mere præcis metode til dette vil eventuelt være at benytte et potentiometer, da der i test $1$ roteres manuelt, hvorfor spændingen fra accelerometret ikke er konstant. Dette kan under test erstatte accelerometret, hvorved det præcise output, der sendes ind i resten af systemet, kan aflæses med et multimeter.\fxnote{NTK: potentiometret sættes op i en spændingstre med to spændingsdelere på hver side, som angiver "arbejdsområdet". Der kan drejes 10 omgange på potentiometret, hvorved den kan fungere som en variabel forstærker inden for arbejdsområdet.} Derudover vil tidspunktet for udløst feedback være mere præcist, da spændingen fra potentiometeret kan holdes konstant og måles mere nøjagtigt. Ved anvendelse af potentiometeret vil der imidlertid være flere tolerancer, som skal tages højde for, da designet er mere omfattende. Ifølge de målte hældningsgrader fra implementeringen ses endnu en afvigelse ift. bestemte hældningsgrader. Jævnfør afsnit \ref{samlet_systemtest_ref}, side \pageref{samlet_systemtest_ref} måles accelerometerets offset til $1.6302$V i den samlede systemtest, hvilket afviger fra offsettet i pilotforsøget, jævnfør bilag \ref{Bilag:Pilotforsoeg}, side \pageref{Bilag:Pilotforsoeg}. Eftersom tærskelværdierne er beregnet ud fra offsettet i pilotforsøget kan disse påvirkes, hvis offsettet i accelerometeret ikke er denne værdi og kan bl.a. forklare afvigelsen på hældningsgraderne. %Men eftersom afvigelsen også ses ved simuleringen gør en anden faktor sig også gældende, såsom afvigelsen på de enkelte komponenter. 
I testen er der derudover undersøgt, hvilken effekt referencespændingen har på offsetjusteringen og dermed hældningsgraderne. Det ses, at hvis der tages højde for kalibrering af offsetjustering, er det generelt muligt at gøre afvigelsen mellem de forventede og målte hældningsgrader lavere. 

Det kan diskuteres på baggrund af den samlede systemtest af den digitale del af systemet, om programmet er brugervenligt og optimalt for det fagkyndige personale. For optimering af brugervenligheden kan der eventuelt tilføjes funktioner, som kan gøre det muligt for personalet at indtaste og hente informationer om patienten igennem programmet, herunder eksempelvis navn, CPR og dato. Derudover kan området for y-aksen evt. afgrænses, således den evt. går fra $\pm25^{\circ}$, hvorved det vil blive lettere at aflæse målingerne præcist, da det forventes, at patienten vil holde sig inden for dette afgrænset område. \\
\clearpage
%\chapter{Syntese}
%\section{Diskussion}
%Formålet med projektet er at udvikle et system, der kan give visuel og somatosensorisk biofeedback samt et digitalt output under apopleksipatienters balancetræning jævnfør afsnit \ref{formaal_anvendelse}, side \pageref{formaal_anvendelse}. Systemet er udviklet på baggrund af fastsatte hældningsgrader for risikozoner jævnfør afsnit \ref{ref:blokdiagram} på side \pageref{ref:blokdiagram}.

%\subsection{Målemetoder og pilotforsøg}
%Der er flere steder i processen, hvor de endelige resultater kan være blevet påvirket. Systemet og de enkelte blokke er primært designet på baggrund af det udførte pilotforsøg i bilag \ref{Bilag:Pilotforsoeg}, side \pageref{Bilag:Pilotforsoeg}. De målte data under pilotforsøget kan være påvirket af forskellige faktorer, herunder lokalets temperatur på dagen for forsøgets udførelse samt støj fra omgivelserne. Dette kan have indflydelse på de beregnede værdier i systemets forskellige blokke.
%Ydermere kan pilotforsøgets metode have væsentlig betydning ift. afvigelser i de beregnede værdier. Eksempelvis er spændingen ved de enkelte hældningsgrader udregnet ved at måle spændingen ved $0^{\circ}$ samt $\pm90^{\circ}$ og efterfølgende udregne spændingen for de enkelte grader, hvor feedbacken skal udløses. Desuden er selve rotationsmålingerne udført ved, at en person har roteret accelerometeret med hånden til hhv. højre og venstre. Disse fejlkilder kunne have været begrænset ved anvendelse af mere præcist udstyr til opmåling af vinkler, hvor det desuden kunne være muligt at måle den præcise hældning ved de udvalgte grader frem for at udregne den.  \\
%De målte data kan desuden være påvirket af afvigelser i de anvendte måleapparater, herunder multimeter og oscilloskop. De forskellige apparater kan kun måle et begrænset antal decimaler, og værdierne kan således variere ift. teorien. Derudover vil der være afvigelse ift. de anvendte komponenter under implementeringen, da der arbejdes med reelle komponenter. Visse komponenter, herunder operationsforstærkere og komparatorer, har i forvejen oplyst en afvigelse i deres datablade. Operationsforstærkeren af typen TL$081$ har f.eks. en inputoffsetspænding på typisk $5$mV \cite{Corporation1995}. %Andre komponenter er blevet målt, herunder modstande og kondensatorer, og en eventuel afvigelse er medkalkuleret i systemet for at minimere afvigelserne, så vidt det var muligt. 

%\subsection{Problemløsning}
%Ved design, simulering, implementering og test af systemets enkelte kredsløb ses det, at alle overholder de fastsatte krav og tolerancer. Der er imidlertid nogle områder indenfor de enkelte blokke, hvor det kan overvejes, hvorvidt andre målemetoder og designs kan anvendes.

%\subsubsection{Referencespænding til offset}
%Implementeringen af en referencespænding til forsyning af offsetkredsløbet tilføjer ekstra komponenter til det samlede system, hvilket giver større risiko for afvigelser. Dette vurderes imidlertid at være den mest optimale løsning sammenlignet med at anvende eksempelvis et batteri, da dette ikke kan afgive en konstant spænding over tid. Der er en uløst problematik ift. referencespændingen, da det er batterier, der fungerer som spændingsforsyning, jævnfør afsnit \ref{subsec:Spaendingsref} på side \pageref{subsec:Spaendingsref}. Dermed forekommer et gradvist fald i den spænding, som batterierne kan yde over tid, hvormed der med tiden heller ikke kan leveres en tilstrækkelig spænding fra referencen. Dette kan have konsekvenser ift. at få justeret offsettet korrekt. Det kan derfor være hensigtsmæssigt at tilføje en funktion, der oplyser brugeren, når referencen ikke længere kan udsende den forventede spænding. Problematikken kan løses ved at tilkoble referencespændingen til elnettet, hvorfor der forsynes med en konstant spænding. Hvis referencespændingen falder tilstrækkeligt, vil det have en påvirkning på hele systemet. %Denne funktion kunne eventuelt være aktivering af en LED eller en auditiv alarm.

%\subsubsection{Opsamling}
%I opsamlingblokken bliver offsettet fra accelerometret ved $0$g påvirkning justeret til $0$V jævnfør afsnit \ref{Offset_Teori_Design} på side \pageref{Offset_Teori_Design}. Da formålet med blokken i denne sammenhæng er at fjerne offsettet helt, kan det være en mulighed at anvende et højpasfilter, som dæmper frekvenserne fra offsettet.\fxnote{NTK: Når accelerometeret detekterer offsetværdien er påvirkningen $0$g, hvilket udgør de laveste frekvenser i signalet. Disse kan identificeres og dermed dæmpes med et højpasfilter, således at offsettets frekvenser er dæmpede på det samlede signal.} Ved denne løsning skal det medregnes, at frekvenserne ikke fjernes helt, og det kræver derfor, at de er dæmpet tilstrækkeligt til, at de ikke har indflydelse på det samlede signal. \\
%Det kan desuden overvejes, om det vil være mere optimalt at sammenlægge offsetjusteringen og forstærkningen i opsamlingsblokken til ét kredsløb, hvilket kan forsimple blokken og dermed begrænse visse afvigelser, da der indgår færre komponenter. Denne løsning er imidlertid ikke valgt, da der i dette projekt ønskes en tydelig afgrænsning imellem de enkelte kredsløb for at gøre systemet mere overskueligt, hvorved identificering af eventuelle fejl muligvis er nemmere.\\
%Accelerometeret skal under træningen være placeret øverst på sternum. Det kan i denne forbindelse overvejes, om der ved målingerne forekommer støj fra kroppen; eksempelvis i form af påvirkning fra hjerte og lunger.\fxnote{opgave: skal diskutteres noget mere.} 

%\subsubsection{Filter}
%På baggrund af \eqref{eq:orden3}, side \pageref{eq:orden} bliver der bestemt, at der skal udarbejdes et 3. ordens lavpasfilter til at dæmpe uønskede frekvenser i det opsamlede signal. Kondensatorerne i kredsløbet bliver valgt ud fra, hvad der er til rådighed i laboratoriet, og modstandene i kredsløbet bliver efterfølgende udregnet ud fra disse. I praksis kan der anvendes en reel kondensator og indsætte dennes målte værdi i ligningen istedet for den teoretiske, hvorved der muligvis vil være behov for andre modstande. \\\fxnote{Der skal diskuttere fordele og ulemper ved butter, Tschebyschev- og Besselfiltre.}%ville det have været mere hensigtsmæssigt at afprøve flere forskellige kondensatorer og vælge dem, der gav den mindste afvigelse ift. sammensætning af modstandene. \\
%Jævnfør afsnit \ref{Filter_afsnit} på side \pageref{Filter_afsnit} blev der valgt at anvende et Butterworth filter frem for de andre former for filter for at opnå maksimal fladhed i pas- og stopbåndet. Hvorvidt dette er blevet opfyldt vides via teorien, at Butterworth filtret opnår den maksimale fladhed i pas- og stopbåndet ift. f.eks. Tschebyschev- og Besselfiltre. \cite{Carter2013} Der kunne dog være forsøgt at designe et 3. ordens Tschebyschev- og Besselfilter for at påvise denne fladhed i Butterworth filtret. Derved kunne der opnås endnu en parameter, der beskriver kvaliteten af det udarbejdede filter og et mere kvalificeret valg kunne blive truffet ift. hvilket filter, der er mest optimalt til dette system. \\  
%Ved simulering af filteret med de beregnede modstande er der opstået nogle afvigelser i pasbåndsfrekvensen, som ligger over de fastsatte tolerancer. Dette problem er imidlertid blevet løst, da de udregnede modstande under simuleringen er blevet erstattet med de teoretiske værdier for de reelle modstande, som skal indgå i implementeringen af kredsløbet. Årsagen til dette er ukendt, da simulering med de præcise udregnede værdier bør give de mindste afvigelser, men det er besluttet at acceptere blokken, da de reelle modstande også kan bruges i implementeringen uden yderligere afrunding.\fxnote{men så havde de reelle værdier jo også afvigelser fra de teoretiske. Havde det ikke være mere korrekt at bruge de reelle modstandes reelle værdier?}

%\subsubsection{Tilpasning}
%I tilpasningsblokken bliver signalet i systemet tilpasset til være $\pm3$V ved $\pm25^{\circ}$. Jævnfør afsnit \ref{Forstaerker_faktor3_afs}, side \pageref{Forstaerker_faktor3_afs} vurderes det, at en hældning over denne værdi ikke vil være relevant ift. at vurdere, hvorvidt patienten er faldet. Dette kan undersøges nærmere for at underbygge denne vurdering yderligere. En eventuel variation fra denne fastsatte hældningsværdi vil betyde, at forstærkningsfaktoren skal ændres. Dette vil medføre, at det efterfølgende arbejdsområde vil blive enten større eller mindre, hvorfor tærskelværdierne i feedbackblokken også skal ændres.\fxnote{NTK: Løsning: variabel forstærker = potensiometer} Ved et større arbejdsområde vil tærskelværdierne kunne indstilles mere præcist, da en grads forskel i hældning derved vil give en større spændingsforskel.

%\subsubsection{Spændingsreference til komparator}
%Spændingsreferencen til feedbackkonfigurationen er opbygget efter de samme principper som referencespændingen til offsettet. Der opstår samme problematik som for spændingsreferencen til offsettet; spændingsforsyningen til referencen er udgjort af batterier, hvorfor referenceværdien med tiden ikke kan opretholdes. For feedbackblokken vil dette det betyde en afvigelse i tærskelværdierne i de enkelte komparatorer, hvorfor der ikke længere vil udløses feedback ved de rigtige hældningsgrader. Det vil også her være essentielt med en indikation til brugeren om, at referencespændingen ikke længere ligger på den krævede værdi. 

%\subsubsection{Feedback}
%Det kan være kompliceret at give feedback til patienter med følger af apopleksi, da de kan have kognitive problemer jævnfør afsnit \ref{Krav_biofeedback}, side \pageref{Krav_biofeedback}. Det skal derfor vurderes, hvorvidt flere typer feedback ad gangen er hensigtsmæssigt for patienterne, samt hvordan feedbacken anvendes. %Det vurderes, at den visuelle feedback vil være essentiel for patienterne i begyndelsen af rehabiliteringsforløbet. 
%For at lære systemet og dets anvendelse at kende, er den visuelle feedback opbygget ud fra samme princip, som et lyskryds, hvilket er et kendt system for patienterne. Dette skyldes, at farverne tydeligt viser, hvilket område patienten befinder sig indenfor, samt hvilken retning hældningen sker i. Desuden er feedbacken sammensat ud fra samme princip som et lyskryds, hvilket er et kendt system for patienten. Dette hjælper til at lære systemet og dets anvendelse af kende% og stiller ikke store krav til patientens evne til at fokusere på feedbacken. Den somatosensoriske feedback vil være at foretrække for patienter, som er længere i rehabiliteringsprocessen og har forbedret balance. Denne feedbackform stiller større krav til patienten end den visuelle feedback, da hældningsintervallerne ikke er tydeligt afgrænset. Derudover er der ingen feedback vedrørende patientens udgangsposition, da der ikke er vibration ved den grønne LED, hvorved patienten skal have fornemmelse for stabil udgangsposition. Desuden er der én vibrationshastighed, hvilket gør, at patienten ikke får feedback om personen er i første eller anden risikozone. De to feedbackformer er adskilt fra hinanden i systemet, hvormed det i praksis vil være muligt at vurdere hvilken feedbacktype, der skal anvendes til den enkelte patient.

%Der bør ift. den somatosensoriske feedback vurderes hvilken side, der er mest hensigtsmæssig at udløse feedbacken i; Det kan i praksis undersøges, om det virker bedst for patienterne, at vibrationen udløses i samme side som hældningen sker til eller, om det fungerer bedre, at vibrationen sker i modsatte side. Hvis den sidste mulighed er mest optimal, vil løsningen være at sætte vibratorene omvendt på patienten. Det kan desuden overvejes, om der kan implementeres en tredje feedbackform, audiotiv feedback, hvor patienten advares vha. lyde, der angiver hældningsgraden. Denne feedbackform kan umiddelbart være vanskelig at anvende, da apopleksipatienter, jævnfør afsnit \ref{Krav_biofeedback}, side \pageref{Krav_biofeedback}, ofte er ældre og derved kan have nedsat hørelse \cite{Sundhedsstyrelsen2011}. Derudover kan betydningen af lydene være svær at huske, da det ikke er et system, der i forvejen er indlært som ved den visuelle feedback.

%Testen af feedbackkonfigurationen er foretaget ved brug af oscilloskop. De målte tærskelværdier er sammenlignet med den værdi, hvor LED'en og vibratoren tænder, hvilket kan ses ved et spændingsfald. Oscilloskopet har indbygget en otte bit ADC-konverter til at afbillede målingerne digitalt. Denne konvertering gør, at oscilloskopet har en gråzone mellem målepunkterne, der er på ca. $0.04$V. I denne gråzone vil oscilloskopet ikke kunne vise den præcise værdi. Da flere af tærskelværdierne samt spændingsfaldene for LED'erne og vibratorerne ligger i denne gråzone, er det vanskeligt at måle den nøjagtige forskel på værdierne. Derved er testen af komparatoren foretaget ved at tilnærme tærskelværdierne samt værdien for spændingsfaldet til det nærmeste målepunkt. Metoden er problematisk ved tærskelværdier med en lavere værdi, da $0.04$V udgør en større del af tærskelværdien, hvilket ses i \tableref{Tab:test-taendsluk}, side \pageref{Tab:test-taendsluk}, hvor værdien  -$2^{\circ}$ afviger med $15.72$\%. Ved denne måling er tærskelværdien -$0.2373$V, hvilket er placeret i gråzonen, som går fra $0.200$V til $0.240$V og det samme er spændingsfaldet. Her bliver spændingsfaldet tilnærmet målepunktet på $0.200$V, da det vurderes, at den ligger nærmest.\fxnote{opgave: skrive hvorfor den grønne LED må have en større afvigelse end den gule og røde LED}

%En målemetode, hvori problematikken med oscilloskopets gråzone ikke gør sig gældende, er at opsætte et spændingstræ. Dette fungerer som to spændingsdelere med et potentiometer, der erstatter den midterste modstand. For at have en fast spænding kan spændingsreferencen på $2.5$V bruges. De to spændingsdeleres tærskelværdier kan udarbejdes således, at de danner et arbejdsområde, hvor det manuelt kan ændres vha. potentiometeret. Et multimeter kan tilkobles for at aflæse den reelle værdi i arbejdsområdet mere nøjagtigt. F.eks. kunne arbejdsområdet være mellem $0.200$V og $0.300$V, og da potentiometeret kan drejes $10$ omgange, vil der pr. omgang kunne justeres med $0.010$V, hvorfor gråzonen vil blive mindre end ved brugen af oscilloskopet. Metoden nøjagtighed afhænger af anvendte modstande i spændingstræet, spændingsreferencen, potentiometeret og multimetret. %Gråzonen afhænger ved denne metode af, hvor præcis værdierne er i spændingstræet. Denne præcision afhænger af modstandene der benyttes. Desuden afhænger metoden af multimeteres nøjagtighed ift. den værdi der måles, og den værdi der vises på interfacet.   

%\subsubsection{ADC}
%I resultaterne fra testen af ADC'en konkluderes det, at en for lav samplingsfrekvens ift. signalet medfører et kantet signal. Dette gøres på baggrund af de grafer, som fremkommer ved målingerne jævnfør afsnit \ref{ADC_afsnit}, side \pageref{ADC_afsnit}. For at underbygge denne konklusion kan det være en mulighed at foretage en test for aliasing.%, hvor inputsignalets frekvens øges yderligere. 
%Hvis resultatet af denne måling skal stemme overens med den nuværende konklusion, skal grafen for den nye måling være mere kantet som et udtryk for, at der opsamles med for få datapunkter til, at der kan opnås et repræsentativt resultat.

%\subsubsection{USB-isolator}
%Da der ikke er udført en decideret test af USB-isolatoren og den opfylder de opstillede krav jævnfør afsnit \ref{USB_afsnit} på side \pageref{USB_afsnit} vurderes det, at der ikke er nogle væsentlige faktorer, der bør overvejes ift. denne blok.

%\subsubsection{Software}
%For at kunne anvende softwaren kræver det, at brugeren har en computer med programmet MATLAB. Derudover skal computeren have mulighed for at åbne målingsfilen, der er gemt som en billedefil. På denne måde kan patienten eller plejepersonalet videresende data til alle, som det kunne være relevant for. %Det er ikke muligt at manipulere med grafen, da det er en billedefil, hvilket stiller en række krav til, at softwaren beregner og viser de nødvendige oplysninger for f.eks. plejepersonalet. 
%Hvorvidt det skal være muligt for fagkyndige personale at kunne redigere grafen kan diskuteres. Som softwaren er nu skal plejepersonalet benytte fotoshop for redigering af grafen, da det er en billedfil. En anden mulighed kunne være, at filen blev gemt i MATLAB, hvilket gør, at plejepersonalet kan redigere og manipulere grafen efter, at filen er gemt. Der kunne f.eks. være behov for at redigere på akserne, hvorved der kan zoomes yderligere ind på et bestemt punkt på grafen. \\
%Softwaren er designet til at gemme en billedfil, men det kan diskuteres, om det vil være hensigtsmæssigt at gemme samtlige datapunkters værdi. Det fagkyndige personale kan muligvis bruge datapunkternes værdier til at udføre yderligere beregninger for patientens øvelse, herunder gennemsnitshældning, antallet af udsving over tærskelværdier og det præcise maksimale udsving uden fald. Dette vil give mere information om patientens progression. \\
%Det vil være optimalt for patienten at have en anden person til rådighed under øvelsen, som kan starte og stoppe optagelsen. Dette skal gøres i softwarens interface, hvilket gør, at øvelsen endnu ikke kan udføres selvstændigt. En løsning på denne problematik kan være, at optagelsen forsinkes med $15$ sekunder efter aktivering i interfacet, hvorved patienten har tid til at tage udgangsposition for øvelsen. Ellers kan brugeren efter øvelsen redigere på x-aksen, hvis redigering igennem MATLAB blev en mulighed. Det kan også være en mulighed, at stop- og startfunktion for interfacet placeres omkring accelerometeret på patienten, hvorved vedkommende selv kan trykke, når udgangspositionen for den pågældende øvelse er taget.

%\subsubsection{Spændingsforsyning}
%Jævnfør afsnit \ref{Spaendingsforsying}, side \pageref{Spaendingsforsying} anvendes spændingsforsyningen til at forsyne hhv. spændingsreferencerne samt de øvrige blokke i systemet. På baggrund af teori og tests fremgår det, at det er essentielt, at spændingsforsyningen leverer den forventede spænding, da det ellers vil påvirke hele systemets funktion. Det kan dermed også være en fordel at tilføje en funktion til systemet, der informerer brugeren, når spændingsforsyningen ligger under den forventede værdi. 
 
%\subsection{Det samlede system}
%Det kan diskuteres, hvorvidt aflæsningen af accelerometerets output under den samlede test er foretaget korrekt, da outputtet aflæses ved forsøgsafviklernes vurdering af tidspunktet for udløst feedback. En mere præcis metode til dette vil eventuelt være at benytte et potentiometer, da der i test $1$ roteres manuelt, hvorfor spændingen fra accelerometret ikke er konstant. Dette kan under test erstatte accelerometret, hvorved det præcise output, der sendes ind i resten af systemet, kan aflæses med et multimeter.\fxnote{NTK: potensiometret sættes op i en spændingstre med to spændingsdelere på hver side, som angiver "arbejdsområdet". Der kan drejes 10 omgange på potensiometret, hvorved den kan fungere som en variabel forstærker inden for arbejdsområdet.} Derudover vil tidspunktet for udløst feedback være mere præcist, da spændingen fra potentiometeret kan holdes konstant og måles mere nøjagtigt. Ved anvendelse af potentiometeret vil der imidlertid være flere tolerancer, som skal tages højde for, da designet er mere omfattende. \\
%Det kan diskuteres på baggrund af den samlede systemtest af den digitale del af systemet, om programmet er brugervenligt og optimalt for det fagkyndige personale. For optimering af brugervenligheden kan der eventuelt tilføjes funktioner, som kan gøre det muligt for personalet at indtaste og hente informationer om patienten igennem programmet, herunder eksempelvis navn, CPR og dato. Derudover kan spektret for y-aksen eventuelt afgrænses, således den evt. går fra $\pm25^{\circ}$, hvorved det vil blive lettere at aflæse målingerne præcist, da det forventes, at patienten vil holde sig inden for feedbackspektret. \\
