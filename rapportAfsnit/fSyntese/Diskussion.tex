% !TeX spellcheck = da_DK
\chapter{Syntese}
\section{Diskussion}
Projektet har til formål at udvikle et samlet system, som kan give visuel og somatosensorisk biofeedback under rehabilitering af apopleskipatienter med balanceproblemer jævnfør \ref{formaal_anvendelse} på side \pageref{formaal_anvendelse}. Projektet er bygget op på baggrund af raske personer, da balancen er meget individuel. Det vurderes, at dette danner det bedste grundlag for udviklingen og systemet kan herfra videreudvikles og tilpasses de enkelte patienter. 

\subsection{Anvendelse af det udviklede system}
Systemet kan være problematisk at anvende til patienter med apopleksi, da de kan have kognitive problemer jævnfør afsnit \ref{Personlige_foelger} på side \pageref{Personlige_foelger}. Det skal derfor vurderes, hvorvidt flere feedbackformer vil være hensigtsmæssigt for patienterne samt hvordan feedbacken anvendes. Det vurderes, at den visuelle feedback vil være essentiel for patienterne i begyndelsen af rehabiliteringsforløbet. Dette skyldes, at farverne tydeligt viser hvilket område patienten befinder sig indenfor, samt hvilken retning hældningen sker i. Desuden er feedbacken sammensat ud fra samme princip som et lyskryds, hvilket er et kendt system for patienten. Dette hjælper til at lære systemet og dets anvendelse at kende, og stiller ikke store krav til patientens evne til at fokusere på feedbacken. Den somatosensoriske feedback vil være at foretrække for patienter som er længere i rehabiliteringsforløbet og har forbedret balance. Denne feedbackform stiller større krav til patienten end den visuelle feedback, da hældningsintervallerne ikke er så tydeligt afgrænset. Dermed er det vigtigt at patienterne kan fokusere og skelne de forskellige vibrationshastigheder fra hinanden. De to feedbackformer er adskilt fra hinanden i systemet, hvormed det i praksis vil være muligt at vurdere hvilken feedbacktype der skal anvendes til den enkelte patient. \fxnote{lyd}
Det skal ift. den somatosensoriske feedback vurderes, hvilken side der er mest hensigtsmæssig at udløse feedbacken i; Det kan i praksis undersøges, om det virker bedst for patienterne at vibrationen udløses i samme side som hældningen sker til, eller om det fungerer bedre, at vibrationen sker i modsatte side. Hvis den sidste mulighed er mest optimal, vil løsningen være at sætte vibratorene omvendt på patienten. 
Accelerometeret skal under træningen være placeret øverst på sternum. Det kan i denne forbindelse overvejes, om der ved målingerne forekommer støj fra kroppen; eksempelvis i form af påvirkning fra hjerte og lunger.
%I systemet er det kun muligt at anvende 5 bestemte tærskelværdier, som symbolisere de grader patienten kan hælde til, dette giver nogle begrænsninger ift. den individuelle balance samt videre udvikling af systemet. En løsningen på dette kunne være at anvende en switch-knap  i systemet som gør det muligt at anvende systemet til flere stadier i rehabiliteringen samt flere øvelser. Dette kan bl.a. gøres ved .... Dette vil gøre at patienterne kan anvende det i saggitalplan, ved at bytte flere akserne på acceleorometeret samt ved ændring af tærskelværdier via switch-funktion ift. balanceniveau. På nuværende tidspunkt vil accelerometerets placering ikke have en påvirkning, men ved ændring fra frontalplan til saggitalplan skal der tages højde for flere faktorer som påvirkning fra hjerte og lunger samt tage højde for at accelerometeret vil sættes mere anteriort på patienten ved placering på sternum. 

\subsection{Målemetoder}
Systemet er primært designet på baggrund af det udførte pilotforsøg i afsnit \ref{Bilag:Pilotforsoeg} på side \pageref{Bilag:Pilotforsoeg}. De målte data under pilotforsøget kan imidlertid være påvirket af forskellige faktorer, herunder lokalets temperatur på dagen for forsøgets udførelse samt støj fra omgivelserne. Dette kan have indflydelse på de beregnede værdier i systemets forskellige blokke.
Ydermere kan pilotforsøgets metode have væsentlig betydning ift. afvigelser i de beregnede værdier. Eksempelvis er spændingen ved de enkelte hældningsgrader udregnet ved at måle spændingen ved $90^{\circ}$ og efterfølgende udregne spændingen for de enkelte grader hvor feedbacken skal udløses. Desuden er selve rotationsmålingerne udført ved, at en person har roteret accelerometeret med hånden til hhv. højre og venstre. Der har således været flere steder i processen hvor de endelige resultater har kunnet blive påvirket. Disse fejlkilder kunne begrænses ved brug af mere præcist udstyr til opmåling af vinkler, hvor det desuden kunne være muligt at måle den præcise hældning ved de udvalgte grader, frem for at udregne den.  
De målte data kan desuden være påvirket af afvigelser i de anvendte måleapparater, herunder oscilloskop, multimeter og optagelse med ADC. De forskellige apparater kan kun måle et begrænset antal decimaler og derved kan værdierne variere ift. teorien. Betydningen af disse afvigelser vil dog variere afhængig af de målte værdiers størrelse. Derudover vil der være afvigelse ift. de anvendte komponenter under implementeringen, da der her arbejdes med reelle komponenter. Visse komponenter, herunder operationsforstærkere og komparatorer, har i forvejen oplyst en afvigelse fra de ideelle værdier i deres datablade som må accepteres. Andre komponenter, herunder modstande og kondensatorer måles og tilpasses til systemet for at minimere afvigelserne så vidt det er muligt. 

\subsection{Problemløsning}
Ved design, simulering, implementering og test af blokkene ses det, at alle overholder de fastsatte krav og tolerancer. Der er imidlertid nogle områder indenfor de enkelte blokke, hvor det kan overvejes om der kunne have været anvendt andre metoder.
\subsubsection{Opsamling}
I opsamlingblokken bliver offsettet justeret til $0$V, men da formålet med blokken i denne sammenhæng er at fjerne offsettet helt, kan det være en mulighed at anvende et højpasfilter, som dæmper frekvenserne fra offsettet.\fxnote{NTK: Når accelerometeret detekterer offsetværdien er påvirkningen $0$g, hvilket udgør de laveste frekvenser i signalet. Disse kan identificeres og dermed dæmpes med et højpasfilter, således at offsettets frekvenser er dæmpede på det samlede signal.} Ved denne løsning skal det imidlertid overvejes, at frekvenserne ikke fjernes helt, og det kræver derfor at de er dæmpet tilstrækkeligt til at de ikke har indflydelse på det samlede signal. Det kan desuden overvejes, om det vil være mere optimalt at sammenlægge offsetjusteringen og forstærkningen i opsamlingsblokken til ét kredsløb, da dette kan forsimple blokken og dermed begrænse visse afvigelser, da der indgår færre komponenter. Denne løsning er imidlertid ikke valgt, da der i dette projekt ønskes en tydelig afgrænsning imellem de enkelte kredsløb for at gøre systemet mere overskueligt, og sikre at det muligt at identificere eventuelle fejl. 
\subsubsection{Filter}
På baggrund af beregninger blev det bestemt at der skulle udarbejdes et 3. ordens lavpasfilter, for at dæmpe uønskede frekvenser i det opsamlede signal. Kondensatorerne i kredsløbet blev valgt ud fra hvad der var til rådighed i laboratoriet, og modstandene blev efterfølgende udregnet ud fra disse. I praksis ville det have været mere hensigtsmæssigt at afprøve flere forskellige kondensatorer og vælge dem, der gav den mindste afvigelse for modstandene.  


 
\subsection{Det samlede system}
Vurdering af testen generelt - dette afhænger af hvordan vi gør det om det er muligt at lave ændringer ved denne som kan give mindre afvigelser. Eller hvorvidt det er afvigelsen af de enkelte blokke. Der er flere faktorer som kan have indflydelse på dette bl.a. de enkelte komponenters afvigelse fra de teoretiske værdier osv....


ADC'ens afvigelse vil kunne påvirke signalet ift. den digitale signal....