% !TeX spellcheck = da_DK
\section{Perspektivering}
Selvom det udviklede system fungerer og opfylder de opstillede krav, er der stadig flere områder, hvor der er plads til ændringer og forbedringer. 
Systemet er udarbejdet til, at apopleksipatienter kan træne deres balance under udførelse af forskellige øvelser. Da formålet med et træningsforløb er, at patienterne skal forbedre deres balance, kan der opstå et behov for at tilpasse sværhedsgraden i de enkelte øvelser, da alle patienter har forskellige udgangspunkter ift. balancen, når de påbegynder forløbet. Dette kan gøres ved at give patienterne eller det fagkyndige personale muligheden for selv at regulere på hvilken hældningsgrad, de forskellige typer af feedback skal udløses ved. Funktionen kan implementeres i systemet ved at tilføje en switch-funktion, der ændrer tærskelværdierne i feedbackblokken. Derudover kan en kalibreringsfunktion implementeres således at brugeren modtager information om, hvorvidt udgangspositionen er $0^{\circ}$ og indstille offsetjusteringen efter dette. Dermed vil tærskelværdierne blive mere præcise, da offsetjusteringen ikke påvirker outputtet i negativ eller positiv retning. % Dermed vil det fagkyndige personale formentlig også kunne få et mere nøjagtigt billede af, hvor langt patienten er i sit rehabiliteringsforløb, da det vil være muligt at give vedkommende maksimal udfordring i øvelserne. 
I denne forbindelse kan det overvejes, hvorvidt endnu en akse på accelerometeret skal benyttes, således der kan måles hældning til begge sider samt frem og tilbage. \\
Til udvikling af softwaren kan der implementeres en funktion, som informerer det fagkyndige personale om, hvor mange gange patienten når de enkelte tærskelværdier.   

Til videreudvikling af systemet bør det også undersøges, hvorvidt løsningerne, der er valgt i designprocessen, kan optimeres. Eksempelvis ift. den første forstærkning med en faktor $9.1$, da det også er muligt at forstærke under offsetjusteringen jævnfør afsnit \ref{Subsec:Forstaerker}, side \pageref{Subsec:Forstaerker}. Hvis modstandene $R_{b}$ og $R_{d}$ på \figref{fig:Forstaerker_faktor18}, side \pageref{fig:Forstaerker_faktor18} bliver udskiftet med to $910$K$\Omega$ modstande \fxnote{NTK: hvilket reelt ikke findes, hvorfor en $120$K$\Omega$ og en $390$K$\Omega$ kan sættes parallelt} i offsetjusteringsblokken, vil denne blok i teorien forstærke med en faktor $9.1$ samtidig med, at den justerer offsettet. Filtret kan desuden også designes til at forstærke. Ved at fjerne en blok i det samlede system kan eventuel støj mindskes. Jævnfør afsnit \ref{Spaendingsforsying}, side \pageref{Spaendingsforsying} skal spændingsforsyningen til flere af systemets blokke ligge på $\pm5.5$V. Derfor vil en implementering af en konfiguration med en LED kunne implementeres, som lyser, når spændingsforsyningen kommer under denne værdi. Herved bliver patienten eller det fagkyndige personale oplyst om, at batterierne skal oplades eller udskiftes.

Det kan være fordelagtigt, især for ældre apopleksipatienter, at tilføje en funktion, som kan alarmere det fagkyndige personale eller alarmcentralen, hvis der detekteres en hældningen for patienten på mere end $\pm85^{\circ}$ over et bestemt tidsinterval. Dette kan give problematikker i eventuelle fejlalarmeringer, hvis patienten f.eks. har afmonteret accelerometret og glemt at deaktivere softwaren.

Det er væsentligt at eksperimentere med placeringen af accelerometeret på patienternes krop; jævnfør afsnit \ref{MekBioFeed} på side \pageref{MekBioFeed} er den mest optimale placering af accelerometeret øverst på sternum. Der er også andre potentielle muligheder ift. accelerometerets placering. Eksempelvis kan placering ved coxa give næsten lige så præcise resultater som placering på sternum, og det bør testes, hvorvidt det er lettere og mere komfortabelt at træne med accelerometeret i denne position \cite{Gjoreski2011}. Derudover kan accelerometeret gøres trådløst, hvorved patienten vil opnå større bevægelsesfrihed ved anvendelse af systemet. Med et sådant design kan der desuden undgås støj fra ledninger. Dette design kan opnås ved en bluetooth-konfiguration, hvor signalet fra accelerometeret sendes til kredsløbet via Bluetooth, som er kendt fra flere mobiler. Der findes allerede accelerometre med implementeret Bluetooth på markedet, hvilket kan indikere, at dette er en realistisk mulighed \cite{Axivity2015, Bioradio2015}. \\
For yderligere optimering af mobiliteten kan der udvikles en app til mobilen, der har samme funktioner som det nuværende system. Mobilen kan måle hældningen og kan evt. placeres ved coxa, hvorved mobilen kan placeres i bukselommen. Ved benyttelse af en mobil med indbygget accelerometer kan der både håndteres et analog og digitalt signal ved at opsamle og behandle data og efterfølgende give feedback i form af vibration og lyde. %Dermed vil patienterne få større fleksibilitet i sin hverdag, og det bliver lettere at arbejde med andre vigtige aspekter ift. rehabiliteringen, herunder sociale kompetencer og den almindelige hverdagsrytme, jævnfør afsnit \ref{Rehabilitering} på side \pageref{Rehabilitering}.

For at gøre træningen af balancen mere interessant for patienterne kan der udvikles et spil til systemet, hvor vedkommende f.eks. skal styre en bil, hvor formålet er at undgå forhindringer. Bilen styres af balancen, så hvis brugeren hælder til venstre vil bilen også bevæge sig til venstre og det samme hvis brugeren hælder sig til højre. Et sådan spil vil gøre brugen af systemet mere interaktiv og patienten vil i den forbindelse muligvis anvende systemet mere. Hvis spillet samtidig indeholder et scoresystem er det en simpel metode for at brugeren også kan følge sin egen progression af balancetræningen. Dette vil også give et incitament for at forbedre sig. 