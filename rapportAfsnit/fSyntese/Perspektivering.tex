\section{Perspektivering}
I dette projekt blev systemet udarbejdet til, at apopleksipatienter kan træne deres balance i forskellige øvelser. Feedbacken gives ved de samme hældningsgrader uanset hvem der anvender systemet, og hvilke øvelser der udføres. Da formålet med træningsforløbet er, at patienterne skal forbedre deres balance, kan der opstå et behov for at tilpasse sværhedsgraden i de enkelte øvelser, da alle patienter har forskellige udgangspunkter ift. balancen når påbegynder forløbet. Dette kan gøres ved at give patienterne eller det fagkyndige personale muligheden for selv at regulere ved hvilken hældningsgrad, de forskellige typer af feedback skal udløses. Denne funktion kan implementeres i systemet ved at tilføje en switch-funktion, der ændrer tærskelværdierne i komparatorkonfigurationen, således at feedbacken kan udløses ved flere forskellige inputsignaler, afhængig af hvilken sværhedsgrad der er valgt. Dermed vil det fagkyndige personale også kunne få et mere nøjagtigt billede af hvor langt patienten er i sit rehabiliteringsforløb, da det vil være muligt at give vedkommende maksimal udfordring i øvelserne. I denne forbindelse kan man også overveje mulighederne for at anvende endnu en af akserne på accelerometeret i systemet, således at der både kan måles hældning til begge sider, men også frem og tilbage, når systemet anvendes.
Det er relevant at overveje, om der er mulighed for at gøre systemet lettere at transportere, således at patienterne kan medbringe og anvende det andre steder end i hjemmet eller på sygehuset. Der kunne eksempelvis udvikles en app til mobilen, der har de samme funktioner som det nuværende system, og hvor enheden der skal måle, stadig kan placeres øverst på kroppen og være koblet til mobilen med en ledning. Dermed vil patienterne få større fleksibilitet i deres hverdag, og det bliver lettere at arbejde med andre vigtige aspekter ift. rehabiliteringen, herunder sociale kompetencer og den almindelige hverdagsrytme, jævnfør afsnit \ref{Rehabilitering} på side \pageref{Rehabilitering}. I denne forbindelse kan det desuden være væsentligt at eksperimentere med placeringen af accelerometeret på patienternes krop - jævnfør afsnit \ref{MekBioFeed} på side \pageref{MekBioFeed} er den mest optimale placering af accelerometeret øverst på sternum. Det viser sig imidlertid, at der også er andre potentielle muligheder ift. accelerometerets placering. Eksempelvis kan placering ved livet give næsten lige så præcise resultater som placering på sternum, og det kan derfor overvejes om det vil være lettere og mere komfortabelt at træne med accelerometeret i denne position\cite{Gjoreski2011}.\\
\noindent Inden systemet eventuelt ville kunne implementeres i sundhedssektoren vil der også være behov for at undersøge andre muligheder ift. opbygningen af de enkelte blokke i systemet, og i det hele taget om de enkelte løsninger der er valgt i designprocessen kan optimeres. Eksempelvis er der i offsetjusteringsblokken valgt at lave en blok der decideret kan justere offsettet til andre værdier, men da formålet med blokken er at fjerne accelerometerets offset helt, kan det være en mulighed at løse opgaven vha. et højpasfilter, som frasorterer offsettet.\fxnote{Denne formulering er virkelig dårlig, men kan ikke komme på en bedre...} Jævnfør afsnit XX på side XX er det essentielt, at spændingsforsyningen til systemet ikke kommer under $5.5$V, hvorfor det desuden kunne være en mulighed at implementere en konfiguration indeholdende en diode, der lyser når spændingen kommer under denne grænse, således at patienten eller det fagkyndige personale ved at batterierne skal oplades eller udskiftes. 
  
 
 