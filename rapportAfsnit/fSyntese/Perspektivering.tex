\section{Perspektivering}
I dette projekt blev systemet udarbejdet til at apopleksipatienter kan træne deres balance i forskellige øvelser, hvilket i sig selv gør det muligt at variere træningen. Da formålet med træningsforløbet med systemet er at patienterne skal forbedre deres balance, kan der opstå et behov for at øge sværhedsgraden i de enkelte øvelser. Omvendt kan det være  Dette kan eksempelvis gøres ved at give patienterne eller det fagkyndige muligheden for selv at regulere ved hvilken hældningsgrad, de forskellige typer af feedback skal udløses. Denne funktion kan føjes til systemet ved at tilføje en switch-funktion, der ændrer tærskelværdierne i komparatoren, således at feedbacken kan udløses ved flere forskellige inputsignaler, afhængig af hvilken sværhedsgrad der er valgt. På denne måde vil det fagkyndige personale også kunne få et mere nøjagtigt billede af hvor lagt patienten er i sit rehabiliteringsforløb, da det vil være muligt at give vedkommende maksimal udfordring i øvelserne.
Da patienterne skal anvende systemet ofte, er det relevant at overveje, om der er mulighed for at gøre det lettere at transportere\fxnote{Dette skal omformuleres - mangler bedre formulering..}, således at det kan medbringes og anvendes andre steder end i hjemmet eller på sygehuset. Der kunne eksempelvis udvikles en app til mobilen, der har de samme funktioner som det nuværende system 