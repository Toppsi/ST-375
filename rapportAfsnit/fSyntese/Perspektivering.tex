% !TeX spellcheck = da_DK
\section{Perspektivering}
Selvom det udviklede system fungerer og opfylder de opstillede krav, er der stadig flere områder, hvor der er plads til ændringer og forbedringer. 
Systemet er udarbejdet til, at apopleksipatienter kan træne deres balance under udførelse af forskellige øvelser. Da formålet med træningsforløbet er, at patienterne skal forbedre deres balance, kan der opstå et behov for at tilpasse sværhedsgraden i de enkelte øvelser, da alle patienter har forskellige udgangspunkter ift. balancen, når de påbegynder forløbet. Dette kan gøres ved at give patienterne eller det fagkyndige personale muligheden for selv at regulere på hvilken hældningsgrad, de forskellige typer af feedback skal udløses ved. Funktion kan implementeres i systemet ved at tilføje en switch-funktion, der ændrer tærskelværdierne i feedbackkonfigurationen. % Dermed vil det fagkyndige personale formentlig også kunne få et mere nøjagtigt billede af, hvor langt patienten er i sit rehabiliteringsforløb, da det vil være muligt at give vedkommende maksimal udfordring i øvelserne. 
I denne forbindelse kan man også overveje mulighederne for at anvende endnu en af akserne på accelerometeret i systemet, således der både kan måles hældning til begge sider samt frem og tilbage, når systemet anvendes. Til udvikling af softwaren kan der implementeres en funktion, som informerer det fagkyndige personale om hvor mange gange patienterne når de enkelte tærskelværdier. Dermed kan personalet sammenligne de forskellige test ud fra reelle talværdier. 

For at optimere transportmulighederne af systemet, kan der udvikles en app til mobilen, der har samme funktioner som det nuværende system. Enheden, der skal måle hældningen, kan stadig placeres øverst på sternum og skal være tilkoblet mobilen med en ledning eller via bluetooth. Ved benyttelse af en mobil med indbygget accelerometer kan mobilen både håndtere det analoge og digitale signal ved at opsamle og behandle data og efterfølgende give feedback i form af vibration og lyde. %Dermed vil patienterne få større fleksibilitet i sin hverdag, og det bliver lettere at arbejde med andre vigtige aspekter ift. rehabiliteringen, herunder sociale kompetencer og den almindelige hverdagsrytme, jævnfør afsnit \ref{Rehabilitering} på side \pageref{Rehabilitering}. 
I denne forbindelse er det væsentligt at eksperimentere med placeringen af accelerometeret på patienternes krop; jævnfør afsnit \ref{MekBioFeed} på side \pageref{MekBioFeed} er den mest optimale placering af accelerometeret øverst på sternum. Det viser sig imidlertid, at der også er andre potentielle muligheder ift. accelerometerets placering. Eksempelvis kan placering ved livet give næsten lige så præcise resultater som placering på sternum, og det bør testes, hvorvidt det er lettere og mere komfortabelt at træne med accelerometeret i denne position\cite{Gjoreski2011}.

Derudover kan accelerometeret gøres trådløst, hvorved patienten vil opnå større bevægelsesfrihed ved anvendelse af systemet. Med et sådant design kan der desuden undgås støj fra ledninger. Dette design kan opnås ved en bluetooth-konfiguration, hvor signalet fra accelerometeret sendes til kredsløbet via Bluetooth, som er kendt fra flere mobiler. Der findes allerede accelerometre med implementeret Bluetooth på markedet, hvilket kan indikere, at dette er en realistisk mulighed \cite{Axivity2015, Bioradio2015}. 

Inden systemet eventuelt kan implementeres i sundhedssektoren, vil der også være behov for at undersøge andre muligheder ift. opbygningen af de enkelte blokke i systemet, og om løsningerne, der er valgt i designprocessen, kan optimeres. Eksempelvis ift. den første forstærkning med en forstærkningsfaktor på $9.1$, da det også er muligt at forstærke under offsetjusteringen jævnfør afsnit \ref{Subsec:Forstaerker} på side \pageref{Subsec:Forstaerker}. Hvis modstandene $R_{b}$ og $R_{d}$ på \figref{fig:Forstaerker_faktor18} på side \pageref{fig:Forstaerker_faktor18} bliver udskiftet med to $910$K$\Omega$ modstande (hvilket reelt ikke findes, hvorfor en $120$K$\Omega$ og en $390$K$\Omega$ kan sættes parallelt) i offsetjusteringsblokken, vil denne blok i teorien forstærke med en faktor $9.1$ samtidig med, at den justerer offsettet. Ved at spare en blok i det samlede system kan eventuel støj mindskes. Jævnfør afsnit \ref{Spaendingsforsying} på side \pageref{Spaendingsforsying} skal spændingsforsyningen til flere af systemets blokke ligge på $5.5$V, hvorfor det desuden kan være en mulighed at implementere en konfiguration indeholdende en diode, der lyser, når spændingsforsyningen kommer under denne værdi. Herved bliver patienten eller det fagkyndige personale oplyst om, at batterierne skal oplades eller udskiftes.

%Der kan f.eks. anvendes et mere sensitivt accelerometer, så de enkelte værdier tilhørende hver grad bliver mere præcise. Dermed vil det være muligt at gøre tærskelværdierne mere præcise, og forstærkerne kan muligvis udelades, hvorfor der vil være færre kilder til støj i systemet.  \\