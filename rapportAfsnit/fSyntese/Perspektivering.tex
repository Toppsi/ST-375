% !TeX spellcheck = da_DK
\section{Perspektivering}
Selvom det udviklede system fungerer og opfylder de opstillede krav, er der stadig flere områder, hvor der er plads til ændringer og forbedringer. 
I projektet er systemet blevet udarbejdet til, at apopleksipatienter kan træne deres balance i forskellige øvelser. Feedbacken gives ved de samme hældningsgrader, uanset hvem der anvender systemet, og hvilke øvelser der udføres. Da formålet med træningsforløbet er, at patienterne skal forbedre deres balance, kan der opstå et behov for at tilpasse sværhedsgraden i de enkelte øvelser, da alle patienter har forskellige udgangspunkter ift. balancen, når de påbegynder forløbet. Dette kan gøres ved at give patienterne eller det fagkyndige personale muligheden for selv at regulere på hvilken hældningsgrad, de forskellige typer af feedback skal udløses ved. Denne funktion kan implementeres i systemet ved at tilføje en switch-funktion, der ændrer tærskelværdierne i komparatorkonfigurationen således, at feedbacken kan udløses ved flere forskellige hældningsgrader afhængig af, hvilken sværhedsgrad der er valgt. Dermed vil det fagkyndige personale formentlig også kunne få et mere nøjagtigt billede af, hvor langt patienten er i sit rehabiliteringsforløb, da det vil være muligt at give vedkommende maksimal udfordring i øvelserne. I denne forbindelse kan man også overveje mulighederne for at anvende endnu en af akserne på accelerometeret i systemet, således der både kan måles hældning til begge sider samt frem og tilbage, når systemet anvendes.\fxnote{Skal vi have noget ift. softwarens videreudvikling tilføjet her?}

Det kan overvejes, om der er mulighed for at gøre systemet lettere at transportere således, at patienterne kan medbringe og anvende det andre steder end i hjemmet eller på sygehuset. Der kan eksempelvis udvikles en app til mobilen, der har de samme funktioner som det nuværende system. Enheden, der skal måle hældningen, kan stadig placeres øverst på kroppen og skal være koblet til mobilen med en ledning eller via bluetooth. Dermed vil patienterne få større fleksibilitet i sin hverdag, og det bliver lettere at arbejde med andre vigtige aspekter ift. rehabiliteringen, herunder sociale kompetencer og den almindelige hverdagsrytme, jævnfør afsnit \ref{Rehabilitering} på side \pageref{Rehabilitering}. I denne forbindelse kan det desuden være væsentligt at eksperimentere med placeringen af accelerometeret på patienternes krop; jævnfør afsnit \ref{MekBioFeed} på side \pageref{MekBioFeed} er den mest optimale placering af accelerometeret øverst på sternum. Det viser sig imidlertid, at der også er andre potentielle muligheder ift. accelerometerets placering. Eksempelvis kan placering ved livet give næsten lige så præcise resultater som placering på sternum, og det kan derfor testes, om det vil være lettere og mere komfortabelt at træne med accelerometeret i denne position\cite{Gjoreski2011}.

Derudover kunne accelerometeret gøres trådløst, hvorved patienten ville have højere bevægelighedsgrad ved brug af systemet. Dette kunne eventuelt gøres ved en bluetooth konfiguration, hvor signalet fra accelerometeret sendes til kredsløbet via bluetooth, som er kendt fra flere mobiler. Der findes allerede accelerometre med bluetooth på markedet, hvilket kan indikere, at dette er en realistisk mulighed \cite{Axivity2015, Bioradio2015}.

Inden systemet eventuelt kan implementeres i sundhedssektoren, vil der også være behov for at undersøge andre muligheder ift. opbygningen af de enkelte blokke i systemet, og i det hele taget om de enkelte løsninger, der er valgt i designprocessen, kan optimeres. Der kunne f.eks. have været anvendt et mere sensitivt accelerometer, så de enkelte værdier tilhørende hver grad vil blive mere præcis. På denne måde ville det være muligt at gøre tærskelværdierne mere præcise, og forstærkerne kunne muligvis have været undgået, hvorfor der ville der være mindre støj i systemet. Jævnfør afsnit \ref{Spaendingsforsying} på side \pageref{Spaendingsforsying} skal spændingsforsyningen til flere af systemets blokke ligge på $5.5$V, hvorfor det desuden kunne være en mulighed at implementere en konfiguration indeholdende en diode, der lyser, når spændingsforsyningen kommer under denne værdi. Herved bliver patienten eller det fagkyndige personale oplyst om, at batterierne skal oplades eller udskiftes. \\
Designet kunne også optimeres ift. den første forstærkning med en faktor $9.1$, da det også er muligt at forstærke under offsetjusteringen jævnfør afsnit \ref{Subsec:Forstaerker} på side \pageref{Subsec:Forstaerker}. Hvis modstandene $R_{b}$ og $R_{d}$ på \figref{fig:Forstaerker_faktor18} blev udskiftet med to $910$K$\Omega$ modstande (hvilket reelt ikke findes, hvorfor en $120$K$\Omega$ og en $390$K$\Omega$ kan sættes parallelt) i offsetjusteringsblokken, ville denne blok i teorien forstærke med en faktor $9.1$ samtidig med, at den justerer offsettet. Herved kan der spares en blok i det samlede system og design, hvilket muligvis vil give mindre støj i signalet.