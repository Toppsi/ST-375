\section{Perspektivering}
I dette projekt blev systemet udarbejdet til, at apopleksipatienter kan træne deres balance i forskellige øvelser. Feedbacken gives ved de samme hældningsgrader uanset hvem der anvender systemet, og hvilke øvelser der udføres. Da formålet med træningsforløbet er, at patienterne skal forbedre deres balance, kan der opstå et behov for at tilpasse sværhedsgraden i de enkelte øvelser, da alle patienter har forskellige udgangspunkter ift. balancen når påbegynder forløbet. Dette kan gøres ved at give patienterne eller det fagkyndige personale muligheden for selv at regulere ved hvilken hældningsgrad, de forskellige typer af feedback skal udløses. Denne funktion kan implementeres i systemet ved at tilføje en switch-funktion, der ændrer tærskelværdierne i komparatorkonfigurationen, således at feedbacken kan udløses ved flere forskellige inputsignaler, afhængig af hvilken sværhedsgrad der er valgt. Dermed vil det fagkyndige personale også kunne få et mere nøjagtigt billede af hvor langt patienten er i sit rehabiliteringsforløb, da det vil være muligt at give vedkommende maksimal udfordring i øvelserne.
Derudover er det væsentligt at overveje accelerometerets placering på patientens krop - 
Det er relevant at overveje, om der er mulighed for at gøre systemet lettere at transportere, således at patienterne kan medbringe og anvende det andre steder end i hjemmet eller på sygehuset. Der kunne eksempelvis udvikles en app til mobilen, der har de samme funktioner som det nuværende system, og hvor enheden der skal måle, stadig kan placeres øverst på kroppen og være koblet til mobilen med en ledning. Dermed vil patienterne få større fleksibilitet i deres hverdag, og det bliver lettere at arbejde med andre vigtige aspekter ift. rehabiliteringen, herunder sociale kompetencer og den almindelige hverdagsrytme, jævnfør afsnit \ref{Rehabilitering} på side \pageref{Rehabilitering}. 
  