% !TeX spellcheck = da_DK
Apopleksi er den tredje største dødsårsag i Danmark og cirka 12.500 mennesker indlægges hvert år pga. apopleksi. Derudover er den det hyppigste handicap i voksenalderen, hvor knap 75.000 mennesker levede i år 2011 med følger af apopleksi. Ud af disse er omkring hver fjerde person afhængig af hjælp fra andre for, at kunne udføre dagligdagens gøremål. \cite{Hjernesagen2015a}

Apopleksi betegnes som en blødning eller blodprop i hjernen \cite{Hjernesagen2015a}. Definitionen på apopleksi er en varighed af over 24 timer, hvis varigheden er under 24 timer betegnes det som transitorisk cerebral iskæmi\cite{Sundhed.dk2014, Ritter2015}. Denne type apopleksi opstår for flere tusinde danskere årligt, men det er sjældent, at folk er klar over det, da symptomerne heraf er milde\cite{Hjernesagen2015a}.

Der findes to former for apopleksi iskæmisk og hæmoragisk.
Den iskæmiske apopleksi er den hyppigste type og opstår i 80-85 \% af tilfældene\cite{Sundhed.dk2014}. Den opstår ved blokering af hjernearterie af en blodprop og stopper derved tilførslen af blod til et bestemt område i hjernen, hvilket medvirker til en blodprop i hjernen\cite{Hjernesagen2015a}.
Hærmoragiske apopleksi finder sted 10-15 \% af tilfældene og opstår ved at en hjerneartiere brister og lækage af blod danner en blodansamling\cite{Sundhed.dk2014, Schulze2011}. Denne blodansamling beskadiger det omkringliggende væv og forøger trykket i hjernen, hvilket medfører en blødning i hjernen\citep{Caplan2006}. Årsagerne til apopleksi er forhøjet blodtryk, rygning, højt kolesteroltal, diabetes og arvelige defekter\cite{Academic2015}.

Det kræver flere former for diagnosticering for at kunne afgøre, hvilken form for apopleksi der er tilfældet. Dette kræver bl.a. en klinisk undersøgelse, hvor sygdomstilfældet vurderes.\citep{Sundhedsstyrelsen2009} Endvidere udføres videre undersøgelser med henblik på, at afgøre via. scanning om det er iskæmisk eller hæmoragisk. Herudover skal lunger, blodsukker og kropstemperatur tjekkes, da alle disse faktorer spiller ind på apopleksipatienters fremtidsprognoser og har betydning for de følger der kan forekomme af apopleksi.\citep{Sundhedsstyrelsen2009}

Følgerne af apopleksi sker ofte gradvist og kan både opleves som fysiske og metale tilstande\cite{Muus2008}. En af de hyppigste faktorer apopleksipatienter oplever er neglekt, hvilket vil sige at patienterne ikke er opmærksomme på den ene side af kroppen\cite{Sundhed.dk}. Derudover oplever nogle sproglige-, sensoriske-, motoriskeskader samt balanceproblemer\cite{Nichols1997, Muus2008}. De nævnte følger har alle alvorlige konsekvenser for apopleksipatienters livskvalitet og livssikring\fxnote{måske et andet ord?}, da det bl.a. kan føre til  begrænsninger i hverdagen og i værre tilfælde faldulykker. 

Det er i nogle tilfælde muligt at behandle apopleksipatienter ved akut behandling med blodpropopløsende medicin. Hvis behandlingen ikke fjerner blodproppen, er det muligt at forebygge nye tilfælde af apopleksi ved behandling med blodfortyndende medicin.\cite{Hjerteforeningen2014} \cite{Kruuse2014a} Forebyggelse er vigtigt, da risikoen for en ny blodprop er stor. Derudover har rehabiliteringen en stor betydning, for at apopleksipaienter kan genfinde eventuelle tabte funktioner som følge af apopleksi.\cite{Kjaergaard2015} Behandlingen og rehabiliteringen afhænger af hinanden rent organisatorisk, da sammenspillet mellem kommuner, sygehuse og praktiserende læger er vigtigt. Apopleksi har omfattende og alvorlige konsekvenser og der er derfor brug for involvering fra flere sundhedsprofessionelle områder.\cite{Sundhedsstyrelsen2010}

Dette sammenspil mellem sundhedsområder og rehabilitering er bl.a. det der gør at apopleksi er så omkostningsfuldt for samfundet. Behandlingen har både konsekvenser direkte som medicin og behandling og indirekte som tabt produktivitet og ekspertise. Det vil for samfundet kunne betyde at patenterne er nødt til at modtage indkomsterstattende ydelser pga. den tabte produktivitet grundet de nedsatte livsfunktioner.\cite{Sundhedsstyrelsen2010} 
Apopleksi påvirker patienters livskvalitet og identitet, da det er svært for patienterne at forholde sig til sygdommen og derved påvirker deres humør, personlighed, færdigheder og sociale relationer. Det er derfor vigtigt at genoptræne patienterne ved at rehabilitering for, at de kan genfinde eller forbedre deres tabte funktioner, f.eks. ved brug af teknologier, som på denne måde er med til at genoprette identitet samt forbedre livskvaliteten for patienten.\cite{Sundhedsstyrelsen2010}  

\fxnote{Jeg ved ikke helt hvordan jeg skal lede op til det initierende problem.}