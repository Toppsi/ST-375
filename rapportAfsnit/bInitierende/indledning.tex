% !TeX spellcheck = da_DK
Apopleksi er den tredje største dødsårsag i Danmark, og ca. 12.500 mennesker indlægges hvert år pga. sygdommen\cite{Hjernesagen2015a}. Antallet der dør af hjerneskader har været stagneret de sidste 10 år fra 2001 til år 2011, hvor 14 \% døde inden for 30 dage \cite{Hjernesagen2015}. I 2010 var der 18.041 indlæggelsesforløb i Danmark forbundet med hjerneskade\cite{Sundhedsstyrelsen2011}. Derudover levede 75.000 danskere i 2011 med følger af apopleksi, og ud af disse er omkring hver fjerde person afhængig af hjælp fra andre for at kunne udføre dagligdagens gøremål\cite{Hjernesagen2015a}. Antallet af indlæggelsesforløb for mænd og kvinder stiger, når de bliver ældre end 65 år\cite{Sundhedsstyrelsen2011}.
På baggrund af dette forventes det at antallet af danskere, der lever med følger af apopleksi er stigende i takt med at der kommer flere ældre. \cite{Sagen2014}. 
Det vil derfor kunne forventes, at der er flere, som kommer ud for en hjerneskade og vil have varige mén herefter, hvilket gør det vigtigt at fokusere på rehabiliteringen for at kunne genoptræne de forskellige kropslige- og mentale mangler. \\
%I 2010 var der som sagt 18.041 indlæggelsesforløb forbundet med hjerneskade i Danmark, og det er langt fra alle, som slipper for varige mén heraf\cite{Sundhedsstyrelsen2011}. % gentagelse af det der skrevet før.
Følgerne af apopleksi opstår ofte gradvist og kan både opleves som fysiske og mentale skader\cite{Muus2008}. Et af de hyppigste mén, som apopleksi patienter oplever er neglekt, hvilket vil sige, at patienterne ikke er opmærksomme på den ene side af kroppen\cite{Sundhed.dk}. Derudover opleves sproglige-, sensoriske-, motoriske skader herunder balanceproblemer. De nævnte følger har alle alvorlige konsekvenser for apopleksi patienters livskvalitet, da det bl.a. kan føre til  begrænsninger i hverdagen og i værre tilfælde faldulykker.\cite{Nichols1997, Muus2008} \\
Alle de fysiske og mentale ændringer medfører, at det er svært for en hjerneskadet patient at vende tilbage til sit gamle hverdagsliv. Forandringerne gør det f.eks. svært at udføre almindelige huslige pligter som rengøring og personlig pleje. Derudover ses det imidlertid, at familierelationerne bliver tættere, mens relationerne til vennerne bliver svækkede for en hjerneskadet patient. Dette er et problem, da gode relationer kan være med til at forbedre rehabiliterings processen og dermed gøre, at patienten hurtigere kan komme tilbage til et normalt liv.\cite{Sundhedsstyrelsen2010} \\
Man kan gisne om, at hjerneskadede patienter, heriblandt apopleksiramte, oplever nedsat livskvalitet pga. deres sygdom. Dette kan ses ved, at apopleksi patienter har dobbelt så stor selvmordsrate som baggrundsbefolkningen. Derudover nævner 16\% af apopleksi patienter, at deres livskvalitet er dårlig\cite{Sundhedsstyrelsen2010}. Den nedsatte livskvalitet kan føre til vanskeligheder senere i livet. En forbedret livskvalitet kan skabes ved hurtigere rehabilitering eller forbedret kropslig funktioner, som den apopleksi ramte patient mistede ved hjerneskaden.\cite{Sundhedsstyrelsen2010} \\
Akut behandling og rehabilitering afhænger organisatorisk af hinanden, da sammenspillet mellem kommuner, sygehuse og praktiserende læger er afgørende. % Apopleksi har omfattende og alvorlige konsekvenser og der er derfor brug for involvering fra flere sundhedsprofessionelle områder.\cite{Sundhedsstyrelsen2010}
De omfatende og alvorglige konsekvenser samt sammenspillet mellem sundhedsområder og rehabilitering er bl.a. det, som gør, at apopleksi er omkostningsfuldt for samfundet.\cite{Sundhedsstyrelsen2010} \\
%Apopleksi påvirker patienters livskvalitet og identitet, da det er svært for patienterne at forholde sig til sygdommen og derved påvirker deres humør, personlighed, færdigheder og sociale relationer. Det er derfor vigtigt at genoptræne patienterne ved at rehabilitering for, at de kan genfinde eller forbedre deres tabte funktioner, f.eks. ved brug af teknologier, som på denne måde er med til at genoprette identitet samt forbedre livskvaliteten for patienten.\cite{Sundhedsstyrelsen2010} 