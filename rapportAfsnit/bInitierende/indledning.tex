% !TeX spellcheck = da_DK
Apopleksi er pludselig opstået fokalneurologiske symptomer forårsaget af vaskulære forstyrrelser i hjernen, der kan være forårsaget af livsstilsfaktorer, såsom forhøjet blodtryk, diabetes eller rygning \cite{Sundhedsstyrelsen2009,Academic2015}. Apopleksi er den tredje største dødsårsag i Danmark og ca. 12.500 personer indlægges hvert år pga. sygdommen \cite{Hjernesagen2015a}. Andelen der dør af hjerneskader, har været stagneret fra 2001 til 2011, hvor 14\% døde inden for 30 dage \cite{Hjernesagen2015}. Derudover levede 75.000 danskere i 2011 med følger af apopleksi, og ud af disse er omkring hver fjerde person afhængig af hjælp for at kunne udføre dagligdagens gøremål \cite{Hjernesagen2015a}. Antallet af indlæggelsesforløb for mænd og kvinder stiger, når de bliver ældre end 65 år \cite{Sundhedsstyrelsen2011}.
Danskere der lever med følger og varige mén af apopleksi forventes at være stigende i takt med, at der kommer flere ældre \cite{Sagen2014}. Apopleksi er den sygdom, der kræver flest plejedøgn i sundhedssektoren. Ud fra et økonomisk perspektiv er det derfor omkostningsfuldt for samfundet ift. behandling, rehabilitering og produktivitetstab.  Udgifterne til sygdommen udgør 4\% af sundhedsvæsenets samlede udgifter, hvor direkte udgifter er estimeret til 2.7 milliard kroner om året \cite{Hjernesagen2015a, Kruuse2014}.
 
%I 2010 var der som sagt 18.041 indlæggelsesforløb forbundet med hjerneskade i Danmark, og det er langt fra alle, som slipper for varige mén heraf\cite{Sundhedsstyrelsen2011}. % gentagelse af det der skrevet før.
Følgerne af apopleksi opstår ofte pludseligt og kan medføre både fysiske og mentale konsekvenser for patienten \cite{Muus2008}. Efter et apopleksitilfælde oplever  patienterne ofte nedsat eller ikke funktionsdygtig balance. Problemer med balancen opstår, da encephalon ikke kan bearbejde de balanceinformationer proprioceptorerne og sansereceptorerne sender. \cite{Karnath2003} Apopleksipatienterne kan desuden opleve neglekt, der også er skade på de sensoriske og motoriske funktioner og et af de hyppigste mén. Der findes forskellige typer af neglekt. Eksempelvis kan patienten opleve ikke at være opmærksom på den ene side af kroppen. \cite{Sundhed.dk} 

Disse to typer af følger har alvorlige konsekvenser for apopleksipatienter, da de bl.a. kan føre til begrænsninger i hverdagen og i nogle tilfælde faldulykker. \cite{Muus2008,Nichols1997} For en apopleksipatient med balanceproblemer kan det være vanskeligt at vende tilbage til sin normale hverdag, da almindelige huslige pligter, såsom rengøring og personlig pleje er svært at klare uden hjælp. \cite{Sundhedsstyrelsen2010} \\  

Balanceproblemer kan medføre nedsat livskvalitet. Generelt oplever hjerneskadede patienter, heriblandt apopleksipatienter med balanceproblemer, nedsat livskvalitet pga. deres sygdom. Dette ses eksempelvis ved, at apopleksipatienter har dobbelt så stor selvmordsrate som baggrundsbefolkningen \cite{Sundhedsstyrelsen2010}. I en kvantitativ undersøgelse nævner 16\% \fxnote{Mangler at rette dette pga. tidspres} af apopleksipatienterne, at deres livskvalitet er dårlig \fxnote{46\% nogenlunde, 38\% god} \cite{Sundhedsstyrelsen2010}. Et apopleksitilfælde medfører en pludselig afbrydelse i patientens livsforløb. Det kan for patienten blive uoverskueligt at opretholde sociale- og familierelationer, hvilket medfører, at de stadigvæk kan opleve nedsat livskvalitet senere i livet. En forbedret livskvalitet kan skabes ved hurtigere rehabilitering samt forbedrede kropslige funktioner, herunder balancen. \cite{Sundhedsstyrelsen2010}

For at apopleksipatienter opnår den bedst mulige behandling og rehabilitering er det afgørende, at der er et fungerende sammenspil mellem kommuner, sygehuse og praktiserende læger \cite{Sundhedsstyrelsen2010}. Det er essentielt, at rehabiliteringen påbegyndes få dages efter apopleksitilfældet er opstået, for så vidt muligt at genskabe den tabte funktionsevne, herunder balancen. \cite{kruuse2015} 

%For apopleksipatienter med balanceproblemer er det vigtigt at fokusere på balancen under rehabiliteringsforløbet for at genoptræne denne tabte funktion \fxnote{havde problemer med denne sætning - denne fører tilbage til balance, men forstår man det. Hjælp til OMF}, der bla. er årsagen til nedsat livskvalitet. 


%Et af de hyppigste mén, som apopleksipatienter oplever, er neglekt. Patienter med neglekt er ikke opmærksomme på den ene side af kroppen \cite{Sundhed.dk}. Derudover opleves sensoriske- og motoriske skader herunder balanceproblemer. De nævnte følger har alle alvorlige konsekvenser for apopleksipatienters livskvalitet, da det bl.a. kan føre til  begrænsninger i hverdagen og i nogle tilfælde faldulykker. \cite{Muus2008,Nichols1997}

%Fysiske og mentale konsekvenser af sygdommen gør, det svært for en apopleksipatient at vende tilbage til sin normale hverdag. Problemer med balancen gør det f.eks. svært at udføre almindelige huslige pligter som rengøring og personlig pleje. \cite{Sundhedsstyrelsen2010} \\

%Hjerneskadede patienter, heriblandt apopleksiramte, oplever nedsat livskvalitet pga. deres sygdom. Dette kan ses ved, at apopleksipatienter har dobbelt så stor selvmordsrate som baggrundsbefolkningen \cite{Sundhedsstyrelsen2010}. I en kvantitativ undersøgelse nævner 16\% af apopleksipatienter, at deres livskvalitet er dårlig \fxnote{46\% nogenlunde, 38\% god} \cite{Sundhedsstyrelsen2010}. Den nedsatte livskvalitet kan føre til vanskeligheder senere i livet \fxnote{Måske skrive hvorfor}. En forbedret livskvalitet kan skabes ved hurtigere rehabilitering eller forbedret kropslige funktioner, som den apopleksiramte mistede ved hjerneskaden. \cite{Sundhedsstyrelsen2010}

%For at patienterne opnår den bedst mulige behandling og rehabilitering er det afgørende, at der er et fungerende sammenspil mellem kommuner, sygehuse og praktiserende læger. Apopleksipatienter er krævende ift. rehabilitering pga. omfattende følger efter hjerneskaden. \cite{Sundhedsstyrelsen2010} Det er derfor vigtigt, at fokusere på patienternes rehabilitering for at kunne genoptræne de forskellige fysiske- og mentale mangler de oplever i dagligdagen samt give dem større livskvalitet. 

%Akut behandling og rehabilitering afhænger organisatorisk af hinanden, da sammenspillet mellem kommuner, sygehuse og praktiserende læger er afgørende. % Apopleksi har omfattende og alvorlige konsekvenser og der er derfor brug for involvering fra flere sundhedsprofessionelle områder.\cite{Sundhedsstyrelsen2010}
%De omfatende og alvorglige konsekvenser samt sammenspillet mellem sundhedsområder og rehabilitering er bl.a. det, som gør, at apopleksi er omkostningsfuldt for samfundet.\cite{Sundhedsstyrelsen2010} \\
%Apopleksi påvirker patienters livskvalitet og identitet, da det er svært for patienterne at forholde sig til sygdommen og derved påvirker deres humør, personlighed, færdigheder og sociale relationer. Det er derfor vigtigt at genoptræne patienterne ved at rehabilitering for, at de kan genfinde eller forbedre deres tabte funktioner, f.eks. ved brug af teknologier, som på denne måde er med til at genoprette identitet samt forbedre livskvaliteten for patienten.\cite{Sundhedsstyrelsen2010} 

%Det vil derfor kunne forventes, at der er flere, som kommer ud for en hjerneskade og vil have varige mén herefter, hvilket gør det vigtigt at fokusere på rehabiliteringen for at kunne genoptræne de forskellige kropslige- og mentale mangler.

%%%%%%%%%%%%%%%   Marias foreslag til indledning
%Det kræver samarbejde fra flere professionelle plejepersonale som kommuner, sygehuse og praktiserende læger for at give patienten den rette rehabilitering, da apopleksi patienter er omfattende og kan have alvorlige konsekvenser. Dette gør også at apopleksi er omkostningsfuldt for samfundet. 