% !TeX spellcheck = da_DK
\section{Overordnede funktionelle krav til systemet}\label{FunkKrav}
\begin{itemize}
	\item Systemet skal være brugervenligt, så det kan anvendes af apopleksipatienter og fagkyndigt personale
	\item Systemet skal kunne måle kropshældning, samt angive hvilken retning hældningen sker mod. Derudover skal det kunne måle statisk acceleration, eftersom vi måler på stående position \fxnote{skal sidste sætning med - referer til Erikas rettelse på side 4 fra statusseminar "skal kunne måle statisk acceleration" - hvorfor?}.
	\item Systemet skal kunne give visuel og somatosensorisk feedback ved forskellige hældningsgrader.
	\begin{itemize}
		\item Grøn diode: Skal lyse, når patienten ikke er ude i risikozonerne og informere patienten om at accelerometeret er placeret korrekt.  
		\item Gul diode: Skal lyse, når den første risikozone defineret i grader indtræffer og slukke, hvis patienten retter sig op.
		\item Rød diode: Skal lyse, når den anden risikozone defineret i grader indtræffer og slukke, hvis patienten retter sig op.
		\item Vibration: Skal aktiveres, når den første risikozone indtræffer og skal slukke, hvis patienten retter sig op. Hvis patienten hælder yderligere, skal vibrationshastigheden stige.
	\end{itemize}
	\item Systemet skal kunne skifte mellem to sværhedsgrader, hvilket vil sige, at systemet skal kunne skifte imellem to forskellige komparator blokke, som har tærskelværdier relateret til sværhedsgraden af den udførte øvelse.
	\item Signalet i systemet må ikke forstærkes til en værdi over 3V, da det så vil nå mætning.
	\item Systemet skal kunne give et digitalt outout, så det er muligt at behandle og gemme patienternes data i et program.
\end{itemize}
