% !TeX spellcheck = da_DK
\section{Overordnede funktionelle krav til systemet}\label{FunkKrav}
\begin{itemize}
	\item Systemet skal være simpelt, så det kan anvendes af både apopleksipatienterne selv samt af fagkyndigt personale
	\item Systemet skal kunne måle kropshældning, samt angive hvilken retning hældningen sker mod. 
	\item Systemet skal kunne give visuel og somatosensorisk feedback ved forskellige hældningsgrader.
	\begin{itemize}
		\item Grøn diode: Skal lyse, når patienten ikke er ude i risikozonerne og informere patienten om at accelerometeret er placeret korrekt.  
		\item Gul diode: Skal lyse, når den første risikozone defineret i grader indtræffer og slukke, hvis patienten retter sig op.
		\item Rød diode: Skal lyse, når den anden risikozone defineret i grader indtræffer og slukke, hvis patienten retter sig op.
		\item Vibration: Skal aktiveres, når den første risikozone indtræffer og skal slukke, hvis patienten retter sig op. Hvis patienten hælder yderligere, skal vibrationshastigheden stige.
	\end{itemize}
	\item Systemet skal kunne skifte mellem to sværhedsgrader, hvilket vil sige, at systemet skal kunne skifte imellem to forskellige komparator blokke, som har tærskelværdier relateret til sværhedsgraden af den udførte øvelse.
	\item Systemet skal kunne give et digitalt outout, så fagkyndigt personale kan behandle og gemme patienternes data i et program.
\end{itemize}
