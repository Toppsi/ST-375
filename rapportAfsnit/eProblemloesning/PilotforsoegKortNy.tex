% !TeX spellcheck = da_DK
\section{Pilotforsøg}\label{Sec:PilotforsoegKort}
Der udføres i dette projekt et pilotforsøg for at identificere og sammenligne væsentlige parametre ift. det anvendte accelerometer. En udførlig beskrivelse af pilotforsøget kan ses i bilag \ref{Bilag:Pilotforsoeg} på side \pageref{Bilag:Pilotforsoeg}.
Formålet med pilotforsøget er at kunne fastsætte det frekvensspektrum, som det ønskede signal ligger indenfor, da det på den måde er muligt at frafiltrere støjsignaler fra det opsamlede signal. Derudover skal det maksimale og minimale outputsignal fra accelerometeret identificeres, for at spændingsværdierne for de valgte hældningsgrader kan udregnes. Pilotforsøget udføres desuden for at sammenligne de målte offset- og sensitivitetsværdier med de oplyste værdier i databladet for accelerometeret. 
\subsection{Konklusion}
På baggrund af målinger og beregninger i pilotforsøget kan det konkluderes, at for outputsignalet fra accelerometret ved den statiske acceleration udgør alt over $0Hz$ støj. Der udregnes en baseline ved $0$g påvirkning til $1.6325$V, som burde være $1.65$V, da dette er halvdelen af spændingsforsyningen. Derved afviger den målte baseline med $1.06$\%. Ved rotationsmålingerne vurderes det, at alt over $25$Hz er støj. Maksimum og minimum outputsignalet fra accelerometret vil for langsom rotation eller svajning henholdsvis være $1.9638$V og $1.3092$V, hvilket bliver til $0.3313$V og -$0.3233$V efter offsettet er blevet justeret, der udregnes til $0.0037$V og -$0.0036$V pr grad. Afvigelserne mellem pilotforsøg og datablad vurderes som værende acceptable ift. videre udførelse af forsøg med accelerometeret. 

