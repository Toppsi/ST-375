% !TeX spellcheck = da_DK
\section{Pilotforsøg}\label{Sec:PilotforsoegKort}
Der udføres i dette projekt et pilotforsøg for at identificere og sammenligne væsentlige parametre ift. det anvendte accelerometer. En udførlig beskrivelse af pilotforsøget kan ses i bilag \ref{Bilag:Pilotforsoeg}, side \pageref{Bilag:Pilotforsoeg}.
Formålet med pilotforsøget er at identificere frekvensspektrummet, det ønskede signal ligger indenfor, da det derved er muligt at frafiltrere støjsignaler fra det opsamlede signal. Derudover skal det maksimale og minimale outputsignal fra accelerometeret identificeres for, at spændingsværdierne for de valgte hældningsgrader kan udregnes. Pilotforsøget udføres desuden for at sammenligne de målte offset- og sensitivitetsværdier med de oplyste værdier i databladet for accelerometeret. 

\subsection{Konklusion}
På baggrund af målinger og beregninger i pilotforsøget kan det konkluderes, at alt over $0$Hz ved statisk acceleration udgør støj, hvorimod det vurderes, at alt over $25$Hz for rotationsmålingerne er støj. Der måles et offset ved $0$g påvirkning på $1.6325$V, hvilket burde være $1.65$V, da dette er halvdelen af spændingsforsyningen. Derved afviger det målte offset med $1.06$\%. Maksimum og minimum outputsignalet fra accelerometret vil for langsom rotation eller svajning hhv. være $1.9638$V og $1.3092$V, der bliver $0.3313$V og -$0.3233$V efter offsetjusteringen. Dette udregnes til at være $0.0037$V pr. grad i positiv retning og -$0.0036$V pr grad i negativ retning. \\
Disse målte data anses for at være de reelle værdier for accelerometret og benyttes derfor i udformningen af det samlede system.
% Afvigelserne mellem pilotforsøg og datablad vurderes som værende acceptable ift. videre udførelse af forsøg med accelerometeret. 

