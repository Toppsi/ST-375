\section{Software}
Når signalet er blevet konverteret fra analog til digitalt signal i ADC’en og har været igennem USB-isolatoren bliver signalet sendt ind i en computer, hvor digital behandling af signalet foregår. Her skal signalet jævnført \pageref{} kunne vises grafisk til patienten og fagkyndigt personale og gemmes til senere brug og analyse. For at imødekomme disse krav, bruges en computer med software programmerne; Scopelogger og Matlab, hvor Scopelogger bruges til at optage det  digitale signal fra NIDAQ'en og Matlab bruges til at behandle den opsamlet data. 

%Vi kunne for at gøre programmet mere brugervenligt udarbejde en manual til systemet. Derudover kunne det være fordelagtigt at det fagkyndige personalet at have mulighed for at ændre akserne, og her tænkes specielt tidsaksen. Dette tænkes da systemet skal kunne optage selvtræning og da det ikke forventes at patienten tænder og slukker systemet efter hver enkelt øvelse. Dette vil både være svært at huske for patientgruppe (gamle og måske med kognitive komplikationer) og så ville de måske glemme at slå den til igen. Derfor tænkes det at de lader systemet optage signaler under hele selvtrænings sessionen. Det vil midlertidigt blive til utrolig meget data, hvor en stor del af dataen vil være pauser for patienten og derved ubrugelig data for det fagkyndige personale. Det ville derfor være smart at hvis de forholdsvis nemt kunne udvælge perioder i dataen, de gerne vil undersøge nærmere. 
%Derudover skal vi også have gjort patienten og det fagkyndige personale opmærksomme på hvilken øvelse der foretages!
%Det kunne måske være en ide, at vi i grafen kunne ligge tærskelværdierne ind som en form for linjer, så det blev nemmere at aflæse grafen
%Skal vi have en løbende visning af signalet i en graf eller er det bare til sidst! 

%Vores program skal altså kunne:
%  -Optage signalet
% - Behandle signalet således, alt efter sværhedsgrad 
%  - Plotte signalet
% - Gem data'en 
