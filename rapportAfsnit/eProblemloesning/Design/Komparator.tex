% !TeX spellcheck = da_DK
\subsection{Komparator}
\subsubsection{Teori og design}
Som nævnt i afsnit \ref{Komparatorafsnit} på side \pageref{Komparatorafsnit} anvendes en komparator til at sammenligne to inputspændinger. I dette tilfælde vil komparatorens output blive tilkoblet hhv. en LED-diode eller/og vibrator, en modstand og den positive spændingsforsyning ($+V_{cc}$). I komparatorens inverterende terminal skal outputtet fra forrige blok tilsluttes. Dens input til den ikke inverterende terminal skal være reference spændingen, hvilket anvendes til at angive den beregnede tærskelværdier. Komparatoren kan derfor have to forskellige outputs afhængig af spændingen af inputtet. Hvis inputsignalet er mindre end den beregnede tærskelværdie, vil outputtet være $0V$, og der vil ikke være et spændingsfald over dioden eller vibratoren, som aktiverer den. Er inputtet højere end tærskelværdien, vil outputtet svare til jord, da strømmen fra $+V_{cc}$ vil løbe igennem komparatoren og derefter til jord. Derved opnås et spændingsfald over den positive og negative pol for LED-dioden og vibratoren på en værdi, der ligger over den minimale spændingsfald, der kræves for en aktivering. LED-dioderne og vibratorerne vil altså blive aktiveret og give et feedback til patienten. \\
Diodernes katode skal tilkobles komparatorens output, hvorimod dens anode skal tilkobles $+V_{cc}$. For aktivering af LED-dioderne vil der blive anvendt fem komparatorer, som skal sammensættes, da der ifølge kravspecifikationerne i afsnit \ref{KomparatorAfs} på side \pageref{KomparatorAfs} er fem forskellige stadier for aktivering. Tærskelværdierne for disse bygges vha. fem parallelforbundede modstande (R1-R5) og en spændingsreference. Derudover placeres fem modstande mellem LED-dioderne og $+V_{cc}$ for at beskytte LED-dioderne mod for høj strømstyrke og undgå, at batteriet drænes. Disse defineres som R6-R10. På figur X vises selve konstruktionen af kredsløbet med de fem komparatorer og tilhørende modstande. \\

indsæt et billede \\

\noindent\textbf{Beregning af tærskelværdier og tilhørende R1-R5 modstande for aktivering af LED-dioder} \\
Ifølge systemets funktionelle krav afsnit X side X ønskes det, at LED-dioderne skal lyse ved en bestemt kropshældning, altså en bestemt tærskelværdi i komparatoren. Spændingsreferencen indgår som del af en spændingsdeler med en spændingsforsyning på 9V i form af et batteri samt modstandene mellem spændingsreferencen og batteriet. Disse modstande har derudover også med det formål at sørge for, at batteriet ikke drænes for strøm, fordi kredsløbet trækker strøm. Hvis modstandene er høje, vil kredsløbet ikke få lov til at trække meget strøm og batteriet vil holde længere. Modstandene beregnes vha. spændingsdelerformlen. \\

indsæt formel \\

- Beregninger af referencespændinger - se datablad \\
- beregninger af modstande \\
- word dokument \\

\noindent\textbf{Beregning af R6-R10 modstande for aktivering af LED-dioder} \\
Ifølge kravspecifikationerne i afsnit \ref{FunkKrav}  på side \pageref{FunkKrav} for komparatoren skal den have en forsyningsspænding på minimum 3V og maksimalt 9V. %Der benyttes derfor en operationsforstærker af typen XX. 
De anvendte LED-dioderne i systemet er: en L-53LG 5mm (grøn), to L-53LI 5mm (rød) og to L-53LY 5mm (gul). Disse LED-dioder kræver en minimum spænding på 2mA for at lyse og spændingsfaldet over dioderne ligger maksimalt i intervallet 2,0 til 2,2 V (rød: 2,0, gul: 2,1 og grøn: 2,2). LED-dioderne forsynes af et 9V batteri og tilkobles, som sagt, tilhørende modstande for bla. at undgå at LED-dioderne brænder af. Spændingsfaldet over dioderne samt den spænding LED-dioderne som minimum skal bruge for at lyse er kendte værdier, dvs. modstandene R6-R10 kan derfor findes vha. Ohms lov. Nedestående udregning giver en værdi af modstandene, hvis batteriet yder maksimal spænding: \\

indsæt ligning \\

I praksis vil batteriet ikke yde en maksimal spænding på 9 V... \fxnote{hvorfor vælger de lige en spænding på 6,2 V}, da spændingen aftager som funktion af tiden... \\

\noindent\textbf{Beregning af tærskelværdier og tilhørende R1-R5 modstande for aktivering af  vibratorerne} \\
For aktivering af vibratorerne vil der blive anvendt to komparatorer, da der ifølge kravsspecifikationerne i afsnit \ref{KomparatorAfs} på side \pageref{KomparatorAfs} skal være 2 tærskelværdier. Disse tærskelværdier konstrueres ligeledes vha. to parallelforbundede modstande og spændingsreferencer, samt de resterende modstande mellem vibratorerne og $+V_{cc}$. På figur X vises konstruktionen af kredsløbet med de to vibratorer og tilhørende modstande. \\

indsæt billede og word dokument \\

\noindent\textbf{Beregning af R6-R10 modstande for aktivering af vibratorerne} \\
Vibratorerne der anvendes i systemet er af typen XX... Afsnit skal skrives, når vi har information om vibratorer.  \\

FÅ SWITCH MED IND I DET? \\

\subsubsection{Simulering}
\subsubsection{Implementering}
\subsubsection{Test}