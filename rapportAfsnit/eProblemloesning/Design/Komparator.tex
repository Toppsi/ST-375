\section{Komparator}
Som nævnt i afsnit x side x anvendes en komparator til at sammenligne to inputsspændinger. Rent teknisk gøres dette ved, at komparatoren inverterer det ene signal $\pm$ 180$^{\circ}$ (V_{-}), hvor det andet signal ikke inverteres (V_{+}). I dette tilfælde vil komparatoren blive tilkoblet hhv. en LED-diode og vibrator samt en modstand og den positive spændingsforsyning (+V_{cc}). Komparatorens input i den inverterende terminal er outputtet fra forrige blok, der filtrer signalet fra accelerometeret og dens input i den ikke-inverterende terminal er beregnede tærskelværdier. Komparatoren kan derfor have to forskellige outputs afhængig af spændingen af inputtet. Hvis beregnede tærskelværdier er højere end inputsignalet vil ouputtet være 0 V og der vil ikke være et spændingsfald over dioden eller vibratoren, som aktiverer den. Er tærskelværdien mindre end inputsignalet vil outputtet svare til jord, da strømmen fra (+V_{cc}) vil løbe igennem komparatoren og derefter til jord. Der vil opstå et spændingsfald over den positive og negative pol for LED-dioden og vibratoren på en værdi, der ligger over den minimale spændingsfald, der kræves for en aktivering. LED-dioderne og vibratorerne vil altså blive aktiveret og give et feedback til patienten. 
For LED-dioderne vil der blive anvendt fem komparatorer, som skal sammensættes, da der ifølge kravspecifikationerne i afsnit X side X skal være fem tærskelværdier. Disse tærskelværdier bygges vha. fem paralleltforbundet modstande og spændingsreferencer, der skal matche hældningsgraden. Derudover placeres modstande mellem LED-dioderne og (+V_{cc}) for at bestykke LED-dioderne mod for høj strømstyrke og for at udgå at batteriet ikke drænes. På figur X vises selve konstruktionen af kredsløbet med de fem komparatorer og tilhørende modstande.

indsæt et billede

\textbf{Beregning af tærskelværdier og tilhørende modstande for LED-dioderne}
Beregning af referencespændinger: 
word dokument

\textbf{Beregning af tærskelværdier og tilhørende modstande for vibratorerne}
For vibratorerne vil der blive anvendt to komparatorer, som skal sammensættes, da der ifølge kravsspecifikationerne i afsnit X side X skal være 2 tærskelværdier. Ligeledes med LED-dioderne konstrueres disse tærskelværdier vha. 2 paralleltforbundet modstande og spændingsreferencer, samt de resterende modstande mellem vibratorerne og (+V_{cc}). PÅ figur X vises konstruktionen af kredsløbet med de to vibratorer og tilhørende modstande.

indsæt billede og word dokument



FÅ SWITCH MED IND I DET? 