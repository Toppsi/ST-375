\section{Komparator}
Som nævnt i afsnit x side x anvendes en komparator til at sammenligne to inputsspændinger. I dette tilfælde vil komparatoren blive tilkoblet hhv. en LED-diode og vibrator samt en modstand og den positive spændingsforsyning (+Vcc). Komparatorens input i den inverterende terminal er outputtet fra forrige blok, der filtrer signalet fra accelerometeret og dens input i den ikke-inverterende terminal er beregnede tærskelværdier. Komparatoren kan derfor have to forskellige outputs afhængig spændingen af inputtet. Hvis tærskelværdien er højere end inputsignalet vil ouput svare til ***+Vcc, hvor spændingsfaldet i komparatoren er fratrukket, dvs. teoretisk set vil der ikke være et spændingsfald over dioden eller vibratoren, som aktiverer den.*** ***Er tærskelværdien mindre end inputsignalet vil outputtet svare til jord, dvs. strømmen fra den positive spændingsforsyning vil løbe igennem komparatoren til jord***. LED-dioden og vibratoren vil blive aktiveret idet der er et spændingsfald over den positive og negative pol på en værdi over den mindste spændingsfald, den kræver for at blive aktiveret***
For LED-dioderne vil der blive anvendt fem komparatorer, som skal sammensættes, da der ifølge kravspecifikationerne i afsnit X side X skal være fem tærskelværdier. Disse tærskelværdier bygges vha. fem paralleltforbundet modstande og en spændingsreference. Derudover placeres modstande mellem LED-dioderne og +Vcc for at bestykke LED-dioderne mod for høj strømstyrke og for at udgå at batteriet ikke drænes. På figur X vises selve konstruktionen af kredsløbet med de fem komparatorer og tilhørende modstande.

indsæt et billede

\textbf{Beregning af tærskelværdier og tilhørende modstande for LED-dioderne}
Beregning af referencespændinger: 

\textbf{Beregning af tærskelværdier og tilhørende modstande for vibratorerne}
For vibratorerne vil der blive anvendt to komparatorer, som skal sammensættes, da der iføl


- hvad sker der med signalet, når det er over og under tærskelværdien? 
- hvordan laves tæskelværdier?
- Vis det visuelt. 
- Beregn værdien af modstandene
- Skriv de dioder, vibratorer og komparatorer vi bruger 
- spænding ift. grader 8 og 13 grader. 
- off-set blok 


FÅ SWITCH MED IND I DET? 