% !TeX spellcheck = da_DK
\subsection{Forstærker i tilpasningsblok}\label{Forstaerker_faktor3_afs}
\subsubsection{Teori og design}
I afsnit \ref{Subsec:Forstaerker}, side \pageref{Subsec:Forstaerker} er teorien samt designet af en forstærker forklaret. Da blokken tilpasning skal tilpasse det filtrerede signal til komparatoren, afgrænses arbejdsintervallet til $\pm25^{\circ}$. Derfor ønskes det, at $V_{out}$ fra denne blok er $\pm3$V, når accelerometret måler $\pm25^{\circ}$. Der skal dermed ske en forstærkning på en faktor $3.6$, hvilket svarer til $11.1261$dB, som beskrevet i afsnit \ref{Tilpasningsblok} på side \pageref{Tilpasningsblok}. \\
For at udregne modstandene defineres R$2$ til $10$K$\Omega$. Ud fra dette er R$1$ blevet bestemt ved følgende udregning:
\begin{eqnarray}
3.6 = 1 + (\frac{R1}{10\text{K}\Omega}) \\
R1 = 26\text{K}\Omega
\end{eqnarray}

\noindent Forstærkerens opbygning kan ses på \figref{fig:Forstaerker_faktor3}.
\begin{figure}[H]
	\centering
	\includegraphics[scale=0.35]{figures/cProblemloesning/Forstaerker_faktor3.PNG}
	\caption{På figuren ses designet af den ikke-inverterende forstærker, der forstærker med en faktor $3.6$.}
	\label{fig:Forstaerker_faktor3}
\end{figure}

\subsubsection{Simulering}
Forstærkeren testes i fire simuleringer for at undersøge, om den opfylder de opstillede krav. Det forstærkede signal ($V_{out}$), skal være $3.6$ gange større end blokkens inputsignal ($V_{in}$). På \figref{fig:faktor3_simulering} ses simuleringen af et inputsignal på $0.8325$V, som ideelt vil komme fra filtreringsblokken, hvis accelerometret hælder $25^{\circ}$.

\begin{figure}[H]
	\centering
	\includegraphics[scale=0.38]{figures/cProblemloesning/Forstaerker_faktor3_simulering.PNG}
	\caption{På figuren ses simuleringen for et inputsignal på $0.8418$V, som giver ca. $3$V i output. Der er således sket en forstærkning med en faktor $3.6$.}
	\label{fig:faktor3_simulering}
\end{figure}
\noindent Resultaterne af fire simuleringer ses i \tableref{tab:forstarker3_simT}. De fire simulerede inputs er den spænding, som forstærkeren vil modtage ved hhv. $\pm90^{\circ}$ og $\pm25^{\circ}$.
\begin{table}[H]
	\centering
	\begin{tabular}{|l|l|l|l|l|l|}
		\hline
		\multicolumn{1}{|c|}{\textit{Input}} & \textit{\begin{tabular}[c]{@{}l@{}}For-\\stærkning\end{tabular}} & \textit{\begin{tabular}[c]{@{}l@{}}Forventet\\outputsignal\end{tabular}} & \multicolumn{1}{c|}{\textit{Output}} & \textit{\begin{tabular}[c]{@{}l@{}}For-\\stærkning\end{tabular}}  & \multicolumn{1}{c|}{\textit{Afvigelse}} \\ \hline
		$3.0148$V     & $3.6$   & \begin{tabular}[c]{@{}l@{}}Forventer mætning\\ ($10.8533$V)\end{tabular} & $3.9765$V  & $\times$ & $\times$     \\ \hline
		$0.8418$V    & $3.6$   & $3.0305$V                                                              & $3.0304$V  & $3.6$ & $0\%$     \\ \hline
		%		$0$V          & 3   & $0$V                                                                   & $33.6527\mu $V        & $\approx 0\%$     \\ \hline
		-$0.8190$V     & $3.6$   & -$2.9484$V                                                             & -$2.9482$V  & $3.6$ & $0\%$     \\ \hline
		-$2.9420$V     & $3.6$   & \begin{tabular}[c]{@{}l@{}}Forventer mætning\\ (-$10.5912$V)\end{tabular}& -$3.97703$V  & $\times$ & $\times$     \\ \hline
	\end{tabular}
	\caption{I tabellen ses resultaterne af de fire simuleringer.}
	\label{tab:forstarker3_simT}
\end{table}
\noindent Der ses i \tableref{tab:forstarker3_simT} en afvigelse, som ligger inde for tolerancerne, jævnfør afsnit \ref{OpsamlingsAfs}, side \pageref{OpsamlingsAfs}. Det er herved påvist, at kredsløbet fungerer teoretisk med ideelle komponenter og kan derfor implementeres.  

\subsubsection{Implementering og test}
Det ses på \figref{fig:Forstaerker_faktor3}, at der skal benyttes to modstande på $10$K$\Omega$ og $26$K$\Omega$ til implementeringen af forstærkeren. Reelt findes der ikke en $26$K$\Omega$, hvorfor der istedet benyttes $27$K$\Omega$- og $680$K$\Omega$ modstande i parallel forbindelse, hvilket teoretisk vil give en $0.12$\% afvigelse fra en ideel $26$K$\Omega$ modstand. De tre modstande blev målt inden implementering, hvilket fremgår i \tableref{Tab:modstand_faktor18}.
\begin{table}[H]
	\centering
	\begin{tabular}{|l|l|l|}
		\hline
		\textit{Teoretisk}  & \textit{Ved måling} & \textit{Afvigelse} \\ \hline
		$10$K$\Omega$       & $9.98$K$\Omega$     & $0.20$\%               \\ \hline
		$26$K$\Omega$      &  $25.964$K$\Omega$   & $0.14$\% \\ \hline
		$680$K$\Omega$      & $684.53$K$\Omega$   & $0.67$\%               \\ \hline
$R_{eq} = 26$K$\Omega$ || $680$K$\Omega$ = $27$K$\Omega$       & $26.99$K$\Omega$    & $0.40$\%               \\ \hline
	\end{tabular}
	\caption{I tabellen ses det, at modstandene har en afvigelse fra deres teoretiske værdi, hvilket er forventet af reelle komponenter. Dette har derved også en effekt på afvigelsen for $R_{eq}$. Det er en acceptabel afvigelse, og modstandene kan derfor anvendes i implementeringen.}
	\label{Tab:modstand_faktor18}
\end{table}
\noindent Herefter implementeres kredsløbet. Til måling af signalet benyttes et multimeter. De aflæste resultater for spændingen efter forstærkningen er angivet under output i \tableref{Tab:faktor3_test}.\
\begin{table}[H]
	\centering
	\begin{tabular}{|l|l|l|l|l|l|}
		\hline
 \textit{Ønsket input} & \textit{Input} & \textit{Forventet output} & \textit{Output}  &  \textit{Forstærkning}  & Afvigelse \\ \hline
   $3.0148$V &  $3.0146$V           & \begin{tabular}[c]{@{}l@{}}mætning\\ $10.8526$V\end{tabular}  & $4.8403$V   &    $\times$     & $\times$  \\ \hline
   $0.8418$V &  $0.8417$V           & $3.0301$V                                                     & $3.0286$V   &    $3.6$        & $0\%$     \\ \hline
  -$0.8190$V & -$0.8192$V           & -$2.9491$V                                                    & -$2.9584$V  &    $3.61$       & $0.28\%$     \\ \hline
  -$2.9420$V &-$2.9439$V            & \begin{tabular}[c]{@{}l@{}}mætning\\ -$10.5980$V\end{tabular} & -$4.1651$V  &    $\times$     & $\times$    \\ \hline
	\end{tabular}
	\caption{I tabellen ses resultaterne fra testen med forstærkeren med en faktor $3.6$.}
	\label{Tab:faktor3_test}
\end{table}
\noindent Det ses i \tableref{Tab:faktor3_test}, at forstærkeren går i mætning, hvis inputsignalet svarer til spændingen fra filtret målt ved $\pm90^{\circ}$ hældning af accelerometeret. Dette sker grundet afgrænsningen af arbejdsområdet. Gruden til at mætningsgrænsen for den negative forsyning er lavere end for den positive forsyning, er at TL$081$ har et input common mode voltage range på hhv. -$12$V og $15$V ved en spændingsforsyning på $\pm 15$V \cite{Corporation1995}. \\
Der ses, at forstærkeren overholder kravene fra afsnit \ref{OpsamlingsAfs} på side \pageref{OpsamlingsAfs} samt ligger inde for tolerancerne, hvorfor forstærkeren i tilpasningsblokken accepteres.