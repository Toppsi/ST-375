% !TeX spellcheck = da_DK
\subsection{Filter}\label{Filter_afsnit}
\subsubsection{Teori og design}
%Efter signalet er blevet forstærket skal det filtreres, så alle de uønskede signaler kan dæmpes. Der benyttes kun et lavpasfilter, da det ønskede signal ifølge litteraturen kan ligge i frekvensområdet $0-10$Hz, som beskrevet i afsnit \ref{FilterAfs} på side \pageref{FilterAfs}. Jævnfør pilotforsøget i afsnit \ref{Sec:PilotforsoegKort}, side \pageref{Sec:PilotforsoegKort} blev der målt et signal i frekvensområdet $0-25$Hz og det frekvensområdet sættes derfor til $0-25$Hz, for at filtrere uønskede signaler fra. 
Filtre kan udarbejdes både i en aktiv og passiv form. Hvis signalet ligger i frekvensområdet under $1$MHz, anbefales det at benytte aktive filtre. Aktive filtre benytter operationforstærkerer, kondensatorer og modstande, hvor passive filtre benytter kondensatorer, modstande og spoler. \cite{Carter2013} Der findes flere forskellige typer filtre, heriblandt Butterworth-, Tschebyschev- og Besselfilter. Butterworthfilteret giver maksimal fladhed i pasbåndet og stopbåndet. Tschebyschevfilteret giver den hurtigste overgang fra pasbåndet til stopbåndet. Besselfilteret giver en lineær faserespons, hvilket vil sige, at fasen er lineær med frekvensen.\fxnote{NTK: Fasen angiver, hvor godt et signals frekvensspektrum bliver gengivet} \cite{Carter2013} I dette projekt anvendes et Butterworthfilter, da der ønskes maksimal fladhed i pasbåndet og stopbåndet. De forskellige typer filtre fremgår på \figref{fig:type_filtre}.
\begin{figure}[H]
	\centering
	\includegraphics[scale=0.7]{figures/cProblemloesning/type_filtre.PNG}
	\caption{På figuren ses egenskaberne for de tre filtertyper: Butterworth-, Tschebyschev- og Besselfilter. \cite{Carter2013}}
	\label{fig:type_filtre}
\end{figure}
Når designet af filtret er valgt, findes der to forskellige måder, hvorpå et filter kan implementeres: Sallen-Key topologien (SKT) og Multiple Feedback toppologien (MFT). SKT-metoden er den mest anvendte og tillader separate gain indstillinger samt inverterende og ikke-inverterende konfigurationer, hvorimod MFT-metoden benyttes i filterdesign med høj gain-nøjagtighed (Q-værdi). I dette projekt benyttes en ikke-inverterende konfiguration grundet den høje indgangsimpedans i den ikke-inverterende terminal, hvorfor SKT-metoden er valgt. Herved forhindres det, at filtret loader fra de forrige blokke. Loading defineres som effekten af, at en komponent trækker strømmen i et kredsløb f.eks. et måleapparat. Loading kan være både ønsket, hvis f.eks. brugen af strøm til aktivering af en LED loader, og uønsket, hvis f.eks. et måleapparat trækker strøm ved afmåling af et signal. Loading vil trække i den samlede strøm fra kredsløbet, og trækker derfor meget strøm fra batterierne. Hvis der vælges et inverterende design for operationsforstærkeren i filterkonfiguration, vil blokken have en lav indgangsimpedans, hvorfor denne blok vil begynde at loade. Der kræves derfor mere strøm for at opretholde outputspændingsniveauet. \cite{webster2009,Carter2013,karni2014}

Jævnfør kravspecifikationer af lavpasfilteret i afsnit \ref{FilterAfs} på side \pageref{FilterAfs} kræves det, at filteret har en minimumsdæmpning af stopbåndet på $(min_{A})$ på $14$dB og der accepteres en maksimal dæmpning af pasbåndet $(max_{A})$ på $3$dB. Derudover skal lavpasfilteret have en pasbåndfrekvens $(\omega_p)$ på $25$Hz, samt en stopbåndfrekvens $(\omega_s)$ på $45$Hz. På \figref{fig:Lavpasfilter_generisk} fremgår en illustration af, hvad de forskellige parametre beskriver. Ydermere skal filteret kunne modtage et signal i området mellem $\pm3$V. Dette afhænger af valget af operations forstærker samt spændingsforsyning til denne. Til filteret vil der blive brugt en TL$081$ operations forstærker, som ifølge databladet vil give et arbejdesområde på $\pm13.5$V med en spændingsforsyning på $\pm15$V  \cite{Corporation1995}. I designet af filteret bliver der brugt en spændingsforsyning på $\pm5.5$V jævnført \ref{Spaendingsforsying} på side \pageref{Spaendingsforsying}. Derfor må det forventes at TL$081$ vil kunne give et arbejdesområde på  minimum $\pm3$V uden signalet vil blive klippet. %når der bruges en spændingsforsyning på $\pm5.5$V. 

\begin{figure}[H]
	\centering
	\includegraphics[scale=1]{figures/cProblemloesning/Lavpasfilter_generisk.PNG}
	\caption{På figuren ses et bodeplot filter, hvor de fire karakteristika $(min_{A})$, $(max_{A})$, $(\omega_p)$ og $(\omega_s)$ af et lavpasfilter er angivet. \cite{Carter2013}}
	\label{fig:Lavpasfilter_generisk}
\end{figure}
\noindent Med udgangspunkt i de enkelte parametre af lavpasfilteret kan den pågældende orden af filteret bestemmes vha. \eqref{eq:lavpasfilter} for overføringsfunktionen:
\begin{equation} \label{eq:lavpasfilter}
A(\omega_s) = 10 \text{log} \cdot \left[1 + \epsilon^2 \cdot (\frac{\omega _s}{\omega _p})^{2N}\right] 
\end{equation}

\noindent I \eqref{eq:lavpasfilter} betegner $A(\omega _s)$ den minimale dæmpning, der kræves af stopbåndet. $(\omega_p)$ og $(\omega_s)$ er pasbåndfrekvensen og stopbåndfrekvensen, som begge er angivet i Hz. N angiver filtrets orden, og $\epsilon$ er udtrykt ved \eqref{eq:epsilon}:
\begin{equation}\label{eq:epsilon}
\epsilon = \sqrt{10^{A_{max} / 10} -1}
\end{equation}

Lavpasfilterets orden kan herefter bestemmes, jævnfør værdierne fra kravspecifikationerne fra afsnit \ref{FilterAfs}, side \pageref{FilterAfs} ved at indsætte disse værdier i \eqref{eq:lavpasfilter}. Udregningerne vil se ud som følgende:
\begin{eqnarray}\label{eq:orden}
\epsilon = \sqrt{10^{3dB /10} -1} = 0.998 \\ 
14\text{dB} = 10 \cdot \text{log} \left[1 + \epsilon ^2 \cdot (\frac{45\text{Hz}}{25\text{Hz}})^{2N}\right] \\
\label{eq:orden3}N = 2.711 \approx 3
\end{eqnarray}
\noindent Det fremgår af \eqref{eq:orden}-\eqref{eq:orden3}, at lavpasfilterets orden bliver $3$. Filterets orden angives i hele tal og derfor afrundes resultatet. Hvis kravene til filteret skal overholdes, skal der altid rundes opad ved udregning af orden. Der anvendes et filter af typen SKT. Et 3. ordens lavpasfilter konstrueres ved at sammensætte et 1. ordens filter med et 2. ordens filter. Af figur \figref{fig:SallenKey1} og \figref{fig:SallenKey2} fremgår hhv. et 1. og 2. ordens lavpasfilter designet efter SKT. \cite{Carter2013}
	
\begin{figure}[H]
	\centering
	\begin{minipage}[b]{0.45\textwidth}
		\includegraphics[width=\textwidth]{figures/cProblemloesning/Lavpasfilter1_teoretisk.PNG}
		\caption{På figuren ses en illustration af et 1. ordens unity-gain Sallen-Key lavpasfilter, hvor værdien C er kondensatoren, R er modstanden. Filterkonfigurationen har en indgangsspænding, Vin, og udgangsspænding, Vout. \cite{Carter2013}}
		\label{fig:SallenKey1}
	\end{minipage}
	\hfill
	\begin{minipage}[b]{0.45\textwidth}
		\includegraphics[width=\textwidth]{figures/cProblemloesning/Sallenlavpas.PNG}
		\caption{På figuren ses en illustration af et 2. ordens unity-gain Sallen-Key lavpasfilter, hvor værdien C er kondensatoren, R er modstandene. Filterkonfigurationen har en indgangsspænding, Vin, og en udgangsspænding, Vout. \cite{Carter2013}}
		\label{fig:SallenKey2}
	\end{minipage}
\end{figure}

\noindent I designet af et 3. ordens filter designes både et 1. og et 2. ordens filter i forlængelse af hinanden, som ses på \figref{fig:filter_Orden}.
\begin{figure}[H]
	\centering
	\includegraphics[scale=0.7]{figures/cProblemloesning/Filter_Orden.PNG}
	\caption{På figuren ses, hvordan et 3. ordens filter designes ved et 1.- og 2. ordens filter i forlængelse af hinanden. Værdierne $a_{1}$, $a_{2}$ og $b_{2}$ er fastsatte værdier, som kan aflæses i en tabel for Butterworth koefficienterne. \cite{Carter2013}}
	\label{fig:filter_Orden}
\end{figure}
\noindent 3. ordens filtret designes efter SKT-metoden bestående af tre modstande, tre kondensatorer og to operationsforstærkere. Et 1. ordens lavpasfilter skal benytte en modstand, kondensator og operationsforstærker, hvoraf værdien for modstanden kan udregnes ved \eqref{eq:Lavpas1Modstande}. I \eqref{eq:LavpasModstande} udregnes modstandene for et  2. ordens lavpasfilter, som skal benytte to modstande, to kondensatorer og en operationsforstærker. \cite{Carter2013}
\begin{eqnarray} \label{eq:Lavpas1Modstande}
R_{1} = \frac{a_1}{2 \cdot \pi \cdot f_c \cdot C_1} \\ 
\label{eq:LavpasModstande}R_{2,3} = \frac{a_2 \cdot C_3 \pm \sqrt{{a_2}^2 \cdot C_3^2 - 4 \cdot b_2 \cdot C_2 \cdot C_3}}{4 \pi \cdot f_c \cdot C_2 \cdot C_3}
\end{eqnarray}
\noindent Værdien for $C_{1}$ i \eqref{eq:Lavpas1Modstande} og \eqref{eq:LavpasModstande} er nødvendigvis ikke det samme, da disse to ligninger er uafhængige af hinanden. Der aflæses i en tabel for Butterworth koefficienterne, at $a_{1}$, $a_{2}$ og $b_{2}$ skal være 1.0000. For at finde reelle værdier under kvadratroden i \eqref{eq:LavpasModstande} skal følgende være opfyldt:
\begin{equation} \label{eq:kondensator}
C_3 \geq C_2 \frac{4 \cdot b_2}{a_2^2}
\end{equation}
I \eqref{eq:LavpasModstande} står C for kondensatorer, R står for modstande og $a_1$ og $b_1$ er konstanter, mens $f_c$ er den valgte pasbåndsfrekvens i Hz, jævnfør kravspecifikationerne afsnit \ref{FilterAfs} på side \pageref{FilterAfs}. 

\noindent For at udregne modstandene fastsættes $C_1$ til 100nF for hhv. både 1. og 2. ordens lavpasfiltrene. Når $C_1$ er bestemt, kan $C_2$ for 2. ordens lavpasfilteret beregnes ved at benytte følgende \eqref{eq:kondensator}. Som grundregel skal $C_3$ være over dobbelt så stor som $C_2$:
\begin{equation}  
C_3 \geq 100\text{nF} \frac{4\cdot 1}{1.000^2} = C_3 \geq 400\text{nF}
\end{equation}

\noindent Ud fra ovenstående ligning vælges $C_3$ til at være 470nF for at opfylde \eqref{eq:kondensator}. Når værdierne for C er bestemt kan modstandene for filteret bestemmes. For et 1. ordens lavpasfilter benyttes \eqref{eq:Lavpas1Modstande} til at beregne $R_1$ og for et 2. ordens lavpasfilter anvendes \eqref{eq:2ordenmodstand} for at beregne $R_1$ og $R_2$. 
\begin{equation} \label{eq:1ordenmodstand}
R_{1} = \frac{1}{2 \cdot \pi \cdot 25 \cdot 100nF} R_{1} = 63661.9772 \Omega
\end{equation}
\begin{equation}
\label{eq:2ordenmodstand}R_{2,3} = \frac{1.0000 \cdot 470\text{nF} \pm \sqrt{1.0000^2 \cdot 470\text{nF}^2 - 4 \cdot 1.0000 \cdot 100\text{nF} \cdot 470\text{nF}}}{4 \pi \cdot 25\text{Hz} \cdot 100\text{nF} \cdot 470\text{nF}} = \begin{cases} R_{2} = 19546.69414 \Omega \\ R_{3} =  44115.28306 \Omega \end{cases}
\end{equation}
\noindent Filterets værdier for kondensatorerne og modstandene er nu udregnet for hhv. 1. og 2. ordens filter. Filterkonfigurationen kan nu simuleres i LTspice for at bestemme kvaliteten af det 3. ordens filter, der er blevet designet.

\begin{figure}[H]
	\centering
	\includegraphics[scale=0.35]{figures/cProblemloesning/Lavpasfilter1_LTspice.PNG}
	\caption{På figuren ses designet af det teoretiske kredsløb for lavpasfilteret med udregnede værdier for de enkelte modstande og kondensatorer. Filteret er designet i LTspice.}
	\label{fig:lavpasfilter1_LTspice}
\end{figure}

\subsubsection{Simulering}
For at udføre en simulering af 3. ordens lavpasfilteret, foretages en AC-analyse, der beskriver forholdet mellem frekvensindholdet og filterets dæmpning. Kredsløbet simuleres med et inputsignal, der har en amplitude på $1$V. Der foretages en simulering i LTspice ud fra kravspecifikationerne i afsnit \ref{FilterAfs} på side \pageref{FilterAfs}. I simuleringen vil der blive undersøgt, hvorledes kravene stemmer overens med kravspecifikationerne, ved at betragte et bodeplot over det simulerede 3. ordens filter.
\begin{figure}[H]
	\centering
	\includegraphics[scale=0.35]{figures/cProblemloesning/Lavpasfilter_LTspice.PNG}
	\caption{På figuren ses designet af 3. ordens lavpasfilteret. Filteret simuleres i LTspice vha. en AC-analyse med et inputsignal, der har en amplitude på $1$V.}
	\label{fig:lavpasfilter_LTspice}
\end{figure}
\begin{figure}[H]
	\centering
	\includegraphics[scale=0.38]{figures/cProblemloesning/Lavpasfiltergraf_LTspice2.PNG}
	\caption{På figuren ses en illustration af et bodeplot, der viser 3. ordens filterets frekvensindhold målt i Hz over dæmpningen målt i dB. Lavpasfilteret er simuleret i LTspice.}
	\label{fig:lavpasfiltergraf_LTspice1}
\end{figure}
\noindent Af bodeplottet fremgår det, at det simulerede filter har en maksimal amplitude i db på $-3.012$ ved en pasbånsfrekvensfrekvens på $25$Hz, hvilket ikke overholder kravspecifikationerne i afsnit \ref{FilterAfs}, side \pageref{FilterAfs}. Grundet den lave afvigelse accepteres afvigelsen i midlertid og der udføres derfor endnu en simulering af filteret med de reelle modstande, der kan benyttes under implementeringen. \\
Der ses på \figref{fig:lavpasfiltergraf_LTspice1}, at der skal benyttes tre modstande på hhv. $63661.9772\Omega$, $44115.2831\Omega$ og $19546.694\Omega$ til opbygningen af filtret. Ingen af disse modstande findes reelt, hvorfor de nærmeste alternativer er udregnet. For at få $63661.9772\Omega$ skal en $1$M$\Omega$  sættes parallelt med $68$K$\Omega$, hvilket teoretisk giver en afvigelse på $0.013\%$. For at få $44115.2831\Omega$ skal $680$K$\Omega$ sættes parallelt med $47$K$\Omega$, hvilket teoretisk giver en afvigelse på $0.349\%$. For at få en $19546.694\Omega$ sættes $39$K$\Omega$ parallelt med $39$K$\Omega$, hvilket teoretisk giver en afvigelse på $0.239\%$. Disse blev målt, hvilket kan ses i \tableref{Tab:Maalingafmodstande_filter}.
\begin{table}[H]
	\centering
	\begin{tabular}{|l|l|l|l|}
		\hline
		\textit{}                                     & \textit{Teoretisk} & \textit{Måling}    & \textit{Afvigelse} \\ \hline
		\multirow{2}{*}{\textit{$R_{1}$(Parallel) :}} & $1$M$\Omega$       & $1.000006$M$\Omega$  & $\approx 0\%$           \\ \cline{2-4} 
		& $68$K$\Omega$      & $68.079$K$\Omega$ & $0.011\%$           \\ \cline{2-4}
		$R1_{eq}$: & $63661.9772\Omega$ & $63633\Omega$ &  $0.046\%$ \\ \hline
		\multirow{2}{*}{\textit{$R_{2}$(Parallel) :}} & $680$K$\Omega$     & $682.840$K$\Omega$  & $0.418\%$        \\ \cline{2-4} 
		& $47$K$\Omega$      & $47.004$K$\Omega$ & $0.010\%$           \\ \cline{2-4}
		$R2_{eq}$: & $44115.2831\Omega$ & $43976\Omega$ & $0.316\%$ \\ \hline
		\multirow{2}{*}{\textit{$R_{3}$(Parallel) :}} & $39$K$\Omega$      & $38.888$K$\Omega$    & $0.287\%$           \\ \cline{2-4} 
		& $39$K$\Omega$     & $39.043$K$\Omega$        & $0.108\%$           \\ \cline{2-4}
		$R3_{eq}$: & $19546.694\Omega$ & $19482\Omega$  & $0.3310\%$ \\ \hline
	\end{tabular}
	\caption{I tabellen ses, at de anvendte modstande afviger fra den teoretiske værdi, hvilket er forventet af reelle komponenter. Disse afvigelser kommer derved også til at have en effekt på $R_{eq}$. Det er de teoretiske værdier, som bliver benyttet i en ny simulering.}
	\label{Tab:Maalingafmodstande_filter}
\end{table}
Der ses på \figref{fig:lavpasfiltergraf_LTspice1}, at der ved en stopbåndsfrekvens på $45$Hz er amplituden i db på $-15.443$. Dette overholder projektets opstillede krav for filterkonfigurationen ved en minimum dæmpning på $14$ dB i stopbåndsfrekvensen. Filterkonfigurationen med reelle modstande fremgår af \figref{fig:Sim_reel_modstande}.
\begin{figure}[H]
	\centering
	\includegraphics[scale=0.4]{figures/cProblemloesning/Sim_reel_modstande.PNG}
	\caption{På figuren ses et 3. ordens filter med modstande, som har de teoretiske $\ref{fig:Sim_reel_modstande}_{eq}$ værdier - dvs. R$1$ består af en parallelforbindelse mellem $68$K $\Omega$ og $1$M $\Omega$, R$2$ består af en parallelforbindelse mellem $47$K $\Omega$ og $680$K $\Omega$ og R$3$ består af en parallelforbindelse mellem to modstande på $32$K $\Omega$.}
	\label{fig:Sim_reel_modstande}
\end{figure}

\begin{figure}[H]
	\centering
	\includegraphics[scale=0.35]{figures/cProblemloesning/Sim_reel_graf.PNG}
	\caption{På figuren ses en illustration af et bodeplot, der viser 3. ordens filterets frekvensindhold målt i Hz over dæmpningen målt i dB med den teoretiske værdi for de reelle modstande. Lavpasfilteret er simuleret i LTspice.}
	\label{fig:Sim_reel_graf}
\end{figure}

\noindent Ud fra bodeplottet ses det, at det simulerede filter med reelle værdier har en maksimal amplitude på $-2.985$dB ved en pasbåndsfrekvens på $25$Hz, hvilket overholder kravspecifikationerne i afsnit \ref{FilterAfs}, side \pageref{FilterAfs}. Der aflæses, at ved stopbåndsfrekvensen på $45$Hz er amplituden $-15.389$dB, hvilket fortsat er acceptabelt ift. kravspecifikationerne for filterkonfigurationen. 

\subsubsection{Implementering og test} 
Der ses på \figref{fig:lavpasfilter_LTspice} at der benyttes 3 kondensatorer til at designe kredsløbet. I \tableref{Tab:Maalingafmodstande_filter} ses målingerne af de benyttede modstande. I \tableref{Tab:Maalingfilter} ses de reelle værdier for de benyttede kondensatorer.
\begin{table}[H]
	\centering
	\begin{tabular}{|l|l|l|l|}
		\hline
		\textit{$C_{1}$ :}                            & $100$n             & $98$n              & $2.00\%$           \\ \hline
		\textit{$C_{2}$ :}                            & $470$n             & $464$n             & $1.29\%$           \\ \hline 
		\textit{$C_{3}$ :}                            & $100$n             & $98$n              & $2.00\%$           \\ \hline
	\end{tabular}
	\caption{I tabellen ses, at de anvendte kondensatorer afviger fra den teoretiske værdi, hvilket er forventet af reelle komponenter. Det er en acceptabel afvigelse, hvorfor disse kan anvendes til implementering}
	\label{Tab:Maalingfilter}
\end{table}
\noindent Herefter implementeres kredsløbet. I testen anvendes spændingsforsyningen, der leverer $5.5530$V og dermed har en afvigelse på $0.96\%$, og en funktionsgenerator som inputsignal, samt et multimeter til aflæsning af dæmpningen. Funktionsgeneratoren sættes til de ønskede frekvenser i intervaller af $1$Hz fra $10$Hz til $50$Hz. Multimeteret måler dæmpningen af signalet ved at indstille et referencepunkt ved inputspændingen. Referencepunktet sættes hvor spændingen ændre sig mindst, dette sker ved $10$Hz. Referencen opfattes som signalets input. Herefter beregner multimeteret forskellen mellem input- og outputspændingen ved ligning \ref{eq:daempningsfaktor} og beregner dæmpningen i dB ved ligning \ref{eq:daempningsfaktoridB}. Outputtet fra multimeteret måles for hvert frekvens og noteres. Ud fra disse målinger plottes en graf af dæmpningen i MatLab, hvilket er illustreret på \figref{fig:Lavpas_Matlab}.  

\begin{figure}[H]
	\centering
	\includegraphics[scale=0.45]{figures/cProblemloesning/Lavpas_Matlab.PNG}
	\caption{På figuren ses dæmpningen som en graf over de målte frekvenser i Hz som funktion af outputtet i dB. På grafen er der angivet knækfrekvens og stopbåndfrekvens.}
	\label{fig:Lavpas_Matlab}
\end{figure}

\begin{table}[H]
	\centering
	\begin{tabular}{|l|l|l|l|}
		\hline
		& \textit{Krav for dæmpning i dB} 	& \textit{Test af dæmpning i dB}  &\textit{Afvigelse} \\ \hline
		Pasbåndsfrekvens & $3$	& $2.7360$	    & $8.8\%$ \\ \hline
		Stopbåndsfrekvens & $14$    & $14.8480$    & $6.1\%$  \\ \hline
	\end{tabular}
	\caption{I tabellen ses afvigelserne for dæmpningen i pasbånds- og stopbåndsfrekvensen ift. kravspecifikationerne og den foretaget test.}
	\label{Tab:Tolerance}
\end{table}
\noindent Ud fra de testede værdier i \tableref{Tab:Tolerance} og kravspecifikationer i afsnit \ref{FilterAfs} på side \pageref{FilterAfs} ses ingen afvigelse på pasbånds- og stopbåndsfrekvensen, men en variation i dæmpningen for både pasbånds- og stopbåndsfrekvensen på hhv. $8.8\%$ og $6.1$\% ift. kravspecifikationerne. Dæmpningen i pasbåndsfrekvensen skal maksimal være $3$dB med en acceptabel afvigelse på -$15\%$, dvs. i intervallet $2.55$-$3$dB. Dette betyder, at testen ligger indenfor tolerancekravet med en dæmpning i pasbåndet på $2.7$dB. Derudover skal lavpasfilteret dæmpe med minimum $14$dB i stopbåndsfrekvensen med en acceptabel afvigelse på +$15\%$. Denne tolerance overholdes i testen med en dæmpning på $14.8480$dB, da de acceptable dæmpningsværdier ligger i intervallet $14$-$16.10$dB.\\

\noindent \textbf{Sammenligning af det simulerede og implementerede lavpasfilter} \\
På \figref{fig:sammenligning_sim_imp} er afvigelsen for det simulerede lavpasfilter ift. det implementerede lavpasfilter visualiseret vha. et bodeplot. 
\begin{figure}[H]
	\centering
	\includegraphics[scale=0.4]{figures/cProblemloesning/sammenligning_sim_imp.PNG}
	\caption{På figuren ses to bodeplots for det simulerede lavpasfilter, markeret med rød, og det implementerede lavpasfilter, markeret med blå. Filternes fase fremgår ikke i bodeplottet men kan ses på de to tidligere figurer. Bodeplottet er udarbejdet i Matlab.}
	\label{fig:sammenligning_sim_imp}
\end{figure} 
Bodeplottet for det simulerede lavpasfilter er udarbejdet ud fra $20$ punkter og det implementerede er udarbejdet ud fra $40$ punkter, i intervallet $10$ til $50$. Det fremgår af grafen, at der er forskel fra at konstruere et teoretisk lavpasfilter ift. at til implementere et. Det er ikke muligt at implementere et lavpasfilter, der er ens med det simulerede, grundet reelle komponenters afvigelse. Det fremgår, at det simulerede lavpasfilter ligger under det implementerede lavpasfilter, hvilket stemmer overens med at dæmpningsværdien er højere i det simulerede ift. det implementerede. \\
Der ses ud fra testen og sammenligningen, at lavpasfilteret overholder kravspecifikationerne afsnit \ref{FilterAfs}, side \pageref{FilterAfs}. Derudover ligger afvigelserne indenfor tolerancekravene og derfor accepteres lavpasfilteret. 