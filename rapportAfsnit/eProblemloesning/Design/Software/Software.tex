% !TeX spellcheck = da_DK
\subsection{Software}
Efter konverteringen af det analoge signal til digitalt i ADC’en indsendes data til en computer. Jævnfør afsnit \ref{subsec:software}, side \pageref{subsec:software} skal patienternes data behandles i form af grafisk visualisering deres hældningsresultater, samt gemmes til senere brug og analyse. For at imødekomme disse krav, bruges en computer med software programmerne; Scopelogger og Matlab, hvor Scopelogger anvendes til at optage det  digitale signal fra NIDAQ'en og Matlab benyttes til at behandle opsamlede data. Til behandling og lagring af data udarbejdes en manual for at gøre pogrammets design brugervenligt for det fagkyndige personale, som skal bruge disse data til analyse af patienternes udvikling ift. balancefunktionen. For at programmet skal kunne fungere med en switch knap, jævnfør afsnit \ref{formål_anvendelse}, side \pageref{formål_anvendelse} skal patienternes øvelsesresultater gemmes alt efter pågældende udførte øvelse hhv. 'anatomisk_data' og 'SRT_dato'. 
Det fagkyndige personale skal åbne scriptet 'Patient_Oevelse' i Matlab for at kunne køre computerprogrammet, der skal behandle apopleksipatienternes data fra Scopelogger. Indsamlede data navngives alt efter den pågældende øvelse, der er udført og hentes ind i Matlab. Af nedestående flowdiagram \ref{Flow_manual} fremgår computerprogrammets manual.

\begin{figure}[H] 
	\centering 
	\includegraphics[scale=0.5]{figures/cProblemloesning/Flow_manual.PNG}
	\caption{af flowdiagrammet fremgår computerprogrammets manual, der omhandler hvorledes det fagkyndige skal behandle og gemme apopleksiatienterne øvelsesresultater}
	\label{Flow_manual}
\end{figure} 



%Vi kunne for at gøre programmet mere brugervenligt udarbejde en manual til systemet. Derudover kunne det være fordelagtigt at det fagkyndige personalet at have mulighed for at ændre akserne, og her tænkes specielt tidsaksen. Dette tænkes da systemet skal kunne optage selvtræning og da det ikke forventes at patienten tænder og slukker systemet efter hver enkelt øvelse. Dette vil både være svært at huske for patientgruppe (gamle og måske med kognitive komplikationer) og så ville de måske glemme at slå den til igen. Derfor tænkes det at de lader systemet optage signaler under hele selvtrænings sessionen. Det vil midlertidigt blive til utrolig meget data, hvor en stor del af dataen vil være pauser for patienten og derved ubrugelig data for det fagkyndige personale. Det ville derfor være smart at hvis de forholdsvis nemt kunne udvælge perioder i dataen, de gerne vil undersøge nærmere. 
%Derudover skal vi også have gjort patienten og det fagkyndige personale opmærksomme på hvilken øvelse der foretages!
%Det kunne måske være en ide, at vi i grafen kunne ligge tærskelværdierne ind som en form for linjer, så det blev nemmere at aflæse grafen
%Skal vi have en løbende visning af signalet i en graf eller er det bare til sidst! 

%Vores program skal altså kunne:
%  -Optage signalet
% - Behandle signalet således, alt efter sværhedsgrad 
%  - Plotte signalet
% - Gem data'en 
