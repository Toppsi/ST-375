% !TeX spellcheck = da_DK
\subsection{Feedbackkonfiguration}
\subsubsection{Teori og design}
Jævnfør \ref{Komparatorafsnit} på side \pageref{Komparatorafsnit} anvendes en komparator til at sammenligne to inputspændinger. Komparatoren der anvendes er af typen LM311. Ved denne type placeres en 100n kondensator ved inputtet for at gøre komparatoren mere præcis og undgå svingninger \cite{Instruments2015}. Der vil blive anvendt flere komparatorer og i dette tilfælde tilkobles komparatorernes output en LED, en vibrator, en modstand og den positive spændingsforsyning ($+V_{cc}$). I komparatorernes ikke-inverterende terminaler tilsluttes outputtet fra forrige blok, hvor inputtet i de inverterende terminaler fungerer som referencespænding, der forsyner med de beregnede tærskelværdier. Komparatorerne kan derfor have to forskellige outputs afhængig af inputspændingen. Hvis inputsignalet ligger udenfor de beregnede tærskelværdier vil outputtet være $0V$ og LEDerne samt de to vibratorer vil ikke blive aktiveret. Er inputtet derimod indenfor de beregnede tærskelværdier, vil outputtet svare til ground, da strømmen fra $+V_{cc}$ vil løbe igennem komparatorerne og derefter til ground\fxnote{det vil vel løbe til komparatorens emitter terminal }. Derved opnås et spændingsfald over den positive (anode) og negative (katode) pol for LEDerne og vibratorerne på en værdi, der ligger over det minimale spændingsfald, der kræves for en aktivering. LEDerne og vibratorerne vil derved blive aktiveret og fungere som feedback til patienten. \\

\noindent\textbf{Design af feedbackkonfiguration} \\
LEDernes katode tilkobles komparatorens output, imens anoden tilkobles $+V_{cc}$. Jævnfør kravspecifikationerne i afsnit \ref{KomparatorAfs} på side \pageref{KomparatorAfs} skal LEDerne aktiveres på fem forskellige stadier. Til denne aktiveren anvendes otte komparatorer, da det første stadie, indeholdende den grønne LED, både har en positiv og negativ tærskelværdi og derfor kræver to komparatorer. De valgte tærskelværdier kan designes på forskellige måder hhv. som to spændingstræer eller otte spændingsdelere, hvoraf der er fordele og ulemper ved begge metoder. I dette projekt anvendes otte spændingsdelere, hvormed to indgår i en vindues-komparatorkonfiguration for den grønne LED, og de seks resterende fungerer som almindelige komparatorer, der aktiveres over en bestemt tærskelværdi. Fordelen ved at vælge dette design er, at modstandene i et spændingstræ påvirker hinanden, hvilket kan ændre tærskelværdierne, hvis én af modstandene ikke fungerer optimalt. Ulempen ved at anvende spændingsdelere er, at der benyttes flere modstande ved denne konfiguration. Feedbackkonfigurationen kan inddeles i to dele; en for hældning i hhv. positiv og negativ retning. De otte spændingsdelere udgøres af i alt $12$ modstande (R$1$-R$12$). Derudover består kredsløbet af en spændingsreference ($+V_{ref}$) på $2.5$V og syv modstande (R$13$-R$19$) mellem LEDerne samt vibratorerne og $+V_{cc}$. Modstandene(R$13$-R$14$, R$16$-R$17$ og R$19$) sikre at batteriet ikke drænes og bestemmer mængden af strøm til LEDerne. Vindues-komparatorkonfigurationen designes ved at placere en LED (D$3$) i emitter-terminalen på komparatorerne (U$3$) og collector-terminalen komparatorerne (U$4$). Feedbackkonfigurationen fremgår af \figref{fig:komparator_uden_vaerdi} 

\begin{figure}[H] 
	\centering
	\includegraphics[scale=0.7]{figures/cProblemloesning/komparator_uden_vaerdi.PNG}
	\caption{På figuren ses feedbackkonfigurationen, hvor kredsløbet består af to dele; en for hældning i positiv og negativ retning. Der er konstrueret en vindues-komparatorkonfigurationen designet vha. komparatorerne (U$3$-U$4$) for den grønne LED, der anvender en tærskelværdi fra hhv. den negative og positive del af kredsløbet. Kredsløbet er konstrueret i LTspice.}
	\label{fig:komparator_uden_vaerdi}
\end{figure}

%%%%%%%%%%%%%%%%%%%%%%%%%%%%%%%%%%%%%%%%%%%%%%%%%%%%%%%%%%%%%%%

\noindent\textbf{Beregning af tærskelværdier og modstandene R$1$-R$12$ i spændingsdelerne} \\
Det ønskes, at LEDerne skal lyse ved bestemte kropshældninger, dvs. ved bestemte tærskelværdier. Inputsignalet afhænger af den pågældende hældningsgrad, hvilket for komparatoren vil være en bestemt spænding. Komparatorens tærskelværdierne kan beregnes, da værdien i volt pr. grad i hhv. positiv og negativ retning, jævnfør \eqref{taerskelvaerdi_pr_grad} i pilotforsøget i bilag \ref{Pilotforsoeg} på side \pageref{Sec_Pilot_Data}, samt forstærkningsværdien af signalet fra forrige blokke, opsamlings- og tilpasningsblokken, er kendte værdier. Jævnfør tærskelværdierne i afsnit \ref{Komparatorafsnit} på side \pageref{Komparatorafsnit} beregnet ud fra følgende formel:
\begin{equation}\label{pr_grad} 
\text{Tærskelværdi i positiv retning} = \frac{0.0037\text{V}*9.1*3.6*\text{hældningsgrad}} = tærskelværdi
\text{Tærskelværdi i negativ retning} = \frac{0.0036\text{V}*9.1*3.6*\text{hældningsgrad}} = tærskelværdi
\end{equation}

De udregnede tærskelværdier fremgår af \figref{taerskelvaerdier}. 
\begin{figure}[H]
	\centering
	\includegraphics[scale=1.]{figures/cProblemloesning/Taerskelvaerdier.PNG}
	\caption{På figuren fremgår de beregnede tærskelværdier og hvilket farve, der lyser ved de de enkelte tærskler.}
	\label{fig:taerskelvaerdier}
\end{figure}


Der anvendes en spændingsreference, bestående af en regulator, der kan levere en konstant spænding på $2.5$V til feedbackkonfigurationen jævnført \ref{subsec:Spaendingsref_Komparator} på side \pageref{subsec:Spaendingsref_Komparator}. Eftersom spændingen ved anvendelse af et batteri vil falde som funktion af tiden benyttes spændingsreferencen, så denne kan holde en fast referencespænding i kredsløbet. Eftersom spændingsreferencen også skal anvendes til de negative tærskelværdier benyttes en inverterende forstærker med et gain på $1$ i designet af feedbackkonfigurationen, hvilket fremgår af figur \figref{fig:komparator_uden_vaerdi}. Ved denne konfiguration vendes signalet, uden at blive forstærket, og kan på denne måde benyttes som referencespænding til negative inputspændinger. For at bestemme R$1$-R$12$ i spændingsdelerne, så LEDerne lyser ved de ønskede tærskelværdier defineres R$1$ til en bestemt værdi. I dette tilfælde fastsættes R$1$ til at være $10$K$\Omega$. Denne værdi er også gældende for R$2$-R$6$, da der benyttes en spændingsdeler for hver tærskelværdi. Derudover er de ønskede tærskelværdier ($V_out$) og spændingsreferencen ($V_in$) også kendte værdier og R$7$-R$12$ kan dermed udregnes vha. den generelle formel for en spændingsdeler jævnført \eqref{eq:Spaendingsdeler} i afsnit \ref{subsec:Spaendingsref} på side \pageref{subsec:Spaendingsref}. 

Dette medfører at R$7$-R$12$ giver følgende resultater:\\
R$7$ = $16821\Omega$ \\
R$8$ = $6285\Omega$ \\
R$9$ = $1068\Omega$ \\
R$10$ = $1042\Omega$ \\
R$11$ = $6039\Omega$ \\
R$12$ = $15765\Omega$ \\

%%%%%%%%%%%%%%%%%%%%%%%%%%%%%%%%%%%%%%%%%%%%%%%%%%%%%%%%

\noindent\textbf{Beregning af modstande for den visuelle del af feedbackkonfigurationen} \\
Til den visuelle del af feedbackkonfigurationen anvendes LEDer. En LED har to terminaler; en anode og en katode. Når en ideel LED tændes skyldes det, at der løber strøm fra anoden til katoden og der vil her være et spændingsfald over LEDen på 0V. Dette kaldes fremadgående spændings. Når strømmen løber fra katoden til anoden vil LEDen være slukket, da der ideelt ikke løber strøm igennem kredsløbet.... \fxnote{mere burde tilføjes.} Jænvfør kravspecifikationerne i afsnit \ref{KomparatorAfs} på side \pageref{KomparatorAfs} for komparatoren skal forsyningsspændingen være $5.5$V. De anvendte LEDer i systemet er: en grøn L-$53$LG $5$mm (D$3$), to gule L-$53$LY $5$mm (D$2$ og D$4$) og to røde L-$53$LI $5$mm (D$1$ og D$5$). LEDerne kræver en minimum spænding på $2$mA for at lyse og $20$mA hvis de skal give et tydeligt lys.  Spændingsfaldet over LED-dioderne ligger maksimalt i intervallet $2.0$V til $2.2$ V (rød: $2.0$, gul: $2.1$ og grøn: $2.2$), men typisk mellem $1.7$V-$1.9$V. LED-dioderne skal derudover forsynes med $2$mA for at fungere, men kan forsynes op til \fxnote{Tjek og 150mA er rigtigt}$150$mA, før de brændes af. LEDerne forsynes af en $5.5$V spændingsforsyning og tilkobles, som sagt, tilhørende modstande for bla. at undgå at LED-dioderne brænder af. \cite{kingbright} Spændingsfaldet over dioderne samt den spænding LEDerne som minimum skal bruge for at lyse er kendte værdier, dvs. modstandene R$12$-R$17$ kan derfor findes vha. Ohms lov. Nedenstående udregning beregnet en værdi af modstandene, hvis spændingsforsyningen forsyner kredsløbet med $5.5$V og LEDernes med $20$mA for at aktiveres:
\begin{equation}
R = \dfrac{5.5V - 2.2V}{0.02A} = 165\Omega
\end{equation}
\noindent Dermed sættes modstandene R$13$-R$14$, R$16$-R$17$ og R$19$ til $165\Omega$ for at sikre, at der er $20$mA til LEDerne, så de kan give et tydeligt lys og derudover sørge for, at der ikke er for meget strøm i kredsløbet, så batterierne drænes. 

%%%%%%%%%%%%%%%%%%%%%%%%%%%%%%%%%%%%%%%%%%%%%%%%%%%%%%%%%%%
\noindent\textbf{Beregning af modstande for den somasensoriske del af feedbackkonfigurationen} \\
Til den somasensoriske del af feedbackkonfigurationen benyttes vibratorer, jævnført \ref{KomparatorAfs} på side \pageref{KomparatorAfs}. En vibrator er en elektrisk motor, der skaber svingninger, hvilket medfører en vibration.\cite{Radaktionen2009}. Vibratorerne der anvendes er af typen C1026B. Den har et drift arbejdsområde på $2.7-3.3$V, der typisk ligger på $3$V og starter ved $2.3$V. Vibratorerne skal derfor have en spændingsforsyning på $3$V. Derudover er startstørmmen $120$mA, og når motoren kører er driftstrømmen på $90$mA. Herefter sker vibrationen med en frekvens på $10-55$Hz, afhængig af spændingsforsyningen, samt strømmen i kredsløbet. \cite{Machinery2009} Jævnfør kravspecifikationerne benyttes vibratorer med dimensionerne; $1$cm i diameter og $0.27$cm i tykkelse, så de kan påsættes patienternes hænder. 
 
For at opnå et tilstrækkeligt strømniveau anvendes en transistor af typen; BS$170$ for at generere en større mængde strøm i kredsløbet. Transistoren placeres mellem vibratoren og komparatoren. Det er designet således, at transistoren tændes og vibratoren aktiveres, når komparatoren er slukket, modsat komparatorerne, der benyttes i forbindelse med LEDerne. Dette skyldes, at når komparatoren er tændt vil den trække spænding tilkoblet et knudepunkt mellem komparatorens collector-terminal og trasistorens gain-terminal. Modstandene R$15$ og R$18$ er placeret ved spændingsforsyningen på $5.5$V ($+V_{cc}$) og bestemmes til at være $100$K. Disse skal sørge for at kredsløbet ikke bruger for meget strøm, hvorfor der er valgt modstande med en høj værdi. Når komparatoren slukker vil spændingen aktivere transistoren og der dannes forbindelse mellem transistorens drain- og source-terminal. Dette vil aktivere vibratorerne. Spændingsforsyningen til vibratorerne er på $3.4$V ($V+$) og forsynes af spændingsregulatoren. Eftersom vibratoren kun skal forsynes med $3$V benyttes en schottky-diode (BAT$41$), der nedsætter spændingen til under $3$V. Schottky-dioden fungerer således, at der sker et spændingsfald afhængig af strømmen, der løber igennem dioden. Spændingsforsyningen på $3.4$V aktiveres, når komparatoren slukker og der er forbindelse mellem transistorens drain- og source-terminal, da strømmen kan løbe til ground i source-terminalen. Den somasensoriske del af feedbackkonfigurationen fremgår af \figret{komparator_uden_vaerdi}.

\fxnote{hvorfor får vi 90mA ud af transistoren?}
 
\subsubsection{Simulering}
Den visuelle og somasensoriske del af feedbackkonfigurationen simuleres med et sinus-signal, der svinger mellem $\pm3$V. Dette gøres for at simulere signalet, der kommer fra den forrige blok med et arbejdsområdet på $\pm3$V.I simuleringen testen, hvorvidt tærskelværdierne kan accepteres ift. kravspecifikationer. Kredsløbet simuleres ikke med en spændingsreference, men derimod en spændingsforsyning på $2.5$V, der indikerer spændingsreferencen. Af figur \figref{fig:komparator_samlet} fremgår hele feedbackkonfigurationen, der simuleres i LTspice ved at sende et DC signal ind.   

\begin{figure}[H]
	\centering
	\includegraphics[scale=0.3]{figures/cProblemloesning/komparator_samlet.PNG}
	\caption{Figuren illustrerer feedbackkonfigurationen, der simuleres i LTspice med beregnede modstande.}
	\label{fig:komparator_samlet}
\end{figure}
 
\noindent\textbf{Simulering af den visuelle del af feedbackkonfigurationen} \\
Til simulering af den visuelle den af feedbackkonfiguration anvendes komparatorer af typen LM$311$ og LEDer af typen xx, da de reelle komponenter ikke kan vælges i LTspice. Kredsløbet simuleres i LTspice for at teste, hvorvidt den visuelle del af feedbackkonfiguration opfylder de opstillede kravspecifikationer, jævnfør afsnit \ref{KomparatorAfs} på side \pageref{KomparatorAfs}. Af figur \figref{fig:komparator_visuel_simulering_samlet1} fremfår tærskelværdierne for den visuelle del af feedbackkonfigurationen. 

%\begin{figure}[H]
%	\centering
%	\includegraphics[scale=0.3]{figures/cProblemloesning/komparator_visuel_simulering_taeerskel.PNG}
%	\caption{Af figuren fremgår de simulerede tærskelværdier, hvor spændingen (V) op ad Y-aksen og tiden i sekunder hen af X-aksen. .....}
%	\label{fig:komparator_visuel_simulering_taerskel}
%\end{figure}
%\noindent De beregende tærskelværdier stemmer overens med de simulerede værdier, hvorfor tolerancekravene accepteres. Herefter simuleres funktionen af den visuelle del af feedbackkonfigurationen, hvilket fremgår af \figref{komparator_visuel_simulering_samlet1}

\begin{figure}[H]
	\centering
	\includegraphics[scale=0.3]{figures/cProblemloesning/komparator_visuel_simulering_samlet1.PNG}
	\caption{På figuren ses simuleringen af den visuelle del af feedbackkonfiguration. Den sorte kurve er sinus-signalet, der illustrerer blokkens inputsignal. Den røde kurve illustrerer de røde dioder, den gule kurve de gule dioder og den grønne kurve den grønne diode. Når inputsignalet når de definerede tærskelværdier, vil kurverne gå i negativ mætning og LED-dioderne vil lyse. Kredsløbet er simuleret i LTspice.}
	\label{fig:komparator_visuel_simulering_samlet1}
\end{figure}
På figur \figref{fig:komparator_visuel_simulering_samlet} fremgår det, at ved de enkelte tærskelværdier går signalet i negativ mætning, hvilket får LED-dioderne til at lyse. Afvigelsen mellem det teoretiske og simulerede reference input beregnes og vil fremgå af \ref{Tab:test_reference}

\begin{table}[H]
	\centering
	\begin{tabular}{|l|l|l|l|} \hline
		& \textit{\begin{tabular}[c]{@{}l@{}}Teoretisk\\ reference\\ input\end{tabular}} & \textit{\begin{tabular}[c]{@{}l@{}}Simuleret\\ reference\\ input\end{tabular}} & \textit{\% afvigelse} \\ \hline
		\textit{$13^{\circ}$}  & $1.5758$V                                                                      & $1.5659$V                                                                  & $0.63\%$              \\ \hline
		\textit{$8^{\circ}$}   & $0.9697$V                                                                      & $0.9654$V                                                                  & $0.44\%$              \\ \hline
		\textit{$2^{\circ}$}   & $0.2424$V                                                                      & $0.2415$V                                                                  & $0.37\%$              \\ \hline
		\textit{-$2^{\circ}$}  & -$0.2359$V                                                                     & -$0.2348$V                                                                 & $0.47\%$              \\ \hline
		\textit{-$8^{\circ}$}  & -$0.9495$V                                                                     & -$0.9341$V                                                                 & $0.92\%$              \\ \hline
		\textit{-$13^{\circ}$} & -$1.5332$V                                                                     & -$1.5320$V                                                                 & $0.08\%$        \\ \hline     
	\end{tabular}
	\caption{Af tabelle fremgår de teoretiske og simulerede tærskelværdier ved de enkelte hældningsgrader, samt afvigelsen mellem disse.}
	\label{Tab:test_reference}
\end{table}

%\begin{table}[H]
%	\centering
%	\begin{tabular}{|l|l|l|l|l|l|}
%		\hline
%					& \textit{Tærskelværdier} 	& \textit{Måling til højre} & \textit{Måling til venstre}		&  \textit{\begin{tabular}[c]{@{}l@{}}Afvigelse\\ for højre\end{tabular}}   &  \textit{\begin{tabular}[c]{@{}l@{}}Afvigelse\\ for venstre\end{tabular}}   \\ \hline
%\textbf{$2^{\circ}$} 		& $0.2412$V  				&$0.23815127$V 			&$-0.17571651$V		& $1.3\%$  &$??\%$ \\ \hline
% \textbf{$8^{\circ}$} 		&$0.9648$V					&$0.96551609$V			&$0.96777598$V		& $0.07\%$	&$0.3\%$\\ \hline
%\textbf{-$8^{\circ}$} 		&-$0.9413$V					&-$0.96784376$V 			&-$0.92251539$V		& $2.8\%$	&$2\%$\\ \hline 		
%\textbf{$13^{\circ}$} 		&$1.5679$V 					&$1.5583895$V 		  	&$1.5705552$V		& $0.6\%$	&$0.2\%$\\ \hline
%\textbf{-$13^{\circ}$} 		&-$1.5297$VV 				&-$1.5297296$V		   	&-$1.5297259$V		& $0.002\%$	&$0.002\%$ \\ \hline
%	\end{tabular}
%	\caption{I tabellen ses der, at de anvendte tærskelværdier afviger fra den teoretiske værdi, hvilket er forventet af reelle komponenter. Det er en acceptabel afvigelse, så tærskelværdierne kan derfor anvendes til implementering}
%	\label{Tab:Maalingtearskelvaerdier}
%\end{table}

\noindent Det kan udfra \ref{Tab:Maalingtearskelvaerdier} konkluderes, at afvigelsen fra de udregnede tærkselværdier stemmer overens med tolerancekravene for komparatorens tærkselværdier, jævnfør afsnit \ref{KomparatorAfs} på side \pageref{KomparatorAfs}, hvorfor den visuelle del af feedbackkonfigurationen accepteres. 

\noindent\textbf{Simulering af somasensoriske del af feedbackkonfigurationen} \\
Til simulering af den somasensoriske del af feedbackkonfigurationen anvendes komparatorer af typen LM$311$ og da der ikke findes vibratorer i LTspice benyttes en modstand på R$33$ udregnet vha. Ohms lov, hvilket fremgår af nedestående \eqref{vibrator_modstand} 

\begin{equation} \label{vibrator_modstand}
R = \dfrac{3V}{0.9A} = 33\Omega
\end{equation}

Kredsløbet simuleres i LTspice for at teste, hvorvidt den somasensoriske del af feedbackkonfiguration opfylder kravspecifikationerne, jævnfør afsnit \ref{KomparatorAfs} på side \pageref{KomparatorAfs}. Af figur \figref{fig:vibration_graf} fremgår tærskelværdierne for den visuelle del af feedbackkonfigurationen.

\begin{figure}[H]
	\centering
	\includegraphics[scale=0.3]{figures/cProblemloesning/vibration_graf.PNG}
	\caption{På figuren ses simuleringen af den somasensoriske del af feedbackkonfiguration. Den sorte kurve er sinus-signalet, der illustrerer blokkens inputsignal. Den blå kurve viser tærskelværdien for den højre vibrator og den pinke kurve viser tærskelværdier for den vesntre vibrator. Kredsløbet er simuleret i LTspice.}
	\label{fig:vibration_graf}
\end{figure}
På figur \figref{fig:vibration_graf} fremgår det, at ved de enkelte tærskelværdier går signalet i negativ mætning, hvilket aktiverer vibratorerne. Derudover fremgår det af figuren, at tærskelværdier for de to vibratorer er de samme tærskelværdier for den gule LED, hvilket gør, at den samme afvigelse vil gøre sig gældende, hvorfor den somasensoriske del af ffedbackkonfigurationen accepteres. 

\subsubsection{Implementering og test}

%Når referencespændingen er kendt, kan modstandene efter LEDerne bestemmes. Da kredsløbet trækker strøm, har modstandene (R$13$-R$17$) \fxnote{er det de rigtige modstande} til formål at sikre, at spændingsforsyningen ikke drænes. Hvis modstanden er høj, vil strømmen til kredsløbet være lav, og batterierne i spændingsreferencen vil derved holde længere, men hvis modstanden er for høj, kan det have indflydelse på LEDernes lysstyrke. 

%Med inputtet til feedback blokken er det placeret en buffer for at stabilisere signalet fra forrige blok, samt adskille blokkene fra hinanden. Blokkene skal adskilles for at sikre at den lave indgangsimpedans ved de feedbackkonfigurationer hvor signalet er forbundet til den inverterende terminal bliver adskilt fra forrige blok.

Ifølge det valgte design skal der benyttes i alt 21 modstande antal modstande. Ved reelle komponenter vil der være en afvigelse, hvilket fremgår af tabel \ref{Tab:komparator_modstande}. Derudover er det ikke muligt at anvende alle beregnede komponenter, hvorfor der anvendes modstande i serie- og parallelforbindelser. I tabellen vil alle modstandene fremgå, samt den afvigelsen mellem den teoretiske og målte modstand. 
\begin{table}[H]
\centering
\begin{tabular}{|l|l|l|l|}
\hline
\textit{}               & \textit{Teoretisk værdi} & \textit{Målt værdi} & \textit{Afvigelse} \\ \hline
\textit{R1}             & $10$K$\Omega$            & $9.973$K$\Omega$    & $0.27\%$           \\ \hline
\textit{R2}             & $10$K$\Omega$            & $9.992$K$\Omega$    & $0.08\%$           \\ \hline
\textit{R3}             & $10$K$\Omega$            & $9.976$K$\Omega$    & $0.24\%$           \\ \hline
\textit{R4}             & $10$K$\Omega$            & $9.950$K$\Omega$    & $0.50\%$           \\ \hline
\textit{R5}             & $10$K$\Omega$            & $9.985$K$\Omega$    & $0.15\%$           \\ \hline
\textit{R6}             & $10$K$\Omega$            & $9.950$K$\Omega$    & $0.50\%$           \\ \hline
\textit{R7 (Serie)}     & $16.821$K$\Omega$        & $16.758$K$\Omega$   & $0.37\%$           \\ \hline
\textit{R8 (Parallel)}  & $6.285$K$\Omega$         & $6.280$K$\Omega$    & $0.08\%$           \\ \hline
\textit{R9 (Serie)}     & $1068\Omega$             & $1065\Omega$        & $0.28\%$           \\ \hline
\textit{R10 (Serie)}    & $1042\Omega$             & $1036\Omega$        & $0.58\%$           \\ \hline
\textit{R11 (Parallel)} & $6.039$K$\Omega$         & $6.044$K$\Omega$    & $0.08\%$           \\ \hline
\textit{R12 (Parallel)} & $15.765$K$\Omega$        & $15.718$K$\Omega$   & $0.30\%$           \\ \hline
\textit{R13 (Serie)}    & $165\Omega$              & $164.59\Omega$      & $0.25\%$           \\ \hline
\textit{R14 (Serie)}    & $165\Omega$              & $164.54\Omega$      & $0.28\%$           \\ \hline
\textit{R15}            & $100$K$\Omega$           & $99.660$K$\Omega$   & $0.34\%$           \\ \hline
\textit{R16 (serie)}    & $165\Omega$              & $164.87\Omega$      & $0.08\%$           \\ \hline
\textit{R17 (Serie)}    & $165\Omega$              & $165.07\Omega$      & $0.04\%$           \\ \hline
\textit{R18}            & $100$K$\Omega$           & $99.722$K$\Omega$   & $0.28\%$           \\ \hline
\textit{R19 (Serie)}    & $165\Omega$              & $163.90\Omega$      & $0.67\%$           \\ \hline
\textit{R20}            & $100$K$\Omega$           & $99.628$K$\Omega$   & $0.37\%$           \\ \hline
\textit{R21}            & $100$K$\Omega$           & $99.676$K$\Omega$   & $0.32\%$           \\ \hline
\end{tabular}
\caption{Af tabellen fremgår de teoretiske og reelle værdier for modstandene benyttet i feedbackkonfigurationen for den visuelle del.}
\label{Tab:komparator_modstande}
\end{table}
\noindent Komparatorkonfigurationen testes ved en spændingsforsyning på $5.5$V med et sinussignal som input. Af \ref{Tab: test_reference} fremgår de beregnede og målte tærskelværdier, samt afvigelsen af disse tærskelværdier. 

\begin{table}[H]
	\centering
	\begin{tabular}{|l|l|l|l|} \hline
		& \textit{\begin{tabular}[c]{@{}l@{}}Teoretisk\\ reference\\ input\end{tabular}} & \textit{\begin{tabular}[c]{@{}l@{}}Målte\\ reference\\ input\end{tabular}} & \textit{\% afvigelse} \\ \hline
		\textit{$13^{\circ}$}  & $1.5758$V                                                                      & $1.5748$V                                                                  & $0.06\%$              \\ \hline
		\textit{$8^{\circ}$}   & $0.9697$V                                                                      & $0.9700$V                                                                  & $0.31\%$              \\ \hline
		\textit{$2^{\circ}$}   & $0.2424$V                                                                      & $0.2427$V                                                                  & $0.12\%$              \\ \hline
		\textit{-$2^{\circ}$}  & -$0.2359$V                                                                     & -$0.2373$V                                                                 & $0.59\%$              \\ \hline
		\textit{-$8^{\circ}$}  & -$0.9495$V                                                                     & -$0.9492$V                                                                 & $0.03\%$              \\ \hline
		\textit{-$13^{\circ}$} & -$1.5332$V                                                                     & -$1.5418$V                                                                 & $0.56\%$        \\ \hline     
	\end{tabular}
	\caption{Af tabelle fremgår de teoretiske og målte tærskelværdier ved de enkelte hældningsgrader. Derudover er der beregnet afvigelsen af den teoretiske og målte tærskelværdi.}
	\label{Tab:test_reference}
\end{table}
Jævnfør kravspecifikationerne afsnit \ref{KomparatorAfs} på side \pageref{KomparatorAfs} overholder de målte tærskelværdier tolerancekravet på $\pm1\%$. For at undersøge, hvornår LEDerne og vibratorerne tænder og slukker måles på input- og outputsignalet vha. et osciolloskop. I \tableref{Tab:test-taendsluk} fremgår de teoretiske og målte tærskelværdier for hvornår LEDen og vibratorerne tænder og slukker, samt afvigelsen mellem disse tærskelværdier. 

%\begin{table}[H]
%\centering
%\begin{tabular}{|l|l|l|l|l|l|l|}
%\hline
%                    & \textit{Målt reference input} & \textit{Målt Tænd}     & \textit{Afvigelse}      & %\textit{Teoretisk sluk}              & \textit{Målt sluk}  & \textit{Afvigelse}      \\ \hline
%\textit{$13\circ$}  & $<1.5758$V              & $1.6000$V              & $1.5357\%$               & $>1.5758$                            & $1.6000$V           & $1.5357$                \\ \hline
%\textit{$8\circ$}   & $<0.9697$V              & $1.0000$V              & $3.1246\%$               & $>0.9697$V                           & $0.9600$V           & $1.0000\%$               \\ \hline
%\textit{$0\circ$}   & -$0.2359 - 0.2424$      & \begin{tabular}[c]{@{}l@{}} -$0.2400$\\ og $0.2800$V\end{tabular} & \begin{tabular}[c]{@{}l@{}}$1.7380\%$\\ og $15.5115\%$\end{tabular} & \begin{tabular}[c]{@{}l@{}}<-$0.2359$\\ og $\textgreater0.2424\%$\end{tabular} & \begin{tabular}[c]{@{}l@{}}-$0.240$\\ og $0.280$\end{tabular} & \begin{tabular}[c]{@{}l@{}}$1.7380\%$\\ og $15.5115\%$\end{tabular} \\ \hline
%\textit{-$8\circ$}  & $<$-$0.9495$V           & -$0.9200$V             & $3.1069\%$               & $>$-$0.9495$V                        & -$0.9200$V          & $3.1069\%$               \\ \hline
%\textit{-$13\circ$} & $<$-$1.5332$V           & -$1.5300$V             & $0.2087\%$               & $>$-$1.5332$V                        & -$1.5400$V          & $0.4435\%$               \\ \hline
%\end{tabular}
%\caption{Af tabellen fremgår det, hvornår LEDerne teoretisk bør slukke og tænde og hvornår de blev målt til at tænde og slukke.}
%\label{Tab:test-taendsluk}
%\end{table}

\begin{table}[H]
\centering
\begin{tabular}{llll}
           & \textit{Målte reference input} & \textit{Målt reference output} & \textit{\% afvigelse eller forskellen i sekunder??} \\
$13\circ$  & $1.5748$V                      & $1.5600$V                      & $1.14$\%                                            \\
$8\circ$   & $0.9700$V                      & $1.0000$V                      & $3.09$ \%                                           \\
$2\circ$   & $0.2427$V                      & $0.2400$V                      & $1.11$ \%                                           \\
-$2\circ$  & -$0.2373$V                     & -$0.2000$V                     & $15.72$ \%                                          \\
-$8\circ$  & -$0.9492$V                     & -$0.9200$V                     & $3.18$ \%                                           \\
-$13\circ$ & -$1.5418$V                     & $1.5200$V                      & $1.44$ \%                                          
\end{tabular}
\caption{Af tabellen fremgår det, hvornår LEDerne teoretisk bør slukke og tænde og hvornår de blev målt til at tænde og slukke}
\label{Tab:test-taendsluk}
\end{table}

De teoretiske værdier for, hvornår LED bør slukke og tænde afviger fra de målte værdier. Under testen blev der anvendt et osciolloskop til at måle, hvornår der sker et spændingsfald over LEDerne og gruppen observerede, at osciolloskopet havde en dårlig opløsning og derudover afrundede værdierne, så disse ikke blev særlig præcise. Eftersom der ses så stor en afvigelse mellem de teoretiske og målte værdier anvendes en computer til at teste feedbackkonfigurationen.????? 
