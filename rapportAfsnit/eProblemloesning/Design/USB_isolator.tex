\section{USB-isolator}
\subsection{Teori}
Efter konvertering fra analogt til digital anvendes en USB-isolator til adskillelse af computeren til systemet samt sikre computerens forbindelse til elnettet. 

\subsection{Design}
Der anvendes USB-isolatoren USI-01 som er godkendt til brug ved sikkerhedsklassifikation BF som isoleringsspænding på 4kV samt fungerer som en galvanisk adskillende af systemet, hvilket har til formål at overføre signal mellem to isolerede kredsløb. For yderligere sikring af computerens forbindelse til elnettet er denne ikke tilkoblet elnettet. 

\subsection{Implementering}
USB-isolatoren implementeres mellem ADC'en og computeren. 

\subsection{Test}
Det skal testet hvor vidt kravet om input er lig med outputspændingen. Dette gøres ved at tilslutte USB-isolatoren til en spændingsforsyning og voltmeter og variationen af strømmen ved at aflæse på voltmeteret, herudfra skal det konkluderes hvorvidt inputtet er lig med outputspænding. 
