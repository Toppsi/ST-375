\subsubsection{Referencespænding}
Der skal til offset og komparator blokken være en konstant spænding, da spændingen skal bruges som sammenligningsgrundlag ift. andre signaler. Denne spændings kaldes en referencespænding. Referencespændingen består af en spændingsforsyning, en modstand og en spændingsreference diode. Som spændingsreference dioden bruges en LM385, som både findes som $1.2$V og $2.5$V. For at udregne modstanden i spændingsreferencen, skal strømmen som løber i samme kredsløb som spændingsreference dioden være kendt. Et eksempel på opsætningen af spændingsreferencen kan ses på figur \figref{}

\begin{figure}[H]
	\centering
	\includegraphics[scale=1.0]{figures/cProblemloesning/}
	\caption{}
	\label{fig:}
\end{figure}

