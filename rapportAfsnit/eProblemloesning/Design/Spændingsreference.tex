\subsubsection{Referencespænding}
Der skal ved offset- og komparatorblokken forsynes med en konstant spænding, da spændingen skal anvendes som sammenligningsgrundlag ift. andre signaler. Denne spænding kaldes en referencespænding. Referencespændingen består af en spændingsforsyning, en modstand og en spændingsreference diode. Der anvendes en referencediode af typen LM385, som både findes som $1.2$V og $2.5$V. Et eksempel på en opsætning af en spændingsreference kan ses på figur \figref{fig:Spaendingsreference}.

\begin{figure}[H]
	\centering
	\includegraphics[scale=1.0]{figures/cProblemloesning/ReferenceEksempel}
	\caption{Figuren illustrerer et eksempel på opsætning af et kredsløb for en spændingsreference med LM$385$. Figuren er en revideret udgave fra kilden. \cite{Instruments2005}}
	\label{fig:Spaendingsreference}
\end{figure}

For at udregne værdien af modstanden, R, i kredsløbet, anvendes følgende generelle formel:
\begin{equation}
R=\dfrac{V_{forsyning}-V_{Reference}}{I_{Z}}
\end{equation}
Hvor V_{forsyning} er forsyningsspændingen som sendes ind i kredsløbet, V_{Reference} er den referencespænding der skal sendes ud af systemet og I_{Z} er strøm forbruget fra de komponenter, der er i referencespændings kredsløbet. 

Først udregnes R for referencespændingen til offsettet. V_{forsyning} er de $5.5$V, der forsynes med fra spændingsforsyningen. V_{Reference} $2.5$V. I kredsløbet for ofsettet indgår én operationsforstærker(TL$081$), der har en maksimal biasstrøm på $200$pA\fxnote{KILDE, datablad påTL081}. Referencedioden har et arbejdsområde mellem $20\muA$ til $20mA$ og for at sikre der er strøm nok til referencedioden er den sat til at bruge $100\muA$\fxnote{ KILDE: Datablad LM385}. Dermed kan strømforbruget, I_Z, for offsettet udregnes som summen af de to biasstrømme og alle de kendte værdier indsættes i formlen:

\begin{equation}
R_{offset}=\frac{5.5V-2.5V}{0.0001000002A}=29999.94\Omega \approx 30K\Omega
\end{equation}  
Da offsettet jævnført 

Herefter kan samme fremgangsmåde anvendes til udregning af R for referencespændingen til komparatoren. I kredsløbet for komparatoren indgår én operationsforstærker(TL$081$), med en maksimal biasstrøm på $200$pA. Biasstrømmen for referencedioden er igen sat til $100\muA$. Derudover indgår \textcolor{red}{otte komparatorer} (LM$311$), der alle har en maksimal biasstrøm på $250$nA. Dermed kan værdierne igen indsættes i formlen og R kan beregnes.

\begin{equation}
R_komparator=\frac{5.5V-2.5V}{0.0001020002A}=29411.70704\Omega \approx 29.5K\Omega 
\end{equation} 
\fxnote{Rigtig afrunding}

\textbf{Test af referencespænding}


 