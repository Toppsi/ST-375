% !TeX spellcheck = da_DK
\subsection{ADC}
\subsubsection{Teori og design}
I dette projekt anvendes en ADC af typen NI USB-6009 til at konvertere det analoge signal til digital som beskrevet i afsnit \ref{ADCafsnit} på side \pageref{ADCafsnit}. %Signalet skal konverteres, for at kunne sendes ind i den digitale del af systemet, så plejepersonalet kan aflæse og behandle målingerne. 
Med denne ADC kan der samples med $13$ bits single-ended\fxnote{Hvad betyder det?}. ADC'en kan dermed inddeles i $2^{13} = 8192 \text{niveauer}$. Den maksimale sampling rate er på $48$kS/s, hvormed det er muligt at sample med minimum det tidobbbelte af båndbredden. Arbejdsområdet for ADC'en ift. spænding ligger på $10Vpp$ og har en typisk præcision på $14.7$mV ved $25^{\circ}$C. \cite{Instruments2014} Ud fra oplysningerne er det muligt at udregne LSB ud fra ligningen i afsnit \ref{ADCafsnit} på side \pageref{ADCafsnit}. Udregningen ser således ud: \\
\begin{equation}
	LSB = \frac{FSR}{2^{n}} 
\end{equation}  
FSR er arbejdsområdet, dvs. $10Vpp$, som indsættes i formlen som $20V$, imens n er antallet af bits, der kan samples med, dvs. $13$.
Værdierne indsættes i formlen: \\
\begin{align}
	LSB = \frac{20V}{2^{13}}
	LSB = 0.00244V = 2.44\text{mV}
\end{align}
Det opsamlede signal bliver forvrænget, og vil dermed ikke være repræsentativt, hvis der opsamles værdier på under $2.44mV$.\\
\subsubsection{Simulering}
Ifølge tolerancerne for ADC'en beskrevet i afsnit \ref{ADCafsnit} på side \pageref{ADCafsnit} skal der testes, om ADC'en kan modtage og konvertere et inputsignal på $\pm4$V samt sample 500 gange i sekundet. \\
Der kan ikke laves en simulering af en ADC i programmet LTSpice, da inputsignalet i LTSpice allerede er et digitalt signal, og der er derfor ikke noget, som skal konverteres.
 
\subsubsection{Implementering og test}
ADC'en skal implementeres og tilkobles efter lavpasfiltret, som det kan ses på \figref{kravblok} på side \pageref{kravblok}. \\
Der benyttes en tonegenerator til at undersøge, om den valgte ADC overholder kravene beskrevet i afsnit \ref{ADCafsnit} på side \pageref{ADCafsnit}. Tonegeneratoren skal sende fem forskellige input til ADC'en;
\begin{enumerate}
	\item Et sinussignal med en frekvens på $100$Hz og den højeste forventede amplitude, dvs. $3.0147 \times 2 = 6.0294\text{Vpp}$.
	\item Et sinussignal med en frekvens på $100$Hz og den laveste forventede amplitude, dvs. -$2.9400 \times 2 = -5.8800 \approx 5.8800\text{Vpp}$.
	\item Et sinussignal med en frekvens på $25$Hz og en amplitude på $8$Vpp, da der ses i \tableref{Tab:faktor3_test}, at operationsforstærkerne går i mætning ved ca. $4$V, når de forsynes med $5.5$V. 
	\item Et sinussignal med en frekvens svarende til knækfrekvensen i lavpasfiltret, dvs. $25$Hz, og den højeste forventede amplitude, dvs. $3.0147 \times 2 = 6.0294\text{Vpp}$.
	\item Et sinussignal med en frekvens svarende til knækfrekvensen i lavpasfiltret, dvs. $25$Hz, og den laveste forventede amplitude, dvs. -$2.9400 \times 2 = -5.8800 \approx 5.8800\text{Vpp}$.
\end{enumerate}
Der samples med $500$Hz, selvom $250$Hz havde været nok, da vores båndbredte er $25$Hz, men ScopeLogger kan enten sample med $200$ eller $500$, så derfor samples der med $500$Hz. På \figref{fig:ADC_Test} ses resultatet af målingerne.

\begin{figure}[H]
	\centering
	\includegraphics[scale=0.45]{figures/cProblemloesning/ADC_Test2_matlab.PNG}
	\caption{På figuren ses resultatet af de fem målinger plottet i Matlab. Der ses, at x-aksen for alle plots er sat til 0.1 sekunder.}
	\label{fig:ADC_Test}
\end{figure}

\noindent Dataen optaget med ADC'en, som er plottet i \figref{fig:ADC_Test}, blev efterfølgende bearbejdet i Matlab. Derudover blev et oscilloskop tilkoblet outputtet fra tonegeneratoren. Resultatet herfra betragtes som det faktiske output fra tonegeneratoren, hvilket indgår i beregningen af afvigelsen i ADC'ens sampling. Resultaterne ses i \tableref{Tab:ADC_resultat}:

\begin{table}[H]
	\centering
	\begin{tabular}{|l|l|l|l|l|l|}
		\hline
		\textit{Måling nr} & \textit{\begin{tabular}[c]{@{}l@{}}Indstillet\\ frekvens\end{tabular}} & \textit{\begin{tabular}[c]{@{}l@{}}Indstillet\\ amplitude\end{tabular}} & \textit{\begin{tabular}[c]{@{}l@{}}Målte amplitude\\ på oscilloscop\end{tabular}} & \textit{\begin{tabular}[c]{@{}l@{}}Målte amplitude\\ via ADC\end{tabular}} & \textit{Afvigelse} \\ \hline
		$1$         & $100$Hz     & $3.0147$V    & $3.0800$V      & $2.5989$V       & $15.62$\%          \\ \hline
		$2$         & $100$Hz     & $2.9400$V    & $2.9200$V      & $2.5284$V       & $13.32$\%          \\ \hline
		$3$         & $25$Hz      & $4$V         & $4.0400$V      & $4.0442$V       & $0.10$\%           \\ \hline
		$4$         & $25$Hz      & $3.0147$V    & $3.0800$V      & $3.0689$V       & $0.36$\%           \\ \hline
		$5$         & $25$Hz      & $2.9400$V    & $2.9200$V      & $2.9207$V       & $0.02$\%           \\ \hline
	\end{tabular}
	\caption{I tabellen ses resultaterne af de beregnede afvigelser.}
	\label{Tab:ADC_resultat}
\end{table}

Der kan ses på \figref{fig:ADC_Test}, at sinussignalet i måling $1$ og $2$ er mere kantet end signalerne i måling $3$-$5$. Årsagen til dette kan være, at vi sampler med $500$Hz, mens signalets frekvens er $100$Hz. Dette gør, at ADC'en sampler med $5$ gange pr. periode og derved bliver signalet kantet. Derfor giver ADC'ens værdi for amplituden en procentafvigelse på henholdsvis $15.62$ og $13.32$, da der ikke samples med nok ift. frekvensen. Det vurderes dermed, at det er samplingsfrekvensen, som er årsagen til de høje afvigelser.\\
I måling $3$-$5$ er sinussignalets frekvens nedsat, hvilket gør, at signalet bliver mere regelmæssigt. Der samples med en samplingshastighed, der er $20$ gange større end frekvensen, og der opsamles derved tilstrækkeligt med datapunkter. Derfor er procentafvigelserne fra måling $3$-$5$ mere repræsentative for ADC'ens afvigelse. Men afvigelserne kan også skyldes afvigelser i oscilloskopet. Der vurderes derfor, at ADC'en viser det forventede og er dermed godkendt.