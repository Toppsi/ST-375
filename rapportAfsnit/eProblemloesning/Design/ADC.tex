% !TeX spellcheck = da_DK
\subsection{ADC}
\subsubsection{Teori og design}
I dette projekt anvendes NI USB-6009 til at konvertere det analoge signal til digital som beskrevet i afsnit \ref{ADCafsnit} på side \pageref{ADCafsnit}. Signalet skal konverteres, for at kunne sendes ind i den digitale del af systemet, så plejepersonalet kan aflæse og behandle målingerne. Med denne ADC kan der samples med $13$ bits single-ended. ADC'en kan dermed inddeles i $2^{13} = 8192 \text{niveauer}$. Den maksimale sampling rate er på $48$kS/s, hvormed det er muligt at sample med minimum det tidobbbelte af båndbredden. Arbejdsområdet for ADC'en ift. spænding ligger på $10Vpp$ og har en typisk præcision på $14.7$mV ved $25^{\circ}$C. \cite{Instruments2014} Ud fra oplysningerne er det muligt at udregne LSB ud fra ligningen i afsnit \ref{ADCafsnit} på side \pageref{ADCafsnit}. Udregningen ser således ud: \\
\begin{equation}
	LSB = \frac{FSR}{2^{n}} 
\end{equation}  
FSR er arbejdsområdet, dvs. $10Vpp$, som indsættes i formlen som $20V$, imens n er antallet af bits, der kan samples med, dvs. $13$.
Værdierne indsættes i formlen: \\
\begin{align}
	LSB = \frac{20V}{2^{13}}
	LSB = 0.00244V = 2.44\text{mV}
\end{align}
Det opsamlede signal bliver forvrænget, og vil dermed ikke være repræsentativt, hvis der opsamles værdier på under $2.44mV$.\\
\subsubsection{Simulering}
Ifølge tolerancerne for ADC'en beskrevet i afsnit \ref{ADCafsnit} på side \pageref{ADCafsnit} skal der testes, om ADC'en kan modtage og konvertere et inputsignal på $\pm4$V samt sample 500 gange i sekundet. \\
Der kan ikke laves en simulering af en ADC i programmet LTSpice, da inputsignalet i LTSpice allerede er et digitalt signal, og der er derfor ikke noget, som skal konverteres.
 
\subsubsection{Implementering og test}
ADC'en skal implementeres og tilkobles efter lavpasfiltret, som det kan ses på \figref{kravblok} på side \pageref{kravblok}. \\
Der benyttes en tonegenerator til at undersøge, om den valgte ADC overholder kravene beskrevet i afsnit \ref{ADCafsnit} på side \pageref{ADCafsnit}. Tonegeneratoren skal sende 4 forskellige input til ADC'en;
\begin{enumerate}
	\item Et sinussignal med en frekvens på $100$Hz og den højeste forventede amplitude, dvs. $0.3313 \times 2 = 0.6626\text{Vpp}$.
	\item Et sinussignal med en frekvens på $100$Hz og den laveste forventede amplitude, dvs. -$0.3233 \times 2 = -0.6466 \approx 0.6466\text{Vpp}$.
	\item Et sinussignal med en frekvens på $100$Hz og en amplitude på $8$Vpp, da der ses i \tableref{Tab:faktor3_test}, at operationsforstærkerne går i mætning ved ca. $4$V, når de forsynes med $5.5$V. 
	\item Et sinussignal med en frekvens svarende til knækfrekvensen i lavpasfiltret, dvs. $25$Hz, og den højeste forventede amplitude, dvs. $0.3313 \times 2 = 0.6626\text{Vpp}$.
	\item Et sinussignal med en frekvens svarende til knækfrekvensen i lavpasfiltret, dvs. $25$Hz, og den laveste forventede amplitude, dvs. -$0.3233 \times 2 = -0.6466 \approx 0.6466\text{Vpp}$.
\end{enumerate}
Der samples med 500 Hz.