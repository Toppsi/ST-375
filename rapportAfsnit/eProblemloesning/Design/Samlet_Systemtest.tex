\section{Samlet systemtest}
Efter de enkelte blokke er blevet testet og godkendt hver for sig, skal det samlede system testes. Formålet med den samlede systemtest er at kontrollere, hvorvidt systemet overholder de overordnede funktionelle krav jævnfør afsnit \ref{FunkKrav} på side \pageref{FunkKrav}. Der anvendes samme fremgangsmåde for test af det samlede system, som af de øvrige blokke; Først simuleres systemet i LTspice, hvorefter det implementeres og testes. Det samlede system vil blive testet på to forskellige måder - først en test efter samme principper som i testen af accelerometeret og efterfølgende en test, hvor accelerometeret er placeret på en person jævnfør afsnit \ref{formaal_anvendelse} på side \pageref{formaal_anvendelse}.

\subsection{Simulering}
I simuleringen af det samlede system kontrolleres det, som i de øvrige simuleringer, om systemet fungerer med ideelle komponenter. Denne kontrol udføres ved at indsende en inputspænding igennem systemet, som teoretisk skal aktivere hhv. de enkelte dioder og vibratorerne. Dermed kan det måles, om systemet teoretisk opfylder de overordnede funktionelle krav jævnfør afsnit \ref{FunkKrav} på side \pageref{FunkKrav}.  

\subsection{Implementering}

\subsection{Test 1}
I den første test af det samlede system tages udgangspunkt i principperne fra testen af accelerometeret beskrevet i afsnit \ref{Acc_afsnit} på side \pageref{Acc_afsnit}. Det måles således, hvorvidt systemet opfylder de overordnede funktionelle krav jævnfør afsnit \ref{FunkKrav} på side \pageref{FunkKrav}, ved at placere accelerometeret i de definerede hældningsgrader, og herudfra vurdere om den korrekte feedback udløses.  

\subsection{Test 2}
I den anden test af det samlede system placeres accelerometeret i stedet på en forsøgsperson, der skal afprøve træningsforløbet med systemet. Formålet med denne test er igen at kontrollere om den korrekte feedback udløses, samt at vurdere, hvorvidt de definerede hældningsgrader er passende under træning med systemet. 
 