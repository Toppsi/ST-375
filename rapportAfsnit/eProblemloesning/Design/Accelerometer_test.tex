% !TeX spellcheck = da_DK
\subsection{Accelerometer}
\subsubsection{Teori og design}
Teorien bag accelerometret ADXL335 ses i afsnit \ref{Subsec:AccTeori} på side \pageref{Subsec:AccTeori}. Der blev ud fra pilotforsøget fundet ud af, at signalet ligger i et frekvensområde fra $0$-$25$Hz og derved er båndbredden $25$Hz. For at få den ønskede båndbrede skal der ændres på kapacitorens størrelse ift. pilotforsøget, da kapacitoren her var tilpasset $50$Hz. Kapacitorens størrelse (C) beregnes ud fra følgende formel fra accelerometerets datablad \cite{Devices2009}:
\begin{equation}
\text{Båndbredde} = \dfrac{5\mu F}{C} \Rightarrow  C = \dfrac{5\mu F}{\text{Båndbredde}}
\end{equation}
Her er kendes båndbredden, som er $25$Hz, hvilket gør at C bliver $0.2\mu$F. For at finde en kapasitor med værdien $0.2\mu$F bruges der to $0.1\mu$F kapacitorer i parallelforbindelse.  

\subsubsection{Simulering}
Ifølge tolerancerne for accelerometret beskrevet i afsnit \ref{OpsamlingsAfs} på side \pageref{OpsamlingsAfs} skal der testes for, hvorledes accelerometret overholder kravet om, at der maks må være en $5\%$ afvigelse i detektionen af hældningsgrad. \\
Der kan ikke laves en simulering af et accelerometer i programmet LTspice, da denne komponent ikke findes der inde. Værdierne beregnet igennem pilotforsøget på side \pageref{Bilag:Pilotforsoeg} anses som de teoretiske værdier for accelerometerets output.

\subsubsection{Implementering og test}
I pilotforsøget blev der arbejdet med reelle komponenter, hvilket også blev benyttet til implementeringen. Derfor antages det, at der vil være en mindre afvigelse imellem resultaterne fra testen og pilotforsøget end imellem resultaterne fra testen og en eventuel simulering, da der i simuleringer arbejdes med ideelle komponenter.\\
Accelerometerets båndbredde bestemmes af to $0.1\mu$F kapacitorer i parallelforbindelse, som derved vil give en samlet kapacitans på  $0.2\mu$F. De to kapacitorer blev målt inden implementering samt den samlede kapacitans. Resultaterne fremgår i  \tableref{Tab:Acc_kondensator}:
\begin{table}[H]
	\centering
	\begin{tabular}{llll}
		& \textit{Teoretisk} & \textit{Målt} & \textit{\% afvigelse} \\
		\textit{C1}       & \textit{$0.1\mu$F} & $0.1009\mu$F  & $0.9\%$               \\
		\textit{C2}       & \textit{$0.1\mu$F} & $0.0989\mu$F  & $1.1\%$               \\
		\textit{C1 || C2} & \textit{$0.2\mu$F} & $0.2002\mu$F  & $1\%$                
	\end{tabular}
		\caption{I tabellen ses det, at de to kondensatorer afviger lidt fra deres teoretiske værdi, hvilket er forventet af reelle komponenter.}
		\label{Tab:Acc_kondensator}
\end{table}
Til opsamling af data fra accelerometret blev der benyttet et multimeter. I testen blev der foretaget en aflæsning med multimetret for $\pm8^\circ$ og $\pm13^\circ$. Disse fire værdier ses i \tableref{Tab:Acc_test_procent}. Den teoretiske stigning af volt pr. grad for henholdsvis negativ og positiv hældning er udregnet i bilag \ref{Bilag:Pilotforsoeg} på side \pageref{Bilag:Pilotforsoeg}. De teoretiske værdier for $8^\circ$ og $13^\circ$ udregnet derfor ud fra disse værdier. Dette gøres i følgende ligninger:
\begin{align}
(-0.0036 \cdot 13) + 1.6325 = 1.5858\text{V} \\
(-0.0036 \cdot 8) + 1.6325 = 1.6038\text{V}  \\
(0.0037 \cdot 8) + 1.6325 = 1.6619\text{V}  \\
(0.0037 \cdot 13) + 1.6325 = 1.6803\text{V}
\end{align}
Disse værdier indsættes og der udregnes en afvigelse.
\begin{table}[H]
	\centering
	\begin{tabular}{|l|l|l|l|}
		\hline
		\textit{\begin{tabular}[c]{@{}l@{}}Vinkel af\\ accelerometer\end{tabular}} &  \textit{\textit{\begin{tabular}[c]{@{}l@{}}Beregnet\\ Output\end{tabular}}} & \textit{Output} & \textit{\begin{tabular}[c]{@{}l@{}}\% afvigelse\\ ift. dektering\\ af hældningsgrad\end{tabular}} \\ \hline
%		$-90^\circ$     & $1.3092$V    & $\times$     & $\times$      \\ \hline
		$-13^\circ$     & $1.5858$V    & $1.5686$V  & 1.08\%      \\ \hline
		$-8^\circ$      & $1.6038$V    & $1.5947$V  & 0.57\%      \\ \hline
%		$0^\circ$       & $1.6325$V    & $\times$     & $\times$      \\ \hline
		$8^\circ$       & $1.6619$V    & $1.6761$V  & 0.85\%      \\ \hline
		$13^\circ$      & $1.6803$V    & $1.7114$V  & 1.85\%      \\ \hline
%		$90^\circ$      & $1.9638$V    & $\times$     & $\times$      \\ \hline
	\end{tabular}
	\caption{I tabellen ses det målte output ved en bestemt grad. Herved kan der beregnes en afvigelse i procent.}
	\label{Tab:Acc_test_procent}
\end{table}
Der ses i tabel \tableref{Tab:Acc_test_procent}, at accelerometret har en maksimal afvigelse i detektionen af hældningsgrad på $1.85\%$. Derved overholder accelerometret tolerancerne, som er blevet stillet i afsnit \ref{OpsamlingsAfs} på side \pageref{OpsamlingsAfs} og accelerometret accepteres derfor.