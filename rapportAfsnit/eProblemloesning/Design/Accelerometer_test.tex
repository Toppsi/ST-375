% !TeX spellcheck = da_DK
\subsection{Accelerometer}
\subsubsection{Teori og design}
Teorien bag accelerometret ADXL335 ses i afsnit \ref{Subsec:AccTeori} på side \pageref{Subsec:AccTeori}, hvorimod designet af accelerometret ses i på \figref{pforsoeg1} på side \pageref{pforsoeg1}.

\subsubsection{Simulering}
Ifølge tolerancerne for accelerometret beskrevet i afsnit \ref{OpsamlingsAfs} på side \pageref{OpsamlingsAfs} skal der testes, hvorledes accelerometret overholder kravet om, at der maks må være en $5\%$ afvigelse i detektionen af hældningsgrad. \\
Der kan ikke laves en simulering af et accelerometer i programmet LT Spice, da denne komponent ikke findes der inde. Værdierne beregnet igennem pilotforsøget på side \pageref{Sec:Pilotforsoeg} anses som de teoretiske værdier for accelerometerets output.

\subsubsection{Implementering og test}
I simuleringen blev der arbejdet med reelle komponenter, hvilket også bliver benyttet til implementeringen. \\
Til opsamling af data fra accelerometret benyttes en computer med ScopeLogger, hvorefter dataen bliver behandlet i Matlab. I testen blev der foretaget 3 målinger for henholdsvis $0^\circ$, $\pm8^\circ$, $\pm13^\circ$ og $\pm90^\circ$. Herefter blev gennemsnittet for hver måling ved hver grad udregnet, som lægges sammen og til slut også tages gennemsnittet af. Herved fås 7 værdier - en for hver grad, men målingerne ved $0^\circ$ samt $\pm90^\circ$ benyttes for at udregne accelerometerets detektion af grader. Værdierne for dette ses i \tableref{Tab:acc_procent}. Den teoretiske stigning af volt pr. grad for henholdsvis negativ og positiv hældning er udregnet i bilag \ref{Bilag:Pilotforsoeg} på side \pageref{Bilag:Pilotforsoeg}. De teoretiske værdier for $8^\circ$ og $13^\circ$ udregnet derfor ud fra disse værdier. Dette gøres i følgende ligninger:
\begin{align}
(-0.0036 \cdot 13) + 1.6325 = 1.5858\text{V} \\
(-0.0036 \cdot 8) + 1.6325 = 1.6038\text{V}  \\
(0.0037 \cdot 8) + 1.6325 = 1.6619\text{V}  \\
(0.0037 \cdot 13) + 1.6325 = 1.6803\text{V}
\end{align}
Disse værdier indsættes og der udregnes en afvigelse.
\begin{table}[H]
	\centering
	\begin{tabular}{|l|l|l|l|}
		\hline
		\textit{\begin{tabular}[c]{@{}l@{}}Vinkel af\\ accelerometer\end{tabular}} & \textit{Output} & \textit{\begin{tabular}[c]{@{}l@{}}Beregnet\\ Output\end{tabular}} & \textit{\begin{tabular}[c]{@{}l@{}}\% afvigelse\\ ift. dektering\\ af hældningsgrad\end{tabular}} \\ \hline
%		$-90^\circ$     & $1.3092$V    & $\times$     & $\times$      \\ \hline
		$-13^\circ$     & $1.5858$V    & $1.5769$V  & 0.56\%      \\ \hline
		$-8^\circ$      & $1.6038$V    & $1.5984$V  & 0.34\%      \\ \hline
%		$0^\circ$       & $1.6325$V    & $\times$     & $\times$      \\ \hline
		$8^\circ$       & $1.6619$V    & $1.6889$V  & 1.62\%      \\ \hline
		$13^\circ$      & $1.6803$V    & $1.7197$V  & 2.34\%      \\ \hline
%		$90^\circ$      & $1.9638$V    & $\times$     & $\times$      \\ \hline
	\end{tabular}
	\caption{I tabellen ses det målte output ved en bestemt grad, hvor outputtet er blevet beregnet ift. outputtet målt i pilotforsøget. Herved kan der beregnes en afvigelse i procent.}
	\label{Tab:Acc_test_procent}
\end{table}
Der ses i tabel \tableref{Tab:Acc_test_procent}, at accelerometret har en maksimal afvigelse i detektionen af hældningsgrad på $2.34\%$. Derved overholder accelerometret tolerancerne, som er blevet stillet i afsnit \ref{OpsamlingsAfs} på side \pageref{OpsamlingsAfs} og kan derved accepteres til videre implementering.