% !TeX spellcheck = da_DK
\subsection{Accelerometer}
\subsubsection{Teori og design}
Teorien bag accelerometret ADXL335 ses i afsnit \ref{Subsec:AccTeori} på side \pageref{Subsec:AccTeori}, hvorimod designet af accelerometret ses i på \figref{pforsoeg1} på side \pageref{pforsoeg1}.

\subsubsection{Simulering}
Ifølge tolerancerne for accelerometret beskrevet i afsnit \ref{OpsamlingsAfs} på side \pageref{OpsamlingsAfs} skal der testes, hvorledes accelerometret overholder kravet om, at der maks må være en $5\%$ afvigelse i detektionen af hældningsgrad. \\
Der kan ikke laves en simulering af et accelerometer i programmet LT Spice, da denne komponent ikke findes der inde. Værdierne beregnet igennem pilotforsøget på side \pageref{Sec:Pilotforsoeg} anses som de teoretiske værdier for accelerometerets output.

\subsubsection{Implementering og test}
I simuleringen blev der arbejdet med reelle komponenter, hvilket også bliver benyttet til implementeringen. \\
Til opsamling af data fra accelerometret benyttes en computer med ScopeLogger, hvorefter dataen bliver behandlet i Matlab. I testen blev der foretaget 3 målinger for henholdsvis $0^\circ$, $\pm8^\circ$, $\pm13^\circ$ og $\pm90^\circ$. Herefter blev gennemsnittet for hver måling ved hver grad udregnet, som lægges sammen og til slut også tages gennemsnittet af. Herved fås 7 værdier - en for hver grad.