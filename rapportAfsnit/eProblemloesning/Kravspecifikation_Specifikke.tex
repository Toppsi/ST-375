% !TeX spellcheck = da_DK
\section{Specifikke kravspecifikationer}
I praksis er det ikke muligt at have ideelle komponenter \cite{Nilsson2011}. Der vurderes derfor ud fra pilotforsøget, litteratur samt relevante datablade, hvilke krav blokkene i blokdiagrammet i afsnit \ref{kravblok} på side \pageref{kravblok} skal opfylde og den tolerance, der accepteres ift. forskellige parametre.

\subsection{Opsamling}\label{OpsamlingsAfs}
Blokken opsamling omfatter accelerometeret, offsetjustering og forstærkning. Accelerometeret skal have en sådan størrelse, at det kan påsættes en patient uden at give fysiske begrænsninger. Accelerometeret skal kunne måle en statisk acceleration i mindst to retninger, da der måles i det frontale plan til hhv. højre og venstre side. Herved kan det bestemmes, hvilken retning patienterne hælder mod i stående position. Det skal kunne detektere en hældningsgrad op til $\pm90^{\circ}$, da en forsøgsperson ikke kan komme over denne hældningsgrad. \\
Offsetjusteringen skal centrere det målte signal omkring $0$V, da accelerometeret typisk måler $1.6325$V ved $0$g påvirkning af den pågældende akse ifølge pilotforsøget. Derfor skal der trækkes $1.6325$V fra alle inputs fra accelerometret. Justeringen af offset skal forekomme inden forstærkning således, at der ikke sker en forstærkning og dermed forskydning af offsetværdien, når accelerometret måler i positiv og negativ retning, hvorved g påvirkningen ændres. Ved justering af offsettet må der ikke ske en ændring af amplitudens peak-to-peak og frekvensen. \\
Forstærkningen skal gøre det nemmere at adskille det ønskede signal fra støj i den efterfølgende filtreringsblok. For at forstærkningskredsløbet ikke dræner strømforsyningen, skal inputimpedansen til forstærkeren være mindst $100$ gange større end outputimpedansen i accelerometeret som kan ses i afsnit \ref{Subsec:AccTeori} på side \pageref{Subsec:AccTeori}. Signalet skal efter forstærkningen ligge indenfor arbejdsområdet $\pm3$V og der ønskes derfor størst mulige forstærkning. Den højeste måling ved $\pm90^{\circ}$ var ifølge pilotforsøget efter offsetjusteringen $0.3313$V. Derfor kan forstærkningen udregnes således:
\begin{equation}
\dfrac{3\text{V}}{0.3313\text{V}} =  9.05 \approx 9.1
\end{equation}
\noindent Der ønskes en forstærkning på faktor $9.1$. Derved vil den maksimale spænding efter denne forstærkning være $3.0148$V, hvilket accepteres, da der vides fra databladet for operationsforstærkeren TL081 og teori, at når TL081 bliver forsynet med en spænding på $5.5$V, så vil den maks give ca. $4$V ud. \cite{Corporation1995} Derfor vil en lille afvigelse i outputsignalet ift. arbejdsområdet ikke få systemet til at gå i mætning. Forstærkningens gain i dB udregnes ud fra følgende formel: 
\begin{equation}
20 \cdot log_{10} (9.1) = 19.1808\text{dB}
\end{equation} 

\noindent\textbf{Krav:}
\begin{itemize}
	\item Opsamlingsblokken skal have en spændingsforsyning på $5.5$V.
	\item Accelerometeret skal minimum have en output akse.%, da der udelukkende måles i det frontale plan.
	\item Accelerometeret skal være under $5x5$ cm i dimensionerne.%, da dette gør implementeringen på patienten nemmere.
	\item Accelerometeret skal kunne måle statisk acceleration i mindst to retninger.
	\item Accelerometeret skal kunne detektere en hældning fra $0^{\circ}$ til $\pm90^{\circ}$.
	\item Offsetjusteringen skal centrere det målte signal omkring $0V$.
	\item Offsetjusteringen skal modtage en referencespænding på $1.6325$V.
	\item Forstærkeren skal have en indgangsimpedans på mindst $3.68M\Omega$\fxnote{NTK: Da man ikke vil dræne blokken før for strøm. Udregnet ved $32K\Omega$, da indputimpedans skal være mindst $100$ gange større end den forrige bloks outputimpedans. Accelerometrets indgangsimpedans har en afvigelse på +$15$\%}.
	\item Forstærkningen skal være på $9.1$, som svarer til $19.1808$dB.
	\item De benyttede modstande til designet af forstærkeren skal besidde den navngivne værdi.
	\item Outputtet fra denne blok skal være mellem $2.9817$V og -$2.9097$V.
\end{itemize}
\textbf{Tolerance:}
\begin{itemize}
	\item Der accepteres en afvigelse i spændingsforsyning på $\pm9\%$.
	\item Der accepteres ingen afvigelse ift. krav for accelerometerets dimension og minimum antal af akser.
	\item Der accepteres en afvigelse i detektionen af hældningsgrad på $\pm5\%$.
	\item Der accepteres en afvigelse af offsetjusteringen på $\pm2\%$.
	\item Der accepteres en afvigelse i referencespændingen på $\pm0.003$V.
	\item Der accepteres kun en indgangsimpedans i forstærkeren på mindst $3.68M\Omega$.
	\item Der accepteres en afvigelse i forstærkningen på $\pm2\%$.
	\item Der accepteres en afvigelse i de benyttede modstandes reelle værdi på $\pm1\%$.
	\item Der accepteres en afvigelse ift. maksimum og minimum output fra blokken på $\pm5\%$.
\end{itemize}
%%%%%%%%%%%%%%%%%%%%%%%%%%%%%%%%%%%%%%%%%%%%%%%%%%%%%%%%%%%%%%%%%%%%%
\subsection{Referencespænding til offset}\label{Ref_Offset_Afs}
Denne blok skal forsyne offsetjusteringen med en konstant spænding, som skal anvendes til sammenligning imellem et inputsignal og denne spændingsværdi. Der skal benyttes både en referencespænding og en spændingsdeler til denne blok, da outputtet fra referencespændingen er for høj og upræcis til at kunne anvendes som sammenligningsgrundlag. \\
Blokken skal levere $1.6325$V, da offsettet i accelerometret ved $0$g påvirkning er dette, jævnført afsnit \ref{Sec_Pilot_Data} på side \pageref{Sec_Pilot_Data}. Outputimpedansen må maks være $1$K$\Omega$, da inputimpedansen i den inverterende kanal er $100$K$\Omega$.

\textbf{Krav:}
\begin{itemize}
	\item Referencespændingsblokken til offsettet skal have en spændingsforsyning på $5.5$V.
	\item Spændingsreferencen skal levere en strøm på $2.5$V.
	\item Blokken skal samlet levere en referencespænding på $1.6325$V.
	\item De benyttede modstande til designet af forstærkeren skal besidde den navngivne værdi.
	\item Outputimpedansen skal være maks $1$K$\Omega$.
\end{itemize}
\noindent \textbf{Tolerance:}
\begin{itemize}
	\item Der accepteres en afvigelse i spændingsforsyning på $\pm1\%$.
	\item Der accepteres en afvigelse i outputtet fra spændingsreferencen på $\pm1\%$.
	\item Der accepteres en afvigelse i den leverede referencespænding på $\pm1\%$. 
	\item Der accepteres en afvigelse i de benyttede modstandes reelle værdi på $\pm1\%$.
	\item Der accepteres ikke en outputimpedans på over $1$K$\Omega$.
\end{itemize}
%%%%%%%%%%%%%%%%%%%%%%%%%%%%%%%%%%%%%%%%%%%%%%%%%%%%%%%%%%%%%%%%%%%
\subsection{Filter}\label{FilterAfs}
Blokken indeholdende et filter anvendes til at frasortere uønskede signaler, der kan forekomme under målinger med accelerometret. Filtreringsblokken inddrages i systemet jævnfør afsnit\ref{Filterafsnit} på side \pageref{Filterafsnit}. Frekvensområdet for kropshældning er ikke klart defineret, men studier anvender et frekvensområde liggende mellem $0.02-10Hz$ \cite{Martinez-Mendez2011}. Alle signaler, der ligger udenfor dette spektrum, ønskes derfor fjernet. I pilotforsøget blev der påvist, at signalet i dette tilfælde ligger mellem $0-25Hz$. Alt signal uden for dette spektrum ønskes derfor frafiltreret. Dette gøres ved at benytte et lavpasfilter, der har en knækfrekvens ved $25$Hz. \\
Den maksimale variation i pasbåndstransmissionen ($A_{max}$) sættes til $3$dB. For fjerne $50Hz$ støj, sættes stopbåndsfrekvensen til $45$Hz. Der beregnes en faktor ud fra LSB, som udregnes i afsnit \ref{ADCafsnit} på side \pageref{ADCafsnit} samt den maksimale amplitude af signalet. Den maksimale amplitude beregnes ved en Root Mean Square (RMS) analyse ud fra rotationerne foretaget i afsnit \ref{Bilag:Pilotforsoeg} på side \pageref{Bilag:Pilotforsoeg} for hhv. højre og venstre side. Dæmpningsfaktoren udregnes ved følgende ligning:
\begin{equation}
\label{eq:daempningsfaktor}
\dfrac{0.00244\text{V}}{0.0162\text{V}} = 0.1506 \approx 0.2 
\end{equation}
\begin{equation}
\dfrac{0.00244\text{V}}{0.0131\text{V}} = 0.1863  \approx 0.2
\end{equation}
Der ønskes en dæmpning på faktor $\dfrac{1}{5}$, da der tages forbehold for et tolerance område. Dæmpningens gain i $dB(A_{min})$ udregnes ud fra følgende formel:   
\begin{equation}
\label{eq:daempningsfaktoridB}
A_{min}=20 \cdot log_{10} \cdot (\frac{1}{5}) = -13,98 \approx -14\text{dB}
\end{equation}
\textbf{Krav:}
\begin{itemize}
	\item Filteret skal have en spændingsforsyning på $5.5$V.
	\item Filteret skal kunne modtage et inputsignal på $\pm3$V.
	\item Der ønskes en knækfrekvens i pasbåndet på $25$Hz.
	\item Der ønskes en stopbåndsfrekvens på $45$Hz.
	\item Der ønskes en minimum dæmpning af stopbåndet på $14$dB.
	\item Der ønskes en maksimal dæmpnning i pasbåndet på $3$dB.
\end{itemize}
\noindent \textbf{Tolerance:}
\begin{itemize}
	\item Der accepteres en afvigelse i spændingsforsyning på $\pm9\%$.
	\item Der accepteres ingen afvigelse ift. hvor stort et inputsignal filtret skal kunne modtage.
	\item Der accepteres ikke en knækfrekvens i pasbåndes på over $25$Hz.
	\item Der accepteres ikke en stopbåndsfrekvens på under $45$Hz eller over $50$Hz.
	\item Der accepteres minimal dæmpning på $14$dB af stopbåndet eller en afvigelse på $+15\%$.
	\item Der accepteres maksimal dæmpning på $3$dB af pasbåndet eller en afvigelse på $-15\%$.
\end{itemize}
%%%%%%%%%%%%%%%%%%%%%%%%%%%%%%%%%%%%%%%%%%%%%%%%%%%%%%%%%%%%%%%%%%%
\subsection{Tilpasning}\label{Tilpasningsblok}
Blokken tilpasning har til formål at tilpasse det filtrerede signal til komparatoren. Dette gøres med en forstærker, således at spektret, der skal reguleres indenfor, øges. Derudover afgrænses måleintervallet til $\pm25^{\circ}$, da et range på $\pm90^{\circ}$ er unødvendigt ift. at vurdere hvorvidt patienten er faldet. Det vurderes, at en hældning på over $25^{\circ}$ medfører fald, da denne hældning ligger væsentligt ud over den naturlige hældningsgrad jævnfør afsnit \ref{BalanceAfsnit}, side \pageref{BalanceAfsnit}. Der ønskes størst mulig forstærkning indenfor det fastsatte arbejdsområde på $\pm3$V. Indgangsimpedansen bestemmes ud fra samme kriterier som for forstærkeren i opsamlingsblokken. I bilag \ref{Bilag:Pilotforsoeg} på side \pageref{Bilag:Pilotforsoeg} er den størst mulige spænding efter justering af offsettet udregnet, hvilket er $0.3313V$. Derudover er der det beregnet, at den maksimale volt pr. grad er $0.0037$V. Ved $\pm25^{\circ}$ vil outputtet være;
\begin{align}
\label{Udreg3} 0.0037 \cdot 25 = 0.0925V
\end{align}
Forstærkerens faktor samt gain i dB udregnes i \eqref{TilpasEq2} og \eqref{TilpasEq3}.
\begin{align}
\label{TilpasEq1} 0.0925V \cdot 9.1 = 0.8418V \\
\label{TilpasEq2} \dfrac{3V}{0.8418V} = 3.5638 \approx 3.6 \\
\label{TilpasEq3} 20 \cdot log_{10} (3.6) = 11.1261\text{dB}
\end{align} 
I \eqref{TilpasEq1}, \eqref{TilpasEq2} og \eqref{TilpasEq3} er de $3V$ grænsen for arbejdsområdet. Der forstærkes med en faktor $3.6$. \\

\noindent\textbf{Krav:}
\begin{itemize}
	\item Forstærkeren skal modtage en spændingsforsyning på $5.5$V.
	\item Forstærkeren skal kunne modtage et inputsignal på $\pm3$V.
	\item Forstærkeren skal forstærke med en faktor $3.6$, som svarer til $11.1261$dB.
	\item De benyttede modstande til designet af forstærkeren skal besidde den navngivne værdi.
	\item Indgangsimpedansen skal være over $3.2M\Omega$.
	\item Outputimpedansen skal være under $32K\Omega$.
\end{itemize}
\noindent\textbf{Tolerance:}
\begin{itemize}
	\item Der accepteres en afvigelse i spændingsforsyning på $\pm9\%$.
	\item Der accepteres ingen afvigelse ift. hvor stort et inputsignal forstærkeren skal kunne modtage.
	\item Der accepteres en afvigelse i forstærkningen på $\pm2\%$.
	\item Der accepteres en afvigelse i de benyttede modstandes reelle værdi på $\pm1\%$.
	\item Der accepteres en indgangsimpedans på mindst $3.2M\Omega$.
	\item Der accepteres en outputimpedans under $32K\Omega$.
\end{itemize}
%%%%%%%%%%%%%%%%%%%%%%%%%%%%%%%%%%%%%%%%%%%%%%%%%%%%%%%%%%%%%%%%%%%%%
\subsection{Referencespænding til komparatoren}\label{Ref_Kompar_Afs}
Denne blok skal forsyne komparatoren med en konstant spænding, som skal anvendes til sammenligning imellem et inputsignal og denne spændingsværdi.
Blokken skal levere $2.5$V, da spændingsdelingen er implementeret i komparatorblokken for at give individuelle tærskelværdier til feedbackblokken. \\ \textbf{Krav:}
\begin{itemize}
	\item Referencespændingsblokken til offsettet skal have en spændingsforsyning på $5.5$V.
	\item Spændingsreferencen skal levere en strøm på $2.5$V.
	\item Blokken skal samlet levere to outputs - $2.5$V og -$2.5$V.
	\item De benyttede modstande til designet af forstærkeren skal besidde den navngivne værdi.
\end{itemize}
\noindent \textbf{Tolerance:}
\begin{itemize}
	\item Der accepteres en afvigelse i spændingsforsyning på $\pm1\%$.
	\item Der accepteres en afvigelse i outputtet fra blokken på $\pm1\%$.
	\item Der accepteres en afvigelse i outputtet fra spændingsreferencen på $\pm1\%$. 
	\item Der accepteres en afvigelse i de benyttede modstandes reelle værdi på $\pm1\%$.
\end{itemize}
%%%%%%%%%%%%%%%%%%%%%%%%%%%%%%%%%%%%%%%%%%%%%%%%%%%%%%%%%%%%%%%%%%%
\subsection{Feedback}\label{KomparatorAfs}
Blokken indeholdende komparatorer skal kunne sammenligne et inputsignal på maks $\pm3$V med referencespændinger. Komparatorerne skal vha. definerede tærskelværdier bestemme, hvilken hældningsgrad patienten har, og i hvilken retning vedkommende hælder mod. Der skal være flere tærskelværdier, så feedbacken kan gives i flere niveauer. Tærskelværdierne er bestemt på baggrund af pilotforsøg, hvor værdierne er beregnet ved først at finde spændingen for $90^{\circ}$, som jævnført \ref{Sec_Pilot_Data} på side \pageref{Sec_Pilot_Data} er $0.3313$V for positiv retning og -$0.3233$V i negativ retning efter forstærkning på hhv. en faktor $9.1$ og derefter en faktor $3.6$. Denne værdi divideres med $90$ for at bestemme spændingen ved $\pm1^{\circ}$ og ganges herefter med hhv. $2^{\circ}$,  $8^{\circ}$ og $13^{\circ}$ for at bestemme spændingen ved de enkelte hældninger og derved definere tærskelværdierne.\\
Visuel og somatosensorisk feedback omfatter den del af systemet, der giver feedback til patienten vha. LED-dioder og vibration. Der skal i alt være fem LED-dioder: En grøn, der placeres i midten, en gul på hver side af den grønne diode og en rød diode på hver side af de gule dioder. Derudover skal der være to vibratorer, som placeres medialt på hver underarm af patienten.\\
Den grønne LED-diode skal lyse, når accelerometret er placeret korrekt på patienten, hvorved der er sikret den rette udgangsposition for hældningen. Dette gør, at den grønne LED-diode er aktiveret fra $0^{\circ}$-$\pm2^{\circ}$. I intervallet $\pm2^{\circ}$-$\pm8^{\circ}$ lyser ingen LED-dioder. Den gule LED samt en vibrator aktiveres, når patienten skal være opmærksom på sin kropshældning ved en hældning på $\pm8^{\circ}$. Den røde LED-diode aktiveres, når patienten er over $\pm13^{\circ}$ og derfor er i fare for at falde. Vibratoren for den pågældende side vil da stadig være aktiveret. Retningen af patientens kropshældning bestemmes vha. placeringen af LED. Hvis aktiveringen af LEDerne går mod højre indikerer dette, at patienten er ved at falde til højre. Det samme gør sig gældende for fald til venstre. \\
\textbf{Krav:} 
\begin{itemize}
	\item Komparatorblokken skal have en spændingsforsyningen på $\pm5.5$V.
	\item Komparatoren skal kunne modtage et inputsignal på $\pm3$V $+5\%$.
	\item Komparatoren skal ved et bestemt spændingsniveauer aktivere en diode og vibrator:
	\begin{itemize}
		\item Grøn diode og ingen vibration: I intervallet indenfor $0.2424$V til -$0.2359$V. Svarende til $\pm2^{\circ}$.
		\item Gul diode og vibration: Fra niveauet $0.9697$V og -$0.9495$V. Svarende til $\pm 8^{\circ}$ eller over.
		\item Rød diode og vibration: Fra nivauet $1.5758$V og -$1.5332$V. Svarende til $\pm13^{\circ}$ eller over.
		%\item Ingen feedback: Over $2.2602$V og under -$2.2168$V. 
	\end{itemize}
	\item LEDerne og vibratorerne skal aktiveres ved angivet tærskelværdier i komparatoren.
\end{itemize}
\textbf{Tolerance:}
\begin{itemize}
	\item Der accepteres en afvigelse i spændingsforsyningen på $\pm5\%$.
	\item Der accepteres ingen afvigelse ift. hvor stort et inputsignal komparatoren skal kunne modtage.
	\item Der accepteres en afvigelse i tærskelværdierne på $\pm1\%$.
	\item Der accepteres ingen afvigelse ift. aktivering af LED og vibratorerne, hvis tærskelværdierne for den bestemte LED-diode er opnået i komparatorblokken.
\end{itemize}
%%%%%%%%%%%%%%%%%%%%%%%%%%%%%%%%%%%%%%%%%%%%%%%%%%%%%%%%%%%%%%%%%%%%%%
\subsection{Spændingsforsyning}\label{Krav_spaending_spicifikt}
I systemet anvendes en spændingsforsyning af to $1.5$V batterier placeret i en spændingsregulator, der skal levere en konstant spænding til hele kredsløbet. Der anvendes batterier for at undgå tilslutning til elnettet, hvilket gør det mere sikkert og anvendeligt for patienten. Spændingsforsyningen kan havde indflydelse på, hvilket outputsignal der opnås, og det er derfor vigtigt, at blokkene i systemet designes efter den fastsatte spænding. \\
\noindent\textbf{Krav:}
\begin{itemize}
	\item Spændingsregulatoren skal levere en spænding på mindst henholdsvis $3.3$V og $5.5$V fra hver terminal.
	\item Spændingsregulatoren skal forsyne samtlige blokke i systemet med den minimale krævede spænding.
	\item Spændingsregulatorens forsyning må ikke forårsage klipning af signalet.
\end{itemize}
\noindent\textbf{Tolerance:}
\begin{itemize}
	\item Der accepteres ikke en spænding under $5.5$V eller $3.3$V.
	\item Der accepteres ingen afvigelse for spændingsregulatorens tilkobling eller klipning af signalet.
	%\item Der accepteres ingen spænding højere end dioderne kan tåle \fxnote{måske en anden måde det kan skrives på}
\end{itemize}
%%%%%%%%%%%%%%%%%%%%%%%%%%%%%%%%%%%%%%%%%%%%%%%%%%%%%%%%%%%%%%%%%%%%%%%%%
\subsection{ADC}
Blokken med en ADC anvendes i systemet, for at konvertere det analoge signal til digitalt. %Den skal kunne sample det forstærkede signal. 
ADC'ens inputsignal vil ligge i området $\pm3V$, men der skal tages højde for afvigelser. Båndbredden på accelerometeret bestemmer, hvilken samplingsfrekvens der skal benyttes. I følge Nyquists sætning skal der samples med minimun det dobbelte af det biologiske signals frekvens. I praksis samples der med minimum det tidobbelte jævnfør afsnit \ref{ADCafsnit}, side \pageref{ADCafsnit}. I dette tilfælde er båndbredden $25Hz$, dvs. at der som minimum skal samples med $100Hz$, og i praksis vil det være hensigtsmæssigt at sample med minimum $250Hz$. \\
\textbf{Krav:}
\begin{itemize}
	\item Komparatoren skal kunne modtage et inputsignal på $\pm3$V $+5\%$.
	\item Skal have en samplingsfrekvens på minimum $250$Hz.
	\item Skal ikke ændre på det signalet, som den optager.
\end{itemize}
\textbf{Tolerance:}
\begin{itemize}
	\item Der accepteres ingen afvigelse ift. kravene om inputsignal og samplingsfrekvens.
	\item Der accepteres en afvigelse på $\pm1\%$ af ADC'ens målinger af signalet sammenlignet med det indsendte signal.
\end{itemize}
%%%%%%%%%%%%%%%%%%%%%%%%%%%%%%%%%%%%%%%%%%%%%%%%%%%%%%%%%%%%%%%%%%%%%%%%
\subsection{USB-isolator}\label{kravspecifikationer_USB}
Blokken med USB-isolatoren anvendes i systemet for at øge patientens sikkerhed. USB-isolatoren isolerer patienten fra kredsløbet og sørger, at der ikke løber lækstrømme fra computeren ind i systemet.\\
\textbf{Krav:}
\begin{itemize}
	\item USB-isolatoren skal være i stand til at kunne modtage et signal på $\pm4$V. 
\end{itemize}
\textbf{Tolerance:} 
\begin{itemize}
	\item Der accepteres ikke en afvigelse ift. kravet til USB-isolatoren. 
\end{itemize}
%%%%%%%%%%%%%%%%%%%%%%%%%%%%%%%%%%%%%%%%%%%%%%%%%%%%%%%%%%%%%%%%%%%%%%%%%55
\subsection{Software}\label{subsec:software}
Blokken indeholdende softwaren implementeres i systemet for at kunne behandle og gemme patienternes øvelsesresultater. Denne del af systemet er brugerfladen for det fagkyndige personale og skal derfor kunne fremvise information omkring patienternes balance i form af grafer eller lignende. Det fagkyndige personale skal vha. af softwaren kunne følge med i patienternes udvikling ift. balancen. \\
\textbf{Krav:}
\begin{itemize}
	\item Skal kunne fremvise data med information om patientens hældning i de enkelte øvelser, herunder hvor ofte patienten har bevæget sig ud i risikozonerne. 
	\item Skal kunne gemme data med information om patientens hældning.
	%\item Skal være brugervenligt for det fagkyndige personale, dvs. designet af programmet skal være enkelt. 
\end{itemize}
\textbf{Tolerance:}
\begin{itemize}
	\item Der accepteres ingen afvigelse ift. software. 
\end{itemize}