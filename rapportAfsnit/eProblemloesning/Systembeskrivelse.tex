% !TeX spellcheck = da_DK
\section{Systembeskrivelse} 
Dette afsnit indeholder en beskrivelse af det system, der skal kunne anvendes af apopleksipatienter som et selvstændigt træningsapparat i rehabiliteringen af balanceproblemer. Systembeskrivelsen indeholder målgruppen for designet, samt hvilket formål og anvendelse det har. Ud fra disse faktorer er systemet blevet designet og illustreret i et blokdiagram. 

\subsection{Systemets bruger}
Systemet udvikles til apopleksipatienter med balanceproblemer mhp. selvtræning af balance i rehabiliteringsfasen i fase $3$ og $4$ i afsnit \ref{Faser} på side \pageref{Faser}. Systemet skal være let anvendeligt, da det udvikles til anvendelse i hjemmet af patienterne selv. Systemets design skal altså være enkelt, så der ikke skabes forvirring blandt patienterne ift. systemets funktioner. Fagkyndigt personale, såsom fysioterapeuter og læger, skal kunne instruere patienten i brugen af systemet samt følge med i udviklingen, som patienten gennemgår. Det skal derfor være muligt for det fagkyndige personale at anvende systemet og aflæse data herfra. 
%Dette gøres ved at have et analogt og digitalt output, hvor den digitale del henvender sig til det fagkyndige personale i form af grafer, mens den analoge del henvender sig til apopleksipatienterne. \fxnote{Erikas kommentar: relevans ift. analogt og digitalt output? Tekniske detaljer! Er det vigtigt, at det er et analog output eller at det output indeholder specifik information?}   

\subsection{Systemets formål og anvendelse}\label{formål_anvendelse}
Systemets input er patienternes kropshældning i det frontale plan, dvs. hvor meget vedkommende svajer, i anatomisk normalstilling og under udførelse en bestemt øvelse i stående udgangsposition, jævnfør afsnit \ref{rehabiliteringbalance}, side \ref{rehabiliteringbalance}. Systemet skal kunne konvertere informationerne vedrørende patienternes kropshældning til visuel og somatosensorisk feedback samt et digitalt output i form af grafer. Den visuelle og somatosensorisk feedback har til formål at gøre apopleksipatienter opmærksomme på, hvornår de har bevæget sig over den normale grænse for krops svaj. Således kan systemet registrere, hvis patienten er i risiko for at falde. Inden et fald sker, udsendes et feedback signal, så patienterne har mulighed for at rette sig op. Selve systemet skal anvendes til selvtræning i hjemmet ved udførelse af de to omtalte øvelser, dvs. træning af statisk balance. Det skal derfor være et brugervenligt system, dvs. systemet skal kunne påsættes uden problemer og fungere uden, at patienten skal navigere rundt i forskellige funktioner for at påbegynde feedbacken. Systemet skal fungere som en hjælp for patienten, da vedkommende bliver bevidst omkring sin balance. Herved kan apopleksipatienten være mere selvstændig i rehabiliteringsprocessen.

Patienterne kan anvende systemet ved to sværhedsgrader ift. kropsposition: normal kropsstilling (anatomisk udgangsposition) og ovenstående omtalte øvelse. %Apopleksipatienternes balance udfordres i højere grad af SRT end ved normal kropsstilling, eftersom kropsvægten fordeles anderledes ved denne øvelse ift. den normale kropsstilling omtalt i afsnit \ref{BalanceAfsnit} på side \pageref{BalanceAfsnit}. 
%SRT udføres i stående udgangsposition med fødderne på en tegnet linje, så den ene fods tæer er mod den anden fods hæl. Derudover holdes armene tæt ind til kroppen og over kors. \cite{Huo1999}.\fxnote{Erikas rettelser til dette skal tilføjes}\\ %Denne position er valgt for at udfordre patientens balance ved at fordele kropsvægten anderledes ift. den normale kropsstilling omtalt i afsnit \ref{BalanceAfsnit} på side \pageref{BalanceAfsnit}. % Patienten påsætter selv systemet øverst på sternum og udfører herefter en kort prøvetest for at kende til de givne feedback parametre.Under prøvetesten svajer patienten langsomt fra side til side. 
%Det er på baggrund af afsnit \ref{MekBioFeed} på side \pageref{MekBioFeed} valgt at hældningen på patienten skal opfanges vha. et accelerometer, der er placeret øverst på sternum. 
Det er på baggrund af afsnit \ref{MekBioFeed}, side \pageref{MekBioFeed} valgt, at systemet skal placeres øverst på sternum for at få bedst mulige målinger ift. patienternes kropshældning. Systemet skal give visuel og somatosensorisk feedback i form af en vibrator samt fem dioder på linje, bestående af en grøn, to gule og to røde. Med denne metode indikeres både, hvilken retning patienten svajer samt graden heraf. Ved benyttelse af to feedback former er der større mulighed for, at patienten kan opfange signalerne. Hvis patientens visuelle sans er begrænset kan systemet stadig benyttes grundet den somatosensorisk feedback. Jævnfør afsnit \ref{BalanceAfsnit}, side \pageref{BalanceAfsnit} er grænsen for, hvornår et fald forekommer ift. hældningsgrad individuel. I praksis bør systemet dermed tilpasses til den enkelte patient på baggrund af testøvelser ift. balancen. Hældningsgraderne vil i dette projekt blive valgt på baggrund af raske forsøgspersoner, da det vurderes udfra problemanalysen afsnit \ref{BalanceAfsnit}, side \pageref{BalanceAfsnit}, at apopleksipatienter har flere sygdomsrelaterede faktorer, der kan påvirke deres hældningsgrad. 

%Hvis patienten hælder i intervallet $8^{\circ}$-$13^{\circ}$ til højre, indikeres dette af den gule diode på højresiden af den grønne diode. Derudover aktiveres en mild vibration mod patientens hænder, når den gule diode lyser. Hvis patienten hælder $13^{\circ}$ eller derover, lyser den røde diode til højre for den gule diode og styrken af vibrationen forøges. Det samme gør sig gældende for hældning mod venstre. Med denne metode indikeres både, hvilken retning patienten svajer samt graden heraf. Ved benyttelse af to feedback former er der større mulighed for, at patienten kan opfange signalerne. Hvis patientens visuelle sans er begrænset kan systemet stadig benyttes grundet den somatosensorisk feedback. \\ 

For at øge øvelsernes sværhedsgrad yderligere kan den visuelle sans udelukkes. Patienten skal under øvelsen forsøge at holde balancen så længe som muligt uden at bevæge sig ud i en risikozonerne. Hvis patienten kommer ud i en risikozone, vil dette blive markeret ved lys i dioderne samt vibration. Træningsøvelsen kan gentages efter behov. Ved at tage flere målinger igennem rehabiliteringsforløbet vil det forventes, at der sker en fremgang ift. tiden, hvori balancen kan opretholdes uden at patienten bevæger sig ud i risikozonerne. 

\subsection{Accelerometer}\label{Subsec:AccTeori}
Der er valgt et accelerometer til måling af det biologiske signal, som der skal gives feedback på. Accelerometeret ADXL335, som ses på \figref{ADXL335} er en treakset sensor, som kan anvendes til måling af statisk balance. Accelerometeret har en single-supply spændingsforsyning, der skal være mellem $1.8$ - $3.6$V, da dette er accelerometerets driftsspændings interval. På dette accelerometer er der tilkoblet en regulator, hvilket forhindrer at spændingsforsyningen kan supplere accelerometret med mere end $3.3$V.  Arbejdsområdet ligger mellem $\pm3.6$g, og outputtet fra accelerometeret er maksimalt på $\pm1.188$V ift. accelerometerets indbyggede offset. Offsettet varierer efter spændingsforsyningen men ligger ved $3$V forsyning på $1500$mV og beregnes som $Off = Vs/2$. Båndbredden for X og Y-akserne ligger mellem $0.5$ til $1600$Hz og for Z-aksen mellem $0.5$ til $550$Hz. Støjen fra Xout og Yout ligger normalt på $150\mu g/\sqrt{Hz * rms}$\fxnote{Ret evt. denne enhed - erikas rettelser}, mens det for Zout ligger på $300\mu g/\sqrt{Hz * rms}$. Sensitiviteten afhænger, ligesom offsettet, af spændingsforsyningen, da accelerometeret er ratiometrisk\fxnote{NTK: Ratiometrisk vil sige, at output er direkte proportionelt med input.} og ligger ved $3$V forsyning mellem $270$ og $330$mV/g. Outputsignalet er en analog spænding der er proportionel med accelerationen. Accelerometerets outputimpedans er $32K\Omega$ med en afvigelse på $\pm15\%$ \cite{Devices2009} %Outputtet ligger normalt ved -1.08g i X-aksen og  1.08g Y-aksen og 1.83 g ved Z-aksen. 

\begin{figure}[H]
	\centering 
	\includegraphics[scale=0.5]{figures/cProblemloesning/ADXL335_2.JPG}
	\caption{På figuren ses accelerometeret af typen ADXL335 i forskellige stillinger. Til venstre på figuren ses accelerometerets pin konfiguration og funktions beskrivelse. Til højre ses, hvordan accelerometret skal placeres for forskellig g påvirkning. \textit{(Revideret)} \cite{Devices2009}}
	\label{ADXL335}
\end{figure}

Når accelerometeret hældes til siden, vil der ske en acceleration ift. tyngdekraften i en given retning og dermed et udslag fra referencepunktet, som er ved en hældning på 0$^{\circ}$. Hvis accelerometeret f.eks. stilles på højkant, som det ses til højre på \figref{ADXL335} i stilling $1$, vil x aksen blive påvirket med $-1$g.\cite{Devices2009} \\
Sammenhængen mellem de enkelte parametre kan udtrykkes ved følgende ligninger: 
\begin{align}
	V_{out} = V_{offset} + sensitiviteten * tyngdekraften * \sin(vinklen) \\
	V_{out} = V_{offset} + \frac{\Delta V}{\Delta g} * g * \sin \Theta
\end{align}
Herudfra er det muligt at isolere og udregne de ukendte parametre, altså kan patientens hældningsgrad bestemmes ud fra accelerometerets output.

\subsection{Overordnede funktionelle krav til systemet}\label{FunkKrav}
\begin{itemize}
	\item Systemet skal være simpelt, så det kan anvendes af både apopleksipatienterne selv samt af fagkyndigt personale
	\item Systemet skal kunne måle kropshældning, samt angive hvilken retning hældningen sker mod. 
	\item Systemet skal kunne give visuel og somatosensorisk feedback ved forskellige hældningsgrader.
	\begin{itemize}
		\item Grøn diode: Skal lyse, når patienten ikke er ude i risikozonerne og informere patienten om at accelerometeret er placeret korrekt.  
		\item Gul diode: Skal lyse, når den første risikozone defineret i grader indtræffer og slukke, hvis patienten retter sig op.
		\item Rød diode: Skal lyse, når den anden risikozone defineret i grader indtræffer og slukke, hvis patienten retter sig op.
		\item Vibration: Skal aktiveres, når den første risikozone indtræffer og skal slukke, hvis patienten retter sig op. Hvis patienten hælder yderligere, skal vibrationshastigheden stige.
	\end{itemize}
	\item Systemet skal kunne skifte mellem to sværhedsgrader, hvilket vil sige, at systemet skal kunne skifte imellem to forskellige komparator blokke, som har tærskelværdier relateret til sværhedsgraden af den udførte øvelse.
	\item Systemet skal kunne give et digitalt outout, så fagkyndigt personale kan behandle og gemme patienternes data i et program.
\end{itemize}

\subsection{Systemets opbygning}\label{ref:blokdiagram}
Systemets opbygning fremgår af \figref{kravblok}.

\begin{figure}[H]
	\centering
	\includegraphics[scale=0.5]{figures/cProblemloesning/blokdiagram.PNG}
	\caption{På figuren ses de enkelte blokke, som systemet skal indeholde.}
	\label{kravblok}
\end{figure}
Det biologiske signal, der opsamles med accelerometeret, skal som det første centreres omkring $0$V ved en offsetjustering og efterfølgende forstærkes. Herefter skal signalet filtreres for at frasortere uønskede frekvenser, der ligger uden for det ønskede signals spektrum. Herefter deler systemet sig i en analog og digital del. I den analoge del følger en tilpasning af signalet i form af en forstærkning. Herefter kommer signalet over i en komparatordel, hvor det bliver sammenlignet med nogle definerede tærskelværdier, der hver især er koblet til en bestemt type feedback, der skal udløses hvis tærskelværdien overskrides. \\
I den digitale del af systemet, bliver signalet ledt ind i en ADC. Denne vil omdanne det analoge biologiske signal til et digitalt signal. Det digitale signal bliver herefter ledt ind i en USB-isolator, så der ikke opstår lækstrømme og for at øge patientens sikkerhed. Til sidst vil det digitale signal overføres til en computer, hvor det efterfølgende kan gemmes. Det bliver herved muligt at databehandle signalet og opstille det som en graf eller lignende. \\
Systemet har altså både et output henvendt til patienterne og det fagkyndige personale. Patienternes output er den feedback, der oplyser om deres hældningsgrad. Programmet, der skal behandle og gemme patienternes øvelsesresultater, er output til det fagkyndige personale.

