% !TeX spellcheck = da_DK
\chapter*{Forord og læsevejledning}
\section{Forord}
Denne rapport er udarbejdet af sundhedsteknologi studerende på 3. semester fra Aalborg Universitet. Ud fra projektets overordnede tema: "Instrumentering til opsamling af fysiologiske signaler" blev der heraf opstillet forskellige projektforslag. Denne rapport tager udgangspunkt i følgende projektforslag opstillet af Erika G. Spaich: "System til detektering af kropsbalance". Gruppens vejleder har under hele projektperioden været Erika G. Spaich.

Projektet rettes mod apopleksipatienter, fagkyndigt personale, der beskæftiger sig med rehabilitering af apopleksipatienter, og medstuderende på Aalborg Universitet samt andre, der har interesse i emnet. \\
Vi vil gerne takke vores vejleder Erika G. Spaich for vejledning og feedback igennem hele projektperioden. Derudover vil vi give en særlig tak til Jan Stavnhøj for hjælp og rådgivning til udarbejdelse af systemet samt Træningsenhed Vest Aalborg Kommune for, at vi måtte komme og observere en dags genoptræning af patienter med balanceproblemer. 

\section{Læsevejledning}
Projektrapporten er udarbejdet efter den problembaserede AAU-model. Selve rapporten er delt op i fire kapitler samt appendiks, således at første kapitel indeholder projektets initierende problem, der ligger til grund for problemanalysen, og det initierende problem. Andet kapitel indeholdende problemanalysen giver viden om apopleksi og apopleksipatienternes følger, rehabiliteringsforløb og nuværende rehabiliteringsmuligheder samt baggrundsviden vedrørende teknologisk behandling af biologiske signaler.  Dette er efterfulgt af en projektafgrænsing samt problemformulering, der ligger til grund for problemløsningen. Problemløsningen beskrives i kapitel tre, indeholdende projektets praktiske del ift. at bygge et system til detektering af kropsbalancen. Herunder beskrives systemets kravspecifikationer samt systemdesign, hvor teori, simulering, implementering og test for hver del i systemet vil være skildret. I fjerde kapitel afsluttes rapporten med en evaluerende diskussion og konklusion af systemets funktion samt perspektivering ift. udvikling af systemet. Herefter findes litteraturlisten, samt appendiks, der henvises til som bilag A, bilag B osv. 

I rapporten benyttes Vancouver-metoden ved litteraturhenvisning, hvor anvendt litteratur tildeles fortløbende numre, således at den første reference i rapporten tildeles nummeret [1], den næste [2] osv. I litteraturlisten skrives den fulde reference, dvs. forfatter navn og årstal samt URL-kode, hvis referencen er en hjemmeside, i den rækkefølge, som referencen anvendes i rapporten. Hvis referencen er placeret efter et punktum i en sætning, tilhører referencen hele afsnittet, hvorimod er referencen placeret før et punktum, tilhører referencen kun den pågældende sætning. Er der placeret flere referencer efter hinanden, betyder dette, at der er anvendt flere referencer til den pågældende sætning eller afsnit. 

Der anvendes det amerikanske komma, når der i rapporten skrives tal - eksempelvis 12,500 og 2.4.%Dvs. det amerikanske komma på dansk er et punktum og omvendt det amerikanske punktum på dansk er et komma.    

