% !TeX spellcheck = da_DK
\chapter*{Forord}
Denne rapport er udarbejdet af gruppe B$203$, der er studerende på $2$. semester Sundhedsteknologi på Aalborg Universitet. Projektet tager udgangspunkt i temaet "Eksperimentiel fysiologi - måling, behandling og præsentation af biologiske signaler"   fra P$2$-projektkataloget, hvorfra emnet "Træthed Vurdering og Køn"  blev valgt. Der er taget inspiration fra hovedtemaet til udarbejdelsen af projektet. Det overordnede formål lyder: "\textit{Formålet med projektet er at undersøge hvilken metode er mest hensigtsmæssig til at evaluere muskeltræthed og hvordan man kan vurdere træthed af en bestemt muskel, når flere agonister kontraherer under fysisk anstrengelse og/eller bevægelse.}"\cite{ProjektKatalog}. \\
I denne rapport anvendes Chicago-metoden til kildereferencer. Dette betyder, at den anvendte litteratur angives med numre. Den fulde reference af kilden kan derefter findes i litteraturlisten bagerst i rapporten. Derudover er figurer og tabeller nummereret ud fra deres respektive afsnit. \\
Når en forkortelse anvendes første gang, er den fulde betegnelse skrevet efterfulgt af forkortelsen angivet i parentes. Forkortelsen af ordet bliver herefter benyttet i resten af rapporten. \\
Gruppens vejleder under projektperioden var Sabata Gervasio, som vi gerne vil takke for godt samarbejde. Desuden skal Aalborg Universitet have tak for lån af apparatur til afvikling af forsøg. %\clearpage
%Projektet er udarbejdet af:\\
%\begin{table}[H]
%	\centering
%	\begin{tabular}{c c c}
%		\underline{\phantom{JAERJAERJAERJAERGO}} & \phantom{cookies} & \underline{\phantom{JAERJAERJAERJAERGO}} \\
%		Cecilie S. R. Topp			& \phantom{cookies} & Christian Ulrich		\\
%		&&\\
%		&&\\
%		\underline{\phantom{JAERJAERJAERJAERGO}} & \phantom{cookies} & \underline{\phantom{JAERJAERJAERJAERGO}} \\
%		Ida V. Johansen			& \phantom{cookies} & Nikoline S. Kristensen		\\
%		&&\\
%		&&\\
%		\underline{\phantom{JAERJAERJAERJAERGO}} & \phantom{cookies} & \underline{\phantom{JAERJAERJAERJAERGO}} \\
%		Sadieh Ghasemi 					& \phantom{cookies} & Sebastian Munk 			\\			
%		&&\\
%		&&\\									
%	\end{tabular}
%\end{table}