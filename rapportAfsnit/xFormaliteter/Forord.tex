% !TeX spellcheck = da_DK
\chapter*{Forord og læsevejledning}
\section*{Forord}
Denne rapport er udarbejdet af en gruppe sundhedsteknologistuderende på 3. semester fra Aalborg Universitet. Ud fra projektets overordnede tema \textit{Instrumentering til opsamling af fysiologiske signaler} blev der heraf opstillet forskellige projektforslag. Denne rapport tager udgangspunkt i følgende projektforslag opstillet af Erika G. Spaich: \textit{System til detektering af kropsbalance}. Gruppens vejleder har under hele projektperioden været Erika G. Spaich.

Projektet er rettet mod apopleksipatienter, fagkyndigt personale, der beskæftiger sig med rehabilitering af apopleksipatienter, medstuderende på Aalborg Universitet og andre, der har interesse for emnet. \\
Vi vil gerne takke vores vejleder Erika G. Spaich for vejledning og feedback igennem hele projektperioden. Derudover vil vi give en særlig tak til Thomas N. Nielsen for hjælp til softwaredelen, Jan K. Stavnshøj for hjælp og rådgivning til udarbejdelse af systemet og Træningsenhed Vest Aalborg Kommune for, at vi måtte observere en dags genoptræning af patienter med balanceproblemer. 

\section*{Læsevejledning}
Projektrapporten er udarbejdet efter den problembaserede AAU-model. Selve rapporten er delt op i fire kapitler samt bilag således, at første kapitel indeholder projektets initierende problemstilling, der ligger til grund for problemanalysen, og det initierende problem. Andet kapitel giver viden om apopleksi og apopleksipatienternes følger, rehabiliteringsforløb og nuværende rehabiliteringsmuligheder samt baggrundsviden vedrørende signalbehandling. Dette er efterfulgt af en projektafgrænsing samt problemformulering, der ligger til grund for problemløsningen. I tredje kapitel beskrives problemløsningen indeholdende projektets praktiske del ift. at udarbejde et system til detektering af kropsbalancen. Herunder beskrives systemets kravspecifikationer samt systemdesign, hvor teori, simulering, implementering og test for hver blok i systemet vil indgå. I fjerde kapitel afsluttes rapporten med en evaluerende diskussion og konklusion af systemets funktion samt perspektivering ift. videreudvikling af systemet. Herefter findes litteraturlisten, samt bilag, der henvises til som bilag A, bilag B osv. 

Til litteraturhenvisning anvendes Vancouver-metoden, hvor anvendt litteratur tildeles fortløbende numre således, at den første reference i rapporten tildeles nummeret [1], den næste [2] osv. Er referencen placeret efter et punktum i en sætning, tilhører den hele afsnittet. Er referencen placeret før et punktum, tilhører den sætningen. Er der placeret flere referencer efter hinanden, betyder dette, at der er anvendt flere referencer til den pågældende sætning eller afsnit. \\
Ved anvendelse af forkortelser skrives ordet helt ud, første gang det nævnes, med forkortelsen i en parentes, som efterfølgende vil blive benyttet. \\
Der anvendes det amerikanske komma, når der i rapporten skrives tal - eksempelvis skrives titusinde som 10,000 og to komma fire som 2.4.%Dvs. det amerikanske komma på dansk er et punktum og omvendt det amerikanske punktum på dansk er et komma.    

