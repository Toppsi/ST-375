\section{Problemafgrænsning}
Apopleksi er den tredje største dødsårsag i Danmark. \fxnote{Mere her - evt. fra indledning}
%Af alle tilfælde af apopleksi er den iskæmiske langt den hyppigste med 80-85\% af tilfældene, imens den hæmoragiske udgør 10-15\%.
Sygdommen har stor indvirkning på blodtilførslen til encephalon, som er det område af hjernen der blandt andet rummer cerebrum. I cerebrum sker behandlingen af bl.a. sensoriske signaler, tanker og følelser. Når blodtilførslen til encephalon, og dermed også cerebrum, er nedsat, vil der kunne opstå både motoriske og sensoriske skader hos patienten. 
Når en patient med formodet apopleksi bliver indlagt er det vigtigt, at lægen laver en grundig undersøgelse af de sensoriske og motoriske evner. Herefter foretages videre undersøgelser, eksempelvis CT- eller MR-scanning, afhængigt af den givne situation. Det er vigtigt, at behandlingen igangsættes hurtigst muligt for at begrænse følgevirkningerne. Der findes flere forskellige behandlingsmetoder. Ved iskæmi foregår behandlingen primært med blodfortyndende medicin, der har til formål at opløse blodproppen. \fxnote{Mangler vi ikke noget om behandlingen af hæmoragi??}
Apopleksi kan, som nævnt, medføre sensoriske og motoriske skader. Disse skader kan have stor indvirkning på patientens liv efter sygdomsforløbet og kan medføre en række psykiske konsekvenser og nedsat livskvalitet, bl.a. pga. sygdommens pludselige opståen.
To væsentlige følger er balanceproblemer og neglekt. Balancen er vigtig for at kunne fungere i dagligdagen, da den sikrer at man holder kroppen oprejst og muliggør bevægelse uden fald. Apopleksipatienter kan f.eks. lide af "Pusher Syndrom", hvor de ubevidst læner sig væk fra deres raske side. Dette påvirker balancen og kan lede til ulykker. Ved neglekt er patienten ikke bevidst om den ene kropshalvdel. Dette kan være enten et visuelt eller et kropsligt problem.
Fælles for de to typer af følger er, at de begrænser patienterne i deres dagligdag og gør dem afhængige af hjælp til mange ting, som de før sygdommen selv kunne udføre. Derfor opleves ofte alvorlige personlige følger, eksempelvis problemer med at opretholde personlige relationer. Derudover kan patienternes identitet og humør også påvirkes. 
For at begrænse de fysiske - og dermed også de personlige - følger mest muligt, er det essentielt at rehabiliteringen påbegyndes hurtigt efter apopleksien. Rehabiliteringen omfatter både genoptræning af fysiske funktioner, men også tilpasning til miljø og styrkelse af sociale kompetencer. 
I Danmark er rehabiliteringen organiseret således at de forskellige aktører i sundhedssektoren arbejder sammen på tværs af hinanden. Dette tværfaglige samarbejde er vigtigt, da apopleksipatienterne er i kontakt med mange forskellige dele af systemet under deres sygdomsforløb. 
Der findes idag flere forskellige metoder til rehabilitering, afhængig af hvilken type følger patienten er ramt af. Metoderne er baseret på forskellige principper, herunder bl.a. træning vha. lyd, mekanisk træning, spejling og sensorisk stimulation.
En anden måde at rehabilitere på er ved hjælp af biofeedback. Ved biofeedback måles på biologiske signaler, der relaterer til det område eller den funktion som skal rehabiliteres hos patienten. Disse data omsættes så til et signal der kan opfattes af patienten. Der kan skelnes imellem fysiologisk biofeedback og biomekanisk biofeedback. Ved fysiologisk biofeedback måles der på kroppens systemer, herunder muskelaktivitet, hjerterytme og respiration. Ved biomekanisk biofeedback måles der på kroppens generelle motoriske egenskaber, eksempelvis hvordan kroppen bevæger sig. Studier har generelt vist positive resultater for biomekanisk biofeedback i forhold til rehabilitering af balanceproblemer.
For at biofeedback er en mulighed er det dog en forudsætning, at patientens kognitive evner er tilstrækkelige til at kunne blive instrueret og til at kunne huske de indlærte øvelser fra gang til gang. Derudover er der også visse krav til de neurologiske og motoriske evner. 

Både når der ses på de samfundsmæssige og personlige omkostninger er apopleksi en krævende sygdom, dels fordi sygdomsforløbet foregår i flere forskellige dele af sundhedssystemet og er den sygdom, der kræver flest plejedøgn. Derudover medfører sygdommen alvorlige fysiologiske og mentale følger. 
Man kan derfor undersøge muligheden for at begrænse både de samfundsmæssige og personlige omkostninger for patienter med balanceproblemer vha. et system baseret på biofeedback. Hvis der kan udvikles et system, der i højere grad tillader patienten at genoptræne sin balance selv, ved at gøre vedkommende opmærksom på eventuel hældning til siden, vil dette muligvis kunne begrænse patientens behov for at være i kontakt med sundhedspersonale, hvormed en normal dagligdag også hurtigere kan påbegyndes igen.  