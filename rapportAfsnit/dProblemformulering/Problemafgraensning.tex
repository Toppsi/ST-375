% !TeX spellcheck = da_DK
\section{Problemafgrænsning}
Apopleksi er en sygdom, der har stor indflydelse på blodtilførslen til encephalon. Hvis tilstrømningen af blod er nedsat, kan der opstå både motoriske og sensoriske skader hos patienten, hvilket kan komme til udtryk som balanceproblemer\cite{Hjernesagen2015a,Kruuse2015a}. Balancen er vigtig for at kunne fungere i hverdagen, da den sikrer, at kroppen holdes oprejst og muliggør bevægelse uden fald. \cite{Nichols1997} Apopleksipatienter med balanceproblemer oplever en begrænsning i deres hverdag, da de kan være afhængige af hjælp til daglige gøremål, som de før sygdommen selv kunne udføre\cite{Sundhedsstyrelsen2010}. De oplever det som et brud på deres tidligere liv, hvilket påvirker deres identitet og livskvalitet. \cite{Sundhedsstyrelsen2010} En hurtigere og/eller bedre rehabilitering kunne muligvis afhjælpe problemet.

\noindent For at begrænse de fysiske, og dermed også de personlige, følger mest muligt, er det essentielt, at rehabiliteringen påbegyndes hurtigt efter apopleksitilfældet \cite{Kruuse2015}. Indenfor rehabilitering af balance tilbydes forskellige metoder, såsom passiv somatosensorisk stimulation, balanceøvelser og biofeedback.\cite{Giggins2013} %En anden mulighed ift. rehabilitering af balancen er biofeedback. 
Studier viser positive resultater ved anvendelse af biomekanisk biofeedback, herunder inerti-sensorer, hvor der måles på kroppens generelle motoriske egenskaber. \cite{Giggins2013} For at biofeedback er en mulighed, er det en forudsætning, at patientens kognitive evner er tilstrækkelige til at kunne blive instrueret og huske de indlærte øvelser imellem træningssessionerne. \cite{Middaugh1989} Dette gør sig især gældende for den ældre del af befolkningen, som systemet skal designes til, da det er denne befolkningsgruppe, der i højere grad rammes af apopleksi. \cite{Sundhedsstyrelsen2011} \\
Det er interessant at undersøge, hvordan et system baseret på biomekanisk biofeedback kan designes således, at det vha. et accelerometer hjælper apopleksipatienter med at genoptræne deres balance. %Der kan benyttes forskellige komponenter til signalbehandling, men til behandling af accelerometrets signal er der valgt en forstærker, filtre, komparator samt ADC. \\
%Det er essentielt at undersøge, om systemet kan designes sådan, at det i højere grad tillader patienterne at bidrage til deres egen rehabilitering ved at benytte visuel, somatosensorisk og/eller audiotiv biofeedback. Det er muligt, at dette kan begrænse nogle af patienternes personlige følger, da kontakten med sundhedspersonale i forbindelse med rehabiliteringen kan begrænses, hvormed det normale hverdagsliv hurtigere kan genoptages. 

\section{Problemformulering}
Hvordan designes et biofeedbacksystem med et accelerometer således, at det hjælper apopleksipatienters balancetræning under rehabilitering?
%Hvordan designes et biofeedbacksystem med et accelerometer således, at det hjælper apopleksipatienter under rehabilitering af balancen?
%Hvordan designes et biofeedback system således, at det hjælper apopleksipatienter under rehabilitering af balancen?


%Apopleksi er den tredje største dødsårsag i Danmark \cite{Hjernesagen2015a}. 
%Af alle tilfælde af apopleksi er den iskæmiske langt den hyppigste med 80-85\% af tilfældene, imens den hæmoragiske udgør 10-15\%.
%Sygdommen har stor indvirkning på blodtilførslen til encephalon, som er det område af hjernen, der bla. rummer cerebrum. I cerebrum sker behandlingen af bla. sensoriske signaler, tanker og følelser. Når blodtilførslen til encephalon, og dermed også cerebrum, er nedsat, vil der kunne opstå både motoriske og sensoriske skader hos patienten.
%Når en patient med formodet apopleksi bliver indlagt, er det vigtigt, at lægen laver en grundig undersøgelse af de sensoriske og motoriske evner. Herefter foretages videre undersøgelser, eksempelvis CT- eller MR-scanning, afhængigt af den givne situation. \cite{Sundhedsstyrelsen2009} Det er vigtigt, at behandlingen igangsættes hurtigst muligt for at begrænse følgevirkningerne\cite{Soenderborg2013}. Der findes flere forskellige behandlingsmetoder. Ved iskæmi foregår behandlingen f.eks. med blodfortyndende medicin, en såkaldt trombolysebehandling, der har til formål at opløse blodproppen \cite{Hjernesagen2015b}. \fxnote{Mangler vi ikke noget om behandlingen af hæmoragi??}
%Apopleksi kan medføre sensoriske og motoriske skader. Disse skader kan have stor indvirkning på patientens liv efter sygdomsforløbet og en række psykiske konsekvenser og nedsat livskvalitet, bla. pga. sygdommens pludselige opståen. \cite{Muus2008}
%To eksempler på følger er balanceproblemer og neglekt. Balancen er vigtig for at kunne fungere i dagligdagen, da den sikrer at man holder kroppen oprejst og muliggør bevægelse uden fald. \cite{Nichols1997} Apopleksipatienter kan f.eks. lide af "Pusher Syndrom", hvor de ubevidst læner sig væk fra deres raske side. Dette påvirker balancen og kan lede til ulykker. \cite{Karnath2003} Ved neglekt er patienten ikke bevidst om den ene kropshalvdel. Dette kan være enten et visuelt eller et kropsligt problem. \cite{Sundhed.dk}
%Fælles for de to typer af følger er, at de begrænser patienterne i deres dagligdag og gør dem afhængige af hjælp til mange ting, som de før sygdommen selv kunne udføre. Derfor opleves ofte alvorlige personlige følger, eksempelvis problemer med at opretholde personlige relationer. Derudover kan patienternes identitet og humør også påvirkes. \cite{Sundhedsstyrelsen2010}
%For at begrænse de fysiske - og dermed også de personlige - følger mest muligt, er det essentielt at rehabiliteringen påbegyndes hurtigt efter apopleksien. Rehabiliteringen omfatter både genoptræning af fysiske funktioner, men også tilpasning til miljø og styrkelse af sociale kompetencer.\fxnote{Ingen kilde på dette i afsnittet..} 
%I Danmark er rehabiliteringen organiseret således, at de forskellige aktører i sundhedssektoren arbejder sammen på tværs af hinanden. Dette tværfaglige samarbejde er vigtigt, da apopleksipatienterne er i kontakt med mange forskellige dele af systemet under deres sygdomsforløb. \cite{Sundhedsstyrelsen2010} 
%Der findes i dag flere forskellige metoder til rehabilitering, afhængig af hvilken type følger patienten er ramt af. Metoderne er baseret på forskellige principper, herunder bla. træning vha. lyd, mekanisk træning, spejling og sensorisk stimulation\cite{Bradt2010,Mehrholz2013,Thieme2012,Sundhedsstyrelsen2010}.
%En anden måde at rehabilitere på er vha. biofeedback. Ved biofeedback måles på biologiske signaler, der relaterer til det område eller den funktion, som skal rehabiliteres hos patienten. Disse data omsættes til et signal, der kan opfattes af patienten. \cite{Giggins2013} Der skelnes mellem fysiologisk- og biomekanisk biofeedback\cite{Giggins2013}. Ved fysiologisk biofeedback måles der på kroppens systemer, herunder muskelaktivitet, hjerterytme og respiration. Ved biomekanisk biofeedback måles der på kroppens generelle motoriske egenskaber, eksempelvis hvordan kroppen bevæger sig. Studier har generelt vist positive resultater for biomekanisk biofeedback ift. rehabilitering af balanceproblemer. \cite{Giggins2013}
%For at biofeedback er en mulighed, er det en forudsætning, at patientens kognitive evner er tilstrækkelige til at kunne blive instrueret og kunne huske de indlærte øvelser fra gang til gang. Derudover er der visse krav til de neurologiske og motoriske evner. \cite{Middaugh1989}

%Både når der ses på de samfundsmæssige og personlige omkostninger, er apopleksi en krævende sygdom. Dels fordi sygdomsforløbet foregår i flere forskellige dele af sundhedssystemet og er den sygdom, der kræver flest plejedøgn. Derudover medfører sygdommen alvorlige fysiologiske og mentale følger.

%Det er derfor interessant at undersøge hvordan der kan udvikles et system baseret på biofeedback, der kan hjælpe patienter med at genoptræne deres balance. Det er relevant at undersøge om der kan designes et system, som i højere grad tillader patienterne at bidrage til deres egen rehabilitering. Det er muligt, at dette kan begrænse nogle af patienternes personlige følger, da kontakten med sundhedspersonale i forbindelse med rehabiliteringen evt. kan begrænses, hvormed det normale hverdagsliv hurtigere kan genoptages. 

%Man kan derfor undersøge muligheden for at begrænse både de samfundsmæssige og personlige omkostninger for patienter med balanceproblemer vha. et system baseret på biofeedback. Hvis der kan udvikles et system, der i højere grad tillader patienten at genoptræne sin balance selv, ved at gøre vedkommende opmærksom på eventuel hældning til siden, vil dette muligvis kunne begrænse patientens behov for at være i kontakt med sundhedspersonale, hvormed en normal dagligdag også hurtigere kan påbegyndes igen.  