\section{Problemafgrænsning}
Apopleksi er den tredje største dødsårsag i Danmark. \fxnote{Mere her - evt. fra indledning}
Af alle tilfælde af apopleksi er den iskæmiske langt den hyppigste med 80-85\% af tilfældene, imens den hæmoragiske udgør 10-15\%.
Apopleksi har stor indvirkning på blodtilførslen til encephalon, som er det område af hjernen der blandt andet rummer cerebrum. I cerebrum sker behandlingen af bl.a. sensoriske signaler, tanker og følelser. Når blodtilførslen til encephalon, og dermed også cerebrum, er nedsat, vil der kunne opstå både motoriske og sensoriske skader hos den ramte. 
Når en patient med formodet apopleksi bliver indlagt er det vigtigt, at lægen optager en grundig anamnese for at høre om sygdomsforløbet og forskellige faktorer, der kan have været medvirkende til udviklingen af sygdommen. Herefter udføres en klinisk undersøgelse af motoriske og sensoriske funktioner ud fra en valgt standardiseret skala. Efter den kliniske undersøgelse vil det være muligt for lægen at beslutte hvilke videre undersøgelser der bør laves, herunder f.eks. CT- eller MR-scanning, afhængigt af hvad der er mest optimalt i den givne situation. Det er vigtigt at behandlingen igangsættes hurtigst muligt for at begrænse følgevirkningerne. Der findes flere forskellige måder at behandle apopleksi på, afhængigt af om der er tale om iskæmisk eller hæmoragisk. Ved blodprop foregår behandlingen primært med blodfortyndende medicin, der har til formål at opløse blodproppen.
Apopleksi kan, som nævnt, medføre sensoriske og motoriske skader. Disse skader kan have stor indvirkning på patientens liv efter behandlingsforløbet og kan medføre en række psykiske konsekvenser og nedsat livskvalitet, bl.a. pga. sygdommens pludselige opståen.
To væsentlige følger er balanceproblemer og neglekt. Balancen er vigtig for at kunne fungere i dagligdagen, da den sikrer at man holder kroppen oprejst og muliggør bevægelse uden fald. Apopleksipatienter kan f.eks. lide af "Pusher Syndrom", hvor de ubevidst læner sig væk fra deres raske side. Dette påvirker balancen og kan lede til ulykker. Det kan således være vanskeligt at udføre daglige gøremål. Ved neglekt er patienten ikke bevidst om den ene kropshalvdel. Dette kan være enten et visuelt eller et kropsligt problem.
Fælles for de to typer af følger er, at de begrænser patienterne i deres dagligdag og gør dem afhængige af hjælp til mange ting, som de før sygdommen selv kunne udføre. Derfor opleves ofte alvorlige personlige følger, eksempelvis problemer med at opretholde personlige relationer. Derudover kan patienternes identitet og humør også påvirkes. 
For at begrænse de fysiske - og dermed også de personlige - følger mest muligt, er det essentielt at rehabiliteringen påbegyndes hurtigt efter apopleksien. Rehabiliteringen omfatter både genoptræning af fysiske funktioner, men også tilpasning til miljø og styrkelse af sociale kompetencer. 
I Danmark er rehabiliteringen organiseret således at de forskellige aktører i sundhedssektoren arbejder sammen på tværs af hinanden. Dette tværfaglige samarbejde er vigtigt, da apopleksipatienterne er i kontakt med mange forskellige dele af systemet under deres sygdomsforløb. 
Der findes idag flere forskellige metoder til rehabilitering, afhængig af hvilken type følger patienten er ramt af. Metoderne er baseret på forskellige principper, herunder bl.a. træning vha. lyd, mekanisk træning, spejling og sensorisk stimulation.
En anden måde at rehabilitere på er ved hjælp af biofeedback. Ved biofeedback måles på biologiske signaler, der relaterer til det område/eller den funktion som skal rehabiliteres hos patienten. Disse signaler      