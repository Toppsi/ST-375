\chapter{Træningsenhed Vest Aalborg Kommune}
I forbindelsen med projektperioden besøgte gruppen, Træningsenhed Vest Aalborg kommune onsdag d. 7 oktober 2015. Gruppen havde været i kontakt med fysioterapeut, Ingrid Hugaas, der havde udarbejdet et program indholdene de borgere gruppen skulle følge. Gruppen blev inddelt i to mindre gruppen. Der er blevet udarbejdet et referat af de observationer gruppen foretog den pågældende dag. Referatet er inddelt i to dele, altså observationer fra gruppe 1 og gruppe 2. 

\section{Observationer fra gruppe 1}
Gruppe 1 observerede en borger, der havde haft iskæmisk apopleksi og var næsten færdig med sit rehabiliteringsforløb. Borgeren havde især haft bivirkninger i sin venstre side, hvor grov- og finmotorikken i armen og hånden blev ramt. I starten sad borgeren i kørestol og kunne hverken åbne eller samle fingrene. Udover de fysiske følger blev det fortalt, at borgeren havde kognitive problemer og ventede på forløbsopstart på et Hjerneskadecenter. Borgeren trænede hovedsageligt maven og ryggen med træningsøvelser, såsom boldkast i yder positioner samt træning af  stående position og gangfunktionen. 
\begin{itemize}
\item Øvelsen med boldkast forløb således, at fysioterapeuten kastede en bold til borgeren i vedkommendes yderpunkter, i stående og siddende position. Idet borgeren skal ændre position for at gribe bolden udfordrer dette vedkommendes balancefunktion. 
\item Øvelsen i stående position blev foretaget på ujævne flader. Borgeren skulle skiftevis stå på forskellige typer af grundlag. Når borgeren blev introduceret for en ujævn flade påvirkede det vedkommendes balancefunktion og der blev observeret hurtige udfald i borgens position. 
\item Øvelsen med borgerens gangfunktion blev ligeledes, øvelsen i stående position, foretaget på ujævne flader, men derudover bar borgeren en bakke, så denne skærmede for hendes udsyn. Den visuelle sansefunktion var altså begrænset, hvilket påvirkede borgerens balacefunktion. 
\end{itemize}

Derudover observerede gruppe 1 en borger, der havde haft iskæmisk apopleksi i cerebellum. Borgeren højre side var ikke lige så funktionsdygtig som den venstre. Borgeren havde problemer med faldulykker, hovedsageligt grundet svimmelhed pga. manglende balancefunktion. 
_________2. Gang____________
John har haft en blodprop i lillehjernen, er på et højt niveau. Hans højre arm er ikke lige så meget i bevægelse, når han går.
At vente på 360 grader er svært, især for den ældre målgruppe.

Theodor - falder, nyt knæ, diabetes.
Hans ledsanser i knæet er
Han har vægten på sin svage side - højre side.
Han kan ikke fokusere på fingrene.  stagnes. 
Han går smalspore, tegn på dårlig balance. Karina ved ikke, om hun kan gøre det bedre for ham. Han skal have tjekket sit indre øre. Han skal lære at stimulere hjernen. 

Feedback afhænger af hvilken øvelse, der er gang i. Der er også reaktiv balance. Karina arbejder meget med hofterne. 

Forslår at vi kan se på Falck linjen, apopleksi afsnittet på sygehuset.

Gruppe 2:
Tager to test af statisk balance
En med åbne øjne 
En med lukket - dårligere balance her
Tager to mere hvor de står på en pude.
Det er sværere på puden!
Fysioterapeuten laver pejlinger af hvad patienten gør for at holde balancen.
Kunne slet ikke holde balancen med lukkede øjne på puden.
Apparatet skal være meget fintfølende og give meget hurtig respons.
Patienten mister balancen i alle retninger.

Laver tests hvor de går
Første test er at gå i normalt tempo
Anden test er at gå i normalt tempo med tempo skift i til hurtigt og langsomt
Tredje test er gå med hoved drejning 
Fjerde test gå i normalt tempo, hvor der skal ses op eller ned eller lige ud
femte test skal gå og vende om på et tidspunkt 
Sjette test gå i normal tempo hvor de skal træde over en pude
Syvende test gå i normalt tempo, hvor der skal gås højre og venstre om to kegler

Test hvor patienten skal gå op af en trappe
Først som normalt
Både op og ned

Test hvor patienten skal rejse sig fra en stol
Skal rejse og sætte sig med så mange gange som muligt på 30 sekunder 

Skal tage en seks minutters gang test skal gå sikkert men så hurtigt patienten kan