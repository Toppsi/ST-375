\chapter{Træningsenhed Vest Aalborg Kommune}
I forbindelsen med projektperioden besøgte gruppen, Træningsenhed Vest Aalborg kommune onsdag d. 7 oktober 2015. Gruppen havde været i kontakt med fysioterapeut, Ingrid Hugaas, der havde udarbejdet et program indholdene de borgere gruppen skulle følge. Gruppen blev inddelt i to mindre gruppen. Der er blevet udarbejdet et referat af de observationer gruppen foretog den pågældende dag. Referatet er inddelt i to dele, altså observationer fra gruppe 1 og gruppe 2. 

\section{Observationer fra gruppe 1}
Gruppe 1 observerede en borger, der havde haft iskæmisk apopleksi og var næsten færdig med sit rehabiliteringsforløb. Borgeren havde især haft bivirkninger i sin venstre side, hvor grov- og finmotorikken i armen og hånden blev ramt. I starten sad borgeren i kørestol og kunne hverken åbne eller samle fingrene. Udover de fysiske følger blev det fortalt, at borgeren havde kognitive problemer og ventede på forløbsopstart på et Hjerneskadecenter. Borgeren trænede hovedsageligt maven og ryggen med træningsøvelser, såsom boldkast i yder positioner samt træning af  stående position og gangfunktionen. 
\begin{itemize}
\item Øvelsen med boldkast forløb således, at fysioterapeuten kastede en bold til borgeren i vedkommendes yderpunkter, i stående og siddende position. Idet borgeren skal ændre position for at gribe bolden udfordrer dette vedkommendes balancefunktion. 
\item Øvelsen i stående position blev foretaget på ujævne flader. Borgeren skulle skiftevis stå på forskellige typer af grundlag. Når borgeren blev introduceret for en ujævn flade påvirkede det vedkommendes balancefunktion og der blev observeret hurtige udfald i borgens position. 
\item Øvelsen med borgerens gangfunktion blev ligeledes øvelsen i stående position, foretaget på ujævne flader, men derudover bar borgeren en bakke, så denne skærmede for hendes udsyn. Den visuelle sansefunktion var altså begrænset, hvilket påvirkede borgerens balacefunktion. 
\end{itemize}

Derudover observerede gruppe 1 en borger, der havde haft iskæmisk apopleksi i cerebellum. Borgeren højre side var ikke lige så funktionsdygtig som den venstre. Borgeren havde problemer med faldulykker, hovedsageligt grundet svimmelhed pga. manglende balancefunktion. Borgeren vurderes at være i en af de tidlige faser af rehabiliteringsperioden, eftersom træningsøvelserne til denne borger var simple. Fokus i disse øvelser var at træne borgerens statiske balance på et jævnt grundlag, hvor vedkommende fik støtte fra fysioterapeuten. 

\section{Observationer fra gruppe 2}
Gruppe 2 observerede forskellige øvelser af balancefunktionen ved forskellige borgere. Der blev foretaget foreskellige øvelser, der hhv. trænede statisk balance i siddende og stående position, samt gangfunktionen på et jævn grundlag og derefter gangfunktion op ad en trappe. 
\begin{itemize}
\item Af den statiske balance i stående position blev der foretaget to forskellige typer af øvelser. Borgeren skulle stå med fødderne på en tegnet linje, så den ene fods tæer var placeret mod den anden fods hæl og armene holdt tæt ind til kroppen og over kors. For at øge sværhedsgraden bedte fysioterapeuten borgeren om at lukke øjnene, hvilket påvirkede borgen balance. Derudover blev borgen bedt om at stille sig på en pude, så grundlaget blev ujævnt. Her blev der igen foretaget to øvelser, hvor borgeren først havde sine øjne åbnede og derefter lukkede. Borgeren kunne ved denne øvelse slet ikke holde balance og mistede sin balancefunktion i alle retninger. 
\item Af den statiske balance i siddende position skulle borgerne træne det at sidde stabilt i en stol samt rejse sig fra stolen. Borgerne skulle på 30 sekunder rejse og sætte sig så mange gange det er muligt. 
\item Borgernes trænede deres balance under gang ved først at gå i normalt tempo og derefter skiftevis normalt og hurtigt tempo. Borgeren skulle herefter dreje hovedet fra side til side samt op og ned imens gang. For at udfordre balancefunktionen i højere grad, skulle borgeren på fysioterapeutens anvisning træde over en pude imens gang og herefter gå højre og venstre om to kegler.
\item Borgerne trænede det at gå op af en krappe. Gruppe 2 observerede at dette udfordrede balancefunktionen, hvilket også kan hænge sammen med at...
\end{itemize} 

