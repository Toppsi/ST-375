\chapter{Referat fra observation hos Træningsenhed Vest Aalborg Kommune}\label{Ref_observation}
I forbindelsen med projektperioden besøgte gruppen Træningsenhed Vest Aalborg Kommune onsdag d. 7 oktober 2015. Gruppen havde været i kontakt med fysioterapeut, Ingrid Hugaas, som havde udarbejdet et ca. 3 timers program, herunder hvilke borgere gruppen skulle følge. Gruppen blev inddelt i to mindre grupper. Der er blevet udarbejdet et referat af de observationer, gruppen foretog den pågældende dag. Referatet er inddelt i to dele, altså observationer fra hver undergruppe. 

\section{Observationer fra den første gruppe}
Gruppen observerede en borger, der havde haft iskæmisk apopleksi og var næsten færdig med sit rehabiliteringsforløb. Borgeren havde haft sequela \fxnote{NTK:bivirkning} i sin venstre side, hvor grov- og finmotorikken i armen og hånden blev ramt. I starten sad borgeren i kørestol og kunne hverken åbne eller samle fingrene. Udover de fysiske følger blev det fortalt, at borgeren havde kognitive problemer og ventede på forløbsopstart på et hjerneskadecenter. Borgeren trænede hovedsagligt truncus med træningsøvelser, såsom boldkast i yder positioner samt træning af stående position og gangfunktionen. 
\begin{itemize}
\item Øvelsen med boldkast forløb således, at fysioterapeuten kastede en bold til borgeren i vedkommendes yderpunkter i stående og siddende position. Idet borgeren skulle ændre position for at gribe bolden, udfordrede dette vedkommendes balancefunktion. 
\item Øvelsen i stående position blev foretaget på ujævne flader. Borgeren skulle skiftevis stå på forskellige typer af underlag. Når borgeren blev præsenteret for en ujævn flade, påvirkede det vedkommendes balancefunktion, og der blev observeret hurtige udfald i borgerens position. 
\item Øvelsen i borgerens gangfunktion blev ligeledes foretaget i stående position og på ujævne flader, men derudover bar borgeren en bakke, så denne skærmede for borgerens udsyn. Den visuelle sansefunktion var altså begrænset, hvilket påvirkede borgerens balancefunktion. 
\end{itemize}
\noindent Derudover observerede gruppe et en anden borger, der havde haft iskæmisk apopleksi i cerebellum. Borgerens højre side var ikke lige så funktionsdygtig som den venstre. Borgeren havde problemer med fald gentagne gange, hovedsageligt grundet svimmelhed og nedsat balance og vurderes at være i en af de tidlige faser af rehabiliteringen pga. den dårlige balancefunktion. Fokus i disse øvelser var at træne borgerens statiske balance på et jævnt underlag, hvor vedkommende fik støtte fra fysioterapeuten. 

\section{Observationer fra  den anden gruppe}
Gruppen observerede forskellige øvelser af balancefunktionen ved forskellige borgere. Der blev foretaget forskellige øvelser, der hhv. trænede statisk balance i siddende og stående position, samt gangfunktionen på et jævnt underlag og derefter gangfunktion op og ned ad en trappe. 
\begin{itemize}
\item Af den statiske balance i stående position blev der foretaget to forskellige typer af øvelser. Borgeren skulle stå med fødderne på en tegnet linje, så den ene fods tæer var placeret mod den anden fods hæl og armene holdt over kors tæt ind til kroppen. For at øge sværhedsgraden bedte fysioterapeuten borgeren om at lukke øjnene, hvilket påvirkede borgerens balance. Derudover blev borgeren bedt om at stille sig på en pude, så underlaget blev ujævnt. Her blev der igen foretaget to øvelser, hvor borgeren først havde sine øjne åbne og derefter lukkede. Borgeren kunne ved denne øvelse ikke opretholde balancen og gav derfor udslag. 
\item For at teste muskelstyrken i underekstremiteten og dermed en prædiktor for fald, skulle borgerne træne at sidde stabilt i en stol samt rejse sig fra siddende til stående position. Borgerne skulle på $30$ sekunder rejse og sætte sig så mange gange som muligt. Denne test kaldes også "chair-stand-test" eller "RSS30" og er en 
\item Borgerne trænede deres balance under gang ved først at gå i normalt tempo og derefter skiftevis normalt og hurtigt tempo. Borgeren skulle herefter dreje hovedet fra side til side samt op og ned imens gang. For at udfordre balancen i højere grad skulle borgeren på fysioterapeutens anvisning træde over en pude imens gang og herefter gå højre og venstre om to kegler.
\item Borgerne trænede balancen ved at gå op og ned af en trappe. Gruppe to observerede at dette udfordrede balancefunktionen mest ved at gå ned ad trappen. Dette kan hænge sammen med at borgeren blev mere balanceusikker ned ad trappen, da balancen på dette tidspunkt vil blive mere udsat pga. fordelingen af kropsvægten \fxnote{Ingrids kommentar: Lidt i tvivl om hvad der menes med fordeling af kropsvægten – er det væsentlig anderledes ift. at gå opad trappen? Er det ift. tyngdekraftens påvirkning på legemet? Ift. center of mass over base of support?}. 
\end{itemize} 