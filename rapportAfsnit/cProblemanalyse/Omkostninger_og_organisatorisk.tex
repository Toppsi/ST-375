\section{Samfundsmæssige omkostninger}
I Danmark opstår 12.000 tilfælde af apopleksi årligt. Apopleksi er den sygdom, hvor der kræves flest plejedøgn i sundhedssektoren [1]. %Uddyb - hvad er et plejedøgn, hvor mange kræver andre syge ? - jeg havde svært ved at finde en difinition på det.
Omkostningerne for patienter med følger af slagstilfælde (%apopleksi) 
er både direkte og indirekte omkostninger. De direkte omkostninger er udgifter til medicinering, sundhedsomkostninger i form af behandling i hospitalsvæsnet, praktiserende læge og speciallæge, samt hospitalsindlæggelse og rehabilitering både fra et samfundsmæssigt, regionalt og kommunalt perspektiv[2]. De indirekte omkostninger  består af udgifter for tabt arbejdsfortjeneste, tabt produktivitet og ekspertise for samfundet som følge af fravær, tidsforbrug, fremtidige relaterende omkostninger og sundhedsomkostninger til personer med hjerneskade ved forlænget levetid[2].

\subsection{Samlede omkostnigner}
% (slet) Hvis der ses på omkostninger for de samlede hjerneskadede, vil omkostningerne for en patient efter hjerneskaden indtræden på en 5 årig periode fra 2005-2009 ligge på sundhedsomkostninger for gennemsnitlig 27.200 kroner i aldersgruppen 18+. Det er vanskeligt at vurdere, hvilken betydning denne omkostningen vil have, da omkostningerne er beregnet som en sum af Diagnose Relaterede Grupper(DRG)-omkostninger, hvilket vil sige at omkostningerne er opgjort efter forskellige takstsystemer [2].

De samlede sundhedsomkostninger i 2008 var 110.000 kroner, hvilket inkluderer incidens-år og året efter. Omkostninger uden for sundhedssektoren, dvs. kommunal rehabilitering ligger på omkring 60.00 kroner pr. patientforløb. I følge sundhedsstyrelsen bærer kommunen ca. 35\% af de offentlige konsekvenser af hjerneskade og hjerneskaderehabilitering rent økonomisk. [2]    

% (slet) Ligeledes er der ikke en konkret overgang fra indlæggelse til påbegyndelse af rehabiliteringsforløbet, da rehabiliteringen forekommer ved indlæggelse (%og derved har en glidende overgang). Den samlede overgang omtales hjerneskaderehabilitering %(forstår ikke helt, hvad der menes med samlede overgang). Det antages, at de kommunale omkostninger ligger gennemsnitlig på 60.000 kr. pr. forløb regnet i 2008-prisniveau [2]. Derudover er der de første to år af behandingsomkostninger i forhold til hjernebehandlings- og rehabiliteringsforløbet en omkostning på 110.000 kr. [2].

Alt efter sværhedsgraden af hjerneskaden kan produktivitetstab forekomme hos patienten. Dette betyder, at patienterne som rammes af et slagstilfælde %(apopleksi)
 i nogle tilfældes er nødsaget til at modtage indkomsterstattende ydelser, såsom arbejdsløshedspenge, sygedagpenge, førtidspension, efterløn osv. En del af patienterne er dog allerede på indkomsterstattende ydelser i form af pension, da alders gennemsnittet for patienter med hjerneskade er gennemsnitlig 62,4 år. Produktivitetstabet formodes at koste samfundet 50.000 kroner årligt, hvilket er beregnet ud fra populationen fra 2004. 
%Der vil dog være en del af patienterne som i forvejen er på indkomsterstattende ydelser (findes der ikke et andet ord for det?) i form af pension, hvilket afspejles i at alders gennemsnittet for patienter med hjerneskade er 62,4 år[2]. Produktivitivtetstabet vil for en hjerneskadet patient i det samlede koste samfundet 50.000 kr. årligt, for de første seks år regnet i 2008-prisniveau[2]. 

% (kan det evt. slettes??) Omkostningerne for både den primære sektor dvs. kommunen og sygehusomkostningerne forskellige. Rehabiliteringen vil for den primære sektor i gennemsnit ligge på 600 kr i perioden 2008-2009 og inden for sygehusomkostninger på 3.200 kr i samme periode, begge for patienter med hjerneskade beregnet i 2008-prisniveau [2].
% mangler en forklaring på, hvad den sekundære sektor er.

I alt vil omkostningerne for hjerneskaderehabilitering og rehabilitering være 270.000 kroner i en to-årig periode bereget ud fra tal fra 2008[2]. De samlede udgifter for patienter med apopleksi udgør 4\% af sundhedsvæsenets samlede udgifter, her direkte udgifter estimeret til 2.7 milliard kroner om året[3]. 
Patienter med apopleksi er altså omkostningsfuld for samfundet, da der som tidligere nævnt blandt andet kræves en del plejedøgn på grund af funktionsnedsættelse, samt udgifter til rehabilitering og pruduktivitetstab.
%Patienter som får en hjerneskade er omkostningsfuld for samfundet, da der som tidligere nævnt bl.a. kræves en del plejedøgn, da patienterne vil have alvorlige følger som vil kunne forårsage nedsat livsfunktioner. Dette vil foruden behandling kræve rehabilitering og kunne medfører produktivitetstab, hvilket yderligere kan være en omkostning, hvis patienterne har følger af hjerneskaden på længere sigt.


\section{Organisatorisk}
I sundhedssektoren arbejder de forskellige dele af organisationen på tværs af hinanden, så der skal være et samarbejde mellem syghuse, kommuner og praktiserende læger. Dette samarbejde skal ske både internt på syghusene, på afdelingerne og kommunalt mellem forvaltningerne [2]. Dette sammenspil mellem de ovennævnte aktører er vigtigt, da hjerneskadede berører flere afdelinger og derfor har brug for involvering af flere sundhedsprofessionelle grupper under sin behandling og rehabilitering på grund af de omfattende og alvorlige konsekvenser.
 
%I sundhedssektoren arbejder de forskellige dele af organisationen på tværs af hinanden, hvilket vil sige, at der skal være et samarbejde mellem sygehuse, kommuner og praktiserende læger. Dette samarbejde skal ske både internt på sygehusene, afdelingerne i mellem og kommunalt, mellem forvaltningerne [2]. Dette sammenspil mellem de ovennævnte aktører er vigtigt, da hjerneskadede kræver involvering af flere sundhedsprofessionelle grupper, da der er omfattende og alvorlige konsekvenser ved tilfælde af apopleksi for både patienter, så vel som for pårørende. 

De ovennævnte aktører er de organisatoriske enheder, der har en central rolle i forløbet. Det er ikke muligt at fastlægge en egentlig organisering af hjerneskaderehabiliteringen i Danmark, da sammenspillet mellem de forskellige aktører er meget %flydende (?) 
og forskellige alt efter hvor i landet man befinder sig og hvor omfattende hjerneskaden er. %Denne forskel opleves regionalt, hvor behandlings- og rehabilitering (mangler der ikke noget - behandlings-hvad) nogle steder i landet foregår på få af sygehusets afdelinger, mens patienter andre steder behandles på et rehabiliteringssygehus, efter den akutte behandling er foretaget[2]. 

Denne forskel opleves regionalt, hvor behandling og rehabilitering enkelte steder foregår på få af sygehusets afdelinger, mens patienter andre steder behandles på et rehabiliteringssygehus, efter den akutte behandling er foretaget[2]. 

I starten af behandlingssforløbet sendes patienterne til neurologiske, geriatriske, neurokirurgiske og medicinske afdelinger på sygehuset[2]. Som tidligere nævnt inddeles patienterne efter sværhedsgrad af hjerneskaden, hvor de sværest ramte, som er patienter med traumatisk hjerneskade og tilgrænsede lidelser, henvendes til Hammel og Hvidovre. Rehabiliteringen kan også ske på rehabiliteringsafsnittet på sygehusene i dele af landet[2].

%Patienterne sendes som regel til den første del af behandlingen hos neurologiske, geriatriske, neurokirurgiske og medicinske afdelinger på sygehuset[2]. Som tidligere nævnt inddeles patienterne efter sværhedsgrad af hjerneskaden, hvor de sværest ramte, som er patienter med traumatisk hjerneskade og tilgrænsede lidelser, henvendes til Hammel og Hvidovre. Rehabiliteringen kan også ske på rehabiliteringsafsnittet på sygehusene i dele af landet[2]. 
% Gøre det mere synligt at behandlingerne er forskellige imellem kommunerne - jeg synes altså det er beskrevet ok i afsnittet over. Så skal man skrive sådan noget som "som nævn ovenfor" - hvilket virker mærkeligt.

Det primære ansvar ligger hos kommunerne i form af genoptræningsplanens afdækning af rehabiliteringsbehov, dvs. at kommunerne holder øje med om dette foregår i praksis, herunder bl.a. patientens genoptræningsbehov. Kommunerne har derudover mulighed for at henvise patienterne til egne tilbud, eller henvise til private[2]. %forstår ikke helt funktionen af dette afsnit - forstår det nok generelt ikke. 

Afslutningsvis gennemgår patienterne et langt og forskelligt behandlingsforløb alt efter hvilken grad hjerneskaden er, som indebærer et samarbejde mellem de forskellige aktører. Efter behandlingen står, som nævnt før, kommunerne for det primære ansvar i forhold til rehabilitering og henvisning for patienten. % Henvisning til hvad?
  
%Patienter vil skulle gennemgå et langt og forskelligt forløb alt efter hvilken grad hjerneskaden har været, dette indebærer et samarbejde mellem de forskellige aktører, som har en flydende overgang mellem hinanden. Efter behandlingen på de forskellige afdelinger og sygehus står, som tidligere nævnt, kommunerne for den primære ansvar i forhold til rehabiliteringen og henvisninger for patienten.

%[1]https://www.sundhed.dk/borger/sygdomme-a-aa/hjerte-og-blodkar/sygdomme/apopleksi/apopleksi-blodprop-eller-bloedning-i-hjernen/
%[2]http://sundhedsstyrelsen.dk/~/media/CB8CCFE77832456C8B1BABF2F558A661.ashx
%[3]http://www.hjernesagen.dk/om-hjerneskader/bloedning-eller-blodprop-i-hjernen/fakta-om-apopleksi

% Mangler en overskuelig tabel over alle de tal, som bliver nævnt. 
% Tallene er roddet. Man kunne lave en delkonklusion, hvis en tabel ikke tydeliggøre det nok.
% Mangler nogle underoverskrifter, så det bliver mere overskueligt.
% Omtales er et dårligt ord - vi skal være definitive i den måde vi skriver på. Vi har kilder på det, så det er det vi tror på.
% Tjek hele dokumentet for "da". Der er nogle sætninger, hvor det findes 2 gange i.
% Er det dyrt for en patient? Man kunne tage ift. andre patienter. Konkluder på tallene.

%kommentar til tekst: Jeg har valgt at slette mange tal, da jeg mente, at budskabet stadigvæk kommer frem uden. Der er derfor ikke så mange tal mere, hvilket er grunden til, at jeg ikke har lavet en tabel. Jeg har prøvet at formulere en konklusion på afsnittet omkostninger, men jeg har haft svært ved at finde apopleksi ift noget. Altså hvor dyr er denne sygdom ift andre sygdomme. Dette kunne måske godt mangle - men jeg har altså ikke kunne finde nogle kilder på det. 