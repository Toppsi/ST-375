\section{Samfundsmæssige omkostninger}
I Danmark opstår 12.000 tilfælde af apopleksi årligt. Apopleksi er den sygdom, hvor der kræves flest plejedøgn i sundhedssektoren [1]. %Uddyb - hvad er et plejedøgn, hvor mange kræver andre syge?
Omkostningerne for patienter med følger af slagstilfælde er både direkte og indirekte omkostninger. De direkte omkostninger er udgifter til medicinering, sundhedsomkostninger ved behandling i hospitalsvæsnet, praktiserende læge og speciallæge, hospitalsindlæggelse og rehabilitering både fra et samfundsmæssigt, regionalt og kommunalt perspektiv[2]. De indirekte omkostninger består af udgifter for tabt arbejdsfortjeneste, tabt produktivitet og ekspertise for samfundet som følge af fravær, tidsforbrug, fremtidige relaterende omkostninger og sundhedsomkostninger til personer med hjerneskade ved forlænget levetid[2].

Hvis der ses på omkostninger for de samlede hjerneskadede, vil omkostningerne for en patient efter hjerneskaden indtræden på en 5 årig periode fra 2005-2009 ligge på sundhedsomkostninger for gennemsnitlig 27.200 kroner i aldersgruppen 18+. Det er vanskeligt at vurdere, hvilken betydning denne omkostningen vil have, da omkostningerne er beregnet som en sum af Diagnose Relaterede Grupper(DRG)-omkostninger, hvilket vil sige at omkostningerne er opgjort efter forskellige takstsystemer [2].
Ligeledes er der ikke en konkret overgangen fra indlæggelse til at rehabiliteringsforløbet igangsættes, da rehabiliteringen sker ved indlæggelse og derved har en glidende overgang. Denne samlede overgang omtales hjerneskaderehabilitering. Det antages at de kommunale omkostninger ligger gennemsnitlig på 60.000 kr. pr. forløb regnet i 2008-prisniveau [2]. Derudover er der de første to år af behandingsomkostninger i forhold til hjernebehandlings- og rehabiliteringsforløbet en omkostning på 110.000 kr. [2].

Alt efter sværhedsgraden af hjerneskaden vil der kunne forekomme et produktivitetstab, hvilket vil sige at patienterne som rammes af et slagstilfælde i nogle tilfældes ville være nødsaget til at modtage indkomsterstattende ydelser, såsom arbejdsløshedspenge, sygedagpenge, førtidspension, efterløn osv. pga. den tabte arbejdsfunktion. Der vil dog være en del af patienterne som i forvejen er på indkomsterstattende ydelser i form af pension, hvilket afspejles i at alders gennemsnittet for patienter med hjerneskade er 62,4 år[2]. Produktivitivtetstabet vil for en hjerneskadet patient i det samlede koste samfundet 50.000 kr. årligt, for de første seks år regnet i 2008-prisniveau[2]. 

Omkostningerne for både den primære sektor dvs. kommunen og sygehusomkostningerne forskellige. Rehabiliteringen vil for den primære sektor i gennemsnit ligge på 600 kr i perioden 2008-2009 og inden for sygehusomkostninger på 3.200 kr i samme periode, begge for patienter med hjerneskade beregnet i 2008-prisniveau [2].
% mangler en forklaring på, hvad den sekundære sektor er.

Patienter som får en hjerneskade er omkostningsfuld for samfundet, da der som tidligere nævnt bl.a. kræves en del plejedøgn, da patienterne vil have alvorlige følger som vil kunne forårsage nedsat livsfunktioner. Dette vil foruden behandling kræve rehabilitering og kunne medfører produktivitetstab, hvilket yderligere kan være en omkostning, hvis patienterne har følger af hjerneskaden på længere sigt.


\section{Organisatorisk}
I sundhedssektoren arbejder de forskellige dele af organisationen på tværs af hinanden, hvilket vil sige, at der skal være et samarbejde mellem sygehuse, kommuner og praktiserende læger. Dette samarbejde skal ske både internt på sygehusene, afdelingerne i mellem og kommunalt, mellem forvaltningerne [2]. Dette sammenspil mellem de ovennævnte aktører er vigtigt, da hjerneskadede kræver involvering af flere sundhedsprofessionelle grupper, da der er omfattende og alvorlige konsekvenser ved tilfælde af apopleksi for både patienter, så vel som for pårørende. 

De ovennævnte aktører er de organisatoriske enheder der har den centrale rolle i forløbet. Det er ikke muligt at fastlægge en reel/endelig/egentlig(ORD?) organisering af hjerneskaderehabiliteringen i Danmark, da sammenspillet mellem de forskellige aktører er meget flydende og forskellige alt efter hvor i landet man befinder sig og hvor omfattende hjerneskaden er. Denne forskel opleves regionalt, hvor behandlings-og rehabilitering nogle steder i landet foregår på få af sygehusets afdelinger, mens patienter andre steder behandles på et rehabiliteringssygehus efter den akutte behandling er foretaget[2]. 

Patienterne sendes som regel til den første del af behandlingen hos neurologiske, geriatriske, neurokirurgiske og medicinske afdelinger på sygehuset[2]. Som tidligere nævnt inddeles patienterne  efter sværhedsgrad af hjerneskaden, hvor de sværest ramte, som er patienter med traumatisk hjerneskade og tilgrænsede lidelser, henvendes til Hammel og Hvidovre. Rehabiliteringen kan også ske på rehabiliteringsafsnit på sygehusene i dele af landet[2]. 
% Gøre det mere synligt at behandlingerne er forskellige imellem kommunerne.

Det primære ansvar ligger hos kommunerne i form af genoptræningsplanens afdækning af rehabiliteringsbehov, dette vil sige at kommunerne holder øje med om dette foregår i praksis, herunder bl.a. patientens genoptræningsbehov. Kommunerne har derudover mulighed for at henvise patienterne til egne tilbud, eller henvise til private[2]. 

Patienter vil skulle gennemgå et langt og forskelligt forløb alt efter hvilken grad hjerneskaden har været, dette indebærer et samarbejde mellem de forskellige aktører, som har en flydende overgang mellem hinanden. Efter behandlingen på de forskellige afdelinger og sygehus står, som tidligere nævnt, kommunerne for den primære ansvar i forhold til rehabiliteringen og henvisninger for patienten. 

%[1]https://www.sundhed.dk/borger/sygdomme-a-aa/hjerte-og-blodkar/sygdomme/apopleksi/apopleksi-blodprop-eller-bloedning-i-hjernen/
%[2]http://sundhedsstyrelsen.dk/~/media/CB8CCFE77832456C8B1BABF2F558A661.ashx

% Mangler en overskuelig tabel over alle de tal, som bliver nævnt. 
% Tallene er roddet. Man kunne lave en delkonklusion, hvis en tabel ikke tydeliggøre det nok.
% Mangler nogle underoverskrifter, så det bliver mere overskueligt.
% Omtales er et dårligt ord - vi skal være definitive i den måde vi skriver på. Vi har kilder på det, så det er det vi tror på.
% Tjek hele dokumentet for "da". Der er nogle sætninger, hvor det findes 2 gange i.
% Er det dyrt for en patient? Man kunne tage ift. andre patienter. Konkluder på tallene.