% !TeX spellcheck = da_DK
\section{Diagnosticering}

Når en patient med apopleksi indlægges, er grundig undersøgelse nødvendig for at identificere, hvilken form for apopleksi patienten har. Dette trin er afgørende for det efterfølgende forløb, da behandling samt rehabilitering planlægges herefter. \cite{Sundhedsstyrelsen2009}

\subsection{Anamnese}
For at blive bekendt med patientens egen subjektive vurdering af det hidtidige sygdomsforløb, optager lægen en anamnese. Her skal sygdomsforløbet  beskrives detaljeret, og der skal desuden spørges ind til faktorer, som kan have været medvirkende til udviklingen af apopleksi såsom livsstil og sygdomme, herunder f.eks. diabetes og hjerteproblemer. \cite{Sundhedsstyrelsen2009}

\subsection{Klinisk undersøgelse}
Den kliniske undersøgelse udføres for at vurdere, hvor alvorligt sygdomstilfældet er. Undersøgelsen udføres ud fra en standardiseret skala, da dette gør det muligt for andre læger, at undersøge patienten på samme måde senere i forløbet. Resultatet af undersøgelsen er en samlet score udregnet fra resultatet af de enkelte undersøgelser. Det er således muligt at vurdere om der sker fremskridt hver gang undersøgelsen foretages. \cite{Sundhedsstyrelsen2009}

Der findes flere forskellige skalaer, som lægen kan anvende til at foretage den kliniske undersøgelse, herunder Scandinavian Stroke Scale, European Stroke Scale og Hemisperic Stroke Scale. Fælles for skalaerne er, at der bliver undersøgt både bevidsthedsniveau samt kognitive og motoriske egenskaber hos patienten \cite{Center, Centera, Centerb, Centerc}. I den danske sundhedssektor benyttes Scandinavian Stroke Scale \cite{Apopleksi2009}\fxnote{Hvilke kritierier vælger man ud fra . Hvorfor anvendes denne i Danmark - står muligvis i kilden}. F.eks. har Region Syddanmark derudover forskellige retningslinjer for, hvilken skala der skal benyttes i et apopleksi forløb \cite{Syddanmark}. Alle apopleksipatienter, der bliver indlagt med mulig akut apopleksi eller TCI, skal have en score på Scandinavian Stroke Scale. Herefter benyttes National Institute of Health Stroke Scale, hvis patienten skal have trombolysebehandling. Barthel Scale anvendes, hvis patienten sendes til videre rehabilitering og beskriver patientens funktionsniveau ift. almindelige dagligdags funktioner. Til sidst benyttes Modificeret Rankin Scale til at give en beskrivelse af graden af handicap. \cite{Syddanmark}

\subsection{Videre undersøgelser}
Ved den videre undersøgelse vil der udføres en scanning for at undersøge, om patienten er ramt af iskæmisk apopleksi eller hæmoragisk apopleksi. Scanningen laves desuden for at lokalisere det ramte område. Enten udføres der en scanning af typen CT eller MR afhængigt af, hvad der er mest hensigtsmæssigt i den givne situation. \cite{Sundhedsstyrelsen2009} %CT-scanning udsender røntgenstråligen, hvilket optages i vævet på forskellige måder. Udfra dette beregner en computer tværsnitsbilleder af kroppens indre \fxnote{https://www.cancer.dk/hjaelp-viden/undersoegelser-for-kraeft/scanninger-billedundersoegelser/ct-scanning/}. Modsat anvendes der ved MR-scanning et kraftigt magnetfelt som sender radiobølger ind i kroppen. Derved registres et ekko og computeren kan derefter beregne et detaljeret billede af kroppens indre organer. 
CT-scanning anvendes f.eks. til undersøgelse af åreforkalkning og indre blødninger, hvor MR-scanning bruges til at undersøge sygdomme i nervesystemet f.eks. i encephalon \cite{Hansen2015,Ammundsen2015}.\fxnote{Kan man finde et andet eksempel på MR? Man kan indlede sætning med CT med, at det er den, der burges hyppigst. - Sofies kilde fra Facebbok "opslagsvæk fagudtrykl".}\\
Det skal derfor vurderes, hvilken form for scanning der skal anvendes ud fra forskellige kriterier, f.eks. lægens mistanke om, hvilket område af encephalon, der er ramt, samt hvor længe symptomerne på apopleksi har optrådt. I visse tilfælde kan lægen vælge at anvende begge scanningstyper.  
Derudover skal patientens blodtryk måles jævnligt i den akutte fase for at sikre, at det falder gradvist til et normalt niveau i løbet af nogle timer til et døgn. Hvis blodtrykket pludselig falder meget, kan dette være et udtryk for en blodprop i hjertet.\fxnote{Mangler en forklaring - hvordan kan apopleksi være årsag til en blodprop i hjertet?} Det er derfor afgørende at følge udviklingen med jævnlige blodtryksmålinger. \cite{Sundhedsstyrelsen2009}
\\
Under forløbet bør andre faktorer også kontrolleres, herunder lungefunktion, blodglukose og kropstemperatur. Disse faktorer kan enten give information om apopleksien eller være væsentlige for patientens fremtidsprognoser og følger efter sygdomsforløbet. \cite{Sundhedsstyrelsen2009}
\\

% [1] Sundhedsstyrelsens rapport
% [2]http://www.strokecenter.org/professionals/stroke-diagnosis/stroke-assessment-scales/
% [3]http://www.strokecenter.org/wp-content/uploads/2011/08/hemispheric.pdf
% [4]http://www.strokecenter.org/wp-content/uploads/2011/08/scandinavian.pdf
% [5]http://www.strokecenter.org/wp-content/uploads/2011/08/European_Stroke_Scale.pdf
% [6]http://www.dsks.dk/filer/hoeringssvar/referenceprogram_for_behandling_af_patienter_med_apopleksi.pdf
% [7] http://ekstern.infonet.regionsyddanmark.dk/Files/dokument90214.htm
% [8] https://www.sundhed.dk/borger/sygdomme-a-aa/hjerte-og-blodkar/sygdomme/apopleksi/apopleksi-blodprop-eller-bloedning-i-hjernen/
% [9] http://academic.eb.com.zorac.aub.aau.dk/EBchecked/topic/569347/stroke
% [10] http://www.netdoktor.dk/sygdomme/fakta/blodprophjerne.htm