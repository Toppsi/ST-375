%\section{Apopleksi}
%Når en person får apopleksi, kan symptomerne vise sig meget hurtigt. Der kan gå et par minutter, men der kan også gå op til et par dage [8 ,9]. Symptomerne der viser sig hos de ramte kan have meget alvorlige konsekvenser. Det kan manifesterer sig som, lammelse af store dele af kroppen og manglende evne til at snakke. Derudover kan apopleksi i alvorlige tilfælde føre til koma, og i værste fald død [9]. Symptomerne er ikke altid så voldsomme, men kan forekomme som en række transient ischemic attacks (Tror nok det er TCI). Symptomerne på disse kan være en følelsesløshed i lemmerne eller i ansigtet samt korte oplevelser af forvirring, synsforstyrrelser og forstyrrelse i tale. Disse vil komme ofte, men leder ofte ud til en større lammelse af en del af kroppen [9].
%Hvis en person oplever nogle af symptomerne, skal der ringes efter en ambulance hurtigst muligt, da chancen for at overleve forøges ved hurtig behandling [10]. 
% Dette kunne godt skrives sammen med "apopleksi" afsnittet.

\section{Diagnosticering}

Når en patient med akut apopleksi indlægges, er grundig undersøgelse nødvendig for at identificere, hvilken form for apopleksi der er tale om. Dette trin er afgørende for det efterfølgende forløb, da behandling samt rehabilitering planlægges herefter. [1]

\subsection{Anamnese}
For at blive bekendt med patientens egen subjektive vurdering af det hidtidige sygdomsforløb, optager lægen en anamnese. Her skal sygdomsforløbet  beskrives detaljeret, og der skal desuden spørges ind til faktorer, som kan have været medvirkende til udviklingen af apopleksi såsom livsstil og sygdomme, herunder f.eks. diabetes og hjerteproblemer. [1]

\subsection{Klinisk undersøgelse}
Den kliniske undersøgelse udføres for at vurdere hvor alvorligt et sygdomstilfælde, der er tale om. Undersøgelsen udføres ud fra en standardiseret skala, da dette gør det muligt for andre læger at undersøge patienten på samme måde senere i forløbet. Resultatet af undersøgelsen er en samlet score udregnet fra resultatet af de enkelte undersøgelser. Det er således muligt at vurdere om der sker fremskridt hver gang undersøgelsen foretages. [1]
Der findes flere forskellige skalaer, som lægen kan anvende til at foretage den kliniske undersøgelse, herunder Scandinavian Stroke Scale, European Stroke Scale og Hemisperic Stroke Scale. Fælles for skalaerne er, at de alle undersøger både bevidsthedsniveau samt motoriske og kognitive egenskaber hos patienten [2, 3, 4, 5]. I den danske sundhedssektor benyttes Scandinavian Stroke Scale, dette blev besluttet i Det Nationale Indikatorprojekt for apopleksi [6]. F.eks. har Region Syddanmark derudover forskellige retningslinjer for hvilken skala, der skal benyttes i et apopleksi forløb [7]. Alle apopleksi patienter, der bliver indlagt med mulig akut apopleksi eller TCI, skal have en score på Scandinavian Stroke Scale. Herefter benyttes National Institute of Health Stroke Scale hvis patienten skal have trombolysebehandling. Barthel Scale anvendes hvis patienten sendes til videre rehabilitering og beskriver patientens funktionsniveau i forhold til almindelige dagligdags funktioner. Til sidst benyttes Modificeret Rankin Scale til at give en beskrivelse af graden af handicap. [7]

\subsection{Videre undersøgelser}
Ved den videre undersøgelse vil der udføres en scanning for at undersøge, om patienten er ramt af en hjerneblødning \fxnote{ret evt til latinske betegnelser - det må vi lige blive enige om} eller en blodprop. Scanningen laves desuden for at lokalisere det ramte område. Enten udføres der en scanning af typen MR eller CT afhængigt af, hvad der er mest hensigtsmæssigt i den givne situation. Dette vurderes ud fra forskellige kriterier, herunder lægens mistanke om, hvilket område af hjernen, der er ramt, samt hvor længe symptomerne på apopleksi har optrådt. I visse tilfælde kan lægen vælge at anvende begge scanningstyper.  
Derudover skal patientens blodtryk måles jævnligt i den akutte fase for at sikre, at det falder gradvist til et normalt niveau i løbet af nogle timer til et døgn. Hvis blodtrykket pludselig falder meget, kan dette være et udtryk for en blodprop i hjertet. Det er derfor afgørende at følge udviklingen med jævnlige blodtryksmålinger. [1]
\\
Under forløbet bør andre faktorer også kontrolleres, herunder lungefunktion, blodsukker og kropstemperatur. Disse faktorer kan enten give information om apopleksien, eller de kan være væsentlige for patientens fremtidsprognoser og følger efter sygdomsforløbet [1].
\\

% [1]Sundhedsstyrelsens rapport
% [2]http://www.strokecenter.org/professionals/stroke-diagnosis/stroke-assessment-scales/
% [3]http://www.strokecenter.org/wp-content/uploads/2011/08/hemispheric.pdf
% [4]http://www.strokecenter.org/wp-content/uploads/2011/08/scandinavian.pdf
% [5]http://www.strokecenter.org/wp-content/uploads/2011/08/European_Stroke_Scale.pdf
% [6]http://www.dsks.dk/filer/hoeringssvar/referenceprogram_for_behandling_af_patienter_med_apopleksi.pdf
% [7] http://ekstern.infonet.regionsyddanmark.dk/Files/dokument90214.htm
% [8] https://www.sundhed.dk/borger/sygdomme-a-aa/hjerte-og-blodkar/sygdomme/apopleksi/apopleksi-blodprop-eller-bloedning-i-hjernen/
% [9] http://academic.eb.com.zorac.aub.aau.dk/EBchecked/topic/569347/stroke
% [10] http://www.netdoktor.dk/sygdomme/fakta/blodprophjerne.htm