\section{Diagnosticering}

Når en patient med akut apopleksi indlægges, er grundig undersøgelse nødvendig for at identificere, hvilken form for apopleksi der er tale om. Dette trin er afgørende for det efterfølgende forløb, da behandling samt rehabilitering skal planlægges herefter.[1]

\subsection{Anamnese}
For at blive bekendt med patientens egen subjektive vurdering af det hidtidige sygdomsforløb, optager lægen en anamnese. Her skal sygdomsforløbet  beskrives detaljeret, og der skal desuden spørges ind til faktorer, som kan have været medvirkende til udviklingen af apopleksi såsom livsstil og sygdomme, herunder f.eks. diabetes og hjerteproblemer.[1]

\subsection{Klinisk undersøgelse}
Den kliniske undersøgelse udføres for at vurdere hvor alvorligt et sygdomstilfælde, der er tale om. Undersøgelsen udføres ud fra en standardiseret skala, da dette gør det muligt for andre læger at undersøge patienten på samme måde senere i forløbet. Resultatet af undersøgelsen er en samlet score udregnet fra resultatet af de enkelte undersøgelser. Det er således hurtigt %ift. hvad? evt. bare slette hurtigt, beghøver ikke sætte tid på.
at vurdere om der er fremskridt hver gang undersøgelsen foretages.[1] 
Der findes flere forskellige skalaer, som lægen kan anvende til at foretage den kliniske undersøgelse, herunder Scandinavian Stroke Scale, European Stroke Scale og Hemisperic Stroke Scale. Fælles for skalaerne er, at de alle undersøger både bevidsthedsniveau samt motoriske og kognitive egenskaber hos patienten[2, 3, 4, 5]. %Hvilke skalaer bruges af den danske sundhedssektor - HVIS MAN KAN FINDE NOGET OM DET?

\subsection{Videre undersøgelser}
Ved den videre undersøgelse vil der udføres en scanning for at undersøge, om patienten er ramt af en hjerneblødning \fxnote{ret evt til latinske betegnelser - det må vi lige blive enige om} eller en blodprop. Scanningen laves desuden for at lokalisere det ramte område. Scanningen der udføres er enten af typen MR eller CT afhængigt af, hvad der er mest hensigtsmæssigt i den givne situation. Dette vurderes ud fra forskellige kriterier, herunder lægens mistanke om, hvilket område af hjernen, der er ramt, samt hvor længe symptomerne på apopleksi har optrådt. I visse tilfælde kan lægen vælge at anvende begge scanningstyper.  
Derudover skal patientens blodtryk måles jævnligt i den akutte fase for at sikre, at det falder gradvist til et normalt niveau i løbet af nogle timer til et døgn. Hvis blodtrykket pludselig falder meget, kan dette være et udtryk for en blodprop i hjertet. Det er derfor afgørende at følge udviklingen med jævnlige blodtryksmålinger.[1]
\\
Under forløbet bør andre faktorer også kontrolleres, herunder lungefunktion, blodsukker og kropstemperatur. Disse faktorer kan enten give information om apopleksien, eller de kan være væsentlige for patientens fremtidsprognoser og følger efter sygdomsforløbet.
[1]
\\
% Hvad sker der, når en patient bliver ramt af apopleksi - hvornår kan personen selv fornemme det? Fylde lidt på omkring den akutte fase? Omkring starten af forløbet?

% [1]Sundhedsstyrelsens rapport
% [2]http://www.strokecenter.org/professionals/stroke-diagnosis/stroke-assessment-scales/
% [3]http://www.strokecenter.org/wp-content/uploads/2011/08/hemispheric.pdf
% [4]http://www.strokecenter.org/wp-content/uploads/2011/08/scandinavian.pdf
% [5]http://www.strokecenter.org/wp-content/uploads/2011/08/European_Stroke_Scale.pdf
