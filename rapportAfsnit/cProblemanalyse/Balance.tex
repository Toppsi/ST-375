\section{Balance}
Efter et apopleksi tilfælde kan patienterne have balanceproblemer. Dette gør, at patienterne har risiko for faldulykker. Balancen er et komplekst system, da flere forskellige kropssystemer samarbejder om at sende balanceinformation til hjernen, hvor de bearbejdes. Balancen er styret af forskellige organer i kroppen vha. sansereceptorer.\fxnote{organer eller lemmer?} Disse er en bestemt slags celler, hvis funktion er at sende balanceinformationer til det centrale nervesystem og hjernen. Sansereceptorerne opfanger sanseindtryk og videresender informationen til områder i cerebral cortex, cerebellum og til centre i hele hjernestammen. Disse områder bearbejder informationen for at konkludere den fysiske position af kroppen og dens lemmer. Sansereceptorerne er placeret i ørerne og øjnene. \cite{Martini2012}   \\
Når hjernen har bearbejdet indtrykkene, udsender den nerveimpulser til skeletmuskulaturen om at foretage jævne og koordinerede bevægelser, hvorved kropsbalancen opretholdes.\cite{Martini2012} For apopleksi patienter opleves der ofte problemer med balancen. Patienterne hænger mod deres syge og svage side, da deres balance ofte er nedsat eller slet ikke funktionsdygtig. [2]

\subsection{Balancen styret af øret} \fxnote{Afsnittene skal muligvis ændres til latinske betegnelser}

Øret består overordnet af tre dele; det ydre øre, mellemøret og det indre øre. Det indre øre er med til at kontrollere balancen vha. hårcellerne, som sættes i bevægelse. Det ydre øre modtager trykbølger, som sætter trommehinden i svingninger. Disse transporteres af mellemørets knogler, der forstærker svingningerne. Væsken i mellemøret modtager svingningerne fra knoglerne, hvilket sætter væsken i bevægelse. Denne bevægelse trækker i hårcellerne, og der skabes derved et aktionspotentiale. Det indre øre er som en knoglelabyrint, hvor der findes et netværk af sammenhængende væskeholdige kanaler, som er indkapslet i knoglen. Det er i disse kanaler receptorerne sidder. 
Det indre øre kan yderligere opdeles i tre underdele; vestibulen, øresneglen og buegangen. De centrale dele, der har med balancen at gøre er vestibulen og buegangen, hvorimod øresneglen kun bidrager til hørelsen.\cite{Martini2012}    
\\
Vestibulen består af to membransække; sacculen og utriclen}, der opfanger sanseindtryk vedrørende tyngdekraft og lineær acceleration. Buegangens sansereceptorer opfanger stimuli omkring hovedets bevægelse, og hvor hurtigt bevægelsen foregår. Sansereceptorerne er placeret i buegangens tre væskefyldte knoglekanaler ved ampulla, der er forbundet til utriclen. Hårcellerne er kun aktive, når kroppen er i bevægelse ved at videregive information vedrørende hovedets bevægelse ift. tyngdekraften. Når hovedet er i bevægelse, sættes væskens i kanalerne også i bevægelse således, at væskebevægelser i den ene retning stimulerer hårcellerne, mens bevægelser i den modsatte retning forhindrer dem. For at få mest mulig information angående hovedets position, stimuleres de tre buegange af forskellige hovedbevægelser. Bevægelsesinformationerne sendes via vestibulocochlearnerven, der sender både information vedrørende balancen og hørelsen til hjernen i områder i pons og medulla oblongata. \cite{Martini2012}    


\fxnote{Indsæt illustration af ørets anatomi, hvor ampulla med hårcellerne kan ses! Kunne være den fra Martini 9th side 577 eller side 579 - OBS, noget er udkommenteret nu, se hvad der er relevant}

\subsection{Øjet og det visuelle}
Synet er en central faktor for, hvordan hjernen holdes informeret omkring kroppens balance og generel orientering. Dette gøres ved at give et indtryk af, hvordan kroppen og dens lemmer er placeret ift. omgivelserne\fxnote{jeg kan ikke finde dette i martini??}. Øjet har tre hinder omkring sig; fibrøs hinde, uvea og retina. Den fibrøse hinde\fxnote{hornhinden} er den yderste, som beskytter og støtter øjet. Den midterste hinde, kaldet uvea, indeholder blod og lymfekar samt regulerer mængden af lys, der kommer ind i øjet. Retina\fxnote{nethinden} er den inderste hinde, som er placeret bagerst i øjet. Den består af en pigmentdel og en indre neuraldel. Den neurale del indeholder fotoreceptorer, kaldet stave og tappe. Stave er følsomme overfor skarp lys og gør, at vi kan se i tusmørke. Tappe er følsomme overfor farvers bølgelængde, hvilket giver os farvesyn. Pigmentdelen absorberer lys, der passerer gennem den neurale del og gør, at lyset ikke har mulighed for at reflektere tilbage. Foto- og lysreceptorerne konverterer lyset fra omgivelserne til elektrisk nervesignal, der giver information omkring det objekt, der betragtes, herunder dets størrelse, form og bevægelser. Informationerne processeres således, at horiensontale celler lokaliserer områdets størrelse. Hvis der er kommet nok signal ind, der kræver en reaktion, sendes informationen først til bipolære celler herefter via synsnerven til den visuelle cortex, hvor informationen bearbejdes. \cite{Martini2012}     

\fxnote{Muligvis indsæt illustration af øjets anatomi f.eks. Martini 9th side 557}

\subsection{Proprioceptorerne og skeletmuskulaturens indvirkning på balancen}
Proprioceptorer monitorer leddenes position, muskelkontraktioners tilstand, samt spændinger i ledbånd og sener og de er placeret i skeletmuskulaturen. Informationerne sendes via nervesignaler til rygmarven og herfra igennem CNS til cerebellum. Proprioceptorer inddeles i tre overordnet grupper; muskelspindlere, golgi-sene organer og receptorer i ledkapsler.\cite{Martini2012}    \\
Muskelspindlere styrer og kontrollerer ændringer af muskellængder og kan udløse en strækrefleks. Den sensoriske nerve er forbundet centralt på muskelspindleren, hvor den kontinuert sender sensoriske impulser til CNS. Hvis den sensoriske nerve modtager stimuli, i form af stræk, vil den motoriske nerve på muskelspindleren blive stimuleret. Stimulation af den motoriske nerve vil forkorte musklens længde. Nogle strækreflekser er holdningsreflekser, som hjælper os med at holde balancen. I en stående position kræves der samarbejde mellem mange forskellige muskelgrupper for at forblive stående. Dette ses f.eks. hvis kroppen lænes forover, vil strækreflekserne i læggene blive aktiveret og kontraherer. Derved vil kroppen læne sig bagud og igen stå i en opret position. Hvis der sker en overkompensation fra lægmusklerne og kroppen læner sig for meget bagud, vil strækreflekser i skinnebenet og lårene aktiveres. Derved vil kroppen læne sig forover igen. Kroppen foretager mange af disse ubevidste korrektioner. \cite{Martini2012}   \\ %(Se Martini 9th side 438 under "monosynaptic reflexes")

Golgi-sene organer sidder i en kløft\fxnote{junction} mellem skeletmusklen og tilhørende sene. Dendritterne fra golgi-sene organet kopler sig på den tætteste sene og stimuleres af spændingen i denne, hvorved den eksterne spænding i en muskelkontraktion bliver målt. \cite{Martini2012}    \\

Ledkapsler er fyldt med frie nerveender, som kaldes receptorer. Disse receptorer detekterer tryk, spænding og bevægelse i leddet. \cite{Martini2012}    \\

Det er kun en lille del, af den information proprioceptererne sender, der opfanges af bevidstheden, eftersom størstedelen foregår på et underbevidst niveau.\cite{Martini2012}   

%Golgi seneorganer (Se Martini 9th side 501 under 15-3 propriocetor) %Receptorer i ledkapsler (Se Martini 9th side 501 under 15-3 propriocetor)
\fxnote{Mangler: Det sidste afsnit propriocerptorerne mangler at blive skrevet helt færdigt. Derudover kunne man have en konklusion hvor der blev kigget på sammenspillet mellem de forskellige dele der har med balancen at gøre. Vi kunne også godt have skrevet til hvilke nervesystemer der sig af signalerne i de forskellige organer. Så skal lange sætninger osv. lige rettes igennem.}

\subsection{Apopleksi og balance}
%Ens balance er et resultat af sammenspillet mellem det indre øres og ledkapslernes receptorer, samt information fra golgi-ene orgaer og muskelspindlere. 

Balancen er styrer flere steder i kroppen og er med til at beskytte kroppen mod f.eks. faldulykker, ved at sikre at kroppen og den lemmer bevæger sig i kontrollerede og jævne bevægelser. Kroppen opretholder balancen ved at bruge ørerne, øjne og proprioceptorer i skeletmuskulaturen. Proprioceptorerne kontrollerer muskler, sener og leds position. Øjne opfanger lys og er med til orienteringen af kroppen og dens lemmer og hårceller i øret register hoveds bevægelser ved hjælp af tyngdekraften. Selvom et balanceorgan er ude af funktion er kroppen stadig i stand til at opretholde balancen ved hjælp fra andre balanceorganer. Det er til gengæld svære for kroppen at opretholde balancen hvis centrene i hjerne, som behandler den information, som kommer fra balanceorganerne, bliver skadet, som det kan ske ved apopleksi patienter. \cite(Martini2012) 
% [1] – Martini, Frederic H and others. Fundamentals of Anatomy & Physiology (Kapitel:13, 14, 15, 17 ). 2012. Pearson. 
% [2] - Karnath2003

\fxnote{Kan man evt skrive et afsnit omkring samarbejde imellem øret og øjet omkring balance?}

