\section{Balance}
Efter et apopleksi tilfælde kan patienterne have balance problemer, hvilket kan give farlige situationer, hvor patienten har risiko for faldulykker. Balancen er styret af forskellige organer i kroppen. Organerne styrer balancen ved at bruge sansereceptorer. Sansereceptorer er en bestemt slags celler, hvis funktion består i at sende informationer til det centralnervesystem og hjerne vedrørende kropsbalancen. Sansereceptorerne opfanger sanseindtryk og videregiver informationen til områder i cerebral cortex, cerebellum og til centre i hele hjernestammen. Disse områder bearbejder informationen, for at konkludere den fysiske position af kroppen og dens lemmer. [1]\\

Balancen er i virkeligheden et komplekst system, da flere forskellige kropssystem samarbejder i forhold til at sende indtryk til hjernen, hvor de bearbejdes. Når hjernen har bearbejdet indtrykkene udsendes nerveimpulser til skeletmuskulaturen om at foretage jævne og koordinerede bevægelser, hvorved kropsbalancen opretholdes. [1] For apopleksipatienter opleves der ofte problemer med balancen. Patienterne hænger mod deres syge og svage side uden de er opmærksomme på det, da deres balance ofte er nedsat eller slet ikke funktionsdygtigt. [2] 

De sansereceptorer, som indvirker til at opretholde balancen, i form af at holde kroppen i en lodret position, og korrigere eventuelle kropshældninger er placeret i ørerne og øjnene.  [1]

\subsection{Øret} \fxnote{Afsnittene skal muligvis ændres til latinske betegnelser}

Øret består overordnet af tre dele. Det ydre øre, mellemøret og det indre øre. Det er i det indre øre, som er med til at kontrollere balancen, da det indeholder de receptorer, som består af hårceller, som reagere på forskellige stimuli og er med opretholder balancen i kroppen. Det indre øre er som en knoglelabyrint, hvor der findes et netværk af sammenhængende væskeholdige kanaler, som er indkapslet i knogle. Det er i disse kanaler receptorerne sidder. Det indre øre kan yderligere opdeles i tre underdele: vestibulen, øresneglen og buegangen. Vestibulen består af et par membransække: Sacculen og utriclen. Disse membransække opfanger sanseindtryk vedrørende tyngdekraft og lineærer accelerationer. \fxnote{Ved ikke om der skal skrives om øresneglen, da den ikke har noget med balancen at gøre, men derimod hørelsen} Buegangen består af væskefyldet knoglekanaler og her sidder også de receptorer, som opfanger stimuli omkring hovedets bevægelse og viderebringer information om hovedet rotationer, og hvor hurtig bevægelsen foregår. Receptorerne består af hårceller, som sidder i buegangenes ampulla, som er placeret, hvor buegangene er forbundet til utriculen. I utriculen og sacculen sidder der øresten, som indeholder hårceller, som opfanger information vedrørende hovedets bevægelse i forhold til tyngdekraften. Dette sker når hovedet bevæger sig, sættes væsken i kanalerne i bevægelse. Væskebevægelser i den ene retning stimulerer hårcellerne, mens bevægelser i den modsatte retning forhindrer dem. De forskellige buegange stimuleres af forskellige hovedbevægelser, for på den måde at få bedst mulig information. Informationen sendes via vestibulocochlearnerven, som sender information til hjernen i områder i pons og medulla oblongata vedrørende balance og hørelse. [1]    

\fxnote{Indsæt illustration af ørets anatomi, hvor ampulla med hårcellerne kan ses! Kunne være den fra Martini 9th side 577 eller side 579 }

\subsection{Øjet og det visuelle}
Øjnene og synet har den funktion at holde hjernen informeret om kroppens balance og generel orientering, ved at give et indtryk af, hvordan kroppen og dens lemmer er placeret i forhold til omgivelserne. Det er fotoreceptorer, som er placeret i nethinden \fxnote{retina}, der udgøre den inderste del af øjet. Nethinden består af et ydre lag, som er pigmentdelen og en indre neural del. Den neurale del indeholder lysreceptorer, støttende celler og neuroner, der udfører behandler visuel information. Pigmentdelen har den funktion at absorberer lys, som passerer gennem den neurale del og gør at lyset ikke har mulighed for at reflektere tilbage til den neurale del. Foto- og lysreceptorerne konverterer lyset fra omgivelserne til elektrisk nervesignal, som sendes via synsnerven til den visuelle cortex i storhjenehemisfæren, hvor informationen bearbejdes. [1]  

\fxnote{Muligvis indsæt illustration af øjets anatomi f.eks. Martini 9th side 557}

\subsection{Proprioceptorerne og skeletmuskulaturens indvirkning på balancen}
Proprioceptorer findes i skeletmuskulaturen og er receptorer, der monitorer leddenes position, spændinger i sener og ledbånd, samt muskelkontraktionernes tilstand. Informationerne som opfanges sendes via nervesignal til rygmarven og herfra igennem centralnervesystemet til cerebellum. Proprioceptorer bliver inddelt i tre overordnet grupper: muskelspindlere, golgi sene organer og receptorer i ledkapsler. [1]

Muskelspindlere er den gruppe, som styre og kontrollerer ændringer i muskellængden og kan udløse en strækrefleks, som derved regulerer skeletmuskulaturens længde. Til muskelspindlerne er der forbundet sensoriske og motoriske nerver. Den sensoriske nerve er forbundet central på muskelspindleren, hvor den sender sensoriske impulser.  (Se Martini 9th side 438 under "monosynaptic reflexes")  

Golgi seneorganer (Se Martini 9th side 501 under 15-3 propriocetor)
Receptorer i ledkapsler (Se Martini 9th side 501 under 15-3 propriocetor)




[1] – Martini, Frederic H and others. Fundamentals of Anatomy & Physiology (Kapitel:13, 14, 15, 17 ). 2012. Pearson. 
[2] - Karnath2003

