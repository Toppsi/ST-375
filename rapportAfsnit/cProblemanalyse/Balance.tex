\section{Balance}
Efter et apopleksi tilfælde kan patienterne have balanceproblemer. Dette gør, at patienterne har risiko for faldulykker. Balancen er et komplekst system, da flere forskellige kropssystemer samarbejder om at sende balanceinformation til hjernen, hvor de bearbejdes. Balancen er styret af forskellige organer i kroppen vha. sansereceptorer.\fxnote{organer eller lemmer?} Disse er en bestemt slags celler, hvis funktion er at sende balanceinformationer til det centralnervesystem og hjernen. Sansereceptorerne opfanger sanseindtryk og videresender informationen til områder i cerebral cortex, cerebellum og til centre i hele hjernestammen. Disse områder bearbejder informationen for at konkludere den fysiske position af kroppen og dens lemmer. Sansereceptorerne er placeret i ørerne og øjnene. [1]\\
Når hjernen har bearbejdet indtrykkene, udsender den nerveimpulser til skeletmuskulaturen om at foretage jævne og koordinerede bevægelser, hvorved kropsbalancen opretholdes.[1] For apopleksi patienter opleves der ofte problemer med balancen. Patienterne hænger mod deres syge og svage side, da deres balance ofte er nedsat eller slet ikke funktionsdygtig. [2]

\subsection{Øret} \fxnote{Afsnittene skal muligvis ændres til latinske betegnelser}

Øret består overordnet af tre dele; det ydre øre, mellemøret og det indre øre. Det indre øre er med til at kontrollere balancen vha. hårcellerne, som sættes i bevægelse. Det ydre øre modtager trykbølger, som sætter trommehinden i svingninger. Disse transporteres af mellemørets knogler, der forstærker svingningerne. Væsken i mellemøret modtager svingningerne fra knoglerne, hvilket sætter væsken i bevægelse. Denne bevægelse trækker i hårcellerne, og der skabes derved et aktionspotentiale. %Det indre øre er som en knoglelabyrint, hvor der findes et netværk af sammenhængende væskeholdige kanaler, som er indkapslet i knogle. Det er i disse kanaler receptorerne sidder. 
Det indre øre kan yderligere opdeles i tre underdele; vestibulen, øresneglen og buegangen. Vestibulen består af to membransække\fxnote{der hedder Sacculen og utriclen}, der opfanger sanseindtryk vedrørende tyngdekraft og lineær accelerationer. Buegangen består af væskefyldt knoglekanaler, hvor nogle specifikke receptorer sidder. De opfanger stimuli omkring hovedets bevægelse, og hvor hurtigt bevægelsen foregår. %Receptorerne består af hårceller, som sidder i buegangenes ampulla, som er placeret, hvor buegangene er forbundet til utriculen. I utriculen og sacculen sidder der øresten, som indeholder hårceller, som opfanger information vedrørende hovedets bevægelse i forhold til tyngdekraften. Dette sker når hovedet bevæger sig, sættes væsken i kanalerne i bevægelse. Væskebevægelser i den ene retning stimulerer hårcellerne, mens bevægelser i den modsatte retning forhindrer dem. De forskellige buegange stimuleres af forskellige hovedbevægelser, for på den måde at få bedst mulig information. Informationen sendes via vestibulocochlearnerven, som sender information til hjernen i områder i pons og medulla oblongata vedrørende balance og hørelse. [1]    
Øresneglen bidrager ikke til balancen men hørelsen.

\fxnote{Indsæt illustration af ørets anatomi, hvor ampulla med hårcellerne kan ses! Kunne være den fra Martini 9th side 577 eller side 579 - OBS, noget er udkommenteret nu, se hvad der er relevant}

\subsection{Øjet og det visuelle}
Øjnene og synet holder hjernen informeret om kroppens balance og generel orientering. Dette gøres ved at give et indtryk af, hvordan kroppen og dens lemmer er placeret ift. omgivelserne. Øjet har tre hinder omkring sig; fibrøs hinde, uvea og retina. Fibrøs hinde\fxnote{hornhinden} er den yderste, som beskytter og støtter øjet. Den midterste hinde, kaldet uvea, indeholder blod og lymfekar samt regulerer mængden af lys, der kommer ind i øjet. Retina\fxnote{nethinden} er den inderste hinde, som er placeret bagerst i øjet. Den består af en pigmentdel og en indre neuraldel. Den neurale del indeholder fotoreceptorer, kaldet stave og tappe. Stave er følsomme overfor skarp lys og gør, at vi kan se i tusmørke. Tappe er følsomme overfor farvers bølgelængde, hvilket giver os farvesyn. Pigmentdelen absorberer lys, der passerer gennem den neurale del og gør, at lyset ikke har mulighed for at reflektere tilbage. Foto- og lysreceptorerne konverterer lyset fra omgivelserne til elektrisk nervesignal, som sendes via synsnerven til den visuelle cortex, hvor informationen bearbejdes\fxnote{Kunne måske godt uddybes lidt mere}. [1]  

\fxnote{Muligvis indsæt illustration af øjets anatomi f.eks. Martini 9th side 557}

\subsection{Proprioceptorerne og skeletmuskulaturens indvirkning på balancen}
Proprioceptorer findes i skeletmuskulaturen og er receptorer, der monitorer leddenes position, spændinger i sener og ledbånd samt muskelkontraktionernes tilstand. Informationerne sendes via nervesignal til rygmarven og herfra igennem centralnervesystemet til cerebellum. Proprioceptorer bliver inddelt i tre overordnet grupper; muskelspindlere, golgi-sene organer og receptorer i ledkapsler.[1] \\
Muskelspindlere styrer og kontrollerer ændringer af muskellængder og kan udløse en strækrefleks. Den sensoriske nerve er forbundet centralt på muskelspindleren, hvor den kontinuert sender sensoriske impulser til CNS. Hvis den sensoriske nerve modtager stimuli, i form af stræk, vil den motoriske nerve på muskelspindleren blive stimuleret. Stimulation af den motoriske nerve vil forkorte musklens længde. Nogle strækreflekser er holdningsreflekser, som hjælper os med at holde balancen. I en stående position kræves der samarbejde mellem mange forskellige muskelgrupper for at forblive stående. Dette ses f.eks. hvis kroppen lænes forover, vil strækreflekserne i læggene blive aktiveret og kontraherer. Derved vil kroppen læne sig bagud og igen stå i en opret position. Hvis der sker en overkompensation fra lægmusklerne og kroppen læner sig for meget bagud, vil strækreflekser i skinnebenet og lårene aktiveres. Derved vil kroppen læne sig forover igen. Kroppen foretager mange af disse ubevidste korrektioner. [1] \\ %(Se Martini 9th side 438 under "monosynaptic reflexes")
Golgi-sene organer sidder i en kløft\fxnote{junction} mellem musklen og senen. Dendritterne fra golgi-sene organet kopler sig på den tætteste sene og stimuleres af spændingen i senen, hvorved den eksterne spænding i en muskel kontraktion bliver målt. [1] \\
Ledkapsler er fyldt med frie nerve ender, som kaldes receptorer. Disse receptorer detekterer tryk, spænding og bevægelse i ledet. [1] \fxnote{evt. skrive mere til her om golgi-sene organer og ledkapsler - er der mere ift. balance?} %Golgi seneorganer (Se Martini 9th side 501 under 15-3 propriocetor) %Receptorer i ledkapsler (Se Martini 9th side 501 under 15-3 propriocetor)
\fxnote{Mangler: Det sidste afsnit propriocerptorerne mangler at blive skrevet helt færdigt. Derudover kunne man have en konklusion hvor der blev kigget på sammenspillet mellem de forskellige dele der har med balancen at gøre. Vi kunne også godt have skrevet til hvilke nervesystemer der sig af signalerne i de forskellige organer. Så skal lange sætninger osv. lige rettes igennem.}
% [1] – Martini, Frederic H and others. Fundamentals of Anatomy & Physiology (Kapitel:13, 14, 15, 17 ). 2012. Pearson. 
% [2] - Karnath2003

\fxnote{Kan man evt skrive et afsnit omkring samarbejde imellem øret og øjet omkring balance?}

