% !TeX spellcheck = da_DK
\section{Rehabilitering}
Når selve slagtilfældet er stabiliseret og behandlet, er det essentielt, at rehabiliteringen af en apopleksipatient indfindes hurtigst muligt - gerne en til to dage efter slagtilfældet. I Danmark dækker rehabilitering af en patient med apopleksi områderne: direkte træning af funktioner, ufrivillig reorganisering af hjernen netværk, kompenserende strategier, ændringer i miljø, social og psykologisk støtte. Genoptræningen omhandler dog ikke kun træning med en ergo- eller fysioterapeut, da plejepersonale til dagens almindelige gøremål også essentiel. Patientens daglige rutiner kan være gået tabt under slagtilfældet, hvorfor det er vigtigt, at få patienten tilbage i sit vante miljø. Plejepersonale skal hjælpe patienten til at genfinde denne rytme og hjælpe patienten til eventuelt at udføre dagligdags ting på en ny måde. Det kan ske, at patienten ikke længere er i stand til at beherske begge sine hænder til en opgave, hvorved plejepersonalet skal bistå patienten i indlæringen af kun at benytte en hånd.

Motoriske og sensoriske funktionsproblemer kan lede til balancebesvær for patienten i både siddende, stående og gående stilling. Der er afprøvet adskillige farmakologiske midler og behandlingsstadegier for at forbedre hjernens rehabilitering og motoriske funktioner. F.eks. er der afprøvet, at tildele apopleksipatienter det antidepressive middel fluoxetin i kombination med fysioterapi. Derudover er kortikal stimulation afprøvet, hvor området af hjernen, som kontrollerer motorstyring, modtager elektriske impulser fra en implanteret anordning. Denne mulighed har haft blandede succesoplevelser, men er udelukkende afprøvet på patienter, der har oplevet et alvorligt slagtilfælde. \cite{Academic2015}
  
Apopleksi patienten skal i samarbejde med lægen, sygeplejersken og andet hjælpepersonale opstille nogle mål for sin rehabilitering. Målene skal hverken være for svære eller for lette, så patienten ikke mister sin motivation til genoptræningen. \cite{Kruuse2015}

\subsection{Forløbsprogram for rehabilitering} 
Sundhedsstyrelsen har udarbejdet et forløbsprogram for rehabilitering af patienter med erhvervet hjerneskade. Forløbsprogrammet strækker sig fra at patienten erhverver hjerneskaden til at patienten har opnået bedst mulig funktionsevne, hvorefter der udføres kontrol og vedligeholdelse af funktionsevnen. Tidsperioden af rehabilitering varierer ift. hjerneskadens sværhedsgrad, samt sværhedsgraden af funktionstabet. %dog kan perioden vare flere år.  
\cite{Sundhedsstyrelsen2011a}

Forløbsprogrammet er essentielt i forhold til at kunne give patienten den korrekte rehabilitering. Patienterne har forskellige behov og er afhængige af hjælp fra plejepersonale. Deruodver kræves der forskellige former for teknologi i de forskellige faser. Det vil derfor være oplagt at undersøge, hvilken form for rehabilitering der er at foretrække i de enkelte faser som ses på \figref{trefaser}.

\begin{figure}[H]
	\centering
	\includegraphics[scale=0.6]{figures/bProblemanalyse/flowdiagram_faser1.png}
	\caption{På figuren ses et overblik over de tre faser, som patienter med apopleksi skal igennem i forløbsprogrammet for rehabilitering \cite{Sundhedsstyrelsen2011a}} 
	\label{trefaser}
\end{figure}

\subsubsection{Den første fase}
Som det vises på \figref{trefaser} afspejler første fase den del af forløbsprogrammet som foregår på sygehusets apopleksiafdeling. På apopleksiafdelingen foretages primært akut behandling for at begrænse skaderne. Når patientens sikkerhed er sikret og skaderne er begrænset påbegyndes den tidlige rehabilitering. Under den tidlige rehabilitering giver en speciallæge i neurologi en vurdering af patientens rehabiliteringsbehov. Derudover bliver patienterne overvåget i forhold til bevidsthed, ændringer og amnesi samt foretaget vurderinger af basale fysiologiske funktioner. Samtidig bliver der iværksat træning i diverse bevægelsesfunktioner, basale egenskaber og kommunikationsfunktioner. Patienterne gennemgår også en tidlig behandling og diagnostik for at undersøge komplicerende tilstande, som f.eks. vaskulære hændelser, blodpropper i ben og lunger og smerter. Patienterne vurderes i denne fase af fagkyndigt personale som ergoterapeut, fysioterapeut og audiologopæd \fxnote{høre og talepædagog}. Disse er med til, at sikre, at patienten udfører træningen korrekt i forhold til stilmulering og træning af bevægelsesfunktioner, taletræning og udførsel af basale daglige aktiviteter.\cite{Sundhedsstyrelsen2011a}

\subsubsection{Den anden fase}
Det fremgår af \figref{trefaser}, at patienten i den anden fase gennemgår rehabilitering på sygehuset, hvor der er fokus på de skadede funktioner. Ligeledes bliver patienten på samme måde som i fase et undervist af fagkyndigt personale. Hvorefter patientens behov for rehabilitering og rehabiliteringens udvikling vurderes. Patienterne bliver i denne fase udredet i forhold til funktionsevne, mentale funktioner, bevægelsesfunktioner herunder bevægelse og mobilitet i led, knogler, reflekser og muskler samt rehabilitering med henblik på daglige aktiviteter. Hvis patienten vurderes til at have en stabil udvikling i rehabiliteringsprocessen, vil patienten blive udskrevet og påbegynde fase tre. \cite{Sundhedsstyrelsen2011a}


\subsubsection{Den tredje fase}
I den tredje og sidste fase er patienten udskrevet fra sygehuset. Derved foregår rehabilitering som ambulant rehabilitering og selvstændig træning som det fremgår af \figref{trefaser}.  Selve rehabiliteringen i tredje fase er bygget op ud fra rehabiliteringsforløbet i den anden fase. Det afgørende er for den tredje faser, hvorvidt patienten skal vedblive rehabilitering på sygehuset, eller om patienten henvises til rehabilitering i de kommunale rehabiliteringscentre. Dette afgøres på baggrund af observationer foretaget i anden fase.  Den selvstændige træning, kan for patienterne med neglekt være en udfordring i forhold til bevægelsesmønstre og kropsholdning. Patienterne går derfor stadig til kontrol og vedligeholdelse for at sikre, at rehabiliteringens udvikling er stabil. Det kan i sidste ende have betydning for, hvor lang tid det tager for patienten at generhverve sine tabte funktioner. Den tredje fase varierer derfor fra patient til patient alt efter udviklingen af rehabiliteringen.\cite{Sundhedsstyrelsen2011a}



% I første og anden fase af rehabiliteringsforløbet bliver patienten undervist og overvåget af fagkyndigt personale. Dette gøres for at sikre, at patienten udfører træningen korrekt f.eks. med bevægelsesmønstre, og korrigere patienten til at bevægelsen og øvelserne udføres korrekt. Dette er vigtigt, da patienten, som sagt i tredje fase, selv skal foretage den nødvendige træning og dermed har fornemmelse af, hvordan træningen udføres korrekt ift. bevægelsesmønstre og kropsholdning. Dette kan midlertidig være en udfordring for apopleksipatienter med neglekt, da de kan have problemer med balancen og opmærksomheden på kroppen. Patienten går derfor stadig til kontrol og vedligeholdelse for at sikre, at rehabiliteringens udvikling er stabil. Det kan i sidste ende have betydning for, hvor lang tid det tager for patienten at generhverve sine tabte funktioner. Den tredje fase varierer derfor fra patient til patient alt efter udviklingen på rehabiliteringen. \cite{Sundhedsstyrelsen2011a}