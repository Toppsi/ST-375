% !TeX spellcheck = da_DK
\section{Rehabilitering}
Når selve apopleksien er stabiliseret og behandlet, er det essentielt, at rehabiliteringen af patienten indfindes hurtigst muligt - gerne en til to dage efter apopleksitilfældet. I Danmark dækker rehabilitering af en patient med apopleksi områderne: direkte træning af funktioner, ufrivillig reorganisering af encephalons netværk, kompenserende bevægelser, ændringer i miljø, social og psykologisk støtte. Genoptræningen omhandler ikke kun træning med en ergo- eller fysioterapeut, da plejepersonale til dagens almindelige gøremål er essentiel. Patientens daglige rutiner kan være gået tabt under apopleksitilfældet, hvorfor det er vigtigt, at få vedkommende tilbage i sit vante miljø. Plejepersonale skal hjælpe patienten til at genfinde rytmen og til evt. at udføre dagligdagsopgaver på en ny måde. Det kan ske, at patienten ikke længere er i stand til at beherske begge sine hænder til en opgave, hvorved plejepersonalet skal bistå patienten i indlæringen af kun at benytte én hånd. \cite{Kruuse2015} \\
%Sensoriske- og motoriske funktionsproblemer kan lede til balancebesvær for patienten i både siddende, stående og gående ling. Der er afprøvet adskillige farmakologiske midler og behandlingsstrategier for at forbedre encephalons rehabilitering og motoriske funktioner. F.eks. er der afprøvet at tildele apopleksipatienter det antidepressive middel fluoxetin i kombination med fysioterapi.\fxnote{Find dette i kilden og forklar, hvordan antidepressiv middel kan hjælpe. Bare skriv det her i Fxnote} Derudover er kortikal stimulation afprøvet, hvor området af encephalon, som kontrollerer motorstyring, modtager elektriske impulser fra en implanteret anordning. Denne mulighed har haft blandede succesoplevelser, men er udelukkende afprøvet på patienter, der har oplevet et alvorligt apopleksitilfælde. \cite{Academic2015} \fxnote{Skal dette afsnit evt. slettes?}
Apopleksipatienten skal i samarbejde med lægen, sygeplejersken og andet hjælpepersonale opstille nogle mål for sin rehabilitering. Målene skal være realistiske, så patienten ikke mister sin motivation til genoptræningen.  \cite{Kruuse2015}

\subsection{Forløbsprogram for rehabilitering} 
Sundhedsstyrelsen har udarbejdet et forløbsprogram for rehabilitering af patienter med erhvervet hjerneskade. Forløbsprogrammet strækker sig fra at patienten erhverver hjerneskaden til at bedst mulig funktionsevne er opnået. Herefter udføres kontrol og vedligeholdelse af funktionsevnen. Tidsperioden for rehabilitering varierer ift. hjerneskadens sværhedsgrad, samt graden af funktionstab. %dog kan perioden vare flere år.  
\cite{Sundhedsstyrelsen2011a}

Forløbsprogrammet er essentielt ift. at kunne give patienten den korrekte rehabilitering. Patienterne har forskellige behov og er afhængige af hjælp fra plejepersonale. Deruodver kræves der forskellige former for teknologi i de enkelte faser. Det vil derfor være oplagt at undersøge, hvilken form for rehabilitering der er at foretrække i de enkelte faser som ses på \figref{firefaser}.

\begin{figure}[H]
	\centering
	\includegraphics[scale=1.2]{figures/bProblemanalyse/flowdiagram_faser1.png}
	\caption{På figuren ses et overblik over de fire faser, som patienter med apopleksi skal igennem i forløbsprogrammet for rehabilitering. \cite{Sundhedsstyrelsen2011a}} 
	\label{firefaser}
\end{figure}

\subsubsection{Den første fase}
Som det vises på \figref{firefaser} afspejler første fase den del af forløbsprogrammet, som foregår på sygehusets apopleksiafdeling. På apopleksiafdelingen foretages primært akut behandling for at begrænse følgerne. Når patientens sikkerhed er opnået og følgerne er begrænset påbegyndes den tidlige rehabilitering. Under den tidlige rehabilitering giver en speciallæge i neurologi en vurdering af patientens rehabiliteringsbehov. Derudover bliver patienterne overvåget ift. bevidsthed, fysiologiske ændringer og amnesi samt foretaget vurderinger af basale fysiologiske funktioner. Samtidig bliver der iværksat træning i diverse bevægelsesfunktioner, basale egenskaber og kommunikationsfunktioner. Patienterne gennemgår også en tidlig behandling og diagnostik for at undersøge komplicerende tilstande, som f.eks. vaskulære hændelser, smerter og blodpropper i ben og lunger. Patienterne vurderes i denne fase af fagkyndigt personale såsom ergoterapeut, fysioterapeut og audiologopæd \fxnote{høre og talepædagog}. Disse er med til at sikre, at patienten udfører træningen korrekt ift. stimulering\fxnote{Kilden siger "herunder stimulering/træning af bevægelsesfunktioner", så det må betyde, at deer sker en stimulering i hjernen af bevægelsesfunktioner, som kan optimere bevægelsen.} og træning af bevægelsesfunktioner, taletræning og udførsel af basale daglige aktiviteter. \cite{Sundhedsstyrelsen2011a}

\subsubsection{Den anden fase}
Det fremgår af \figref{firefaser}, at patienten i den anden fase gennemgår rehabilitering på sygehuset, hvor der er fokus på de skadede funktioner. Ligeledes bliver patienten på samme måde som i den første fase undervist af fagkyndigt personale, hvor patientens behov for rehabilitering og rehabiliteringens udvikling vurderes. Patienterne bliver i denne fase udredt ift. funktionsevne, mentale funktioner, bevægelsesfunktioner herunder bevægelse og mobilitet i led, knogler, reflekser og muskler samt rehabilitering med henblik på daglige aktiviteter. Hvis patienten vurderes til at have en stabil udvikling i rehabiliteringsprocessen, vil vedkommende blive udskrevet og påbegynde fase tre. \cite{Sundhedsstyrelsen2011a}


\subsubsection{Den tredje fase}
I den tredje fase er patienten udskrevet fra sygehuset. Derved foregår rehabiliteringen ambulant og som selvstændig træning, hvilket fremgår af \figref{firefaser}.  Selve rehabiliteringen i tredje fase er bygget op ud fra rehabiliteringsforløbet i den anden fase. Det afgørende for den tredje fase er, hvorvidt patienten skal vedblive rehabilitering på sygehuset eller henvises til de kommunale rehabiliteringscentre. Dette afgøres på baggrund af observationer foretaget i anden fase. Den selvstændige træning kan for patienter med neglekt og balanceproblemer være en udfordring ift. bevægelsesmønstre og kropsholdning. \cite{Sundhedsstyrelsen2011a}

\subsubsection{Den fjerde fase}
Det fremgår på \figref{firefaser}, at fjerde fase er den afsluttende fase for behandlingsforløbet. Patienterne går stadig til kontrol og vedligeholdelse for at sikre, at rehabiliteringens udvikling er stabil. Det kan i sidste ende have betydning for, hvor lang tid det tager for patienten at generhverve sine tabte funktioner. Den fjerde fase varierer derfor yderligere fra patient til patient alt efter udviklingen af rehabiliteringen. \cite{Sundhedsstyrelsen2011a} \\

\subsection{Organisering af rehabiliteringsprocessen}
I sundhedssektoren arbejder de forskellige organisatoriske aktører på tværs af hinanden, hvilket vil sige, at der er et samarbejde mellem sygehuse, kommuner og praktiserende læger. Dette samarbejde sker både internt på sygehusene, på afdelingerne og kommunalt mellem forvaltningerne. \cite{Sundhedsstyrelsen2010}
De nævnte aktører er de centrale enheder i forbindelse med hjerneskaderehabilitering.\fxnote{Hvilke øvrige aktører indgår, som ikke er de centrale? Svar: Forskellige faggrupper med neurofaglige kompetencer. Andre aktører - kommunikationscentre (kommunal/regionalt eller foreningsejet/privat) med specialiserede hjerneskadetilbud, hvis de centrale aktører ikke kan leverer det de skal. Ellers er der VISO, ViHS, UU og borgerorganisationer. Tjek kilde "Sundhedsstyrelsen2011a" for yderligere info} De har opgaver i alle faser i varierende grad. Sygehuset har flest opgaver i første og anden fase, mens kommuner og praktiserende læger har flest opgaver i tredje og fjerde fase. \cite{Sundhedsstyrelsen2011a}
Det er vigtigt, at det organisatoriske samspil fungerer, da hjerneskadede patienter, som beskrevet ovenfor, er i kontakt med flere forskellige organisatoriske aktører under deres sygdomsforløb. De enkelte forløb kan desuden være forskellige, afhængig af hvor i landet patienten befinder sig, samt hvor omfattende hjerneskaden er. \cite{Sundhedsstyrelsen2010}



%Denne forskel opleves regionalt, hvor behandling og rehabilitering enkelte steder foregår på få af sygehusets afdelinger, mens patienter andre steder behandles på et rehabiliteringssygehus, efter den akutte behandling er foretaget.\cite{Sundhedsstyrelsen2010}

%I starten af behandlingssforløbet sendes patienterne til neurologiske, geriatriske, neurokirurgiske og medicinske afdelinger på sygehuset. Som tidligere nævnt inddeles patienterne efter sværhedsgrad af hjerneskaden, hvor de sværest ramte, som er patienter med traumatisk hjerneskade og tilgrænsede lidelser, vidererstilles til Hammel og Hvidovre. Rehabiliteringen kan også ske på rehabiliteringsafsnittene på landets sygehuse.
%Det primære ansvar ligger hos kommunerne i form af genoptræningsplanens afdækning af rehabiliteringsbehov, dvs. at kommunerne holder øje med om dette foregår i praksis, herunder bl.a. patientens genoptræningsbehov. Kommunen har derudover mulighed for at henvise patienterne til dens egne tilbud, samt at henvise til private.\cite{Sundhedsstyrelsen2010} 

%Afslutningsvis gennemgår patienterne et langt og forskelligt behandlingsforløb alt efter hjerneskadens omfang. Forløbet indebærer et samarbejde mellem de forskellige aktører. Efter behandlingen står kommunerne, som førnævnt, for det primære ansvar i forhold til rehabilitering og henvisning for patienten\fxnote{hvor kommer denne info fra?}.



% I første og anden fase af rehabiliteringsforløbet bliver patienten undervist og overvåget af fagkyndigt personale. Dette gøres for at sikre, at patienten udfører træningen korrekt f.eks. med bevægelsesmønstre, og korrigere patienten til at bevægelsen og øvelserne udføres korrekt. Dette er vigtigt, da patienten, som sagt i tredje fase, selv skal foretage den nødvendige træning og dermed har fornemmelse af, hvordan træningen udføres korrekt ift. bevægelsesmønstre og kropsholdning. Dette kan midlertidig være en udfordring for apopleksipatienter med neglekt, da de kan have problemer med balancen og opmærksomheden på kroppen. Patienten går derfor stadig til kontrol og vedligeholdelse for at sikre, at rehabiliteringens udvikling er stabil. Det kan i sidste ende have betydning for, hvor lang tid det tager for patienten at generhverve sine tabte funktioner. Den tredje fase varierer derfor fra patient til patient alt efter udviklingen på rehabiliteringen. \cite{Sundhedsstyrelsen2011a}