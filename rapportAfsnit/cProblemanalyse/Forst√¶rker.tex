\section{Forstærker}
For at forstærke det svage signal, som kommer fra accelerometeret, benyttes en operationsforstærker. En operationsforstærker skalerer input spændingen til en ønsket output spænding. Dette gøres for at opnå et bestemt output, hvis den næste komponent skal bruge et specifikt input, eller for at forstærke svage signaler f.eks. fra sensorer eller fysiologiske signaler. Der kan f.eks. bruges en inverterende forstærker, som ses på \figref{inverterendeforstærker}, hvor Vs er det målte signal, der ønsket forstærket og Vo er output. Inputtets forstærkning kaldes gain og er en ratio mellem Rf/Rs, som er de to modstande, der kan ses på \figref{inverterendeforstærker}. \cite{Nilsson2011}

\begin{figure}
\centering
\includegraphics[scale=0.8]{figures/bProblemanalyse/inverterendeforstærker.png}
\caption{En ideel operationsforstærker, som er inverterende koblet, og som kan forstærke input signalet Vs, til et ønsket output signal Vo. \cite{Nilsson2011}
\label{inverterendeforstærker}
\end{figure}
