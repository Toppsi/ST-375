% !TeX spellcheck = da_DK
\subsection{Komparator}\label{Komparatorafsnit}
En analog komparator er et kredsløb, der sammenligner en inputspænding eller -strøm med en eller flere referencespændinger eller -strømme. Rent teknisk gøres dette ved, at komparatoren inverterer inputsignalet $\pm$ 180$^{\circ}$ (V_{-}), imens referencesignalet ikke inverteres (V_{+}). Hvis inputsignalet passer med en eller flere af referenceværdierne, vil de tilknyttede komponenter aktiveres.\fxnote{kilde - Cecilie} Den simpleste komparator er en operationsforstærker.  
Komparatorens output går fra en mætningsgrænse til en anden, når det negative input af operationsforstærkeren passerer igennem 0 V. Dette betyder, at ved et inputsignal på mere end tærskelniveauet vil outputsignalet opnå negativ mætningsgrad. Omvendt ved et inputsignal, som er lavere end tærskelniveauet, vil outputsignalet opnå positiv mætningsgrad \fxnote{skriv hvorfor dette sker rent teknisk}. \cite{webster2009} 

%Det kan være fordelagtig at placere modstanden R1 ved inputsignalet, som det kan ses på \figref{komparator}, da dette minimerer overstyringen\fxnote{Et bedre ord end overstyring} af operationsforstærkeren.  Når inputtet, ved en simpel komparator, når tærskelniveauet og der forekommer støj på inputtet, kan outputsignalet svinge kraftigt. Dette kan imidlertid undgås ved at tilføje to modstande, R2 og R3. \cite{webster2009}
