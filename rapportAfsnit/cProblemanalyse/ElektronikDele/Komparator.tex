H% !TeX spellcheck = da_DK
\subsection{Komparator}\label{Komparatorafsnit}
En komparatorkonfiguration er et kredsløb, der sammenligner en inputspænding  med en referencespænding, som er tærskelværdien for aktiveringen af enten en operationsforstærker eller en komparator. %Dette gøres  ved, at komparatoren inverterer inputsignalet $\pm180^{\circ}$, imens referencespændingen ikke inverteres. 
Hvis inputspændingen er over eller under referencespændingen, hvilket afhænger af designet på komparatorkonfigurationen, vil komparatoren aktiveres, og give en outputspænding. En komparatorkonfiguration kan designes ved brug af komparatorer eller operationsforstærkere.
Komparatorkonfigurationens output går fra en mætningsgrænse til en anden, når det negative input af operationsforstærkeren passerer igennem $0$V. Dette betyder, at ved et inputsignal på mere end tærskelniveauet vil outputsignalet opnå en negativ mætningsgrad. Omvendt ved et inputsignal, som er lavere end tærskelniveauet, vil outputsignalet opnå en positiv mætningsgrad. \cite{webster2009} 

Outputspændingen ved en komparatorkonfiguration med operationsforstærkere, vil ideelt svarende til operationsforstærkerens spændingsforsyningen. Reelt vil outputspændingen være mindre, hvilket afhænger af operationsforstærkerens specifikationer. Ved en komparator vil aktiveringen gøre at den tændes og strømmen vil løbe igennem collectorterminalen og til ground i emitterterminalen. Her skal den positive spændingsforsyning være tilkoblet komparatorens collectorterminal. Den simpleste komparator er en operationsforstærker eller komparator uden modkobling. Derudover kan det ved brug at to komparatorer eller operationsforstærkere designes en vindueskonfiguration. Ved brug af operationsforstærkerer, hvor den ene skal være inverterende og den anden ikke-inverterende,  forbindes de to output så den ene operationsforstæker aktiverer og den anden deaktiverer. Ved komparatoren desigens vindueskonfiguration, således at den ene komparators emitterterminal er forbundt til den andens collectorterminal. Derved skal begge komparatorer være tændt før en strøm kan løbe og f.eks. en LED kan blive aktiveret. En komparatorkonfiguration kan desuden designes med hysterese, hvliket er en modkolblingsfunktion. Hysterese gør at noget outputspændingen løber tilbage til input over en modstand, og derved bliver komparatorkonfiguration ved med at være tændt en kort tid, efter inputsignalet falder under tærskelværdien. Dette vil betyde at tærskelværdien for slukning er lidt lavere end tærskelværdien for aktivering og derved opnås en mere stabil aktivering, hvis inputsignalet svinger meget omkring tærskelværdien. \cite{webster2009, Storr2015} \\
  


%Det kan være fordelagtig at placere modstanden R1 ved inputsignalet, som det kan ses på \figref{komparator}, da dette minimerer overstyringen\fxnote{Et bedre ord end overstyring} af operationsforstærkeren.  Når inputtet, ved en simpel komparator, når tærskelniveauet og der forekommer støj på inputtet, kan outputsignalet svinge kraftigt. Dette kan imidlertid undgås ved at tilføje to modstande, R2 og R3. \cite{webster2009}

