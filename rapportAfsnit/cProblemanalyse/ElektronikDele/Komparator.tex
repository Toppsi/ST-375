% !TeX spellcheck = da_DK
\subsection{Komparator}\label{Komparatorafsnit}
En analog komparator er et kredsløb, der sammenligner en inputspænding eller -strøm med en eller flere referencespændinger eller -strømme. Rent teknisk gøres dette ved, at komparatoren inverterer det ene signal $\pm$ 180$^{\circ}$ (V_{-}), hvor det andet signal ikke inverteres (V_{+}).
Komparatorens output går fra en mætningsgrænse til en anden, når det negative input af operationsforstærkeren passerer igennem 0 V. Dette betyder, at ved en inputspænding på mere end tærskelniveauet vil outputspændingen opnå negativ mætningsgrad. Omvendt ved en inputspænding, som er lavere end tærskelniveauet, vil outputspændingen opnå positiv mætningsgrad \fxnote{skriv hvorfor dette sker rent teknisk}. Den simpleste komparator er en operationsforstærker. \cite{webster2009} 
Når en komparator modtager et inputsignal vil det blive sammenlignet med nogle definerede tærskelværdier angivet i volt eller ampere. Hvis en eller flere af værdierne opnås, vil komparatoren aktivere de tilknyttede komponenter.\fxnote{kilde - Cecilie} \\
%Det kan være fordelagtig at placere modstanden R1 ved inputsignalet, som det kan ses på \figref{komparator}, da dette minimerer overstyringen\fxnote{Et bedre ord end overstyring} af operationsforstærkeren.  Når inputtet, ved en simpel komparator, når tærskelniveauet og der forekommer støj på inputtet, kan outputsignalet svinge kraftigt. Dette kan imidlertid undgås ved at tilføje to modstande, R2 og R3. \cite{webster2009}
