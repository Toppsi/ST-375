% !TeX spellcheck = da_DK
\subsection{Accelerometer}\label{Accafsnit}
Et accelerometer er en elektromekanisk enhed, som måler hastigheden for et objekt, når det ændre sin position. Dette kaldes accelerationen og kan måles enten dynamisk eller statisk. Et accelerometer bennyttes til mange forskellige ting i f.eks. biler, mobiltelefoner, fly og computere, hvor de har betydning for forskellige ting. I Appels nye bærbare computere benyttes et accelerometer f.eks. til at dektere, hvis computeren tabes. I så fald vil accelerometret koble hard driven fra, så den ikke bliver beskadiget. \cite{Inc.2015,Academic2015c}\\
Et accelerometer benytter sig af kraften omkring sig til at udregne accelerationen vha. Newtons anden lov, der lyder således: \textit{kraft = masse x acceleration}. Der måles på en kendt masse, hvis udsving er proportional med accelerationen. Massen påvirker en strain-gauge eller et piezoelektrisk krystal, hvorved udsvinget omsættes til et elektrisk signal. \cite{Academic2015c,Krag2015} 

Outputtet fra et accelerometer er typisk en varierende elektrisk spænding eller en forskydning af en graf hen over en fast skala. For dette projekt vil det være mest relevant at måle dynamisk acceleration, da der fokuseres på opretholdelse af balance. Derved vil begge outputs være essentielle, da det elektriske signal kan være den analoge detektor, mens det grafiske signal kan være det digitale signal. \cite{Academic2015c}

%EVT. FINDE BEDRE EKSEMPEL END DET MED APPLE'S COMPUTERE
%NÅR VI FÅR VORES ACCELEROMETER KAN DER SKRIVES NOGET TIL FRA DATABLADENE - F.EKS. TABELLER
%HVILKET SIGNAL MÅLES? (ENHED PÅ DET DER KOMMER UD)
%EVT. KOMME MED EKSEMPEL PÅ HVORDAN SYSTEMET KAN ANVENDES IFT. BALANCEMÅLING
