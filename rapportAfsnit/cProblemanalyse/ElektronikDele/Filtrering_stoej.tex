% !TeX spellcheck = da_DK
\subsection{Filtrering}\label{Filterafsnit}
Filtrering er et værktøj indenfor signalbehandling, som anvendes i det biologiske signals frekvensdomæne. Formålet med at filtrere et signal er at dæmpe uønskede signaler, der ikke tilhører det signal, der undersøges. Filtret opdeler signalet i såkaldte bånd: Pasbånd, hvor frekvenserne frit passerer igennem filteret uden påvirkning, samt stopbånd, hvor frekvenserne dæmpes, så de ikke har indflydelse på signalet. 
Der findes flere forskellige typer af filtre, der afhænger af, hvilke frekvenser der skal fjernes fra det målte signal \cite{Devasahayam2000}:

\begin{itemize}
	\item Lavpasfiltret: Anvendes til at dæmpe frekvenser over den valgte knækfrekvens. 
	\item Højpasfilteret: Anvendes, modsat lavpasfiltret, til at dæmpe frekvenser under den valgte knækfrekvens. 
	\item Båndpasfilteret: Er en kombination af et lav- og højpasfilter.  Her defineres et interval, hvormed de frekvenser der ligger udenfor intervallet vil blive dæmpet.
	\item Båndstopfilteret: Fungerer, modsat båndpasfilteret, ved at dæmpe signaler indenfor det definerede frekvensområder. 
\end{itemize}
  
\noindent I forbindelse med signalbehandling kan flere af filtrene anvendes samtidig \cite{Devasahayam2000}. Princippet i de fire filtertyper er illustreret på \figref{filtertyper}.
\begin{figure}[H]
\centering
\includegraphics[scale=0.5]{figures/bproblemanalyse/filtertyper2.png}
\caption{På figuren ses de fire nævnte filtertyper. \cite{Sedra2010}}
\label{filtertyper}
\end{figure}
\noindent Filtrene kan desuden inddeles i forskellige ordener afhængigt af, hvor stejl filtreringskurven er, dvs. hvor meget signalet skal dæmpes pr. dekade \cite{Sedra2010}.

\subsection{Støj}\label{StoejAfsnit}
Støj er den uønskede del af et opsamlet signal, der ikke har nogen relation til signalet der undersøges. Signaler, der er fordelt udover et frekvensspektrum, kan filtreres for støj vha. de tidligere beskrevne filtre. \cite{Devasahayam2000,Wolf2004}
Støj kan inddeles i forskellige typer, som typisk vil forekomme:

\begin{itemize}
\item Elektriske signaler: Dette er bl.a. $50$Hz støj, som er en frekvens fra elnettet. Denne frekvens kan gå ind og påvirke de biologiske signaler, der måles på. Hvis der er flere $50$Hz kilder, der interagerer, kan det give ekko ved eksempelvis $100$Hz og $150$Hz. \fxnote{kilde??}
\item Ledninger: Kan fungere som antenner, der opfanger $50$Hz støj og andre former for støj. Problemet bliver større jo længere ledningen er. Derudover kan der forekomme støj, hvis ledningerne bevæges under optagelse af signalet. Denne form for støj er lavfrekvent. \cite{webster2009}
\item Magnetfelt: Kan påvirke ledningerne og inducere strømmen, hvilket skaber støj. Jordens magnetfelt kan f.eks. påvirke ledningerne. For at mindske støjen kan ledningerne snoes eller flettes sammen. \cite{Wolf2004}
\end{itemize} 