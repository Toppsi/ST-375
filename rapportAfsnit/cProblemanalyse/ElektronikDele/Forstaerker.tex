% !TeX spellcheck = da_DK
\section{Signalbehandling}
Et biologisk signal skal behandles for at kunne give et feedback samt et digitalt output. For at kunne behandle et signal fra et accelerometer kræves der hhv. en forstærker, filtre, komparator samt ADC. Der kan anvendes andre komponenter til signalbehandling ift. hvad accelerometret skal benyttes til, men de nævnte vil blive benyttet i dette projekt. 

\subsection{Forstærker}
En forstærker kan benyttes til at ændre inputtet til et ønsket output. Dette kan gøres ved at kombinere en operationsforstærker med modstandere, der derved kan skalere, ændre fortegn på, addere og subtrahere signalet. Der findes fire forskellige forstærkningskredsløb til at udføre de nævnte opgaver: \cite{Nilsson2011}
\begin{itemize}
\item Inverterende forstærkningskredsløb: Benyttes til at invertere signalet, samtidig med det skaleres. Inverteringen af signalet betyder, at der ændres fortegn på signalet.
\item Summerende forstærkningskredsløb: Fungerer ligesom det inverterende forstærkningskredsløb med den undtagelse, at input signaler summeres.
\item Ikke-inverterende forstærkningskredsløb: Benyttes kun til at skalere input signalet.
\item Differens forstærkningskredsløb: Benyttes til at trække to input signaler fra hinanden, så det bliver muligt at se forskellen\cite{Nilsson2011}. Der findes forskellige typer af differensforstærkning, herunder et kredsløb med en enkelt operationsforstærker samt en såkaldt instrumenteringsforstærker. I instrumenteringsforstærkeren indgår yderligere to operationsforstærkere, for at lave inputbuffere til den oprindelige operationsforstærker.\fxnote{Spørg Erika om bogens titel}  
\end{itemize} 

For at forstærke signalet fra accelerometeret benyttes operationsforstærkeren, der skalerer input-spændingen til en ønsket output-spænding. Dette gøres for at opnå et bestemt output, hvis den næste komponent skal bruge et specifikt input eller for at forstærke signaler med lav frekvens eller amplitude. Der kan f.eks. bruges en inverterende forstærker, som ses på \figref{invf}, hvor Vs er det målte signal, der ønskes forstærket og Vo er output. Inputtets forstærkning kaldes gain og er en ratio mellem Rf/Rs, som er de to modstande, der kan ses på \figref{invf}. \cite{Nilsson2011}

\begin{figure}[H]
\centering
\includegraphics[scale=0.6]{figures/bProblemanalyse/inverterendeforstaerker.png}
\caption{En ideel operationsforstærker, som er inverterende koblet, og som kan forstærke input signalet Vs, til et ønsket output signal Vo. \cite{Nilsson2011}}
\label{invf}
\end{figure}


%INDLEDE MED KORT FORKLARING PÅ HVAD EN FORSTÆRKER EGENTLIG ER OG HVILKE FORSKELLIGE TYPER DER FINDES
%UNDERSTREGE AT VI GODT VED AT DER ER FLERE FORSKELLIGE TYPER AF FORSTÆRKERE..