\subsection{Filtrering}
Filtrering er et værktøj indenfor databehandling, som anvendes i det målte signals frekvensdomæne. Formålet med at filtrere et målt signal er at fjerne uønskede frekvenser, som ikke tilhører det signal der ønskes undersøgt. Filtret kan opdele signalet i såkaldte bånd: Pasbånd, hvor frekvenserne frit passerer igennem filteret uden påvirkning, samt stopbånd hvor frekvenserne dæmpes, så de ikke har indflydelse. \fxnote{Knækfrekvens!}
Der findes flere forskellige typer af filtre, afhængigt af hvilke frekvenser der skal fjernes fra det målte signal \cite{Devasahayam2000}:

\begin{itemize}
	\item Lavpasfiltret anvendes til at frasortere høje frekvenser. Dette gøres ved at dæmpe de frekvenser som ligger over knækfrekvensen.
	\item Højpasfilteret anvendes, modsat lavpasfiltret, til at frasortere lave frekvenser ved at dæmpe signalet under knækfrekvensen.
	\item Båndpasfilteret er en kombination af et lav- og højpasfilter.  Her defineres et interval, hvormed de frekvenser der ligger udenfor intervallet vil blive dæmpet.
	\item Båndstopfilteret fungerer modsat båndpasfilteret ved at dæmpe specifikt definerede frekvensområder. Frekvenserne udenfor det definerede område påvirkes ikke. 
\end{itemize}
  
Princippet i de fire filtertyper er illustreret på \fxnote{Referer til figur}.

I forbindelse med databehandling kan flere af filtrene anvendes samtidig. \cite{Devasahayam2000}

\subsubsection{Støj}
Støj er en uønsket del af et opsamlet signal, som ikke har nogen relation til det signal som ønskes målt\fxnote{Indsæt kilde}. Signaler der er fordelt udover et frekvensspektrum kan filtreres for støj vha. de tidligere beskrevne filtre. \cite{Devasahayam2000}
Støj kan inddeles i flere forskellige typer.  
 

%RET AFSNITTET OM FILTRERING, SÅ DER IKKE STÅR AT 'HØJE' OG 'LAVE' FREKVENSER FRASORTERES - ISTEDET SKRIVE AT DET ER FREKVENSERNE OVER ELLER UNDER KNÆKFREKVENSEN
%STØJ: SKRIVE OM STØJ GENERELT - 50HZ OG STØJ FRA LEDNINGER
%KOBLE FILTRERING OG STØJ SAMMEN
%FIND FIGURER - KIG I DET GAMLE PROJEKT
% 
 
