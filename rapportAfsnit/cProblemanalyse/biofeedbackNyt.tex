\subsection{Biofeedback}
Biofeedback blev introduceret i slutningen af 1960\fxnote{ved ikke hvilken kilde der er brugt her - står i første afsnit om biofeedback}. Metoden har siden været anvendt i forbindelse med rehabilitering af patienter, ved at give dem informationer om biologiske parametre i deres krop, som relaterer til deres sygdom/skade. 
Overordnet kan responsen fra biofeedback inddeles i to grupper: Direkte feedback, hvor det målte signal udtrykkes som eksempelvis en nummerisk værdi, eller transformeret feedback, hvor det målte signal kontrollerer et stykke udstyr, der kan give patienten et bestemt signal. Dette signal kan f.eks. være auditivt eller visuelt.
Derudover kan biofeedback også deles ind i en fysiologisk og en biomekanisk del.\cite{Giggins2013}

\subsubsection{Fysiologisk biofeedback}
Fysiologisk biofeedback omfatter måling på forskellige systemer. Der kan måles på det neuromuskulære system, det kardiovaskulære system samt respirationssystemet. 
Herunder hører f.eks. EMG-feedback, hvor myoelektriske signaler omsættes til et signal til patienten, hvormed der kan opnås bevidsthed om f.eks. svage muskler. Desuden findes HR-feedback, hvor patientens hjerterytme måles og udtrykkes på et stykke udstyr der er synligt for patienten. Dette anvendes f.eks. i forbindelse med træning. Til patienter med sygdom i respirationssystemet kan anvendes respiratorisk feedback, som udtrykker værdier for de respiratoriske evner.\cite{Giggins2013}
\fxnote{Noget om hvordan signalerne kan måles - elektroder} 

\subsubsection{Biomekanisk biofeedback}
Ved biomekanisk biofeedback måles der ikke på kroppens enkelte systemer, men på generelle mekaniske egenskaber\fxnote{vil gerne have et bedre udtryk her..}, såsom hvordan kroppen bevæger sig og på selve holdningen.  
Ved biomekanisk biofeedback findes der flere forskellige typer måleudstyr, herunder inerti-sensorer, kraftplader og kamerasystemer, der alle kan måle forskellige mekaniske parametre. Studier har vist, at de forskellige metoder indenfor biomekanisk biofeedback generelt har en positiv effekt på rehabiliteringen af patienter med balanceproblemer. Eksempelvis viste et studie\fxnote{rigtig kildehenvisning??} positive resultater, da effekten af inerti-sensorer i forbindelse med rehabilitering af patienter med kropssvaj, blev testet. Her skulle patienterne på samme tid udføre motoriske og kognitive handlinger imens de gik. Imens modtog de biofeedback ud fra gyroskopmålinger.\cite{Giggins2013} Gyroskopet måler accelerationen i en bestemt retning\cite{Hjaelpemiddelbasen}. Det viste sig, at især de yngre patienter havde gavn af at modtage signaler omkring deres kropshældning imens de udførte opgaverne. De ældre patienter havde gavn af biofeedbacken imens de kun udførte én af opgaverne - det blev forvirrende for dem at skulle udføre to, imens de skulle fokusere på balancen.\cite{Giggins2013} 
Et andet eksempel på positiv effekt af inddragelse af biomekanisk biofeedback er kraftmåling i forbindelse med rehabilitering af patienter med pushersyndrom.
Der findes et hånddynamometer, der kan benyttes til at måle styrkeforskellen i hhv. højre og venstre hånd, hvilket i nogen tilfælde kan være fordelagtigt at benytte for patienter med pusher-syndrom, da patienterne derved gøres opmærksomme på styrken i deres følelsesløse side.\cite{Hjaelpemiddelbasen}

\subsubsection{Krav til patienter ved anvendelse af biofeedback}
Hvis en patient skal have gavn af biofeedback kræver det, at patienten har en kognitiv kapacitet til at følge instruktionerne under behandlingssessioner og fastholde læring fra session til session. Derudover kræves en neurologiske kapacitet til at genskabe frivillig kontrol. \cite{Middaugh1989} \\


 


