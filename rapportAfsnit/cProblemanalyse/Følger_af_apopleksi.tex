\section{Følger af apopleksi }
Apopleksi kan forekomme pludseligt og uden mulighed for, at den ramte kan forberede sig på følgerne og deres eventuelle reduceret fysiske evner og formåen. Dette kan være tilfældet ved andre kroniske sygdomme, hvor progressionen ofte sker gradvist. Derfor kan der være flere og/eller alvorlige følger af en hjerneblødning eller blodprop. \citep{Muus2008}

Der findes forskellige måder, hvorpå patienten bliver hæmmet, hvilket både kan være sproglige følger, som afasi, og ikke-sproglige følger. Af de ikke-sproglige følger kan der forekomme; nedsat opmærksomhed eller lammelse i dele af kroppen, neglekt, manglende funktion i arme og hånd og apraksi. Desuden kan patienten opleve mangel på sygdomserkendelse og manglende initiativ. Der kan forekomme en kombination af flere af følgerne på samme tid. \citep{Christensen&Zielke2008}\citep{Sundhed.dk}

Hvilke problemer patienten oplever, som følge af apopleksi, afhænger af, hvor apopleksien forekommer i hjernen, da der er flere forskellige blodårer, der leverer blod til hjernen. Det har derfor stor betydning, hvilken blodåre apopleksien rammes i, for hvilke områder, og dermed hvilke funktioner, der bliver hæmmet eller skadet. Derfor er omfanget af hæmmelsen eller skaden afhængig af f.eks. tiden, der går uden ilt, størrelsen af blødningen, trykket i blodåren og hvor apopleksien rammer.\citep{Michael-Titus2010}

De sproglige konsekvenser, som følge af apopleksi, kaldes afasi.  Afasi opleves efter en måned hos 20\% af de ramte og forekommer oftest ved skade i venstre temporal- og frontallap. De sproglige følger af apopleksi kan skade funktionen til at skrive og tale, men også evnen til at forstå og læse andres tale og skrift. I hvilken grad de sproglige funktioner er berørt kan varierer mellem de apopleksi ramte, da nogle oplever formuleringsproblemer, mens andre oplever global afasi. Dette gør, at de ramte i nogle tilfælde er helt ude af stand til at kommunikerer verbalt eller kun kan fremsige enkle ord med en vis usikkerhed omkring ordenes betydning. De sproglige konsekvenser kan midlertidig bedres med tiden, og flere patienter opnår et kommunikationsniveau, som gør det muligt at begå i hverdagen.\citep{Muus2008}

En anden konsekvens af apopleksi er neglekt. Neglekt er en følge, hvor den ramte ikke er opmærksom på egen krop, dvs. den ramte opfatter ikke den ene side af kroppen. Neglekt kan forekomme på to måder; en visuel neglekt og en kropslig neglekt, som ofte forekommer samtidig. Ved visuelle neglekt mangler patienten sanseindtryk fra den påvirket side af F.eks. er patienter ikke opmærksom på den ene side af teksten, når de skal læse, på trods af at selve synet er normalt. Derudover kan patienten opleve, at man kun spiser fra den ene del af tallerkenen. Ved den kropslige neglekt har patienten manglende kropsbevidsthed. Patienter med kropslig neglekt har ofte normal følelse i den syge side af kroppen - indtrykkene bemærkes men registres ikke i hjernen. Patienten glemmer at klæde den syge side af kroppen ordentligt på eller barberer kun halvdelen af ansigtet. En alvorlig følge af dette er, at patienten kan udføre ubevidst skade på sig selv, f.eks. ved at støde ind i ting med den syge side eller ikke være opmærksom på, at benene ikke kan bære. Derved kan der på længere sigt forekomme ergonomiske skader andre steder i kroppen. \citep{Sundhed.dk} % Der skal lægges meget vægt på dette afsnit, fordi det er det, som er vigtigt for vores projekt. Kunne evt. bygges mere på dette afsnit, eller gøre sådan at det bliver mere centreret?

Endvidere kan apopleksi skade sensoriske såvel som motoriske funktioner. Dette opleves som manglende funktionerne i arm, hænder, ben og fødder. De motoriske og sensoriske konsekvenser er de hyppigst forekommende følger hos apopleksi ramte og kan medføre, at patienten har problemer med at udføre orienterede handlinger. Dvs. at patienten kan have problemer med forholdet mellem egen krop og ting omkring sig, med kropdelenes indbyrdes forhold og de kan have problemer med højre og venstre. Desuden oplever nogle patienter at have problemer med at afstands bedømme. De motoriske følger kan medføre upræcise bevægelser, general stivhed i bevægelserne, problemer ved opstart af bevægelser, langsommere bevægelser, færre spontane bevægelser samt rystelser. Patienten kan derved risikere at være ude af stand til eller har store problemer med at gå uden hjælp.\citep{Sundhed.dk} \citep{DSfA2009}

\section{Balance og Pusher syndrom}
Som følge af apopleksi oplever patienter problemer med balancen. Balancen er vigtig for mennesker, da den opretholder kropstillingen ved hjælp af ukontrollerede bevægelser, hvilket også er med til at bevægelse er muligt uden fald. For at opretholde balancen bliver kropsvægten så vidt mulig fordelt omkring kroppens akse og de vægtbærende legemer, såsom fødder i oprejst position og gluteal musklerne i siddende position.\citep{Nichols1997}   
For apopleksi ramte patienter opleves der ofte problemer med balancen i den transversale retning i det frontale plan, hvor patienterne hænger mod deres syge og svage siden uden de er opmærksomme på det. Dette kaldes for ”Pusher Syndrom”. Da patienter med Pusher Syndrom ikke registrerer, at deres krop hænger, kan dette være med til at besværliggøre funktioner i dagligdagen og giver øget risiko for faldeulykker. Derudover har patienten ofte problemer med at læne sig fremad, samt risiko for faldeulykker i både stående, gående og siddende stilling. \citep{Karnath2003}

% En visuel skala på, hvor mange der får de forskellige ting - sproglig skade, pusher-syndrom osv? (Hvis man kan finde det) Vise hvor det sker henne i hjernen?
% OBS - Tjek for lange sætninger og evt. omformuler dem.
% Fokusere lidt mere på balancen i neglect afsnittet.
% Mangler lidt overskrifter på de forskellige afsnit. Pas på at vi ikke fokuserer på pusher-syndom i det sidste afsnit.
% Hvorfor er det egentlig farligt af have apopleksi / pusher-syndrom?