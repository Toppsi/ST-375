% !TeX spellcheck = da_DK
\section{Biofeedback}
Biofeedback har været anvendt i forbindelse med rehabilitering af patienter ved at give dem informationer om biologiske parametre i deres krop, som relaterer til sygdommen eller skaden. \cite{Giggins2013} Metoden blev introduceret i slutningen af 1960 \cite{Glanz1995} .
Overordnet kan responsen fra biofeedback inddeles i to grupper: Direkte feedback, hvor det målte signal udtrykkes som eksempelvis en nummerisk værdi, eller transformeret feedback, hvor det målte signal kontrollerer et udstyr, der kan give patienten et bestemt signal. Dette signal kan f.eks. være auditivt eller visuelt.
Derudover kan biofeedback også deles ind i en fysiologisk og en biomekanisk del. \cite{Giggins2013}

\subsection{Fysiologisk biofeedback}
Fysiologisk biofeedback omfatter måling på forskellige kropslige systemer. Det kan blandt andet måles på det neuromuskulære system, det kardiovaskulære system samt respirationssystemet. 
Herunder hører f.eks. elektromyografisk feedback, hvor myoelektriske signaler omsættes til et signal til patienten, hvormed der kan opnås bevidsthed om f.eks. svage muskler. Desuden findes hjerterytme feedback, hvor patientens hjerterytme måles og udtrykkes på et udstyr der er synligt for patienten. Til patienter med sygdom i respirationssystemet kan anvendes respiratorisk feedback, som udtrykker værdier for de respiratoriske evner.\cite{Giggins2013}
\fxnote{Noget om hvordan signalerne kan måles - elektroder} 

\subsection{Biomekanisk biofeedback}
Ved biomekanisk biofeedback måles der ikke på kroppens enkelte systemer, men på generelle motoriske egenskaber, såsom hvordan kroppen bevæger sig og på selve kropsholdningen.  \\
Ved biomekanisk biofeedback findes der flere forskellige typer måleudstyr, herunder inerti-sensorer, kraftplader og kamerasystemer, der alle kan opfange forskellige motoriske parametre. %Studier har vist, at de forskellige metoder indenfor biomekanisk biofeedback generelt har en positiv effekt på rehabiliteringen af patienter med balanceproblemer. 
Eksempelvis viste et studie positive resultater, da effekten af inerti-sensorer i forbindelse med rehabilitering af patienter med kropssvaj, blev testet. \cite{Giggins2013} Her skulle patienterne på samme tid udføre kognitive- og motoriske handlinger imens de gik. Imens modtog de biofeedback ud fra gyroskopmålinger. \cite{Giggins2013} Gyroskopet blev brugt til at måle accelerationen i en bestemt retning \cite{Hjaelpemiddelbasen}. Det viste sig, at især de yngre patienter havde gavn af at modtage signaler omkring deres kropshældning imens de udførte opgaverne. De ældre patienter havde gavn af biofeedbacken imens de kun udførte én af opgaverne - det blev forvirrende for dem at skulle udføre to, imens de skulle fokuserer på balancen. \cite{Giggins2013} \\
Et andet eksempel på positiv effekt med inddragelse af biomekanisk biofeedback er kraftmåling i forbindelse med rehabilitering af patienter med Pusher Syndrom.
Der kan benyttes et hånddynamometer til at måle styrkeforskellen i hhv. højre og venstre hånd, hvilket i nogle tilfælde kan være fordelagtigt at benytte for patienter med Pusher Syndrom, da patienterne derved gøres opmærksomme på styrken i deres følelsesløse side. \cite{Hjaelpemiddelbasen}

\subsection{Krav til patienter ved anvendelse af biofeedback}
Hvis en patient skal have gavn af biofeedback kræver det, at patienten har en kognitiv kapacitet til at følge instruktionerne under behandlingssessioner og fastholde læring fra session til session. Derudover kræves en neurologiske kapacitet til at genskabe frivillig kontrol, samt motorisk kapacitet, hvis patienten skal opnå genskabelse af evt. tabte fysiske egenskaber. \cite{Middaugh1989}  Kravene til patienten ved anvendelse af et medicinsk instrument kan være yderst forskellige alt efter instruments virkemåde. Det er midlertidig vigtigt at når instrumentet bliver designet at det bliver tilpasset til de patienter, som det skal anvendes på, og de begrænsninger patienterne har, både sensoriske og motoriske.\\

%begrænsning af teknologiafsnit: 
% - vi skal begrænse os til accelerometeret. Så derfor bliver vi også nødt til at beskrive det her, som det sidste. Vi beskriver andre biofeedbacksystemer, men ikke accelerometeret. 


%\section{Biofeedback}
%Biofeedback teknologien blev introduceret i slutningen af 1960. Denne teknologi gør det muligt for en patient at opnå bevidst kontrol over latent nervefunktion, samt forbedre rehabiliteringen. Dette sker ved et auditiv eller visuel tegn på, at deres bevægelse har aktiveret en neuromuskulær genvej. Biofeedback teknologien har bla. haft stor succes i behandling af apopleksi patienter med fækal- og urin inkontinens. Biofeedback kan også være gavnligt for en apopleksipatient med balanceproblemer, herunder pusher-syndrom forårsaget af følelsesløshed i den ene side af kroppen. Patienten har stadigvæk kontakt til hele sin krop, men kan ikke fornemme den ene halvdel. Hvis patienten underrettes om, at han/hun har forkert kropsholdning, som vil lede til mistet balance, kan patienten pga. et signal nå at rette op på dette, inden et uheld indtræffer. \\ 

%Hvis en patient skal have gavn af biofeedback teknologien kræver det, at patienten har en kognitiv kapacitet til at følge instruktionerne under behandlingssessioner og fastholde læring fra session til session. Derudover kræves en neurologiske kapacitet til at genskabe frivillig kontrol. \cite{Middaugh1989} \\
 
%Der findes apparater og sensorer til at opfange et fysiologisk signal, der kan bruges til bevægelses-, styrke- og balancetræning, og som afgiver et biofeedback tilbage til patienten. Der findes f.eks. et hånd dynamometer, der kan benyttes til at måle styrkeforskellen i hhv. højre og venstre hånd. Dette kan være fordelagtigt at benytte for en apopleksipatient med balanceproblemer, herunder eksempelvis pusher-syndrom, da patienten derved gøres opmærksom på styrken i sin følelsesløse side. Herudover findes et smerte algometer, som kan gemme målinger af smerte, en trykplade, som kan måle fordelingen af en persons kropsvægt under forskellige øvelser, gyroskop, som kan måle accelerationen i en bestemt retning, accelerometer, som kan måle kropshældning, når sensoren er placeret på patienten, samt diverse elektroder, som f.eks. kan måle muskelaktivitet. Valget af hvilke apparater og sensorer der er fordelagtige afhænger af patientens tilstand f.eks. hørelse og følsomhed, samt sværhedsgraden af hjerneskaden og hvilke funktioner, der skal genoptrænes. \cite{Hjaelpemiddelbasen} %mangler en kilde på, hvad de forskellige gør. - skal vi ikke have dette i en tabel?

%\begin{table}[Biofeedback]
%\centering
%\caption{My caption}
%\label{my-label}
%\begin{tabular}{lllll}
%{\bf Apparatur}  & {\bf Funktion}                                                                                                                                                                                                                                         %&  &  &  \\
%Hånd dynamometer & Måler styrkeforskellen i hhv. højre og venstre hånd. Dette kan være fordelagtigt at benytte for en %apopleksipatient med balanceproblemer, herunder eksempelvis pusher-syndrom, da patienten derved gøres opmærksom på %styrken i sin følelsesløse side. &  &  &  \\
%Smerte algometer & Målinger af smerte, en trykplade, som kan måle fordelingen af en persons kropsvægt under %forskellige øvelser                                                                                                                                           %&  &  &  \\
%Gyroskop         & Måler accelerationen i en bestemt retning                                                                                                                                                                                                              %&  &  &  \\
%Accelerometer    & Måler kropshældning, når sensoren er placeret på patienten, samt diverse elektroder, som f.eks. kan måle muskelaktivitet                                                                                                                               &  &  & 
%\end{tabular}
%\end{table}


%[1] - http://academic.eb.com.zorac.aub.aau.dk/EBchecked/topic/569347/stroke
%[2] - https://www.sundhed.dk/borger/sygdomme-a-aa/hjerte-og-blodkar/sygdomme/apopleksi/		   apopleksi-rehabilitering/
%[3] - http://link.springer.com.zorac.aub.aau.dk/article/10.1007/BF00999338    http://www.archives-pmr.org/article/S0003-9993(95)80503-6/abstract
%[4] - http://www.hmi-basen.dk/r4x.asp?linktype=iso&linkinfo=044824&P=1              
%[5] - Sundhedsstyrelsen, 2011. Sundhedsstyrelsen. Forløbsprogram for rehabilitering af voksne med erhvervet hjerneskade. Sundhedstyrrelsen, 2011.


%\section{Rehabilitering}
%Når selve slagtilfældet er stabiliseret og behandlet, er det essentielt, at rehabiliteringen af en apopleksi patient indfindes hurtigst muligt - gerne 1 til 2 dage efter slagtilfældet. Rehabiliteringen kan indebære fysisk, erhvervsmæssig eller tale terapi. Motoriske og sensoriske funktionsproblemer kan lede til balancebesvær for patienten i både siddende, stående og gående stilling. Der er afprøvet adskillige farmakologiske midler og behandlingsstadegier for at forbedre hjernens rehabilitering og motoriske funktioner. F.eks. er der afprøvet, at tildele apopleksi patienter det antidepressive middel Prozac (fluoxetin) i kombination med fysioterapi. Derudover er kortikal stimulation afprøvet, hvor området af hjernen, som kontrollerer motorstyring, modtager elektriske impulser fra en implanteret anordning. Denne mulighed har haft blandede succes oplevelser, men er udelukkende blevet afprøvet på patienter, der har oplevet et alvorligt slagtilfælde. \\ %http://academic.eb.com.zorac.aub.aau.dk/EBchecked/topic/569347/stroke                             % Hvor foregår rehabiliteringen ift. hvor slemt det er?
%Genoptræningen af en apopleksi patient i Danmark dækker områderne direkte træning af funktioner, reorganisering af netværk (ufrivviligt - hjernens proces), kompenserende strategier, ændringer i miljø, social og psykologisk støtte. Genoptræningen omhandler dog ikke kun træning med en ergo- eller fysioterapeut, da plejepersonale til dagens almindelige gøremål også spiller en stor rolle. Patientens daglige rutiner kan være gået tabt under slagtilfældet, og det er derfor utrolig vigtigt at få patienten tilbage i sit vante miljø. Plejepersonale skal hjælpe patienten til at genfinde denne rytme og hjælpe patienten til eventuelt at udføre dagligdags ting på en ny måde. Det kan ske, at patienten ikke længere er i stand til at beherske begge sine hænder til en opgave, hvorved plejepersonalet skal bistå patienten i indlæringen af kun at benytte en hånd. \\
%Apopleksi patienten skal i samarbejde med lægen, sygeplejersken og andet hjælpepersonale opstille nogle mål for sin rehabilitering. Målene skal hverken være for svære eller for lette, så patienten ikke mister sin motivation til genoptræningen. \\ %https://www.sundhed.dk/borger/sygdomme-a-aa/hjerte-og-blodkar/sygdomme/apopleksi/apopleksi-rehabilitering/

%indledning nr. 2
%Når selve slagtilfældet er stabiliseret og behandlet, er det essentielt, at rehabiliteringen af en apopleksi patient indfindes hurtigst muligt - gerne 1 til 2 dage efter slagtilfældet. Rehabiliteringen kan indebære fysisk, erhvervsmæssig eller tale terapi. Motoriske og sensoriske funktionsproblemer kan lede til balancebesvær for patienten i både siddende, stående og gående stilling. Der er afprøvet adskillige farmakologiske midler og behandlingsstadegier for at forbedre hjernens rehabilitering og motoriske funktioner. F.eks. er der afprøvet, at tildele apopleksi patienter det antidepressive middel Prozac (fluoxetin) i kombination med fysioterapi. Derudover er kortikal stimulation afprøvet, hvor området af hjernen, som kontrollerer motorstyring, modtager elektriske impulser fra en implanteret anordning. Denne mulighed har haft blandede succes oplevelser, men er udelukkende blevet afprøvet på patienter, der har oplevet et alvorligt slagtilfælde. [1] \\    
                         
%Rehabiliteringen af en apopleksi patient i Danmark dækker områderne direkte træning af funktioner, reorganisering af netværk (ufrivilligt - hjernens proces), kompenserende strategier, ændringer i miljø, social og psykologisk støtte. Genoptræningen omhandler dog ikke kun træning med en ergo- eller fysioterapeut, da plejepersonale til dagens almindelige gøremål også spiller en stor rolle. Patientens daglige rutiner kan være gået tabt under slagtilfældet, og det er derfor utrolig vigtigt, at få patienten tilbage i sit vante miljø. Plejepersonale skal hjælpe patienten til at genfinde denne rytme og hjælpe patienten til eventuelt at udføre dagligdags ting på en ny måde. Det kan ske, at patienten ikke længere er i stand til at beherske begge sine hænder til en opgave, hvorved plejepersonalet skal bistå patienten i indlæringen af kun at benytte en hånd. \\ 

% Kan de to overstående afsnit skrives sammen, så man starter mere bredt og derefter indskærper sig. Båske bytte om på rækkefælgen i det præsenterede? Husk at lave det gamle stå, hvis man forsøger sig med en omformulering. - 

%\section{Biofeedback}
%Biofeedback teknologien blev introduceret i slutningen af 1960. Denne teknologi gør det muligt for en patient at opnå bevidst kontrol over en voluntær men latent nervefunktion. Dette sker ved en auditiv eller visuel tegn på, at deres bevægelse har aktiveret en neuromuskulær genvej. Biofeedback teknologien har bla. haft stor succes i behandling af apopleksi patienter med fækal- og urin inkontinens. Teknologien kunne også være gavnligt for en apopleksi patient, som har udviklet pusher-syndrom, da dette skaber balanceproblemer pga. følelsesløshed i den ene side af kroppen. Patienten har altså stadig kontakt til hele sin krop, men kan ikke fornemme den ene halvdel. Hvis patienten blev underrettet om, at han/hun havde forkert kropsholdning, som ville lede til at miste balancen, kunne patienten nå at rette op på dette, inden et uheld vil indtræffe. \\
%Hvis en patient skulle have gang af biofeedback teknologien kræver det dog, at patienten har en kognitive kapacitet til at følge instruktionerne under behandlings sessioner og fastholde læring fra session til session. Derudover kræves en neurologiske kapacitet til at genskabe frivillig kontrol. \\ %   http://link.springer.com.zorac.aub.aau.dk/article/10.1007/BF00999338    http://www.archives-pmr.org/article/S0003-9993(95)80503-6/abstract
%Der findes apparater på markedet til bevægelses-, styrke- og balancetræning, som afgiver et biofeedback til patienten. Der findes f.eks. et hånd dynamometer, som kan benyttes til at måle styrkeforskellen i henholdsvis højre og venstre hånd. Dette kan være fordelagtigt at benytte for en apopleksi patient, som har udviklet pusher-syndrom, da patienten derved bliver opmærksom på styrken i sin følelsesløse side. Herudover findes et smerte algometer, som kan gemme målinger af smerte. %http://www.hmi-basen.dk/r4x.asp?linktype=iso&linkinfo=044824&P=1               % Vurder, om det med hånddynamometer er "to far off" og skal slettes.

% Bruges det kun på latente nervefunktioner? Skal ændres lidt, så det ikke kun er latente nerfunktioner, der snakkes om i biofeedback afsnittet. Udbyg mere her.
% Udbyg de forskellige rehabiliteringstyper - undersøg og vurder, om der kan skrives mere på? Gerne med underoverskrifter.
% Hvordan fungerer rehabiliteringen der hjemme?
% Vi skal have fokus på, hvor vi gerne vil hen. Få strikket den røde tråd (balance) ind i teksten, så fokus kommer på det.
% Mangler et overblik over, hvor lang tid rehabiliteringen tager - nogle eksempler på, hvor lang tid rehabiliteringen kan tage.
% Biofeedback afsnittet er måske lidt for specifikt på pusher-syndrom. Skal muligvis skrives om, så det bliver mere generelt med fokus på balance.
% Hvordan skal man give biofeedback til en ældre person, hvis de har dårlig hørelse og følelelse?

