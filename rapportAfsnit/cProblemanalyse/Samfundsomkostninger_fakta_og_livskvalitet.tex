\section{Omkostninger}

I 2010 var der i Danmark 18.041 indlæggelsesforløb forbundet med apopleksi. 2.450 af disse var sequelae, som er følgetilstande fra en hjerneskade. De 18.041 indlæggelsesforløb er inddelt i seks kategorier; Apopleksi: Spontan blødning i hjernen, stod for 1547 af tilfældene; Apopleksi: Spontan infarkt i hjernen, stod for 6832 af tilfældene; Uspecificeret apopleksi, stod for 4049 af tilfældene; diverse, stod for 141 af tilfældene; Sequelae, stod for 2450 af tilfældene; TCI, stod for 4860 af tilfældene [1]. Sequelae tilfældene er ikke nødvendigvis et apopleksi tilfælde, men en anden sygdom der forekom på grund af apopleksi [1]. Indlæggelsesforløbene for apopleksi (ekslusiv sequale) er faldet siden år 2000 [1], hvilket kan ses på Figur 1: %Hele denne beskrivelse er snørlet og uforståeligt med alle de forskellige ; : tegn. Hvilke besrivelser hører til hvad?

Figur 1, graf fra kilden hvor der ses at tilfældene er faldet 

Fordelingen af person, der rammes af apopleksi, er også blevet undersøgt. Det ses her, at mænd står for 9.710 af apopleksi tilfælde, mens kvinderne står for 8.562 [1]. For mændene er 3.436 af tilfældene udgjort af personer under 65 år, mens 6.274 af tilfældene finder sted for personer over 65 år [1]. For kvinderne står personerne over 65 år for 6.300 af tilfældene, mens dem under 65 år står for 2.262 af tilfældene [1]. Det ses ud fra dette, at mænd og kvinder bliver ramt af apopleksi på lige fod, altså er apopleksi et lige stort problem for kvinder som det er for mænd. 

Mænds antal af indlæggelsesforløb stiger voldsomt, når de kommer 55 år. Dette ses ved at mænd fra 45-54 år kun havde 948 indlæggelsesforløb, mens mænd fra 55-64 år havde 2.024 indlæggelsesforløb [1]. Resten af indlæggelsesforløbene kan ses på Figur 2 inddelt efter alder.

Figur 2, tabel med tal

Kvinders antal af indlæggelsesforløb følger mønsteret fra mænds, men det stiger ikke lige så voldsomt. Antallet af indlæggelsforløb fra 45-54 år er på 646, mens det fra 55-64 år er på 1.209 [1]. Det kan ses at der sker en stigning når kvinder bliver over 55, men den er ikke lige så udtalt som mænds stigning. Resten af indlæggelsesforløbene efter aldersinddeling kan ses på figur 3.
% Forkert konklusion herpå med mænd og kvinder, skal omformuleres.

Figur 3, tabel med tal

Indlæggelsesforløbene har forskellig længde. Størstedelen af indlæggelsesforløbene er under 15 dage. Det ses ved at 16.460 af indlæggelsesforløbene er under 15 dage. Herefter kommer indlæggelsforløbene fra 15-28 dage, hvor der finder 1.191 sted af disse. Efter dette er tallet støt faldende, med kun 3 indlæggelsesforløb der er over 150 dage [1]. 



\section{Livskvalitet}

Dette afsnit er baserer på hjerneskader generelt. Dvs. det ikke er sikkert, at apopleksi er årsagen, men det antages, at de samme udfordringer gør sig gældende hos personer, der fik hjerneskader af apopleksi. Derudover skal det noteres, at det ikke er sikkert, at en apopleksiramt får en hjerneskade. 

Det første personer, der har fået en hjerneskade, beskriver er, at hjerneskaden er et brud i deres liv, som de skal lære at forholde sig til [2]. Derudover tager det også tid for nogle ramte at indse, de er blevet ramt af en sygdom, de skal forholde sig til. Hjerneskader går ind og influerer de ramtes humør, personlighed, færdigheder, aktiviteter, samt deres sociale relationer [2]. Uvisheden omkring hjerneskaden, som patienten har fået, skaber usikkerhed hos patienten, som han/hun kan risikere at leve med i lang tid [2]. En hjerneskadets identitet ændres også, da patienten ikke er i stand til at udføre de samme opgaver som tidligere. Derfor bliver en hjerneskadet nødt til at skabe en ny identitet, hvilket for mange kan være svært [2]. Kroppens funktionsændringer gør også, at den ramte kommer til at leve et mere inaktivt og hjemmeorienteret liv end før. En yngre patient er mere ramt af denne forandring i forhold til en ældre patient [2]. Apopleksi ramte kan derudover opleve en kropsspaltning, hvor kroppen bliver et fremmede objekt for den ramte. Et objekt, som kan være svært at styre og ikke gør, som patienten vil [2]. 

Der findes skjulte vanskeligheder for patienter med hjerneskade. Disse omfatter vanskelighed med hukommelse, læsning, regning, samt andre færdigheder, der ikke er let synlige. Disse skjulte vanskeligheder har også en indflydelse på, hvordan patienten opfatter sig selv og kan være med til at nedsætte livskvaliteten for den enkelte [2]. 

Alle disse kropslige og mentale ændringer gør, at det er svært for en hjerneskadet at vende tilbage til sit gamle hverdagsliv. Forandringerne gør det svært at udføre almindelige huslige pligter, såsom rengøring og personlig pleje [2]. De ramte oplever det også som en svær oplevelse at vende tilbage på arbejde. Dette skyldes, udover de kropslige og mentale ændringer, også den træthed, der kan opleves hos en hjerneskadet. Det er derfor vigtigt at følge sig værdsat på jobbet [2]. Den hjerneskadede patient skal også vende tilbage til sine sociale relationer. Dette kan også opleves som en meget hård oplevelse pga. de forandringer, kroppen har gennemgået [2]. Det ses imidlertid, at familierelationerne bliver tættere, mens relationerne til vennerne bliver mindre. Dette er et problem, da gode relationer kan være med til at forbedre rehabiliteringsprocessen og dermed gøre, at den hjerneskade ramte hurtigere kan komme tilbage til et normalt liv [2].

Ud fra ovenstående kan det konkluderes, at hjerneskadede patienter, heriblandt apopleksi ramte, får en nedsat livskvalitet pga. deres sygdom. Dette kan også ses ved, at apopleksi patienter har dobbelt så stor selvmordsrate som baggrundsbefolkningen [2]. Derudover nævner 16\% af apopleksi patienter, at deres livskvalitet er dårlig, 46\% synes den er nogenlunde, mens 38\% synes den er god [2]. Den nedsatte livskvalitet er noget kan føre til vanskeligheder senere i livet, hvilket selvfølgelig skal prøves at undgås. En forbedret livskvaliteten kan skabes ved hurtigere rehabilitering eller forbedret kropslig funktion, som kroppen mistede ved hjerneskaden.  

% Mangler nogle deloverskrifter igennem hele afsnittet.
% Tjek igennem for, hvor kilderne er sat. Man skal kun skrive kilderne, når man kommer med et postulat, ellers behøver det kun stå til sidst i hele afsnittet. Find evt. flere kilder til afsnittet.
% Kan man evt. skrive nogle delkonklusioner på tallene?
% Ligner meget omkostningsafsnittet - er der noget, som skal rykkes rundt eller slettes fra det ene, fordi det ligner det andet?
% Find et specifikt tal på, hvor mange der får følger/mén efter apopleksi.
% Livskvalitet - Nogle af de gamle projekter har skrevet om livskvalitet. Tjek deres kilder og se, om man kan skrive mere til dette afsnit. Der står også noget om det i MTV "hjerneskade rehabilitering" 

[1] - Beskrivelse af dataopgørelse for voksne med apopleksi og TCI med tabeller og grafik. 
[2] - Hjerneskaderehabilitering, Sundhedsstyrelsen
