\section{Omkostninger}
% Der skal skrives en form for indledning her, mainen vil ikke accpetere en section og så en subsection.

\subsection{Indlæggelsesforløb}
I 2010 var der i Danmark 18.041 indlæggelsesforløb forbundet med hjerneskade. Indlæggelsesforløb er inddelt i seks kategorier, spontan blødning i hjernen, spontan infarkt i hjernen, uspecificeret apopleksi, diverse apopleksitilfælde, sequelae og TCI. \fxnote{Ydermere er forløbet for alle apopleksitilfælde inddelt i gruppe for sig selv, dette vil sige alle foruden sequelae og TCI.} Den hyppigste diagnose iblandt apopleksitilfælde var spontan infarkt i hjernen i 2010 med 6832 indlæggelsesforløb, hvilket svare til omkring 38\% af alle forløbene. Den næst hyppigste er TCI som står for omkring 27 \% af forløbene, hvor den laveste andel af indlæggelsesforløb, som er diverse, står for 0,8\% af det samlede antal[1]. \fxnote{tænkte at vi kunne lave en tabel over de forskellige antal og sige at for yderligere kan ses i appendix, ellers kan vi sætte den ind} Indlæggelsesforløbene for apopleksi (ekslusiv sequale) er faldet siden år 2000 [1], hvilket kan ses på Figur 1:

Figur 1, graf fra kilden hvor der ses at tilfældene er faldet  - se figur i dropboxen --> projekt --> problemfasen --> problemanalyse

Indlæggelsesforløbene har forskellig længde. Størstedelen af indlæggelsesforløbene varede under 15 dage, med 16.460 af indlæggelsesforløb. For perioden fra 15-28 dage var der 1.191 forløb. Efter dette er tallet støt faldende, med kun 3 indlæggelsesforløb der er over 150 dage [1].

\subsection{Køn og aldersfordeling}
Hvis der ses på fordelingen af køn og aldersgrupper der rammes af apopleksi, står mændene for 9.710 af alle tilfælde, mens kvinder står for 8.562 [1]. Dette svarer til at mænd står for 53\% af alle tilfældene[3]. Selve aldersfordelingen af personer som rammes af apopleksi, fordeler sig forholdsvis ens hos både mænd og kvinder. Den største andel ses for begge køn for personer over 65 år, hvilket svarer til 65 \% for mænd og  73 \% for kvinder. Ud af dette kan det konkluderes at der sker 8\% flere tilfælde af apopleski for mænd end kvinder. Aldersfordelingen er for begge køn et overtal af personer over 65 år, det ses dog at der er flere kvinder der rammes i en alder over 65 år. % Kan man lave en graf på dette? 

\subsection{Indlæggelsesforløb delt på køn}
Mænds antal af indlæggelsesforløb stiger voldsomt, når de kommer over 55 år. Dette ses ved at mænd fra 45-54 år kun havde 948 indlæggelsesforløb, mens mænd fra 55-64 år havde 2.024 indlæggelsesforløb [1].Dette svarer til en stigning på 113,5 \%. Resten af indlæggelsesforløbene kan ses på \figref{MK} inddelt efter alder.

Kvinders antal af indlæggelsesforløb følger mønsteret fra mænds, men det stiger ikke lige så voldsomt. Antallet af indlæggelsforløb fra 45-54 år er på 646, mens det fra 55-64 år er på 1.209 [1]. Dette svarer til en stigning på 87 \%.

Resten af indlæggelsesforløbene efter aldersinddeling kan ses på \figref{MK}. %%%%%% AFSNITTET SKAL SKRIVES OM SÅ DET PASSER TIL DEN NYE FIGUR. SKriv en beskrivlese i \caption{------HER-----} og sæt en kilde på.

\begin{figure}[H]
	\centering
	\includegraphics[scale=1.0]{figures/bProblemanalyse/MaendKvinder.png}
	\caption{}
	\label{MK}
\end{figure}

Det kan ud fra dette konkluderes at mænd har længere indlæggelsesforløb end kvinder og vil på denne måde koste samfundet mest. Yderligere kan det konkluderes at alderen har indflydelse på både mænd og kvinder i forhold til hvor længe deres indlæggelsesforløb vil blive.

\subsection{Følger af apopleksi}
Patienternes kan som tidligere nævnt rammes af en række følger som blandt andet kan have indflydelse på det kropslig og mentale. Ud fra alle tilfælde af apopleksi lever 75.000 over 18 i år 2011 med følger efter et slagstilfælde[3]. Dette forventes dog at være stigende i takt med at der bliver flere ældre [4]. Antallet der dør af hjerneskader har været stagneret de sidste 10 år før 2011, hvor 14 \% dør inden for 30 dage[3]. Det vil derfor kunne forventes, at der er flere, som kommer ud for en hjerneskade og vil have mén herefter, hvilket gør det vigtigt at fokusere på rehabiliteringen for at kunne genoptræne de forskellige kropslige og mentale mangler.
% Denne subsection skal omdøbes.

\section{Livskvalitet}
Dette afsnit er baseret på hjerneskader generelt. Dvs. det ikke er sikkert, at apopleksi er årsagen, men det antages, at de samme udfordringer gør sig gældende hos personer, der får hjerneskader af apopleksi. Derudover skal det noteres, at det ikke er sikkert, at en apopleksiramt får en hjerneskade. 

Personer der første gang rammes af en hjerneskade, beskriver hjerneskaden, som et brud i deres liv, som de skal lære at forholde sig til. Derudover kan det tage lang tid for patienten at indse, at de er ramt af en sygdom[2]. 
Hjerneskaden går ind og influerer den ramtes humør, personlighed, færdigheder, aktiviteter, samt deres sociale relationer. Ud fra de ovennævnte skader påvirkes patienterne af  en uvished og usikkerhed, som patienterne kan risikere at leve i lang tid med, afhængigt af hvilken grad hjerneskaden har haft[2]. 

\subsection{Identitets ændring}
En hjerneskadets identitet ændres, da patienten ikke er i stand til at udføre de samme opgaver som tidligere. Derfor bliver en hjerneskadet nødt til at skabe en ny identitet, hvilket for mange kan være svært [2]. Kroppens funktionsændringer gør, at den ramte kommer til at leve et mere inaktivt og hjemmeorienteret liv end før. En yngre patient er mere ramt af denne forandring i forhold til en ældre patient [2]. Apopleksi ramte kan derudover opleve en kropsspaltning, hvor kroppen bliver et fremmede objekt for den ramte. Et objekt, som kan være svært at styre og ikke gør, som patienten vil [2]. 

Der findes skjulte vanskeligheder for patienter med hjerneskade. Disse omfatter vanskelighed med hukommelse, læsning, regning, samt andre færdigheder, der ikke er let synlige. Disse skjulte vanskeligheder har også en indflydelse på, hvordan patienten opfatter sig selv og kan være med til at nedsætte livskvaliteten for den enkelte [2]. 

\subsection{Patienternes påvirkning}
Alle disse kropslige og mentale ændringer gør, at det er svært for en hjerneskadet at vende tilbage til sit gamle hverdagsliv. Forandringerne gør det svært at udføre almindelige huslige pligter, såsom rengøring og personlig pleje [2]. De ramte oplever det også som en svær oplevelse at vende tilbage på arbejde. Dette skyldes, udover de kropslige og mentale ændringer, også den træthed, der kan opleves hos en hjerneskadet. Det er derfor vigtigt at følge sig værdsat på jobbet. Den hjerneskadede patient skal vende tilbage til sine sociale relationer,  dette kan opleves som en meget hård opgave pga. de forandringer, kroppen har gennemgået [2]. Det ses imidlertid, at familierelationerne bliver tættere, mens relationerne til vennerne bliver mindre. Dette er et problem, da gode relationer kan være med til at forbedre rehabiliteringsprocessen og dermed gøre, at den hjerneskade ramte hurtigere kan komme tilbage til et normalt liv [2].

Ud fra  det ovenstående kan det konkluderes, at hjerneskadede patienter, heriblandt apopleksi ramte, får en nedsat livskvalitet pga. deres sygdom. Dette kan også ses ved, at apopleksi patienter har dobbelt så stor selvmordsrate som baggrundsbefolkningen [2]. Derudover nævner 16\% af apopleksi patienter, at deres livskvalitet er dårlig, 46\% syntes den er nogenlunde, mens 38\% synes den er god [2]. Den nedsatte livskvalitet er noget der kan føre til vanskeligheder senere i livet, hvilket selvfølgelig skal prøves at undgås. En forbedret livskvaliteten kan skabes ved hurtigere rehabilitering eller forbedret kropslig funktion, som den apopleksiramte patient mistede ved hjerneskaden.  

% Måske skrive et nyt afsnit omkring alder istedet for køn - koble dem sammen og forklare, hvorfor kvinder bliver ramt senere end mænd, og hvorfor der er flest mænd.
% Ændre lidt i rækkefølgen på de forskellige afsnit.
% Se på tiden i afsnittene - nutid, datid.

[1] - Beskrivelse af dataopgørelse for voksne med apopleksi og TCI med tabeller og grafik. 
[2] - Hjerneskaderehabilitering, Sundhedsstyrelsen
[3] - Hjertesagen.dk
[4] - aeldresagen

%Det ses her, at mænd står for 9.710 af apopleksi tilfælde, mens kvinderne står for 8.562 [1]. For mændene er 3.436 af tilfældene udgjort af personer under 65 år, mens 6.274 af tilfældene finder sted for personer over 65 år [1]. For kvinderne står personerne over 65 år for 6.300 af tilfældene, mens dem under 65 år står for 2.262 af tilfældene [1]. Det ses ud fra dette, at mænd og kvinder bliver ramt af apopleksi på lige fod, altså er apopleksi et lige stort problem for kvinder som det er for mænd.
% Spontan blødning i hjernen, stod for 1547 af tilfældene; Apopleksi: Spontan infarkt i hjernen, stod for 6832 af tilfældene; Uspecificeret apopleksi, stod for 4049 af tilfældene; diverse, stod for 141 af tilfældene; Sequelae, stod for 2450 af tilfældene; TCI, stod for 4860 af tilfældene [1]. Sequelae tilfældene er ikke nødvendigvis et apopleksi tilfælde, men en anden sygdom der forekom på grund af apopleksi [1]. 