% !TeX spellcheck = da_DK
\subsection{Kroppens kompenserende bevægelser}
Efter et slagtilfælde kompenserer kroppen for tabt funktionsevne med nye bevægelsesmønstre, der skal erstatte de gamle. Kompensatoriske bevægelser er et resultat af, at kroppen stadigvæk har brug for en givet funktion, men pga. tabt sensorisk- og motorisk funktion ikke kan udføre bevægelsen. Disse kompenserende bevægelsesmønstre kan medføre et funktionelt dårligt resultat og kan være associeret med langsigtede konsekvenser såsom smerte og reduceret funktionsevne. \cite{Takeuchi2012,Leea2009} %F.eks. bevægelsesmønstre under en gangcyklus kan forekomme ukoordinerede og ukontrollerede ift. en gangcyklus foretaget af en rask person. Disse karakteristika er af tre typer og kan identificere gangmønstret for en patient efter et apopleksitilfælde  \cite{Lamontagne2006}:
%\begin{itemize}
%\item Type 1: Spasticitet, karakteriseres ved tidlig aktivitet af legmusklerne under standfasen.
%\item Type 2: Paretisk, er nedsat aktivitet i fleste muskelgrupper.
%\item Type 3: Co-aktivation af muskler, hvilket betyder, at muskelgrupper med forskellig effekt aktiveres på samme tid.
%\end{itemize}
%\fxnote{Angiv navnet først og så funktionen bagefter. Lav selve navnet til en punkt istedet for en prik.}