\section{Behandlinger}
Når en patient rammes af apopleksi er det vigtigt, at patienten kommer i behandling hurtigst muligt. Ved akut blodprop, hvilket vil sige at symptomerne er til stede inden for 4 1/2 time anvendes blodprop-opløsende medicin. Ved at behandle hurtigt vil det være muligt at opløse blodproppen. Andre tilfælde fjernes blodproppen ved brug af et tyndt kateter, som indføres gennem pulsåren op til hjernen. Derudover anvendes blodfortyndende medicin, for at undgå nye tilfælde af apopleksi \cite{Hjerteforeningen2014} \cite{Sundhed2014}.

\subsection{Akut behandling}
Ved mistanke om blodpropper hos patienten er det vigtigt, at der tages kontakt til sygehuset omgående. Patienten bliver her undersøgt ved blodtryk, blodprøver, neurologisk undersøgelse og skanning af hjernen, for at kunne udelukke om der er opstået en blødning eller der er en anden årsag til funktionstabet som er følgen af blodproppen. Dette gøres for at sikre, at patienten får den rette behandling. Hvis blodprop er tilfældet igangsættes en behandling med blodpropopløsende eller blodpropshæmmende medicin. \cite{Hjerteforeningen2014} \cite{Sundhed2014} 

\subsubsection{Trombolyse}
Standard behandling for akutte blodpropper har siden år 2006 været den blodpropsopløsende behandling trombolyse \cite{Hjernesagen2015b}. Selve behandling foregår ved, at der sprøjtes blodpropløsende medicin ind i en blodåre, ofte i armen, hvorefter blodproppen opløses. Denne behandlingen skal helst foregå 6 timer efter blodproppen tilstedeværelse og senest 12 timer efter, da behandlingen ikke vil have nogen indvirkning efter længere tid. Jo hurtigere behandlingen bliver foretaget, jo flere områder i hjernen vil kunne reddes og jo bedre vil patienterne have det. Den akutte behandling med trombolyse finder sted på 12 sygehuse fordelt over de 5 regioner. En risici ved trombolyse kan være blødninger.\cite{Hjernesagen2015b} 

\subsection{Forebyggelse}
En stor del af behandlingen er forebyggelse, da risikoen for en ny blodprop er stor. Der anvendes antikoagulationsbehandlingen (AK-behandling), som er en behandling med blodfortyndende medicin \cite{Kjaergaard2015}. Normalt har kroppen et system som får blodet til at størkne ved blødninger, derudover medvirker kroppen også til at opløse eventuelle blodpropper i blodåresystemet. For en patient med apopleksi, er det nødvendigt at behandles med antikoagulation, da det hæmmer blodets evne til at størkne, hvilket modvirker dannelsen af blodpropper. Der findes to former for AK-behandling, warfarin(Marevan) og nyre orale antikoaglulantia(NOAK) \cite{Kjaergaard2015}. Den primære forskel mellem de to mediciner er, at behandlingen med NOAK gør, at den blodfortyndende effekt ikke behøver at blive kontrolleret ved blodprøver.\cite{Kjaergaard2015}

%[1]http://www.hjerteforeningen.dk/alt-om-dit-hjerte/hjerte-kar-sygdomme/apopleksi/
%[2]http://www.hjernesagen.dk/om-hjerneskader/behandling/trombolyse
%[3] https://www.sundhed.dk/borger/sygdomme-a-aa/hjerte-og-blodkar/sygdomme/apopleksi/behandling-ved-apopleksi/
%[4]https://www.sundhed.dk/borger/sygdomme-a-aa/hjerte-og-blodkar/sygdomme/behandlinger/antikoagulationsbehandling-blodfortyndende-medicin/