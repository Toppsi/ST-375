\section{Behandlinger}
Når en patient rammes af apopleksi, er det vigtigt at komme i behandling hurtigst muligt. Ved akut blodprop, hvilket vil sige at symptomerne er til stede inden for 4 1/2 time, anvendes blodpropopløsende medicin. Ved hurtig behandling vil det være muligt at opløse blodproppen. I andre tilfælde fjernes blodproppen ved brug af et tyndt kateter, som indføres gennem pulsåren op til hjernen. Derudover anvendes blodfortyndende medicin for at undgå nye tilfælde af apopleksi. \cite{Hjerteforeningen2014} \cite{Kruuse2014a}

\subsection{Akut behandling}
Ved mistanke om blodpropper hos patienten er det vigtigt, at der tages kontakt til sygehuset omgående. Patienten bliver her undersøgt ved blodtryksmåling, blodprøver, neurologisk undersøgelse og skanning af hjernen. Dermed kan det udelukkes om der er eventuelle blødninger eller andre årsager til funktionstabet. Dette  sikrer, at patienten får den rette behandling. I tilfælde af blodprop igangsættes en behandling med blodpropopløsende eller blodpropshæmmende medicin. \cite{Hjerteforeningen2014} \cite{Kruuse2014a} 

\subsubsection{Trombolyse}
Standardbehandling for blodpropper har siden år 2006 været trombolyse. Selve behandlingen foregår ved, at der sprøjtes blodpropopløsende medicin ind i en blodåre, ofte i armen, hvorefter blodproppen opløses. Denne behandling skal helst foregå 6 timer efter blodproppens forekomst og senest 12 timer efter, da behandlingen ikke vil have nogen indvirkning efter længere tid. Jo hurtigere behandlingen bliver foretaget, jo flere områder i hjernen vil kunne reddes. Dermed vil patientens fremtidige livskvalitet forbedres. Trombolysebehandlingen finder sted på 12 sygehuse fordelt over de 5 regioner. En risiko ved behandlingen kan være blødninger grundet den blodpropopløsende medicin.\cite{Hjernesagen2015b} 

\subsection{Forebyggelse}
En stor del af behandlingen er forebyggelse, da risikoen for en ny blodprop er stor. Til forebyggelse anvendes antikoagulationsbehandling(AK-behandling), som er en behandling med blodfortyndende medicin. Normalt har kroppen sit eget koagulationsssystem som får blodet til at størkne. Derudover medvirker koagulationssystemet også til at opløse eventuelle blodpropper i blodåresystemet. For apopleksipatienter fungerer koagulationssystemet ikke optimalt, hvilket gør det nødvendigt at behandle med antikoagulation. Dette hæmmer blodets evne til at størkne, hvilket modvirker dannelsen af blodpropper. Der findes to former for AK-behandling, warfarin(Marevan) og nyre orale antikoaglulantia(NOAK). Den primære forskel mellem de to mediciner er, at der ved behandling med NOAK ikke kræver kontrol ved blodprøver.\cite{Kjaergaard2015}

%[1]http://www.hjerteforeningen.dk/alt-om-dit-hjerte/hjerte-kar-sygdomme/apopleksi/
%[2]http://www.hjernesagen.dk/om-hjerneskader/behandling/trombolyse
%[3] https://www.sundhed.dk/borger/sygdomme-a-aa/hjerte-og-blodkar/sygdomme/apopleksi/behandling-ved-apopleksi/
%[4]https://www.sundhed.dk/borger/sygdomme-a-aa/hjerte-og-blodkar/sygdomme/behandlinger/antikoagulationsbehandling-blodfortyndende-medicin/