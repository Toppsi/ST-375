% !TeX spellcheck = da_DK
\section{Behandlinger}
Når en patient rammes af apopleksi, er det vigtigt at komme i behandling hurtigst muligt. Ved akut iskæmisk apopleksi, hvilket vil sige at symptomerne er til stede inden for fire en halv time, anvendes blodpropopløsende medicin. Ved hurtig behandling vil det være muligt at opløse blodproppen. I andre tilfælde fjernes blodproppen ved brug af et tyndt kateter, som indføres gennem arterien op til encephalon. Derudover anvendes blodfortyndende medicin for at undgå nye tilfælde af apopleksi. \cite{Hjerteforeningen2014, Kruuse2014a} 

\subsection{Akut behandling}
Ved mistanke om iskæmisk apopleksi er det vigtigt, at der tages kontakt til sygehuset omgående. Patienten bliver her undersøgt ved blodtryksmåling, blodprøver, neurologisk undersøgelse og scanning af encephalon. Dermed kan det udelukkes om der er eventuelle blødninger eller andre årsager til funktionstabet. Dette  sikrer, at patienten får den rette behandling. I tilfælde af blodprop igangsættes en behandling med blodpropopløsende eller blodpropshæmmende medicin. \cite{Hjerteforeningen2014, Kruuse2014a} 

\subsection{Trombolyse}
Standardbehandling for blodpropper har siden år 2006 været trombolyse. Selve behandlingen foregår ved, at der sprøjtes blodpropopløsende medicin ind i en arterie, ofte i armen, hvorefter blodproppen opløses. Denne behandling skal helst foregå seks timer efter blodproppens forekomst og senest 12 timer efter, da behandlingen ikke vil have nogen indvirkning efter længere tid. Hurtig behandling vil betyde at flere områder af encpehalon vil kunne reddes. Dermed vil patientens fremtidige livskvalitet forbedres. Trombolysebehandlingen finder sted på 12 sygehuse fordelt over de fem regioner. En risiko ved behandlingen kan være blødninger grundet den blodpropopløsende medicin. \cite{Hjernesagen2015b} \fxnote{Kig på de tre afsnit i forhold til om de minder for meget om hinanden}

\subsection{Forebyggelse}
En væsentlig del af behandlingen er forebyggelse, da risikoen for en ny blodprop er betydelig. Til forebyggelse anvendes antikoagulationsbehandling, som er en behandling med blodfortyndende medicin. Normalt har kroppen sit eget koagulationsssystem som får blodet til at koagulere. Derudover medvirker koagulationssystemet også til at opløse evt. blodpropper i kardiovaskulæresystem. For apopleksipatienter fungerer koagulationssystemet ikke optimalt, hvilket gør det nødvendigt at behandle med antikoagulation. Dette hæmmer blodets evne til at koagulere, hvilket modvirker dannelsen af blodpropper. Der findes to former for antikoagulationsbehandling, warfarin og nyre orale antikoaglulantia. Den primære forskel mellem de to mediciner er, at der ved behandling med nyre orale antikoaglulantia ikke kræver kontrol ved blodprøver.\cite{Kjaergaard2015}

%[1]http://www.hjerteforeningen.dk/alt-om-dit-hjerte/hjerte-kar-sygdomme/apopleksi/
%[2]http://www.hjernesagen.dk/om-hjerneskader/behandling/trombolyse
%[3] https://www.sundhed.dk/borger/sygdomme-a-aa/hjerte-og-blodkar/sygdomme/apopleksi/behandling-ved-apopleksi/
%[4]https://www.sundhed.dk/borger/sygdomme-a-aa/hjerte-og-blodkar/sygdomme/behandlinger/antikoagulationsbehandling-blodfortyndende-medicin/