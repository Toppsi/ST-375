% !TeX spellcheck = da_DK
\section{Undersøgelse og behandling}
Det er vigtigt, at patienter med formodet apopleksi får den rette undersøgelse og behandling. Undersøgelse og diagnosticering er afgørende for det efterfølgende forløb, da behandling samt rehabilitering planlægges herefter. \cite{Sundhedsstyrelsen2009}

\subsection{Undersøgelse}
Når en patient med apopleksi indlægges, er grundig undersøgelse nødvendig for at identificere, hvilken form for apopleksi patienten har. 
Diagnosticeringsprocessen består af flere trin. Først optages en anamnese, hvor lægen stiller patienten spørgsmål omkring sygdomsforløbet og eventuelle risikofaktorer. Herefter anvendes en udvalgt, standardiseret skala til at foretage en klinisk undersøgelse af patientens almene tilstand. Den valgte skala gør det muligt for lægen at vurdere, hvordan patientens tilstand udvikler sig i perioden efter indlæggelsen.
Der kan efterfølgende foretages enten en CT- eller MR-scanning for at undersøge, om patienten er ramt af iskæmisk eller hæmoragisk apopleksi. MR-scanning er at foretrække, hvis lægen mistænker blødning i cerebellum eller truncus encephalius, hvorimod CT-scanning anvendes til at bedømme området og omfanget af blødningen. Under forløbet kontrollerer lægen også andre fysiologiske faktorer, eksempelvis blodtryk og temperatur. Disse faktorer kan give information om apopleksien. \cite{Sundhedsstyrelsen2009,Schulze2011} 

\subsection{Behandling}
Ved både iskæmisk og hæmoragisk apopleksi er det vigtigt at komme i behandling hurtigst muligt. \cite{Soenderborg2013}  
Standardbehandling for iskæmisk apopleksi har siden 2006 været trombolyse. Selve behandlingen foregår ved, at der sprøjtes blodpropopløsende medicin ind i en arterie, hvorefter blodproppen opløses. Denne behandling skal senest foregå 12 timer efter, da behandlingen derved ikke vil have nogen indvirkning efter længere tid. Ved hurtig behandling kan områder af encpehalon reddes, hvormed patientens fremtidige livskvalitet forbedres.\fxnote{NTK: Trombolysebehandling finder sted på 12 sygehuse fordelt over de fem regioner.} En risiko ved behandling med blodpropopløsende medicin er, at det kan skabe nye blødninger i dele af hjernen eller andre steder i kroppen. \cite{Hjernesagen2015b} Behandlingen af patienter, der rammes af hæmoragisk apopleksi, afhænger af hæmatomets placering samt størrelse. Primært vil lægerne dræne blodet ud, såfremt det er muligt. Der kan desuden behandles med blodtrykssænkende medicin for at begrænse blødningen. \cite{Caplan2006} 

\subsection{Forebyggelse}
En væsentlig del af iskæmisk apopleksi behandling er forebyggelse, da der er risiko for en ny blodprop. Til dette anvendes antikoagulationsbehandling, som er blodfortyndende medicin. Normalt har kroppen sit eget koagulationsssystem, som får blodet til at størkne. Derudover medvirker koagulationsssystem til at opløse evt. blodpropper i det kardiovaskulære system. For iskæmiske apopleksipatienter fungerer koagulationssystemet ikke optimalt, og det er dermed nødvendigt at hæmme blodets evne til at koagulere. Dette modvirker dannelsen af nye blodpropper. \cite{Kjaergaard2015}
For både iskæmisk og hæmoragisk apopleksi er en væsentlig del af forebyggelsen at undgå diverse risikofaktorer ift. livsstil. \cite{Christensen2015}

 %Der findes to former for antikoagulationsbehandling, warfarin og nyre orale antikoaglulantia. Den primære forskel mellem de to mediciner er, at der ved behandling med nyre orale antikoaglulantia ikke kræves kontrol ved blodprøver.\cite{Kjaergaard2015}

%Hvorimod ved akut iskæmisk apopleksi, hvilket vil sige at symptomerne er til stede inden for fire en halv time, anvendes blodpropopløsende medicin. Ved hurtig behandling vil det være muligt at opløse blodproppen. I andre tilfælde fjernes blodproppen ved brug af et tyndt kateter, som indføres gennem arterien op til encephalon. Derudover anvendes blodfortyndende medicin for at undgå nye tilfælde af apopleksi. \cite{Hjerteforeningen2014, Kruuse2014a} 

%\subsection{Akut behandling}
%Ved mistanke om iskæmisk apopleksi er det vigtigt, at der tages kontakt til sygehuset omgående. Patienten bliver her undersøgt ved blodtryksmåling, blodprøver, neurologisk undersøgelse og scanning af encephalon. Dermed kan det udelukkes om der er eventuelle blødninger eller andre årsager til funktionstabet. Dette  sikrer, at patienten får den rette behandling. I tilfælde af blodprop igangsættes en behandling med blodpropopløsende eller blodpropshæmmende medicin. \cite{Hjerteforeningen2014, Kruuse2014a} 