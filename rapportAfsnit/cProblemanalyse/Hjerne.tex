% !TeX spellcheck = da_DK
\subsection{Påvirkning på hjernen}
Cerebrum er den største region af encephalon og kan deles op i to hjernehalvdele. Her sker en processering af sanserne, tale, tanker, synet, hukommelsen og følelser. \cite{Martini2012} For en yderligere beskrivelse af hjernen, nervefysiologi samt biologisk kommunikation se bilag \ref{AppNerve}. De forskellige motoriske og sensoriske regioner kan ses på \figref{Enc}. Som tidligere nævnt er 80-85\% af apopleksitilfældene blodpropper og rammer hyppigst i media arterien, der forsyner det meste af cerebrum. Derfor er det ofte motoriske og sensoriske områder, som bliver skadet ved et apopleksitilfælde. \cite{Sundhed.dk,Gade2004,Boss2010} \\
De motoriske og sensoriske nervebaner fra motor- og sensorisk cortex løber ned gennem rygmarven og leder derved impulser ud til target organer og muskler og tilbage igen. Nervebanerne fra hhv. højre og venstre hjernehalvdel krydser i medulla oblongata\fxnote{den forlængede marv} eller i medulla spinalis\fxnote{rygmarven}. Denne krydsning gør, at afferente signaler fra højre side af kroppen behandles i venstre hjernehalvdel, der sender efferente signaler tilbage til højre side af kroppen. \cite{Martini2012,Stanfield2014} Dette medfører, at et apopleksitilfælde i højre hjernehalvdel kan give motoriske eller sensoriske skader i venstre kropsdel og omvendt med venstre hjernehalvdel. Et apopleksitilfælde kan derved lede til neglekt eller problemer med balancen \cite{Sundhedsstyrelsen2009,Nichols1997}.

\begin{figure}[H]
	\centering
	\includegraphics[scale=0.6]{figures/bProblemanalyse/Encephalon.png}
	\caption{På figuren ses de motoriske og sensoriske regioner på den venstre hjernehalvdel af cerebrum. \cite{Martini2012}}
	\label{Enc}
\end{figure}

Hver muskelgrupper har bestemte nerveceller. Antallet af nerveceller til hver muskel afhænger af, hvor præcis legemets bevægelse skal være - jo færre desto mere præcis \cite{Stanfield2014}. Nervecellerne har en bestemt placering i cerebral cortex. Derfor vil et apopleksitilfælde et helt bestemt sted ramme en bestemt muskel.\fxnote{Man kunne skrive et appendix med nervefysiologi - altså CNS, PNS, nerver, aktionspotentiale osv.} Encephalon har dog en naturlig tilpasning, hvilket gør, at den i nogle tilfælde kan genskabe tabte nerver eller finde en anden vej for funktionen, som den tabte nerve skulle udføre. \cite{Martini2012} Denne mekaniske kaldes plasticitet. \cite{Ramanathan2006}

\subsection{Plasticitet}
Encephalon kan ændre eller tilpasse sig de stimuli, den udsættes for, hvilket kaldes encephalons plasticitet eller nerveplasticitet. Dette sker kontinuerligt igennem hele livet, men encephalon / corpus kan dog ikke danne nye nerver. \cite{Stanfield2014} Under et apopleksitilfælde forekommer der iltmangel til encephalon, og nervecellerne kan derved blive skadet eller gå tabt. \cite{Schulze2011} Denne celledød gør, at den døde nerve mister sine forbindelser til raske nerver. Denne forbindelsesafbrydelse i encephalon bevirker, at der kan opstå en kaskade af mistet kommunikation i de eksisterende nerver, hvorved en nerves celledød altså kan påvirke andre områder af encephalon end blot der, hvor skaden er sket. \cite{Raine2009} Herved vil encephalon benytte sig af sin plasticitet og omlægge det eksisterende nervenetværk. Hjernen vil aktivere nogle signalstoffer, som kan finde en alternativ metode til at gennemføre den ønskede handling. \cite{Rugnett2015}  Som sagt kan encephalon ikke danne nye nerver efter celledød, hvilket betyder, at der ikke kan generhverves præcis samme funktion som tidligere men evt. en lignende funktion. Men encephalon vil forsøge at kompensere for de tabte nerver ved at danne nye forbindelser og kommunikationsveje, hvilket kan deles op i tre fænomener: \cite{Raine2009}

\begin{itemize}
	\item En mulig er, at en afbrydelse imellem et akson og synapse medfører, at synapsen bliver overfølsom og derved lettere påvirket til at lave nye synapseforbindelser. Dette fænomen kaldes “Denervation supersensitivity”.
	\item En anden mulighed er, at synapser, der har fuld funktionalitet men ingen effekt på slutstedet, “afsløres”, hvorefter der opstår en aktivitet og effekt. Dette kaldes “unmasking of silent (latent) synapses”.
	\item En sidste mulighed er, hvis to nerver innerverer på samme slutsted og den ene nerve dør, så vil den anden nerve spire ind i den skadet nerves telodendron, og funktionen vil derved genvindes. Dette kaldes “Collateral sprouting”.
\end{itemize}

Man kan ud fra disse tre fænomener finde en fysiologisk baggrund for rehabilitering. Nerveplasticitet er særlig øget i op til en måned efter et apopleksitilfælde, hvorved det er vigtigt, at foretage genoptræning i denne periode, så encephalon kan danne disse nye forbindelser og kommunikationsveje. \cite{Rugnett2015} Gentagelser af en færdighed effektiviserer synapseforbindelser, hvilket betyder, at den kompenserende færdighed kan styrkes. \cite{Stanfield2014}.