% !TeX spellcheck = da_DK
\subsection{Påvirkning på hjernen}\label{HjerneSenMot}

Cerebrum er den største region af encephalon og kan deles op i to hjernehalvdele. Her sker en processering af sanserne, tale, tanker, synet, hukommelsen og følelser. \cite{Martini2012} For en yderligere beskrivelse af hjernen, nervefysiologi samt biologisk kommunikation se bilag \ref{AppNerve}. De forskellige sensoriske- og motoriske regioner kan ses på \figref{Enc}. Som tidligere nævnt i afsnit \ref{IskaemiskApp} er 80-85\% af apopleksitilfældene iskæmiske og rammer hyppigst i media arterien, der forsyner det meste af cerebrum med blod. Derfor er det ofte sensoriske- og motoriske områder, som bliver skadet ved et apopleksitilfælde. \cite{Sundhed.dk,Gade2004,Boss2010} \\

\begin{figure}[H]
	\centering
	\includegraphics[scale=0.6]{figures/bProblemanalyse/Encephalon.png}
	\caption{På figuren ses de sensoriske og motoriske regioner på den venstre hjernehalvdel af cerebrum. \cite{Martini2012}}
	\label{Enc}
\end{figure}

De sensoriske- og motoriske nervebaner fra sensorisk- og motorisk cortex løber ned gennem medulla spinalis og leder derved impulser ud til target organer og muskler og tilbage igen. Nervebanerne fra hhv. højre og venstre hjernehalvdel krydser i medulla oblongata eller i medulla spinalis. Denne krydsning betyder, at afferente signaler fra højre side af kroppen behandles i venstre hjernehalvdel, der sender efferente signaler tilbage til højre side af kroppen. \cite{Martini2012,Stanfield2014} Dette medfører, at et apopleksitilfælde i højre hjernehalvdel kan give sensoriske- og motoriske skader i venstre kropsdel og omvendt med venstre hjernehalvdel. \cite{Sundhedsstyrelsen2009,Nichols1997} %Et apopleksitilfælde kan derved lede til neglekt eller problemer med balancen .

\fxnote{Sæt et billede ind af krydsningen i medulla oblongata eller medulla spinalis}

Hver muskelgruppe har sine egne dedikerede nerveceller. Antallet af nerveceller til hver muskel afhænger af, hvor præcis legemets bevægelse skal være. Flere nerveceller gør musklens bevægelse mere præcis. \cite{Stanfield2014} Nervecellerne har en bestemt placering i cerebral cortex. Derfor vil et apopleksitilfælde et bestemt sted ramme en bestemt muskel.\fxnote{Kunne man evt. give et eksempel her?} Encephalon har en naturlig tilpasning. Dette medfører, at den i nogle tilfælde kan genskabe skadede nerver eller finde en anden vej for funktionen, som en eventuelt tabt nerve skulle udføre. \cite{Martini2012} Denne mekanisme kaldes plasticitet \cite{Ramanathan2006}. 

\subsection{Plasticitet}
Encephalon kan ændre eller tilpasse sig de stimuli, den udsættes for, hvilket kaldes encephalons plasticitet eller nerveplasticitet. Dette sker kontinuerligt igennem hele livet, men encephalon kan ikke danne nye nerver. \cite{Stanfield2014} Under et apopleksitilfælde forekommer der iltmangel til encephalon, og nervecellerne kan derved blive skadet eller gå tabt \cite{Schulze2011}. Celledød gør, at den døde nerve mister sine forbindelser til fungerende nerver. Denne forbindelsesafbrydelse i encephalon bevirker, at der kan opstå en kaskade af mistet kommunikation i de eksisterende nerver. Herved kan en nerves celledød påvirke andre områder af encephalon end blot der, hvor skaden er sket. \cite{Raine2009} Encephalon vil benytte sig af sin plasticitet og omlægge det eksisterende nervenetværk.\fxnote{Hvad omlægges det til?} Encephalon vil aktivere nogle signalstoffer, som kan finde en alternativ metode til at gennemføre den ønskede handling. \cite{Rugnett2015}  Som nævnt kan encephalon ikke danne nye nerver efter celledød, hvilket betyder, at der ikke kan generhverves præcis samme funktion som tidligere men evt. en lignende funktion. %Encephalon vil forsøge at kompensere for de tabte nerver ved at danne nye forbindelser og kommunikationsveje, hvilket 
Plasticitet kan deles op i tre fænomener: \cite{Raine2009}

\begin{itemize}
	\item En afbrydelse imellem akson og synapse medfører, at synapsen bliver overfølsom og derved lettere påvirket til at lave nye synapseforbindelser. Dette fænomen kaldes Denervation Supersensitivity.
	\item Synapser, der har fuld funktionalitet men ingen effekt på slutstedet, afsløres, hvorefter der opstår en aktivitet og effekt. Dette kaldes Unmasking of Silent (Latent) Synapses.
	\item Hvis to nerver innerverer på samme slutsted, og den ene nerve dør, så vil den anden nerve spire ind i den skadede nerves telodendron, og funktionen vil derved genvindes. Dette kaldes Collateral Sprouting.
\end{itemize}
\fxnote{Angiv navnet først og så funktionen bagefter. Lav selve navnet til en punkt istedet for en prik.}
Ud fra disse tre fænomener findes der en fysiologisk baggrund for rehabilitering. Nerveplasticitet er særlig øget op til en måned efter et apopleksitilfælde. Det er derfor vigtigt at foretage genoptræning i denne periode, så encephalon kan danne nye forbindelser og kommunikationsveje. \cite{Rugnett2015} Gentagelser af en færdighed effektiviserer synapseforbindelser, hvilket betyder, at den kompenserende færdighed kan styrkes. \cite{Stanfield2014}.\fxnote{Afsnittet "Kroppens kompenserende bevægelser" kunne evt. komme her efter istedet for nede under "følger". Hvad synes folk om denne rækkefølge?}


%\subsection{Plasticitet}
%Encephalon kan ændre eller tilpasse sig de stimuli, den udsættes for, hvilket kaldes encephalons plasticitet eller nerveplasticitet. Dette sker kontinuerligt igennem hele livet, men encephalon kan ikke danne nye nerver. \cite{Stanfield2014} Under et apopleksitilfælde forekommer der iltmangel til encephalon, og nervecellerne kan derved blive skadet eller gå tabt. \cite{Schulze2011} Denne celledød gør, at den døde nerve mister sine forbindelser til raske nerver. Denne forbindelsesafbrydelse i encephalon bevirker, at der kan opstå en kaskade af mistet kommunikation i de eksisterende nerver. Herved kan en nerves celledød altså påvirke andre områder af encephalon end blot der, hvor skaden er sket. \cite{Raine2009} Encephalon vil benytte sig af sin plasticitet og omlægge det eksisterende nervenetværk. Encephalon vil aktivere nogle signalstoffer, som kan finde en alternativ metode til at gennemføre den ønskede handling. \cite{Rugnett2015}  Som sagt kan encephalon ikke danne nye nerver efter celledød, hvilket betyder, at der ikke kan generhverves præcis samme funktion som tidligere men evt. en lignende funktion. Men encephalon vil forsøge at kompensere for de tabte nerver ved at danne nye forbindelser og kommunikationsveje, hvilket kan deles op i tre fænomener: \cite{Raine2009}

%\begin{itemize}
	%\item En afbrydelse imellem akson og synapse medfører, at synapsen bliver overfølsom og derved lettere påvirket til at lave nye synapseforbindelser. Dette fænomen kaldes “Denervation Supersensitivity”.
%	\item Synapser, der har fuld funktionalitet men ingen effekt på slutstedet, “afsløres”, hvorefter der opstår en aktivitet og effekt. Dette kaldes “Unmasking of Silent (Latent) Synapses”.
%	\item Hvis to nerver innerverer på samme slutsted, og den ene nerve dør, så vil den anden nerve spire ind i den skadede nerves telodendron, og funktionen vil derved genvindes. Dette kaldes “Collateral Sprouting”.
%>>>>>>> origin/master
%\end{itemize}

