% !TeX spellcheck = da_DK
%\subsection{Nuværende metoder til rehabilitering}
\subsection{Metoder til rehabilitering af balance}

Inden patienten udskrives fra sin behandling, skal der være udarbejdet en genoptræningsplan fra sundhedssektorens side. I denne plan besluttes det, hvilken form for rehabilitering og teknologi patienten skal benytte sig af. \cite{Sundhedsstyrelsen2011a} \\
Der findes flere forskellige metoder og teknologier til at hjælpe med balanceproblemer. Disse omfatter: \cite{Sundhedsstyrelsen2011a}  

\begin{itemize}
\item Platform feedback
\item Passiv sensorisk stimulation
\end{itemize}

Svimmelhed

Platform feedback er en biofeedback metode, hvor patienten står på en platform. Platformen vil herefter måle graden af patientens svajning \fxnote{centre of pressure}. Når platformen har målt svajningen af patienten, kan denne enten få visuel eller auditiv feedback. Feedbacken skal gøre patienten mere opmærksom på, hvor meget kroppen svajer, hvilket gør det muligt at opretholde en stående position. \cite{Barclay-Goddard2004}
Denne form for teknologi benyttes særligt i de tidlige faser af rehabiliteringen \cite{Sundhedsstyrelsen2011a}. 

Passiv sensorisk stimulation er en rehabiliteringsform, hvor patienten modtager elektrisk stimulation, der ikke medfører aktivitet i musklerne. Stimulationen er der for at fortælle patienten om, hvad kroppen foretager sig, så det bliver muligt at korrigere bevægelserne. \cite{Sundhedsstyrelsen2010} Denne form for metode tilbydes under hele rehabiliteringsforløbet \cite{Sundhedsstyrelsen2011a}.\fxnote{NTK: Når musklerne stimuleres passivt, kan patienten 'mærke' den lammede kropsdel, og selvom den ikke kan styres fuldt ud kan der stadig gradvist skabes kontakt til den igen, da det er muligt at føle hvad den foretager sig.}