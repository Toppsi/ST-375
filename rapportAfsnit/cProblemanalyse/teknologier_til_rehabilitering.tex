% !TeX spellcheck = da_DK
\subsection{Nuværende metoder til rehabilitering}

Inden patienten udskrives fra sin behandling skal der fra sundhedssektorens side være udarbejdet en genoptræningsplan. I denne plan besluttes det hvilken form for rehabilitering og teknologi patienten skal benytte sig af. \cite{Sundhedsstyrelsen2011a} \\
Der findes flere forskellige metoder og teknologier til at hjælpe med balance og gangproblemer. Disse omfatter: \cite{Sundhedsstyrelsen2011a}  

\begin{itemize}
\item Platform feedback
\item Fokuseret gangtræning
\item Konditionstræning
\item Auditorisk rytmestøtte
\item Elektromekanisk fysioterapistøttet gangtræning
\item Opgavespecifik repetitiv træning
\item Spejlterapi
\item Programmer til motorisk visualisering
\item Passiv sensorisk stimulation
\end{itemize}

Platform feedback er en biofeedback metode, hvor patienten står på en platform. Platformen vil herefter måle hvor meget patienten svajer \fxnote{centre of pressure}. Når platformen har målt svajningen af patienten, kan denne enten få visuel eller auditiv feedback. Feedbacken skal gøre patienten mere opmærksom på, hvor meget kroppen svajer, hvilket gør det muligt at opretholde en stående position. \cite{Barclay-Goddard2004}
Denne form for teknologi benyttes særligt i de tidlige faser af rehabiliteringen \cite{Sundhedsstyrelsen2011a}. 

Det er med høj evidens blevet påvist, at fokuseret gangtræning medfører en forbedringen af gangfunktionen hos apopleksiramte\cite{Sundhedsstyrelsen2010}. Derudover er det også vist, at konditionstræning kan være med til at forbedre gangfunktionen\cite{Sundhedsstyrelsen2010}.

Auditorisk rytmestøtte er en metode, hvor patienten lytter eller udfører handlinger til en form for musik. Det mest centrale ved musikken er rytmen, der findes i den. \cite{Bradt2010} Det er vist, at denne metode kan øge både ganghastigheden, skridtlængden og gangsymmetrien \cite{Sundhedsstyrelsen2010}.

Ved elektromekanisk fysioterapistøttet gangtræning benytter fysioterapeuten sig af et apparat til hjælp af patientens gang. Apparatet består af enten et robot-drevet exoskelet eller to drevne mekaniske plader, der simulerer gang hos patienten. \cite{Mehrholz2013} Denne form for metode har vist at kunne øge skridthastighed, skridtlængde og skridtsymmetri \cite{Sundhedsstyrelsen2010}. Denne metode benyttes til patienter med svært nedsat gangfunktion, med særlig fokus på de tidlige faser af rehabiliteringen \cite{Sundhedsstyrelsen2011a}.

Opgavespecifik repetitiv træning omfatter aktivitetsbestemte motoriske opgaver, som er bestemt til den enkelte patient. Disse opgaver tager udgangspunkt i hverdagsaktiviteter. Denne metode kan have effekt på patienternes gangfunktion, gangdistance og -hastighed. \cite{Sundhedsstyrelsen2010}

Spejlterapi er en træning af bevægelser, hvor patienten udfører en række bevægelsesmønstre med den raske side af kroppen. Der er imens placeret et spejl foran patienten i det midtsagittale plan, hvormed der skabes en illusion om at den syge side udfører den raske sides bevægelser. Således stimuleres kroppen visuelt til at forsøge at bevæge den syge side.\cite{Thieme2012} Denne metode skal tilbydes under hele rehabiliteringen \cite{Sundhedsstyrelsen2011a}.

Programmer til motorisk visualisering kan bl.a. være virtual reality \cite{Sundhedsstyrelsen2010}. Dette er et program, hvor en computer modellerer og simulerer et miljø, som patienten kan placeres i vha. briller og andre interaktive apparater \cite{Lowood2015}. Dette gør at patienten kan simulere bevægelser, som de ikke nødvendigvis er i stand til i virkeligheden. Det er uafklaret, hvilken effekt denne form for rehabilitering har \cite{Sundhedsstyrelsen2010}.

Passiv sensorisk stimulation er en rehabiliteringsform, hvor patienten modtager elektrisk stimulation, der ikke medfører aktivitet i musklerne. Stimulationen er der for at fortælle patienten om, hvad kroppen foretager sig, så det bliver muligt at korrigere bevægelserne. \cite{Sundhedsstyrelsen2010} Denne form for metode tilbydes under hele rehabiliteringsforløbet \cite{Sundhedsstyrelsen2011a}.\fxnote{NTK: Når musklerne stimuleres passivt, kan patienten 'mærke' den lammede kropsdel, og selvom den ikke kan styres fuldt ud kan der stadig gradvist skabes kontakt til den igen, da det er muligt at føle hvad den foretager sig.}
%
%
%[1] - http://onlinelibrary.wiley.com.zorac.aub.aau.dk/doi/10.1002/14651858.CD004129.pub2/full
%[2] - http://onlinelibrary.wiley.com.zorac.aub.aau.dk/doi/10.1002/14651858.CD006787.pub2/full
%[3] - http://onlinelibrary.wiley.com/doi/10.1002/14651858.CD006185.pub3/full
%[4] - http://academic.eb.com.zorac.aub.aau.dk/EBchecked/topic/630181/virtual-reality-VR/