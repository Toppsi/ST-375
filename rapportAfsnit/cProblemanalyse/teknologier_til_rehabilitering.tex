% !TeX spellcheck = da_DK
%\subsection{Nuværende metoder til rehabilitering}
\subsection{Metoder og teknologier til rehabilitering af balancen}
Inden patienten udskrives fra sin behandling, udarbejdes en genoptræningsplan fra sundhedssektorens side. I denne plan besluttes det, hvilke metoder og teknologier patienten skal benytte sig af til rehabiliteringen. Disse omfatter: \cite{Sundhedsstyrelsen2011a}  

\begin{itemize} \label{rehabiliteringbalance}
\item Platform feedback: En metode baseret på biofeedback, hvor patienten står på en platform, der måler graden af patientens svajning\fxnote{NTK: Centre of pressure}. Patienten modtager visuel eller auditiv feedback på baggrund af den information, platformen giver. Feedbacken skal gøre patienten mere opmærksom på kroppens hældning, hvilket gør det muligt at opretholde balancen i stående position. Metoden har vist at forbedre en symmetrisk kropsholdning for patienten. \cite{Barclay-Goddard2004} Denne form for teknologi benyttes særligt i forløbsprogrammets tidlige faser \cite{Sundhedsstyrelsen2011a}.
\item Passiv sensorisk stimulation: En rehabiliteringsform, hvor patienten modtager elektrisk stimulation, der ikke medfører aktivitet i musklerne. Stimulationen underretter patienten omkring kroppens bevægelser, så det bliver muligt at korrigere disse og opretholde balancen. \cite{Sundhedsstyrelsen2010} Metoden tilbydes under hele rehabiliteringsforløbet og har effekt på gangfunktionen \cite{Sundhedsstyrelsen2011a}.\fxnote{NTK: Når musklerne stimuleres passivt, kan patienten 'mærke' den lammede kropsdel, og selvom den ikke kan styres fuldt ud kan der stadig gradvist skabes kontakt til den igen, da det er muligt at føle hvad den foretager sig.}
\item Balancetræning med fysioterapeut: Er en træningsform, der indebærer forskellige træningsmetoder, så som øvelser med vippebræt eller skumpude. Øvelserne indebærer, at patienten skal stå på brættet eller puden mens der foretages andre øvelser, eksempelvis boldkast eller rotation på stedet. Derudover træner patienterne den statiske balance ved at udføre en øvelse i stående udgangsposition med fødderne på en tegnet linje, så den ene fods tæer er mod den anden fods hæl og armene holdes tæt ind til kroppen og over kors. I nogle tilfælde kan fysioterapeuten bede patienten om at lukke øjnene eller blokere udsynet ned til fødderne, så patienten skal stole på kroppens egne signaler til opretholdelse af balancen. \cite{Joergensen2004} Øvelserne blev observeret på Træningsenhed Vest Aalborg Kommune. Et referat fra observationen kan ses i bilag \ref{Ref_observation} på side \pageref{Ref_observation}. Denne træningsform tilbydes under hele rehabiliteringsforløbet og har en god effekt på patientens balance samt sociale kompetencer. \cite{Sundhedsstyrelsen2011a}
\item Styrketræning: Træning af kroppens styrke og især muskelkraft har en dokumenteret effekt på balancen, da en god koordinering er essentiel for musklernes sammenspil. Hvis musklerne er stærke, er kroppen bedre til at stå imod udefrakommende påvirkninger, hvilket giver en bedre balance. \cite{Joergensen2004}
\end{itemize}

\noindent Efter rehabiliteringsforløbet er det besværligt at måle, hvorvidt genoptræningen har været succesfuld af flere årsager. Nogle sygdomme kan læges over tid uden nogen form for behandling, hvorfor nogle patienter muligvis vil opnå samme resultat uden rehabiliteringen. Derudover skal faldulykker ikke altid opfattes som et tegn på, at patienten ikke har gjort fremskridt. Når patienten opnår bedre balance og stoler mere på sine egne signaler, vil der naturligt foregå mere aktivitet i hverdagen, hvilket indebærer en højere risiko for at komme i ubalance. \cite{Hain2008} \\
%Man mener, at den bedste målemetode for succes af rehabilitering er spørgeskemaer til patienter \cite{Hain2008}.