% !TeX spellcheck = da_DK
%\subsection{Nuværende metoder til rehabilitering}
\subsection{Metoder til rehabilitering af balance}

%Inden patienten udskrives fra sin behandling, skal der være udarbejdet en genoptræningsplan fra sundhedssektorens side. I denne plan besluttes det, hvilken form for rehabilitering og teknologi patienten skal benytte sig af. \cite{Sundhedsstyrelsen2011a} \\
Der findes flere forskellige metoder og teknologier til at hjælpe med balanceproblemer  under rehabiliteringsprocessen. Disse omfatter: \cite{Sundhedsstyrelsen2011a}  

\begin{itemize}
\item Platform feedback: En metode baseret på biofeedback, hvor patienten står på en platform, der måler graden af patientens svajning\fxnote{centre of pressure}. Når platformen har målt svajningen af patienten, kan vedkommende enten få visuel eller auditiv feedback. Feedbacken skal gøre patienten mere opmærksom på, hvor meget kroppen svajer, hvilket gør det muligt at opretholde balancen en stående position. Denne metode har vist sig at forbedre en symmetrisk holdning for patienten. \cite{Barclay-Goddard2004} Denne form for teknologi benyttes særligt i de tidlige faser af rehabiliteringen \cite{Sundhedsstyrelsen2011a}.
\item Passiv sensorisk stimulation: En rehabiliteringsform, hvor patienten modtager elektrisk stimulation, der ikke medfører aktivitet i musklerne. Stimulationen underretter patienten om, hvad kroppen foretager sig, så det bliver muligt at korrigere bevægelserne og opretholde balancen. \cite{Sundhedsstyrelsen2010} Denne metode tilbydes under hele rehabiliteringsforløbet og har effekt for gangfunktionen \cite{Sundhedsstyrelsen2011a}.\fxnote{NTK: Når musklerne stimuleres passivt, kan patienten 'mærke' den lammede kropsdel, og selvom den ikke kan styres fuldt ud kan der stadig gradvist skabes kontakt til den igen, da det er muligt at føle hvad den foretager sig.}
\item Balancetræning med fysioterapeut: Denne træningsform indebærer forskellige træningsmetoder med f.eks. et vippebræt eller skumpude. Her skal patienten stå på brættet eller puden mens der foretages andre øvelser, eksempelvis boldkast eller rotation på stedet. I nogle tilfælde kan fysioterapeuten bede patienten om at lukke øjnene eller blinde vinklen ned til fødderne, så patienten skal stole på kroppens egne signaler til opretholdelse af balancen. \cite{Joergensen2004} Denne form for rehabilitering tilbydes under hele rehabiliteringsforløbet og har en god effekt på patientens balance samt sociale kompetencer. \cite{Sundhedsstyrelsen2011a}
\item Styrketræning: Træning af kroppens styrke og især muskelpower har en dokumenteret effekt på balancen, da en god koordinering for musklernes sammenspil er essentiel. Hvis musklerne er stærke, er kroppen bedre til at stå imod udefrakommende påvirkninger som f.eks. tyngdekraften, hvilket giver en bedre balance. \cite{Joergensen2004}
\end{itemize}

Efter rehabiliteringsforløbet er det besværligt at måle, om genoptræningen har været succesfuld af flere årsager. Nogle sygdomme kan læges over tid uden nogen form for behandling. Derfor vil nogle patienter muligvis opnå samme resultat uden rehabiliteringen. Derudover skal man ikke altid opfatte faldulykker som et tegn på, at patienten ikke har gjort fremskridt. Når patienten får bedre balance og stoler mere på sine egne signaler, vil der naturligt foregå mere aktivitet i hverdagen, hvilket indebærer en højere risiko for at komme ud af balance. \cite{Hain2008} \\
Man mener, at den bedste målemetode for succes af rehabilitering er spørgeskemaer til patienter. \cite{Hain2008}