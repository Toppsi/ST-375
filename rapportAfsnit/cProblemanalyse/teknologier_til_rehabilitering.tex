% !TeX spellcheck = da_DK
%\subsection{Nuværende metoder til rehabilitering}
\subsection{Metoder til rehabilitering af balance}

Inden patienten udskrives fra sin behandling, skal der være udarbejdet en genoptræningsplan fra sundhedssektorens side. I denne plan besluttes det, hvilken form for rehabilitering og teknologi patienten skal benytte sig af. \cite{Sundhedsstyrelsen2011a} \\
Der findes flere forskellige metoder og teknologier til at hjælpe med balanceproblemer. Disse omfatter: \cite{Sundhedsstyrelsen2011a}  

\begin{itemize}
\item Biofeedback - Platform feedback
\item Passiv sensorisk stimulation
\item Balancetræning med fysioterapeut
\item Styrketræning
\end{itemize}

Platform feedback er en biofeedback metode, hvor patienten står på en platform. Platformen vil herefter måle graden af patientens svajning\fxnote{centre of pressure}. Når platformen har målt svajningen af patienten, kan denne enten få visuel eller auditiv feedback. Feedbacken skal gøre patienten mere opmærksom på, hvor meget kroppen svajer, hvilket gør det muligt at opretholde en stående position. \cite{Barclay-Goddard2004} Denne form for teknologi benyttes særligt i de tidlige faser af rehabiliteringen \cite{Sundhedsstyrelsen2011a}. 

Passiv sensorisk stimulation er en rehabiliteringsform, hvor patienten modtager elektrisk stimulation, der ikke medfører aktivitet i musklerne. Stimulationen er der for at fortælle patienten om, hvad kroppen foretager sig, så det bliver muligt at korrigere bevægelserne og opretholde balancen. \cite{Sundhedsstyrelsen2010} Denne form for metode tilbydes under hele rehabiliteringsforløbet \cite{Sundhedsstyrelsen2011a}.\fxnote{NTK: Når musklerne stimuleres passivt, kan patienten 'mærke' den lammede kropsdel, og selvom den ikke kan styres fuldt ud kan der stadig gradvist skabes kontakt til den igen, da det er muligt at føle hvad den foretager sig.}

Balancetræning med en fysioterapeut indebærer forskellige træningsmetoder med f.eks. et vippebræt eller skumpude. Her skal patienten stå på brættet eller puden mens der foretaget andre øvelser som at kaste med en boldt eller dreje rundt på stedet. I nogle tilfælde kan fysioterapeuten også bede patienten om at lukke øjnene eller blinder vinklen ned til fødderne, så patienten skal stole på kroppens egne signaler til opretholdelse af balancen. \cite{Joergensen2004} Denne form for rehabilitering tilbydes under hele rehabiliteringsforløbet. \cite{Sundhedsstyrelsen2011a}

Styrketræning og især muskel power har en dokumenteret effekt på balancen, da en god koordinering for musklernes sammenspil er essentiel. Hvis musklerne er stærke, er kroppen bedre til at stå imod udefrakommende påvirkninger som f.eks. tyngdekraften. \cite{Joergensen2004}

Efter rehabiliteringsforløbet er det besværligt at måle, om det har væres en succes eller ej af flere årsager. Nogle sygdomme kan forbedre sig på egen hånd over tid uden nogen form for behandling. Derfor vil nogle patienter muligvis have opnået samme resultat uden rehabiliterings behandlingen, som de ville have fået med. Derudover skal man ikke altid opfatte et fald som et egn på, at patienten ikke har gjort fremskridt. Når patienten får bedre balance og stoler mere på sine egne signaler, vil der naturligt foregå mere aktivitet i hverdagen, hvilket indebærer en højrer risiko. \cite{Hain2008} \\
Man mener, at den bedste målemetode for succes af rehabilitering er spørgeskemaer til patienter. \cite{Hain2008}