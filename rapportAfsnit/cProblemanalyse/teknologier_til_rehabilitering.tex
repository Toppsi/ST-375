\section{Nuværende metoder til rehabilitering}

Inden patienten udskrives fra sin behandling, skal der fra sundhedssektorens side være udarbejdet en genoptræningsplan. Det er i denne plan at det besluttes hvilken form for rehabilitering, samt hvilken teknologi patienten skal benytte sig af. \cite{Sundhedsstyrelsen2011a}

Der findes flere forskellige metoder og teknologier til at hjælpe med balance og gangproblemer. Disse omfatter: Platform feedback, fokuseret gangtræning, konditionstræning, auditorisk rytmestøtte, elektromekanisk fysioterapistøttet gangtræning, opgavespecifik repetitiv træning, spejlterapi, programmer til motorisk visualisering og passiv sensorisk stimulation. \cite{Sundhedsstyrelsen2011a}

Platform feedback er en biofeedback metode, hvor patienten står på en platform. Platformen vil herefter måle hvor meget patienten svajer (centre of pressure). Når platformen har målt svajningen af patienten, kan denne enten få visuel eller auditiv feedback. Feedbacken skal gøre patienten mere opmærksom på hvor meget patienten svajer, så det bliver lettere at opretholde en stående position. [1]
Denne form for teknologi benyttes særligt i de tidlige faser af rehabiliteringen med en moderat effekt \cite{Sundhedsstyrelsen2011a}.

Fokuseret gangtræning er blevet vist, med god evidens, at have en moderat grad af bedring på gangfunktionen hos apopleksiramte \cite{Sundhedsstyrelsen2010}. Derudover er det også vist at konditionstræning kan være med til at forbedre gangfunktionen i moderat grad \cite{Sundhedsstyrelsen2010}. 

Auditorisk rytmestøtte er en metode hvor patienten lytter til eller udfører en form for musik. Det mest centrale ved musikken er den rytme der findes i den. [2] Det er vist at denne form for metode kan øge både ganghastigheden, skridtlængden og gangsymmetrien \cite{Sundhedsstyrelsen2010}. 

Elektromekanisk fysioterapistøttet gangtræning er gangtræning hvor fysioterapeuten benytter sig af en maskine til at hjælpe med patientens gang. Maskiner består af enten et robot-drevet exoskelet eller en maskine med to drevne mekaniske plader der simulerer gang hos patienten. [3] Denne form for metode er vist at kunne øge skridthastighed, skridtlængde og skridtsymmetri \cite{Sundhedsstyrelsen2010}. Denne metode benyttes til patienter med svært nedsat gangfunktion, med særlig fokus på de tidlige faser af rehabiliteringen \cite{Sundhedsstyrelsen2011a}.

Opgavespecifik repetitiv træning er træning af aktivitetsbestemte motoriske opgaver, som er bestemt til den enkelte patient. Disse opgaver tager udgangspunkt i hverdagsaktiviteter. Det er vist at denne metode har en moderat effekt på patienternes gangfunktion, gangdistance og -hastighed. \cite{Sundhedsstyrelsen2010}

Spejlterapi er en træning af bevægelser hvor patienten laver en række bevægelsesmønstre med den raske side af kroppen. Herefter bliver disse bevægelser spejlet til den syge side af kroppen. Dette vil skabe en illusion af et normalt bevægelsesmønster af den syge side hos patienten. \cite{Sundhedsstyrelsen2010} Denne metode skal tilbydes under hele rehabiliteringen \cite{Sundhedsstyrelsen2011a}.

Programmer til motorisk visualisering kan bl.a. være "virtual reality" \cite{Sundhedsstyrelsen2010}. Virtual reality er en form for et program, hvor en computer modellerer og simulerer er miljø, som patienten ved brug af briller og andre interaktive apparater kan placeres i [4]. Dette gør at patienten kan simulere bevægelser, som de ikke nødvendigvis er i stand til i virkeligheden. Det er uafklaret hvilken effekt denne form for rehabilitering har \cite{Sundhedsstyrelsen2010}.

Passiv sensorisk stimulation er en form for rehabilitering hvor patienten modtager elektrisk stimulation, der ikke aktiverer muskelaktivitet. Stimulationen er der for at fortælle patienten om hvad deres krop egentlig foretager sig, så det bliver muligt at korrigere bevægelserne. \cite{Sundhedsstyrelsen2010} Denne form for metode tilbydes under hele rehabiliteringsforløbet \cite{Sundhedsstyrelsen2011a}.



[1] - http://onlinelibrary.wiley.com.zorac.aub.aau.dk/doi/10.1002/14651858.CD004129.pub2/full
[2] - http://onlinelibrary.wiley.com.zorac.aub.aau.dk/doi/10.1002/14651858.CD006787.pub2/full
[3] - http://onlinelibrary.wiley.com/doi/10.1002/14651858.CD006185.pub3/full
[4] - http://academic.eb.com.zorac.aub.aau.dk/EBchecked/topic/630181/virtual-reality-VR/

