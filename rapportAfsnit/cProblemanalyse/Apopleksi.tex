\section{Apopleksi}

Et apopleksi tilfælde kan være forårsaget af enten en blodprop i hjernen (iskæmisk) eller hjerneblødning (hæmoragisk).
Apopleksi er af World Health Organization (WHO) defineret som pludseligt opstået fokale neurologiske symptomer pga. forstyrrelser i hjernens blodcirkulation, der varer mere end 24 timer eller fører til døden[1]. Hvis varigheden er under 24 timer, betegnes det som transitorisk cerebral iskæmi (TCI), hvor de fleste tilfælde varer under 1 time[2] uden permanent hjerneskade [3].

(Billeder)

Iskæmisk apopleksi forekommer hyppigst%ift. hvad? 
[2] og opstår, når en hjernearterie blokeres af en blodprop (infarkt), der stopper tilførslen af blod til et bestemt område i hjernen, hvilket ses på figur xx. Infarkterne dannes primært pga. åreforkalkning enten ved en trombe, der dannes på stedet, eller emboli fra hjertet. Nervecellerne skades efter få minutter pga. stoppet blodtilførsel og vil gå tabt [5].

Hæmoragisk apopleksi skyldes hovedsageligt forhøjet blodtryk eller i sjældnere tilfælde bristede svagheder på arterier (aneurismer) eller misdannede kar[5]. Hæmoragisk apopleksi opstår, når en hjernearterie brister og lækage af blod danner en blodansamling (hæmatom), der beskadiger det omkringliggende væv og forøger trykket i hjernen, hvilket ses på figur xx. Blødning i selve hjernen (intracerebral hæmoragi) kommer af forhøjet blodtryk, der danner et pres på de små arterier, som får dem til at briste[4] og forekommer i 10-12\% af tilfældene[2]. %ift. hvilke tilfælde? 
Blødning i rummet mellem de to hjernehinder (subaraknoidalrummet) skyldes bristning af et aneurisme på en pulsåre i hjernen [5] og forekommer 3-5\% af tilfældene[2]. Symptomerne ved subaraknoidalblødning er generel tab af hjernefunktion, da der forekommer et øget pres på hjerneskallen, hvorimod ved intracerebral hæmoragi er hæmatomet lokaliseret et bestemt sted i hjernen og forårsager nedsat funktion ved én bestemt hjernefunktion[4]. 

% WHO - find en dansk difination istedet.
% Fjern paranteserne - skrev enten det rigtige ord først eller lav en anden måde at skrive det på.
% Mangler lidt en forklaring på, hvordan og hvorfor apopleksi det opstår. Man kan godt uddybe mere i det, der allerede er skrevet.
% Når der er bygget mere på, kan man godt lave nogle forskellige overskrifter.
% Fakta omkring, hvad der er årsagen til apopleksi - hjertesagen har nogle forksellige info om det. Fakta om apopleksi.