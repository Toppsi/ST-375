% !TeX spellcheck = da_DK
\section{Følger af apopleksi }
Apopleksi kan forekomme pludseligt, og dermed uden at den ramte kan forberede sig på følgerne. Dette er modsat andre kroniske sygdomme, hvor progressionen ofte sker gradvist. Derfor kan der opstå andre psykiske konsekvenser forsaget af hæmoragisk apopleksi eller iskæmisk apopleksi som f.eks. depression. \cite{Muus2008} Udover de psykiske konsekvenser giver apopleksi andre følger, som afhænger af, hvilken del af hjernen der rammes og hvor omfangsrig hjerneskaden er. Her tænkes på tiden, hvor en del af hjernen ikke får ilt, størrelse af blødningen og trykket i arterien \cite{Michael-Titus2010}. Følgerne kan både gå ud over patientens fysiske- og mentale tilstand. \\
% ALT DETTE STÅR I INDLEDNINGEN - 75.000 mennesker over 18 år levede i 2011 med følger efter et slagstilfælde \cite{Hjernesagen2015}. Dette tal forventes at være stigende i takt med, at der kommer flere ældre \cite{Sagen2014}. Antallet der dør af hjerneskader har været stagneret de sidste 10 år før 2011, hvor 14 \% døde inden for 30 dage[3]. Det vil derfor kunne forventes, at der er flere, som kommer ud for en hjerneskade og vil have mén herefter, hvilket gør det vigtigt at fokusere på rehabiliteringen for at kunne genoptræne de forskellige kropslige og mentale mangler.

\subsection{Sensoriske og motoriske skader} %Sproglige, sensoriske og motoriske skader
%De sproglige konsekvenser fra apopleksi kaldes afasi. Afasi opleves efter en måned hos 20\% af de apopleksiramte og forekommer oftest ved skade i venstre temporal- og frontallap. De sproglige følger af apopleksi kan skade funktionen til at skrive og tale, men også evnen til at forstå og læse andres tale og skrift. I hvilken grad de sproglige funktioner er berørt kan variere mellem de enkelte patienter, da nogle oplever enkelte formuleringsproblemer, mens andre oplever global afasi. Global afasi gør, at de ramte i nogle tilfælde er helt ude af stand til at kommunikere verbalt eller kun kan fremsige enkelte ord med en vis usikkerhed omkring ordenes betydning.
%De sproglige konsekvenser kan midlertidig bedres med tiden, og flere patienter opnår et kommunikationsniveau, som gør det muligt at begå sig i hverdagen.\cite{Muus2008} \\
Som tidligere nævnt\fxnote{reference} kan apopleksi skade sensoriske såvel som motoriske funktioner. Dette kan bl.a. opleves som manglende funktion i arme, hænder, ben og fødder. De sensoriske- og motoriske konsekvenser er de hyppigst forekommende følger hos apopleksiramte og kan medføre problemer med udførsel af orienterende handlinger. Patienten kan således have problemer med forholdet mellem egen krop og objekter omkring sig samt kropdelenes indbyrdes forhold. Desuden kan patienterne have problemer med højre og venstre, derudover oplever nogle problemer med afstandsbedømmelse. De motoriske følger kan medføre upræcise bevægelser, generel stivhed i bevægelserne, problemer ved opstart af bevægelser, langsommere bevægelser, færre spontane bevægelser samt rystelser. Patienten kan derved have problemer med at gå uden hjælp, pga. balanceproblemer eller uopmærksomhed på den ene side af kroppen. \cite{DSfA2009, Sundhed.dk}

\subsection{Balance}
Som følge af apopleksi oplever nogle patienter problemer med balancen. Dette kan opfattes som både en sensorisk- og motorisk skade, da både kroppens sanser samt motorik hjælper til opretholdelse af balance. Balancen er vigtig for mennesket, da den opretholder kropsstillingen ved hjælp af ubevidste bevægelser, hvilket også er med til at bevægelse er muligt uden fald. For at opretholde balancen bliver kropsvægten så vidt mulig fordelt omkring kroppens akse og de vægtbærende legemer, herunder fødder i oprejst position og gluteal musklerne i siddende position.\cite{Nichols1997}

Balancen er et komplekst system, da flere forskellige kropssystemer samarbejder om at sende balanceinformation til hjernen, hvor de bearbejdes. Balancen styres f.eks. af proprioceptorer og sansereceptorer i øjne og øre. Proprioceptorerne kontrollerer muskler, sener og ledenes position - altså styrer proprioceptorerne de ubevidste bevægelser, som hjælper til balancen. \cite{Martini2012} Sansereceptorer opfanger sanseindtryk og videresender informationen til områder i cerebral cortex, cerebellum og til centre i hele hjernestammen. Disse områder bearbejder informationen for at konkludere den fysiske position af kroppen og dens lemmer. \cite{Martini2012,Karnath2003} Proprioceptorer og sansereceptorerne, samt hvor de findes, bliver uddybet nærmere i bilag \ref{app-Balance}, hvor der også er en anatomisk forklaring heraf.

Apopleksipatienter kan opleve problemer med balancen i den transversale retning i det frontale plan. Patienterne kan hænge mod deres syge side uden de er opmærksomme på det, hvilket besværliggør funktioner i dagligdagen og giver øget risiko for faldulykker. Et eksempel på, hvordan balancen påvirkes, er Pusher Syndrom. Dette er en lidelse, hvor halvsidigt lammede patienter aktivt skubber deres kropsvægt mod den lammede kropsside \cite{Karnath2003}. Lidelsen kan opstå som følge af både højre- og venstresidig hjerneskade. Patienter med Pusher Syndrom registrerer ikke, at deres krop hænger, hvilket kan være med til at besværliggøre funktioner i dagligdagen og give øget risiko for faldulykker i både stående, gående og siddende stilling. \cite{Karnath2003} \fxnote{evt. billede af en patient med pusher syndrom}

\subsection{Neglekt}
Ved indlæggelse er cirka 25\% af apopleksipatienterne ramt af neglekt \cite{Sundhedsstyrelsen2009}. Patienter med neglekt er ikke opmærksomme på den ene side af kroppen. Sygdommen kan forekomme på to måder; visuelt og kropsligt. Neglekt kan derfor opfattes som en sensorisk- eller motorisk skade\fxnote{http://gade.psy.ku.dk/bogkap/neglekt.htm - skal IKKE indsættes i kildeliste, da det ikke er en "pålidelig" kilde, men den er god for os at læse.}. De to typer forekommer ofte samtidig. Ved visuel neglekt mangler patienten sanseindtryk fra den påvirkede side af kroppen. Patienten er eksempelvis ikke opmærksom på den ene side af teksten, når vedkommende skal læse, selvom synet er normalt. Derudover kan patienten opleve kun at spise fra den ene del af tallerkenen, eftersom hjernen ikke registrerer den anden halvdel. Ved den kropslige neglekt har patienten manglende kropsbevidsthed. Patienten har ofte normal følelse i den syge side af kroppen - indtrykkene bemærkes, men registres ikke i hjernen. Dette kan komme til udtryk idet patienten glemmer at klæde den syge side af kroppen ordentligt på eller kun barbere halvdelen af ansigtet. En alvorlig følge af dette er, at patienten kan udføre ubevidst skade på sig selv, f.eks. ved at støde ind i ting med den syge side eller ikke være opmærksom på, at benene ikke kan bære kropsvægten. Derved kan der på længere sigt forekomme ergonomiske skader andre steder i kroppen. \cite{Sundhed.dk}

\subsection{Personlige følger}
Dette afsnit er baseret på hjerneskader generelt. Dvs. det ikke er sikkert, at apopleksi er årsagen, men det antages, at de samme udfordringer gør sig gældende hos personer, der får hjerneskader af apopleksi. Derudover skal det noteres, at det ikke er sikkert, at en apopleksiramt får en hjerneskade.

Personer der første gang rammes af en hjerneskade, beskriver hjerneskaden som et brud i deres liv, som de skal lære at forholde sig til. Derudover kan det tage tid for patienterne at indse, at de er ramt af en sygdom. Hjerneskaden går ind og influerer den ramtes humør, personlighed, færdigheder, aktiviteter samt sociale relationer. Patienterne er ikke i stand til at udføre de samme opgaver som tidligere, hvilket påvirker deres identitet. Kroppens funktionsændringer gør, at den ramte kommer til at leve et mere inaktivt og hjemmeorienteret liv end før. En yngre patient er mere ramt af denne forandring i forhold til en ældre patient. Dette kan bl.a. skyldes vanskeligheden ift. at opretholde sociale relationer og begå sig i hverdagen. Apopleksiramte kan derudover opleve en kropsspaltning, hvor kroppen opleves som et fremmet objekt. Et objekt, som kan være svært at styre og ikke gør, som patienten vil. \cite{Sundhedsstyrelsen2010}

Der findes skjulte vanskeligheder for patienter med hjerneskade. Disse omfatter vanskelighed med hukommelse, læsning, regning samt andre færdigheder. Disse skjulte vanskeligheder har også en indflydelse på patientens selvopfattelse og  kan derved være med til at nedsætte livskvaliteten for den enkelte. \cite{Sundhedsstyrelsen2010} 

\subsection{Kroppens kompenserende bevægelser}
Efter et slagtilfælde kompenserer kroppen for tabt funktionsevne med nye bevægelsesmønstre, der skal erstatte de gamle. Kompensatoriske bevægelser er et resultat af, at kroppen stadigvæk har brug for en givet funktion, men pga. tabt sensorisk- og motorisk funktion ikke kan udføre bevægelsen. Disse kompenserende bevægelsesmønstre medfører ikke blot et funktionelt dårligt resultat men er også associeret med langsigtede konsekvenser såsom smerte og reduceret funktionsevne. \cite{Takeuchi2012,Leea2009} F.eks. bevægelsesmønstre under en gangcyklus kan forekomme ukoordinerede og ukontrollerede ift. en gangcyklus foretaget af en rask person. Disse karakteristika er af tre typer og kan identificere gangmønstret for en patient efter et apopleksitilfælde  \cite{Lamontagne2006}:
\begin{itemize}
\item Type 1: Spastsitet, karakteriseres ved tidlig aktivitet af legmusklerne under standfasen.
\item Type 2: Paretisk, er nedsat aktivitet i fleste muskelgrupper.
\item Type 3: Co-aktivation af muskler, hvilket betyder, at muskelgrupper med forskellig effekt aktiveres på samme tid.
\end{itemize}

%\subsection{Identitet}
%En hjerneskadets identitet ændres, da patienten ikke er i stand til at udføre de samme opgaver som tidligere. Derfor bliver den hjerneskadede nødt til at skabe en ny identitet, hvilket for mange kan være svært. Kroppens funktionsændringer gør, at den ramte kommer til at leve et mere inaktivt og hjemmeorienteret liv end før. En yngre patient er mere ramt af denne forandring i forhold til en ældre patient. Apopleksiramte kan derudover opleve en kropsspaltning, hvor kroppen opleves som et fremmet objekt. Et objekt, som kan være svært at styre og ikke gør, som patienten vil.\cite{Sundhedsstyrelsen2010} 

%Der findes skjulte vanskeligheder for patienter med hjerneskade. Disse omfatter vanskelighed med hukommelse, læsning, regning samt andre færdigheder, der ikke er let synlige. Disse skjulte vanskeligheder har også en indflydelse på, hvordan patienten opfatter sig selv og kan være med til at nedsætte livskvaliteten for den enkelte.\cite{Sundhedsstyrelsen2010} 

%\subsection{Patienternes påvirkning}
%Alle de fysiske og mentale ændringer medfører, at det er svært for en hjerneskadet patient at vende tilbage til sit gamle hverdagsliv. Forandringerne gør det svært at udføre almindelige huslige pligter, såsom rengøring og personlig pleje. De ramte oplever det også som en svær oplevelse at vende tilbage på arbejde. Dette skyldes, udover de kropslige og mentale ændringer, også den træthed, der kan opleves. Det er derfor vigtigt at føle sig værdsat på jobbet. Den hjerneskadede patient skal vende tilbage til sine sociale relationer. Dette kan opleves som en meget hård opgave pga. de forandringer, kroppen har gennemgået. Det ses imidlertid, at familierelationerne bliver tættere, mens relationerne til vennerne bliver mindre. Dette er et problem, da gode relationer kan være med til at forbedre rehabiliteringsprocessen og dermed gøre, at den hjerneskadede patient hurtigere kan komme tilbage til et normalt liv.\cite{Sundhedsstyrelsen2010}

%Ud fra  det ovenstående kan det konkluderes, at hjerneskadede patienter, heriblandt apopleksiramte, oplever nedsat livskvalitet pga. deres sygdom. Dette kan også ses ved, at apopleksipatienter har dobbelt så stor selvmordsrate som baggrundsbefolkningen. Derudover nævner 16\% af apopleksi patienter, at deres livskvalitet er dårlig, 46\% syntes den er nogenlunde, mens 38\% synes den er god. Den nedsatte livskvalitet er noget der kan føre til vanskeligheder senere i livet, hvilket selvfølgelig skal forsøges undgået. En forbedret livskvalitet kan skabes ved hurtigere rehabilitering eller forbedret kropslig funktion, som den apopleksiramte patient mistede ved hjerneskaden.\cite{Sundhedsstyrelsen2010}