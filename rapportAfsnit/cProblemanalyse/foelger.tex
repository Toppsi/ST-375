\section{Følger af apopleksi }
Apopleksi kan forekomme pludseligt, og dermed uden at den ramte kan forberede sig på følgerne. Dette er modsat andre kroniske sygdomme, hvor progressionen ofte sker gradvist. Derfor kan der være flere alvorlige følger af en hjerneblødning eller blodprop som f.eks. depression. \cite{Muus2008} Hvilke følger patienten oplever afhænger af, hvilken del af hjernen, der rammes.\cite{Michael-Titus2010} Følgerne kan dog både gå ud over patientens fysiske og mentale tilstand. Der er flere væsentlige faktorer, som kan have en afgørende effekt på følgerne. Disse faktorer kan f.eks. være. tiden, hvor noget af hjernen ikke får ilt, størrelsen af blødningen, trykket i blodåren og hvor apopleksien rammer.\cite{Michael-Titus2010} \\
75.000 mennesker over 18 år levede i 2011 med følger efter et slagstilfælde \cite{Hjernesagen2015}. Dette tal forventes at være stigende i takt med, at der kommer flere ældre \cite{Sagen2014}. \fxnote{Det følgende er "fremtidsprognoser" fra samf. afsnittet}%Antallet der dør af hjerneskader har været stagneret de sidste 10 år før 2011, hvor 14 \% døde inden for 30 dage[3]. Det vil derfor kunne forventes, at der er flere, som kommer ud for en hjerneskade og vil have mén herefter, hvilket gør det vigtigt at fokusere på rehabiliteringen for at kunne genoptræne de forskellige kropslige og mentale mangler.

\subsection{Neglekt}
Ved indlæggelse er cirka 25\% af apopleksipatienterne ramt af neglekt\fxnote{Indsæt kilde - Sunhedsstyrelsens rapport}.
Patienter med neglekt er ikke opmærksomme på den ene side af kroppen. Neglekt kan forekomme på to måder; visuelt og kropsligt. De to typer forekommer ofte samtidig. Ved visuel neglekt mangler patienten sanseindtryk fra den påvirkede side af kroppen. F.eks. er der ikke opmærksomhed på den ene side af teksten, når patienten skal læse, selvom synet er normalt. Derudover kan patienten opleve kun at spise fra den ene del af tallerkenen. Ved den kropslige neglekt har patienten manglende kropsbevidsthed. Patienten har ofte normal følelse i den syge side af kroppen - indtrykkene bemærkes men registres ikke i hjernen. Patienten glemmer at klæde den syge side af kroppen ordentligt på eller barberer kun halvdelen af ansigtet. En alvorlig følge af dette er, at patienten kan udføre ubevidst skade på sig selv, f.eks. ved at støde ind i ting med den syge side eller ikke være opmærksom på, at benene ikke kan bære. Derved kan der på længere sigt forekomme ergonomiske skader andre steder i kroppen. \cite{Sundhed.dk}

\subsection{Balance}
Som følge af apopleksi oplever nogle patienter problemer med balancen. Balancen er vigtig for mennesker, da den opretholder kropstillingen ved hjælp af ubevidste bevægelser, hvilket også er med til at bevægelse er muligt uden fald. For at opretholde balancen bliver kropsvægten så vidt mulig fordelt omkring kroppens akse og de vægtbærende legemer, herunder fødder i oprejst position og gluteal musklerne i siddende position.\cite{Nichols1997}   
For apopleksipatienter opleves der ofte problemer med balancen i den transversale retning i det frontale plan. Patienterne hænger mod deres syge og svage side uden de er opmærksomme på det, hvilket besværliggør funktioner i dagligdagen og giver øget risiko for faldulykker. Et eksempel på, hvordan balancen påvirkes er ”Pusher Syndrom”. Da patienter med Pusher Syndrom ikke registrerer, at deres krop hænger, kan dette være med til at besværliggøre funktioner i dagligdagen og giver øget risiko for faldeulykker. Derudover har patienten ofte problemer med at læne sig fremad, samt risiko for faldeulykker i både stående, gående og siddende stilling. \cite{Karnath2003}

% Skal skrives sammen med det nye afnsit, som kommer til at omhandle balance.

\subsection{Sproglige, sensoriske og motriske skader}
De sproglige konsekvenser som følge af apopleksi, kaldes afasi. Afasi opleves efter en måned hos 20\% af de apopleksiramte og forekommer oftest ved skade i venstre temporal- og frontallap. De sproglige følger af apopleksi kan skade funktionen til at skrive og tale, men også evnen til at forstå og læse andres tale og skrift. I hvilken grad de sproglige funktioner er berørt kan variere mellem de enkelte patienter, da nogle oplever enkelte formuleringsproblemer, mens andre oplever global afasi. Global afasi gør, at de ramte i nogle tilfælde er helt ude af stand til at kommunikere verbalt eller kun kan fremsige enkelte ord med en vis usikkerhed omkring ordenes betydning.
De sproglige konsekvenser kan midlertidig bedres med tiden, og flere patienter opnår et kommunikationsniveau, som gør det muligt at begå sig i hverdagen.\cite{Muus2008}

Apopleksi kan skade sensoriske såvel som motoriske funktioner. Dette opleves som manglende funktion i arme, hænder, ben og fødder. De motoriske og sensoriske konsekvenser er de hyppigst forekommende følger hos apopleksiramte og kan medføre problemer med at udføre orienterede handlinger. Patienten kan således have problemer med forholdet mellem egen krop og ting omkring sig samt med kropdelenes indbyrdes forhold. De kan desuden have problemer med højre og venstre. Nogle patienter oplever at have problemer med at afstandsbedømme. De motoriske følger kan medføre upræcise bevægelser, generel stivhed i bevægelserne, problemer ved opstart af bevægelser, langsommere bevægelser, færre spontane bevægelser samt rystelser. Patienten kan derved risikere at være ude af stand til eller have store problemer med at gå uden hjælp.\cite{Sundhed.dk} \cite{DSfA2009}

% Ny mangler:
% Vise hvor det sker henne i hjernen (evt. når bogen er kommet)
% DER MANGLER NOGLE KILDER I DETTE AFSNIT, SOM JEG (CECILIE) IKKE KENDER NOGET TIL????


%Mangler:

% En visuel skala på, hvor mange der får de forskellige ting - sproglig skade, pusher-syndrom osv? (Hvis man kan finde det) Vise hvor det sker henne i hjernen?