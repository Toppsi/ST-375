\section{Rehabilitering}
Når selve slagtilfældet er stabiliseret og behandlet, er det essentielt, at rehabiliteringen af en apopleksipatient indfindes hurtigst muligt - gerne 1 til 2 dage efter slagtilfældet. I Danmark dækker rehabilitering af en patient med apopleksi områderne: direkte træning af funktioner, reorganisering af netværk (ufrivilligt - hjernens proces), kompenserende strategier, ændringer i miljø, social og psykologisk støtte. Genoptræningen omhandler dog ikke kun træning med en ergo- eller fysioterapeut, da plejepersonale til dagens almindelige gøremål også essentiel. Patientens daglige rutiner kan være gået tabt under slagtilfældet, hvorfor det er vigtigt, at få patienten tilbage i sit vante miljø. Plejepersonale skal hjælpe patienten til at genfinde denne rytme og hjælpe patienten til eventuelt at udføre dagligdags ting på en ny måde. Det kan ske, at patienten ikke længere er i stand til at beherske begge sine hænder til en opgave, hvorved plejepersonalet skal bistå patienten i indlæringen af kun at benytte en hånd. \\ 
Motoriske og sensoriske funktionsproblemer kan lede til balancebesvær for patienten i både siddende, stående og gående stilling. Der er afprøvet adskillige farmakologiske midler og behandlingsstadegier for at forbedre hjernens rehabilitering og motoriske funktioner. F.eks. er der afprøvet, at tildele apopleksipatienter det antidepressive middel Prozac (fluoxetin) i kombination med fysioterapi. Derudover er kortikal stimulation afprøvet, hvor området af hjernen, som kontrollerer motorstyring, modtager elektriske impulser fra en implanteret anordning. Denne mulighed har haft blandede succesoplevelser, men er udelukkende afprøvet på patienter, der har oplevet et alvorligt slagtilfælde. [1] \\  
Apopleksi patienten skal i samarbejde med lægen, sygeplejersken og andet hjælpepersonale opstille nogle mål for sin rehabilitering. Målene skal hverken være for svære eller for lette, så patienten ikke mister sin motivation til genoptræningen. [2] \\ 

\subsection{Forløbsprogram for rehabilitering} %Lav et flow diagram over de forskellige faser - done!.
Sundhedsstyrelsen har udarbejdet et forløbsprogram for rehabilitering af patienter med erhvervet hjerneskade. Forløbsprogrammet deler rehabiliteringen op i en række faser, der strækker sig fra at patienten erhverver hjerneskaden til at patienten har opnået bedst mulig funktionsevne, hvorefter der udføres kontrol og vedligeholdelse af funktionsevnen. Tidsperioden af rehabilitering varierer ift. hjerneskadens sværhedsgrad, samt sværhedsgraden af funktionstabet, dog kan perioden vare flere år. [5] \\
Den første fase i forløbsprogrammet foregår på sygehusets apopleksiafdeling. På apopleksiafdelingen foretages primært akut behandling for at begrænse skaderne. Når patientens sikkerhed er sikret og skaderne er begrænset påbegyndes den tidlige rehabilitering. Under den tidlige rehabilitering giver en speciallæge i neurologi en vurdering af patientens rehabiliteringsbehov. Derudover bliver der foretaget en vurdering af basale funktioner, samt iværksat træning i diverse bevægelsesfunktioner, basale egenskaber og kommunikationsfunktioner. [5] \\
I den anden fase gennemgår patienten rehabilitering på sygehuset, hvor der er fokus på de skadede funktioner. Her vurderes patientens behov for rehabilitering og rehabiliteringens udvikling. Hvis patienten vurderes til at have en stabil udvikling i rehabiliteringsprocessen, vil patienten blive udskrevet og påbegynde fase tre. [5] \\
I den tredje fase er patienten udskrevet fra sygehuset, og derved forgår rehabilitering som ambulant rehabilitering og selvstændig træning. Hvorvidt patienten skal vedblive rehabilitering på sygehuset, eller om patienten henvises til rehabilitering på kommunale rehabiliteringscentre afgøres af de vurderinger, som er blevet foretaget i anden fase. I første og anden fase af rehabiliteringsforløbet bliver patienten undervist og overvåget af fagkyndigt personale. Dette gøres for at sikre, at patienten udfører træningen korrekt f.eks. med bevægelsesmønstre, og korrigere patienten til at bevægelsen og øvelserne udføres korrekt. Dette er vigtigt, da patienten, som sagt i tredje fase, selv skal foretage den nødvendige træning og dermed har fornemmelse af, hvordan træningen udføres korrekt ift. bevægelsesmønstre og kropsholdning. Dette kan midlertidig være en udfordring for apopleksipatienter med neglekt, da de kan have problemer med balancen og opmærksomheden på kroppen. Patienten går derfor stadig til kontrol og vedligeholdelse for at sikre, at rehabiliteringens udvikling er stabil. Det kan i sidste ende have betydning for, hvor lang tid det tager for patienten at generhverve sine tabte funktioner. Den tredje fase varierer derfor fra patient til patient alt efter udviklingen på rehabiliteringen. [5]

\section{Biofeedback}
Biofeedback teknologien blev introduceret i slutningen af 1960. Denne teknologi gør det muligt for en patient at opnå bevidst kontrol over latent nervefunktion, samt forbedre rehabiliteringen. Dette sker ved et auditiv eller visuel tegn på, at deres bevægelse har aktiveret en neuromuskulær genvej. Biofeedback teknologien har bla. haft stor succes i behandling af apopleksi patienter med fækal- og urin inkontinens. Biofeedback kan også være gavnligt for en apopleksipatient med balanceproblemer, herunder pusher-syndrom forårsaget af følelsesløshed i den ene side af kroppen. Patienten har stadigvæk kontakt til hele sin krop, men kan ikke fornemme den ene halvdel. Hvis patienten underrettes om, at han/hun har forkert kropsholdning, som vil lede til mistet balance, kan patienten pga. et signal nå at rette op på dette, inden et uheld indtræffer. \\ 

Hvis en patient skal have gavn af biofeedback teknologien kræver det, at patienten har en kognitiv kapacitet til at følge instruktionerne under behandlingssessioner og fastholde læring fra session til session. Derudover kræves en neurologiske kapacitet til at genskabe frivillig kontrol.[3] \\
 
Der findes apparater og sensorer til at opfange et fysiologisk signal, der kan bruges til bevægelses-, styrke- og balancetræning, og som afgiver et biofeedback tilbage til patienten. Der findes f.eks. et hånd dynamometer, der kan benyttes til at måle styrkeforskellen i hhv. højre og venstre hånd. Dette kan være fordelagtigt at benytte for en apopleksipatient med balanceproblemer, herunder eksempelvis pusher-syndrom, da patienten derved gøres opmærksom på styrken i sin følelsesløse side. Herudover findes et smerte algometer, som kan gemme målinger af smerte, en trykplade, som kan måle fordelingen af en persons kropsvægt under forskellige øvelser, gyroskop, som kan måle accelerationen i en bestemt retning, accelerometer, som kan måle kropshældning, når sensoren er placeret på patienten, samt diverse elektroder, som f.eks. kan måle muskelaktivitet. Valget af hvilke apparater og sensorer der er fordelagtige afhænger af patientens tilstand f.eks. hørelse og følsomhed, samt sværhedsgraden af hjerneskaden og hvilke funktioner, der skal genoptrænes. [4] %mangler en kilde på, hvad de forskellige gør. - skal vi ikke have dette i en tabel?

%\begin{table}[Biofeedback]
%\centering
%\caption{My caption}
%\label{my-label}
%\begin{tabular}{lllll}
%{\bf Apparatur}  & {\bf Funktion}                                                                                                                                                                                                                                         %&  &  &  \\
%Hånd dynamometer & Måler styrkeforskellen i hhv. højre og venstre hånd. Dette kan være fordelagtigt at benytte for en %apopleksipatient med balanceproblemer, herunder eksempelvis pusher-syndrom, da patienten derved gøres opmærksom på %styrken i sin følelsesløse side. &  &  &  \\
%Smerte algometer & Målinger af smerte, en trykplade, som kan måle fordelingen af en persons kropsvægt under %forskellige øvelser                                                                                                                                           %&  &  &  \\
%Gyroskop         & Måler accelerationen i en bestemt retning                                                                                                                                                                                                              %&  &  &  \\
%Accelerometer    & Måler kropshældning, når sensoren er placeret på patienten, samt diverse elektroder, som f.eks. kan måle muskelaktivitet                                                                                                                               &  &  & 
%\end{tabular}
%\end{table}


%[1] - http://academic.eb.com.zorac.aub.aau.dk/EBchecked/topic/569347/stroke
%[2] - https://www.sundhed.dk/borger/sygdomme-a-aa/hjerte-og-blodkar/sygdomme/apopleksi/		   apopleksi-rehabilitering/
%[3] - http://link.springer.com.zorac.aub.aau.dk/article/10.1007/BF00999338    http://www.archives-pmr.org/article/S0003-9993(95)80503-6/abstract
%[4] - http://www.hmi-basen.dk/r4x.asp?linktype=iso&linkinfo=044824&P=1              
%[5] - Sundhedsstyrelsen, 2011. Sundhedsstyrelsen. Forløbsprogram for rehabilitering af voksne med erhvervet hjerneskade. Sundhedstyrrelsen, 2011.


%\section{Rehabilitering}
%Når selve slagtilfældet er stabiliseret og behandlet, er det essentielt, at rehabiliteringen af en apopleksi patient indfindes hurtigst muligt - gerne 1 til 2 dage efter slagtilfældet. Rehabiliteringen kan indebære fysisk, erhvervsmæssig eller tale terapi. Motoriske og sensoriske funktionsproblemer kan lede til balancebesvær for patienten i både siddende, stående og gående stilling. Der er afprøvet adskillige farmakologiske midler og behandlingsstadegier for at forbedre hjernens rehabilitering og motoriske funktioner. F.eks. er der afprøvet, at tildele apopleksi patienter det antidepressive middel Prozac (fluoxetin) i kombination med fysioterapi. Derudover er kortikal stimulation afprøvet, hvor området af hjernen, som kontrollerer motorstyring, modtager elektriske impulser fra en implanteret anordning. Denne mulighed har haft blandede succes oplevelser, men er udelukkende blevet afprøvet på patienter, der har oplevet et alvorligt slagtilfælde. \\ %http://academic.eb.com.zorac.aub.aau.dk/EBchecked/topic/569347/stroke                             % Hvor foregår rehabiliteringen ift. hvor slemt det er?
%Genoptræningen af en apopleksi patient i Danmark dækker områderne direkte træning af funktioner, reorganisering af netværk (ufrivviligt - hjernens proces), kompenserende strategier, ændringer i miljø, social og psykologisk støtte. Genoptræningen omhandler dog ikke kun træning med en ergo- eller fysioterapeut, da plejepersonale til dagens almindelige gøremål også spiller en stor rolle. Patientens daglige rutiner kan være gået tabt under slagtilfældet, og det er derfor utrolig vigtigt at få patienten tilbage i sit vante miljø. Plejepersonale skal hjælpe patienten til at genfinde denne rytme og hjælpe patienten til eventuelt at udføre dagligdags ting på en ny måde. Det kan ske, at patienten ikke længere er i stand til at beherske begge sine hænder til en opgave, hvorved plejepersonalet skal bistå patienten i indlæringen af kun at benytte en hånd. \\
%Apopleksi patienten skal i samarbejde med lægen, sygeplejersken og andet hjælpepersonale opstille nogle mål for sin rehabilitering. Målene skal hverken være for svære eller for lette, så patienten ikke mister sin motivation til genoptræningen. \\ %https://www.sundhed.dk/borger/sygdomme-a-aa/hjerte-og-blodkar/sygdomme/apopleksi/apopleksi-rehabilitering/

%indledning nr. 2
%Når selve slagtilfældet er stabiliseret og behandlet, er det essentielt, at rehabiliteringen af en apopleksi patient indfindes hurtigst muligt - gerne 1 til 2 dage efter slagtilfældet. Rehabiliteringen kan indebære fysisk, erhvervsmæssig eller tale terapi. Motoriske og sensoriske funktionsproblemer kan lede til balancebesvær for patienten i både siddende, stående og gående stilling. Der er afprøvet adskillige farmakologiske midler og behandlingsstadegier for at forbedre hjernens rehabilitering og motoriske funktioner. F.eks. er der afprøvet, at tildele apopleksi patienter det antidepressive middel Prozac (fluoxetin) i kombination med fysioterapi. Derudover er kortikal stimulation afprøvet, hvor området af hjernen, som kontrollerer motorstyring, modtager elektriske impulser fra en implanteret anordning. Denne mulighed har haft blandede succes oplevelser, men er udelukkende blevet afprøvet på patienter, der har oplevet et alvorligt slagtilfælde. [1] \\    
                         
%Rehabiliteringen af en apopleksi patient i Danmark dækker områderne direkte træning af funktioner, reorganisering af netværk (ufrivilligt - hjernens proces), kompenserende strategier, ændringer i miljø, social og psykologisk støtte. Genoptræningen omhandler dog ikke kun træning med en ergo- eller fysioterapeut, da plejepersonale til dagens almindelige gøremål også spiller en stor rolle. Patientens daglige rutiner kan være gået tabt under slagtilfældet, og det er derfor utrolig vigtigt, at få patienten tilbage i sit vante miljø. Plejepersonale skal hjælpe patienten til at genfinde denne rytme og hjælpe patienten til eventuelt at udføre dagligdags ting på en ny måde. Det kan ske, at patienten ikke længere er i stand til at beherske begge sine hænder til en opgave, hvorved plejepersonalet skal bistå patienten i indlæringen af kun at benytte en hånd. \\ 

% Kan de to overstående afsnit skrives sammen, så man starter mere bredt og derefter indskærper sig. Båske bytte om på rækkefælgen i det præsenterede? Husk at lave det gamle stå, hvis man forsøger sig med en omformulering. - 

%\section{Biofeedback}
%Biofeedback teknologien blev introduceret i slutningen af 1960. Denne teknologi gør det muligt for en patient at opnå bevidst kontrol over en voluntær men latent nervefunktion. Dette sker ved en auditiv eller visuel tegn på, at deres bevægelse har aktiveret en neuromuskulær genvej. Biofeedback teknologien har bla. haft stor succes i behandling af apopleksi patienter med fækal- og urin inkontinens. Teknologien kunne også være gavnligt for en apopleksi patient, som har udviklet pusher-syndrom, da dette skaber balanceproblemer pga. følelsesløshed i den ene side af kroppen. Patienten har altså stadig kontakt til hele sin krop, men kan ikke fornemme den ene halvdel. Hvis patienten blev underrettet om, at han/hun havde forkert kropsholdning, som ville lede til at miste balancen, kunne patienten nå at rette op på dette, inden et uheld vil indtræffe. \\
%Hvis en patient skulle have gang af biofeedback teknologien kræver det dog, at patienten har en kognitive kapacitet til at følge instruktionerne under behandlings sessioner og fastholde læring fra session til session. Derudover kræves en neurologiske kapacitet til at genskabe frivillig kontrol. \\ %   http://link.springer.com.zorac.aub.aau.dk/article/10.1007/BF00999338    http://www.archives-pmr.org/article/S0003-9993(95)80503-6/abstract
%Der findes apparater på markedet til bevægelses-, styrke- og balancetræning, som afgiver et biofeedback til patienten. Der findes f.eks. et hånd dynamometer, som kan benyttes til at måle styrkeforskellen i henholdsvis højre og venstre hånd. Dette kan være fordelagtigt at benytte for en apopleksi patient, som har udviklet pusher-syndrom, da patienten derved bliver opmærksom på styrken i sin følelsesløse side. Herudover findes et smerte algometer, som kan gemme målinger af smerte. %http://www.hmi-basen.dk/r4x.asp?linktype=iso&linkinfo=044824&P=1               % Vurder, om det med hånddynamometer er "to far off" og skal slettes.

% Bruges det kun på latente nervefunktioner? Skal ændres lidt, så det ikke kun er latente nerfunktioner, der snakkes om i biofeedback afsnittet. Udbyg mere her.
% Udbyg de forskellige rehabiliteringstyper - undersøg og vurder, om der kan skrives mere på? Gerne med underoverskrifter.
% Hvordan fungerer rehabiliteringen der hjemme?
% Vi skal have fokus på, hvor vi gerne vil hen. Få strikket den røde tråd (balance) ind i teksten, så fokus kommer på det.
% Mangler et overblik over, hvor lang tid rehabiliteringen tager - nogle eksempler på, hvor lang tid rehabiliteringen kan tage.
% Biofeedback afsnittet er måske lidt for specifikt på pusher-syndrom. Skal muligvis skrives om, så det bliver mere generelt med fokus på balance.
% Hvordan skal man give biofeedback til en ældre person, hvis de har dårlig hørelse og følelelse?

