\section{Rehabilitering}
Når selve slagtilfældet er stabiliseret og behandlet, er det essentielt, at rehabiliteringen af en apopleksi patient indfindes hurtigst muligt - gerne 1 til 2 dage efter slagtilfældet. Rehabiliteringen kan indebære fysisk, erhvervsmæssig eller tale terapi. Motoriske og sensoriske funktionsproblemer kan lede til balancebesvær for patienten i både siddende, stående og gående stilling. Der er afprøvet adskillige farmakologiske midler og behandlingsstadegier for at forbedre hjernens rehabilitering og motoriske funktioner. F.eks. er der afprøvet, at tildele apopleksi patienter det antidepressive middel Prozac (fluoxetin) i kombination med fysioterapi. Derudover er kortikal stimulation afprøvet, hvor området af hjernen, som kontrollerer motorstyring, modtager elektriske impulser fra en implanteret anordning. Denne mulighed har haft blandede succes oplevelser, men er udelukkende blevet afprøvet på patienter, der har oplevet et alvorligt slagtilfælde. \\ %http://academic.eb.com.zorac.aub.aau.dk/EBchecked/topic/569347/stroke                             % Hvor foregår rehabiliteringen ift. hvor slemt det er?
Genoptræningen af en apopleksi patient i Danmark dækker områderne direkte træning af funktioner, reorganisering af netværk (ufrivviligt - hjernens proces), kompenserende strategier, ændringer i miljø, social og psykologisk støtte. Genoptræningen omhandler dog ikke kun træning med en ergo- eller fysioterapeut, da plejepersonale til dagens almindelige gøremål også spiller en stor rolle. Patientens daglige rutiner kan være gået tabt under slagtilfældet, og det er derfor utrolig vigtigt at få patienten tilbage i sit vante miljø. Plejepersonale skal hjælpe patienten til at genfinde denne rytme og hjælpe patienten til eventuelt at udføre dagligdags ting på en ny måde. Det kan ske, at patienten ikke længere er i stand til at beherske begge sine hænder til en opgave, hvorved plejepersonalet skal bistå patienten i indlæringen af kun at benytte en hånd. \\
Apopleksi patienten skal i samarbejde med lægen, sygeplejersken og andet hjælpepersonale opstille nogle mål for sin rehabilitering. Målene skal hverken være for svære eller for lette, så patienten ikke mister sin motivation til genoptræningen. \\ %https://www.sundhed.dk/borger/sygdomme-a-aa/hjerte-og-blodkar/sygdomme/apopleksi/apopleksi-rehabilitering/

\section{Biofeedback}
Biofeedback teknologien blev introduceret i slutningen af 1960. Denne teknologi gør det muligt for en patient at opnå bevidst kontrol over en voluntær men latent nervefunktion. Dette sker ved en auditiv eller visuel tegn på, at deres bevægelse har aktiveret en neuromuskulær genvej. Biofeedback teknologien har bla. haft stor succes i behandling af apopleksi patienter med fækal- og urin inkontinens. Teknologien kunne også være gavnligt for en apopleksi patient, som har udviklet pusher-syndrom, da dette skaber balanceproblemer pga. følelsesløshed i den ene side af kroppen. Patienten har altså stadig kontakt til hele sin krop, men kan ikke fornemme den ene halvdel. Hvis patienten blev underrettet om, at han/hun havde forkert kropsholdning, som ville lede til at miste balancen, kunne patienten nå at rette op på dette, inden et uheld vil indtræffe. \\
Hvis en patient skulle have gang af biofeedback teknologien kræver det dog, at patienten har en kognitive kapacitet til at følge instruktionerne under behandlings sessioner og fastholde læring fra session til session. Derudover kræves en neurologiske kapacitet til at genskabe frivillig kontrol. \\ %   http://link.springer.com.zorac.aub.aau.dk/article/10.1007/BF00999338    http://www.archives-pmr.org/article/S0003-9993(95)80503-6/abstract
Der findes apparater på markedet til bevægelses-, styrke- og balancetræning, som afgiver et biofeedback til patienten. Der findes f.eks. et hånd dynamometer, som kan benyttes til at måle styrkeforskellen i henholdsvis højre og venstre hånd. Dette kan være fordelagtigt at benytte for en apopleksi patient, som har udviklet pusher-syndrom, da patienten derved bliver opmærksom på styrken i sin følelsesløse side. Herudover findes et smerte algometer, som kan gemme målinger af smerte. %http://www.hmi-basen.dk/r4x.asp?linktype=iso&linkinfo=044824&P=1               % Vurder, om det med hånddynamometer er "to far off" og skal slettes.

% Bruges det kun på latente nervefunktioner? Skal ændres lidt, så det ikke kun er latente nerfunktioner, der snakkes om i biofeedback afsnittet. Udbyg mere her.
% Udbyg de forskellige rehabiliteringstyper - undersøg og vurder, om der kan skrives mere på? Gerne med underoverskrifter.
% Hvordan fungerer rehabiliteringen der hjemme?
% Vi skal have fokus på, hvor vi gerne vil hen. Få strikket den røde tråd (balance) ind i teksten, så fokus kommer på det.
% Mangler et overblik over, hvor lang tid rehabiliteringen tager - nogle eksempler på, hvor lang tid rehabiliteringen kan tage.
% Biofeedback afsnittet er måske lidt for specifikt på pusher-syndrom. Skal muligvis skrives om, så det bliver mere generelt med fokus på balance.
% Hvordan skal man give biofeedback til en ældre person, hvis de har dårlig hørelse og følelelse?

